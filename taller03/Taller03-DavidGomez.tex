\documentclass{article}

% Librerías:
\usepackage{amsmath, mathtools, amssymb, mathrsfs, amsthm, nicematrix}
\usepackage{lmodern, graphicx, fancyhdr}
\usepackage[margin = 2cm, top = 2.5cm, includefoot]{geometry}
\usepackage{xcolor}
\usepackage[hidelinks]{hyperref}
\usepackage[T1]{fontenc}
\usepackage[spanish]{babel}
\usepackage{listings}
\usepackage{tikz}
\usetikzlibrary{shapes, snakes, fit, tikzmark}
\usepackage{color, colortbl}
\usepackage[table]{xcolor}
\usepackage[most]{tcolorbox}

% Configuraciones:
\pagestyle{fancy}
\fancyhf{}
\setlength{\headheight}{55.34027pt}
\rhead{\textit{David Gómez}}
\lhead{\includegraphics[width = 4cm]{\logo}}
\lfoot{Página \thepage}
\rfoot{Taller 02}
\renewcommand{\headrule}{\hbox to \headwidth{\color{rojoEci}\leaders\hrule height \headrulewidth\hfill}}
\renewcommand{\footrulewidth}{0.4pt}

\hyphenpenalty=10000

\newcommand{\logo}{C:/Users/usuario/Documents/U/logo-eci.jpg}

\newcommand{\q}[1]{``#1''}
\newcommand{\und}[2]{#1\textbf{\_}#2}

\newcommand\drawCodeBox[2]{%
\begin{tikzpicture}[remember picture,overlay]
  \coordinate (start) at ([yshift = 2ex]pic cs:#1);
  \coordinate (end) at ([yshift = -1.5ex]pic cs:#2);
  \node[inner sep=2pt,draw=black,fit=(start) (end)] {};
\end{tikzpicture}
}

\newcommand\drawCodeLine[2]{%
\begin{tikzpicture}[remember picture,overlay]
  \coordinate (start) at ([yshift = 2ex]pic cs:#1);
  \coordinate (end) at ([yshift = -1.5ex]pic cs:#2);
  \node[inner sep=0pt,draw=black,fit=(start) (end)] {};
\end{tikzpicture}
}

\tcbset{metateorema/.style={
    enhanced, title=title,
    attach boxed title to top
    left = {xshift = .5pt, yshift = -\tcboxedtitleheight -.5pt},
    top = 1pt,
    coltitle = black,
    colback = defini,
    colframe = defini,
    arc = 0mm,
    outer arc = 0mm,
    boxed title style={
        colback = white,
        colframe = deftitlmarg,
        boxrule = .5pt,
        arc = 0mm,
        outer arc = 0mm
    }
}}

% Colores
\definecolor{rojoEci}{RGB}{225, 70, 49}
\definecolor{defini}{HTML}{ede9e6}
\definecolor{deftitlmarg}{HTML}{cfcfcf}

% Documento
\begin{document}
\begin{titlepage}
    \begin{center}
        \vspace*{1cm}

        \textbf{\Huge{Taller 03}}

        \vspace{1.5cm}

        \textbf{\large{David Gómez}}

        \vspace{4cm}

        \includegraphics[width=\textwidth]{\logo}

        \vspace{5cm}

        Matemáticas\linebreak
        Escuela Colombiana de Ingeniería Julio Garavito\linebreak
        Colombia\linebreak
        \today

    \end{center}
\end{titlepage}
\clearpage
\tableofcontents
\clearpage

\section{Punto 1: tabla de verdad y función}
\subsection{$p = \phi$}
\begin{center}
    \begin{NiceTabular}{|c|}
        \hline
        $p$\\
        \hline
        \texttt{F}\\
        \texttt{T}\\
        \hline
    \end{NiceTabular}
\end{center}
\begin{alignat*}{2}
    &H_{\phi}(\mathtt{T}) = \mathtt{T}\\
    &H_{\phi}(\mathtt{F}) = \mathtt{F}
\end{alignat*}
\subsection{$(p \equiv r) = \phi$}
\begin{center}
    \begin{NiceTabular}{|c|c|c|}
        \hline
        $p$ & $r$ & $(p \equiv r)$\\
        \hline
        \ttfamily F   &   \ttfamily F  &    \ttfamily  T\\
        \ttfamily F   &   \ttfamily T   &   \ttfamily F\\
        \ttfamily T   &   \ttfamily F   &   \ttfamily F\\
        \ttfamily T   &   \ttfamily T   &   \ttfamily T\\
        \hline
    \end{NiceTabular}
\end{center}
\begin{alignat*}{2}
    & H_{\phi}(\mathtt{F}, \mathtt{T}) = H_{\phi}(\mathtt{T}, \mathtt{F}) = \mathtt{F}\\
    & H_{\phi}(\mathtt{F}, \mathtt{F}) = H_{\phi}(\mathtt{T}, \mathtt{T}) = \mathtt{F}
\end{alignat*}
\subsection{$((p \to (\lnot q)) \to r) = \phi$}
\begin{center}
    \begin{NiceTabular}{|c|c|c|c|c|c|}
        \hline
        $p$ & $q$ & $r$ & $(\lnot q)$ & $(p \to (\lnot q))$ & $((p \to (\lnot q)) \to r)$\\
        \hline
        \ttfamily F & \ttfamily F & \ttfamily F & \ttfamily T & \ttfamily T & \ttfamily F \\
        \ttfamily F & \ttfamily F & \ttfamily T & \ttfamily T & \ttfamily T & \ttfamily T \\
        \ttfamily F & \ttfamily T & \ttfamily F & \ttfamily F & \ttfamily T & \ttfamily F \\
        \ttfamily F & \ttfamily T & \ttfamily T & \ttfamily F & \ttfamily T & \ttfamily T \\
        \ttfamily T & \ttfamily F & \ttfamily F & \ttfamily T & \ttfamily T & \ttfamily F \\
        \ttfamily T & \ttfamily F & \ttfamily T & \ttfamily T & \ttfamily T & \ttfamily T \\
        \ttfamily T & \ttfamily T & \ttfamily F & \ttfamily F & \ttfamily F & \ttfamily T \\
        \ttfamily T & \ttfamily T & \ttfamily T & \ttfamily F & \ttfamily F & \ttfamily T \\
        \hline
    \end{NiceTabular}
\end{center}
\begin{alignat*}{2}
    & H_{\phi}(\mathtt{F}, \mathtt{F}, \mathtt{T}) = H_{\phi}(\mathtt{F}, \mathtt{T}, \mathtt{T}) = H_{\phi}(\mathtt{T}, \mathtt{F}, \mathtt{T}) = H_{\phi}(\mathtt{T}, \mathtt{T}, \mathtt{F}) = H_{\phi}(\mathtt{T}, \mathtt{T}, \mathtt{T})\\
    &H_{\phi}(\mathtt{F}, \mathtt{F}, \mathtt{F}) = H_{\phi}(\mathtt{F}, \mathtt{T}, \mathtt{F}) = H_{\phi}(\mathtt{T}, \mathtt{F}, \mathtt{F})
\end{alignat*}
\subsection{$((p \to q) \vee ((\lnot p) \to (\lnot q))) = \phi$}
\begin{center}
    \begin{NiceTabular}{|c|c|c|c|c|c|c|}
        \hline
        $p$ & $q$ & $(\lnot p)$ & $(\lnot q)$ & $(p \to q)$ & $((\lnot p) \to (\lnot q))$ & $((p \to q) \vee ((\lnot p) \to (\lnot q)))$\\
        \hline
        \ttfamily F & \ttfamily F & \ttfamily T & \ttfamily T & \ttfamily T & \ttfamily T & \ttfamily T\\
        \ttfamily F & \ttfamily T & \ttfamily T & \ttfamily F & \ttfamily T & \ttfamily F & \ttfamily T\\
        \ttfamily T & \ttfamily F & \ttfamily F & \ttfamily T & \ttfamily F & \ttfamily T & \ttfamily T\\
        \ttfamily T & \ttfamily T & \ttfamily F & \ttfamily F & \ttfamily T & \ttfamily T & \ttfamily T\\
        \hline
    \end{NiceTabular}
\end{center}
\begin{alignat*}{2}
    & H_{\phi}(\mathtt{F}, \mathtt{F}) = H_{\phi}(\mathtt{F}, \mathtt{T}) = H_{\phi}(\mathtt{T}, \mathtt{F}) = H_{\phi}(\mathtt{T}, \mathtt{T}) = \mathtt{T}
\end{alignat*}
\subsection{$(p \to (q \to p)) = \phi$}
\begin{center}
    \begin{NiceTabular}{|c|c|c|c|}
        \hline
        $p$ & $q$ & $(q \to p)$ & $(p \to (q \to p))$\\
        \hline
        \ttfamily F & \ttfamily F & \ttfamily T & \ttfamily T\\
        \ttfamily F & \ttfamily T & \ttfamily F & \ttfamily T\\
        \ttfamily T & \ttfamily F & \ttfamily T & \ttfamily T\\
        \ttfamily T & \ttfamily T & \ttfamily T & \ttfamily T\\
        \hline
    \end{NiceTabular}
\end{center}
\begin{alignat*}{2}
    & H_{\phi}(\mathtt{F}, \mathtt{F}) = H_{\phi}(\mathtt{F}, \mathtt{T}) = H_{\phi}(\mathtt{T}, \mathtt{F}) = H_{\phi}(\mathtt{T}, \mathtt{T}) = \mathtt{T}
\end{alignat*}
\subsection{$((p \vee r) \wedge (p \to q)) = \phi$}
\begin{center}
    \begin{NiceTabular}{|c|c|c|c|c|c|}
        \hline
        $p$ & $q$ & $r$ & $(p \vee r)$ & $(p \to q)$ & $((p \vee r) \wedge (p \to q))$\\
        \hline
        \ttfamily F & \ttfamily F & \ttfamily F & \ttfamily F & \ttfamily T & \ttfamily F\\
        \ttfamily F & \ttfamily F & \ttfamily T & \ttfamily T & \ttfamily T & \ttfamily T\\
        \ttfamily F & \ttfamily T & \ttfamily F & \ttfamily F & \ttfamily T & \ttfamily F\\
        \ttfamily F & \ttfamily T & \ttfamily T & \ttfamily T & \ttfamily T & \ttfamily T\\
        \ttfamily T & \ttfamily F & \ttfamily F & \ttfamily T & \ttfamily F & \ttfamily F\\
        \ttfamily T & \ttfamily F & \ttfamily T & \ttfamily T & \ttfamily F & \ttfamily F\\
        \ttfamily T & \ttfamily T & \ttfamily F & \ttfamily T & \ttfamily T & \ttfamily T\\
        \ttfamily T & \ttfamily T & \ttfamily T & \ttfamily T & \ttfamily T & \ttfamily T\\
        \hline
    \end{NiceTabular}
\end{center}
\begin{alignat*}{2}
    &H_{\phi}(\mathtt{F}, \mathtt{F}, \mathtt{F}) = H_{\phi}(\mathtt{F}, \mathtt{T}, \mathtt{F}) = H_{\phi}(\mathtt{T}, \mathtt{F}, \mathtt{F}) = H_{\phi}(\mathtt{T}, \mathtt{F}, \mathtt{T}) = \mathtt{F}\\
    &H_{\phi}(\mathtt{F}, \mathtt{F}, \mathtt{T}) = H_{\phi}(\mathtt{F}, \mathtt{T}, \mathtt{T}) = H_{\phi}(\mathtt{T}, \mathtt{T}, \mathtt{F}) = H_{\phi}(\mathtt{T}, \mathtt{T}, \mathtt{T}) = \mathtt{T}
\end{alignat*}
\subsection{$(\lnot((r \to (r \wedge (p \vee s))) \equiv (\lnot((p \to q) \vee (r \wedge (\lnot r)))))) = \phi$}

Debido a la longitud de la proposición, decidí añadir subíndices a los conectores, para así no tener que hacer mención de las variables que hacen parte del mismo:

$$(\lnot_{0}((r \to_{2} (r \wedge_{3} (p \vee_{4} s))) \equiv_{1} (\lnot_{2}((p \to_{4} q) \vee_{3} (r \wedge_{4} (\lnot_{5} r))))))$$

\begin{center}
    \begin{NiceTabular}{|c|c|c|c|c|}
        \hline
        Posición & $p$ & $q$ & $r$ & $s$\\
        \hline
        \ttfamily 0  & \ttfamily F & \ttfamily F & \ttfamily  F & \ttfamily F\\
        \ttfamily 1  & \ttfamily F & \ttfamily F & \ttfamily  F & \ttfamily T\\
        \ttfamily 2  & \ttfamily F & \ttfamily F & \ttfamily  T & \ttfamily F\\
        \ttfamily 3  & \ttfamily F & \ttfamily F & \ttfamily  T & \ttfamily T\\
        \ttfamily 4  & \ttfamily F & \ttfamily T & \ttfamily  F & \ttfamily F\\
        \ttfamily 5  & \ttfamily F & \ttfamily T & \ttfamily  F & \ttfamily T\\
        \ttfamily 6  & \ttfamily F & \ttfamily T & \ttfamily  T & \ttfamily F\\
        \ttfamily 7  & \ttfamily F & \ttfamily T & \ttfamily  T & \ttfamily T\\
        \ttfamily 8  & \ttfamily T & \ttfamily F & \ttfamily  F & \ttfamily F\\
        \ttfamily 9  & \ttfamily T & \ttfamily F & \ttfamily  F & \ttfamily T\\
        \ttfamily 10 & \ttfamily T & \ttfamily F & \ttfamily  T & \ttfamily F\\
        \ttfamily 11 & \ttfamily T & \ttfamily F & \ttfamily  T & \ttfamily T\\
        \ttfamily 12 & \ttfamily T & \ttfamily T & \ttfamily  F & \ttfamily F\\
        \ttfamily 13 & \ttfamily T & \ttfamily T & \ttfamily  F & \ttfamily T\\
        \ttfamily 14 & \ttfamily T & \ttfamily T & \ttfamily  T & \ttfamily F\\
        \ttfamily 15 & \ttfamily T & \ttfamily T & \ttfamily  T & \ttfamily T\\
        \hline
    \end{NiceTabular}
\end{center}
\begin{center}
    \begin{NiceTabular}{|c|c|c|c|c|c|c|c|c|c|c|}
        \hline
        Posición & $\lnot_{5}$ & $\vee_{4}$ & $\to_{4}$ & $\wedge_{4}$ & $\wedge_{3}$ & $\vee_{3}$ & $\to_{2}$ & $\lnot_{2}$ & $\equiv_{1}$ & $\lnot_{0}$\\
        \hline
        \ttfamily 0  &  \ttfamily T & \ttfamily F & \ttfamily T & \ttfamily F & \ttfamily F & \ttfamily T & \ttfamily T & \ttfamily F & \ttfamily F & \ttfamily T\\
        \ttfamily 1  &  \ttfamily F & \ttfamily T & \ttfamily T & \ttfamily F & \ttfamily F & \ttfamily T & \ttfamily T & \ttfamily F & \ttfamily F & \ttfamily T\\
        \ttfamily 2  &  \ttfamily T & \ttfamily F & \ttfamily T & \ttfamily F & \ttfamily F & \ttfamily T & \ttfamily F & \ttfamily F & \ttfamily T & \ttfamily F\\
        \ttfamily 3  &  \ttfamily F & \ttfamily T & \ttfamily T & \ttfamily F & \ttfamily T & \ttfamily T & \ttfamily T & \ttfamily F & \ttfamily F & \ttfamily T\\
        \ttfamily 4  &  \ttfamily T & \ttfamily F & \ttfamily T & \ttfamily F & \ttfamily F & \ttfamily T & \ttfamily T & \ttfamily F & \ttfamily F & \ttfamily T\\
        \ttfamily 5  &  \ttfamily F & \ttfamily T & \ttfamily T & \ttfamily F & \ttfamily F & \ttfamily T & \ttfamily T & \ttfamily F & \ttfamily F & \ttfamily T\\
        \ttfamily 6  &  \ttfamily T & \ttfamily F & \ttfamily T & \ttfamily F & \ttfamily F & \ttfamily T & \ttfamily F & \ttfamily F & \ttfamily T & \ttfamily F\\
        \ttfamily 7  &  \ttfamily F & \ttfamily T & \ttfamily T & \ttfamily F & \ttfamily T & \ttfamily T & \ttfamily T & \ttfamily F & \ttfamily F & \ttfamily T\\
        \ttfamily 8  &  \ttfamily T & \ttfamily T & \ttfamily F & \ttfamily F & \ttfamily F & \ttfamily F & \ttfamily T & \ttfamily T & \ttfamily T & \ttfamily F\\
        \ttfamily 9  &  \ttfamily F & \ttfamily T & \ttfamily T & \ttfamily F & \ttfamily F & \ttfamily F & \ttfamily T & \ttfamily T & \ttfamily T & \ttfamily F\\
        \ttfamily 10 &  \ttfamily T & \ttfamily T & \ttfamily T & \ttfamily F & \ttfamily T & \ttfamily F & \ttfamily T & \ttfamily T & \ttfamily T & \ttfamily F\\
        \ttfamily 11 &  \ttfamily F & \ttfamily T & \ttfamily F & \ttfamily F & \ttfamily T & \ttfamily F & \ttfamily T & \ttfamily T & \ttfamily T & \ttfamily F\\
        \ttfamily 12 &  \ttfamily T & \ttfamily T & \ttfamily T & \ttfamily F & \ttfamily F & \ttfamily T & \ttfamily T & \ttfamily F & \ttfamily F & \ttfamily T\\
        \ttfamily 13 &  \ttfamily F & \ttfamily T & \ttfamily T & \ttfamily F & \ttfamily F & \ttfamily T & \ttfamily T & \ttfamily F & \ttfamily F & \ttfamily T\\
        \ttfamily 14 &  \ttfamily T & \ttfamily T & \ttfamily T & \ttfamily F & \ttfamily T & \ttfamily T & \ttfamily T & \ttfamily F & \ttfamily F & \ttfamily T\\
        \ttfamily 15 &  \ttfamily F & \ttfamily T & \ttfamily T & \ttfamily F & \ttfamily T & \ttfamily T & \ttfamily T & \ttfamily F & \ttfamily F & \ttfamily T\\
        \hline
    \end{NiceTabular}
\end{center}
\begin{alignat*}{2}
    &H_{\phi}(\mathtt{F}, \mathtt{F}, \mathtt{F}, \mathtt{F}) = 
    H_{\phi}(\mathtt{F}, \mathtt{F}, \mathtt{F}, \mathtt{T}) = 
    H_{\phi}(\mathtt{F}, \mathtt{F}, \mathtt{T}, \mathtt{F}) = 
    H_{\phi}(\mathtt{F}, \mathtt{T}, \mathtt{F}, \mathtt{F}) = 
    H_{\phi}(\mathtt{F}, \mathtt{T}, \mathtt{F}, \mathtt{T}) =
    H_{\phi}(\mathtt{F}, \mathtt{T}, \mathtt{T}, \mathtt{T})\\ 
    &\hspace{2cm} = H_{\phi}(\mathtt{T}, \mathtt{T}, \mathtt{F}, \mathtt{F}) = 
    H_{\phi}(\mathtt{T}, \mathtt{T}, \mathtt{F}, \mathtt{T}) = 
    H_{\phi}(\mathtt{T}, \mathtt{T}, \mathtt{T}, \mathtt{F}) = 
    H_{\phi}(\mathtt{T}, \mathtt{T}, \mathtt{T}, \mathtt{T}) = \mathtt{T}\\[1cm]
    &H_{\phi}(\mathtt{F}, \mathtt{F}, \mathtt{T}, \mathtt{F}) =
    H_{\phi}(\mathtt{T}, \mathtt{F}, \mathtt{F}, \mathtt{F}) =
    H_{\phi}(\mathtt{T}, \mathtt{T}, \mathtt{F}, \mathtt{T}) =
    H_{\phi}(\mathtt{T}, \mathtt{F}, \mathtt{T}, \mathtt{F}) =
    H_{\phi}(\mathtt{T}, \mathtt{F}, \mathtt{T}, \mathtt{T}) = \mathtt{F}
\end{alignat*}
\section{Punto 2: Tabla de verdad}

$((p \vee (q \vee r)) \equiv ((p \vee q) \vee r))$
\begin{center}
    \begin{NiceTabular}{|c|c|c|c|c|c|c|c|}
        \hline
        $p$ & $q$ & $r$ & $(q \vee r)$ & $(p \vee q)$ & $(p \vee (q \vee r))$ & $((p \vee q) \vee r)$ & $((p \vee (q \vee r)) \equiv ((p \vee q) \vee r))$\\
        \hline
        \ttfamily F & \ttfamily F & \ttfamily F & \ttfamily F & \ttfamily F & \ttfamily F & \ttfamily F & \ttfamily T\\
        \ttfamily F & \ttfamily F & \ttfamily T & \ttfamily T & \ttfamily F & \ttfamily T & \ttfamily T & \ttfamily T\\
        \ttfamily F & \ttfamily T & \ttfamily F & \ttfamily T & \ttfamily T & \ttfamily T & \ttfamily T & \ttfamily T\\
        \ttfamily F & \ttfamily T & \ttfamily T & \ttfamily T & \ttfamily T & \ttfamily T & \ttfamily T & \ttfamily T\\
        \ttfamily T & \ttfamily F & \ttfamily F & \ttfamily F & \ttfamily T & \ttfamily T & \ttfamily T & \ttfamily T\\
        \ttfamily T & \ttfamily F & \ttfamily T & \ttfamily T & \ttfamily T & \ttfamily T & \ttfamily T & \ttfamily T\\
        \ttfamily T & \ttfamily T & \ttfamily F & \ttfamily T & \ttfamily T & \ttfamily T & \ttfamily T & \ttfamily T\\
        \ttfamily T & \ttfamily T & \ttfamily T & \ttfamily T & \ttfamily T & \ttfamily T & \ttfamily T & \ttfamily T\\
        \hline
    \end{NiceTabular}
\end{center}
\clearpage
\section{Punto 3: Justificar que la implicación no es asociativa}
\begin{alignat*}{2}
    (\forall \phi, \psi, \tau \in \mathbb{B} &: ((\phi \to \psi) \to \tau )\equiv (\phi \to (\psi \to \tau))) \tag*{propiedad a refutar}\\
    (\exists \phi, \psi, \tau \in \mathbb{B} &: \mathbf{v}[((\phi \to \psi) \to \tau )] \not= \mathbf{v}[(\phi \to (\psi \to \tau))]) \tag*{negación de la propiedad}\\
    \makebox[3cm]{\hrulefill}& \makebox[6.5cm]{\hrulefill}\\
    \mathbf{v}[((\phi \to \psi) \to \tau)] &= \mathtt{F} \tag*{suposición}\\
    \equiv\\
    (\mathbf{v}[(\phi \to \psi)] &= \mathtt{T}) \wedge (\mathbf{v}[\tau] = \mathtt{F}) \tag*{Metateorema 2.23}\\
    \equiv\\
    ((\mathbf{v}[\phi] = \mathtt{F}) &\vee (\mathbf{v}[\psi] = \mathtt{T})) \wedge (\mathbf{v}[\tau] = \mathtt{F})\\[0.4cm]
    \tikzmark{A}\text{Tomando } \mathbf{v} &= \{\phi  \mapsto \mathtt{F}, \psi \mapsto \mathtt{T}, \tau \mapsto \mathtt{F}\} \tikzmark{B}\\[0.4cm]
    \mathbf{v}[(\phi \to (\psi \to \tau))] &= H_{\to}[\mathtt{F}, \mathbf{v}[(\psi \to \tau)]]\\
    &= H_{\to}[\mathtt{F}, H_{\to}(\mathtt{T}, \mathtt{F})]\\
    &=H_{\to}[\mathtt{F}, \mathtt{F}]\\
    &= \mathtt{T}\\
    \makebox[3cm]{\hrulefill}& \makebox[6.5cm]{\hrulefill}\\
    \therefore (\exists \phi, \psi, \tau \in \mathbb{B} &: \mathbf{v}[((\phi \to \psi) \to \tau )] \not= \mathbf{v}[(\phi \to (\psi \to \tau))]) \tag*{por suposición}
\end{alignat*}
\drawCodeBox{A}{B}
\section{Punto 4: Tabla de verdad}

$(((p \to q) \wedge (\lnot \textrm{\textit{true}})) \equiv (r \vee q))$

\begin{center}
    \begin{NiceTabular}{|c|c|c|c|c|c|c|c|c|}
        \hline
        $p$ & $q$ & $r$ & \textit{true} & $(\lnot \textrm{\textit{true}})$ & $(p \to q)$ & $(r \vee q)$ & $((p \to q) \wedge (\lnot \textrm{\textit{true}}))$ & $(((p \to q) \wedge (\lnot \textrm{\textit{true}})) \equiv (r \vee q))$\\
        \hline
        \ttfamily F & \ttfamily F & \ttfamily F & \ttfamily T & \ttfamily F & \ttfamily T & \ttfamily F & \ttfamily F & \ttfamily T\\
        \ttfamily F & \ttfamily F & \ttfamily T & \ttfamily T & \ttfamily F & \ttfamily T & \ttfamily T & \ttfamily F & \ttfamily F\\
        \ttfamily F & \ttfamily T & \ttfamily F & \ttfamily T & \ttfamily F & \ttfamily T & \ttfamily T & \ttfamily F & \ttfamily F\\
        \ttfamily F & \ttfamily T & \ttfamily T & \ttfamily T & \ttfamily F & \ttfamily T & \ttfamily T & \ttfamily F & \ttfamily F\\
        \ttfamily T & \ttfamily F & \ttfamily F & \ttfamily T & \ttfamily F & \ttfamily F & \ttfamily F & \ttfamily F & \ttfamily T\\
        \ttfamily T & \ttfamily F & \ttfamily T & \ttfamily T & \ttfamily F & \ttfamily F & \ttfamily T & \ttfamily F & \ttfamily F\\
        \ttfamily T & \ttfamily T & \ttfamily F & \ttfamily T & \ttfamily F & \ttfamily T & \ttfamily T & \ttfamily F & \ttfamily F\\
        \ttfamily T & \ttfamily T & \ttfamily T & \ttfamily T & \ttfamily F & \ttfamily T & \ttfamily T & \ttfamily F & \ttfamily F\\
        \hline
    \end{NiceTabular}
\end{center}
\section{Punto 5: Nuevo conector}
\subsection{Definir $*$ en conectores lógicos clásicos}
\begin{alignat*}{2}
    H_{*}(\phi, \psi) &= H_{\lnot}[H_{\vee} (\phi , \psi)]\\
    \makebox[1.5cm]{\hrulefill}& \makebox[2.5cm]{\hrulefill}\\
    (\phi * \psi)  &\equiv (\lnot (\phi \vee \psi)) 
\end{alignat*}
\subsection{Hallar una proposición equivalente a $(\lnot p)$ usando únicamente $\{p, *\}$}
\begin{alignat*}{2}
    (\phi * \psi)  &\equiv (\lnot (\phi \vee \psi)) \tag*{sub-punto 1}\\
    (p \vee p) &\equiv p \tag*{tabla de verdad}\\
    \makebox[1.3cm]{\hrulefill}&\makebox[2.3cm]{\hrulefill}\\
    (p * p) &\equiv (\lnot p)
\end{alignat*}
\subsection{Hallar una proposición equivalente a $(p \wedge q)$ usando únicamente $\{p, q, *\}$}
\begin{alignat*}{2}
    (\phi * \psi)  &\equiv (\lnot (\phi \vee \psi)) \tag*{sub-punto 1}\\
    (p * p) &\equiv (\lnot p) \tag*{sub-punto 2}\\
    (\lnot (\phi \vee \psi)) &\equiv ((\lnot \phi) \wedge (\lnot \psi)) \tag*{tabla de verdad}\\
    \makebox[2.5cm]{\hrulefill}&\makebox[3cm]{\hrulefill}\\
    ((p * p) * (q * q)) &\equiv (p \wedge q)
\end{alignat*}
\subsection{Justificar o refutar propiedades  de $*$}

\begin{tcolorbox}[metateorema, title = Metateorema 2.23 para $*$]
    \vspace*{1cm}
    $\mathbf{v}[(\phi * \psi)] = \mathtt{T}$ si y solo si $\mathbf{v}[\phi] = \mathtt{F}$ y $\mathbf{v}[\psi] = \mathtt{F}$ ; de lo contrario $\mathbf{v}[(\phi * \psi)] = \mathtt{F}$
\end{tcolorbox}

\subsubsection{Conmutativa}
\begin{alignat*}{2}
    (\forall \phi, \psi \in \mathbb{B} &: \mathbf{v}[(\phi * \psi)] = \mathbf{v}[(\psi * \phi)]) \tag*{def. propiedad conmutativa}\\
    \mathbf{v}[(\phi * \psi)] &= \mathtt{T} \tag*{suposición}\\
    (\mathbf{v}[\phi] = \mathtt{F}) &\wedge (\mathbf{v}[\psi] = \mathtt{F})\\
    (\mathbf{v}[\psi] = \mathtt{F}) &\wedge (\mathbf{v}[\phi] = \mathtt{F}) \tag*{Propiedad conmutativa de $\wedge$}\\
    \mathbf{v}[(\psi * \phi)] &= \mathtt{T} \tag*{Metateorema 2.23}\\
    \makebox[2.5cm]{\hrulefill}&\makebox[4.5cm]{\hrulefill}\\
    (\forall \phi, \psi \in \mathbb{B} &: \mathbf{v}[(\phi * \psi)] = \mathbf{v}[(\psi * \phi)]) \tag*{Es conmutativa}
    \intertext{Se puede ver que cualquier otra valuación cumple la propiedad (negar el metatorema y aplicar propiedad conmutativa sobre $\vee$)}
\end{alignat*}

\subsubsection{Asociativa}
\begin{alignat*}{2}
    (\forall \phi, \psi, \tau \in \mathbb{B} &: \mathbf{v}[(\phi * \psi) * \tau] = \mathbf{v}[(\phi * (\psi	 * \tau))]) \tag*{def. propiedad asociativa}\\
    \mathbf{v}[(\phi * \psi) * \tau] &= \mathtt{T} \tag*{suposición}\\
    (\mathbf{v}[(\phi * \psi)] = \mathtt{F}) &\wedge (\mathbf{v}[\tau] = \mathtt{F})\\
    (\mathbf{v}[\phi] = \mathtt{T} \vee \mathbf{v}[\psi] = \mathtt{T}) &\wedge (\mathbf{v}[\tau] = \mathtt{F})\\
    \intertext{Tomando $\mathbf{v} = \{\phi \mapsto \mathtt{T}, \psi \mapsto \mathtt{T}, \tau \mapsto \mathtt{F}\}$}
    \mathbf{v}[(\phi * (\psi * \tau))] &= H_{*}[\phi, (\psi * \tau)]\\
    &= H_{*}[\mathtt{T}, H_{*}(\mathtt{T}, \mathtt{F})]\\
    &= H_{*}[\mathtt{T}, \mathtt{F}]\\
    &= \mathtt{F}\\
    \makebox[3cm]{\hrulefill}&\makebox[6cm]{\hrulefill}\\
    \therefore (\exists \phi, \psi, \tau \in \mathbb{B} &: \mathbf{v}[(\phi * \psi) * \tau] \not= \mathbf{v}[(\phi * (\psi * \tau))]) \tag*{No es asociativa}
\end{alignat*}

\section{Punto 6: Nuevo conector}
\subsection{Definir $*$ en conectores lógicos clásicos}
\begin{alignat*}{2}
    H_{*}(\phi, \psi) &= H_{\lnot}[H_{\gets}(\phi, \psi)]\\
    \makebox[1.5cm]{\hrulefill}&\makebox[2.5cm]{\hrulefill}\\
    (\phi * \psi) &\equiv (\lnot (\phi \gets \psi))
\end{alignat*}

\subsection{Justificar o refutar propiedades de $*$}
\begin{tcolorbox}[metateorema, title = Metateorema 2.23 para $*$]
        \vspace*{1cm}
        $\mathbf{v}[(\phi * \psi) = \mathtt{T}$ si y solo si $\mathbf{v}[\phi] = \mathtt{F}$ y $\mathbf{v}[\psi] = \mathtt{T}$ ; de lo contrario $\mathbf{v}[(\phi * \psi)] = \mathtt{F}$]
\end{tcolorbox}
\subsubsection{Asociativa}

\begin{alignat*}{2}
    (\forall \phi, \psi, \tau \in \mathbb{B} &: \mathbf{v}[(\phi * \psi) * \tau] = \mathbf{v}[(\phi * (\psi	 * \tau))]) \tag*{def. propiedad asociativa}\\
    \mathbf{v}[(\phi * \psi) * \tau] &= \mathtt{T} \tag*{suposición}\\
    (\mathbf{v}[\phi] = \mathtt{F}) &\wedge (\mathbf{v}[\psi] = \mathtt{T})\\
    (\mathbf{v}[\psi] = \mathtt{T}) &\wedge (\mathbf{v}[\phi] = \mathtt{F}) \tag*{Propiedad conmutativa de $\wedge$}\\
    \mathbf{v}[(\psi * \phi)] &= \mathtt{F}\\
    \makebox[2.5cm]{\hrulefill}&\makebox[5.5cm]{\hrulefill}\\
    \therefore (\exists \phi, \psi, \tau \in \mathbb{B} &: \mathbf{v}[(\phi * \psi) * \tau] \not= \mathbf{v}[(\phi * (\psi * \tau))]) \tag*{No es asociativa}
\end{alignat*}

\subsubsection{Transitiva}

\begin{alignat*}{2}
    \tikzmark{A}(\forall \phi, \psi, \tau \in \mathbb{B} &: ((\mathbf{v}[(\phi * \psi)] = \mathtt{T}) \wedge (\mathbf{v}[(\psi * \tau)] = \mathtt{T})) \Rightarrow (\mathbf{v}[(\phi * \tau)] = \mathtt{T}))\\
    &\wedge\\ 
    (\forall \phi, \psi, \tau \in \mathbb{B} &: ((\mathbf{v}[(\phi * \psi)] = \mathtt{F}) \wedge (\mathbf{v}[(\psi * \tau)] = \mathtt{F})) \Rightarrow (\mathbf{v}[(\phi * \tau)] = \mathtt{F}))\tag*{def. propiedad transitiva}\tikzmark{B}\\[0.3cm]
    (\mathbf{v}[(\phi * \psi)] = \mathtt{T}) &\wedge (\mathbf{v}[(\psi * \tau)] = \mathtt{T})\tag*{suposición}\\
    ((\mathbf{v}[\phi] = \mathtt{F}) \wedge (\mathbf{v}[\psi] = \mathtt{T})) &\wedge ((\mathbf{v}[\psi] = \mathtt{F}) \wedge (\mathbf{v}[\tau] = \mathtt{T}))\\
    (\mathbf{v}[\phi] = \mathtt{F}) &\wedge \textrm{\textit{false}} \wedge (\mathbf{v}[\tau] = \mathtt{T}) \tag*{Propiedad asociativa de $\wedge$}\\
    &\textrm{\textit{false}}\\
    \makebox[4cm]{\hrulefill}&\makebox[9.5cm]{\hrulefill}\\
    \tikzmark{C}(\exists \phi, \psi, \tau \in \mathbb{B} &: ((\mathbf{v}[(\phi * \psi)] = \mathtt{T}) \wedge (\mathbf{v}[(\psi * \tau)] = \mathtt{T})) \not\Rightarrow (\mathbf{v}[(\phi * \tau)] = \mathtt{T}))\\
    &\vee\\ 
    (\exists \phi, \psi, \tau \in \mathbb{B} &: ((\mathbf{v}[(\phi * \psi)] = \mathtt{F}) \wedge (\mathbf{v}[(\psi * \tau)] = \mathtt{F})) \not\Rightarrow (\mathbf{v}[(\phi * \tau)] = \mathtt{F}))\tag*{No es transitiva}\tikzmark{D}
\end{alignat*}
\drawCodeBox{A}{B}
\drawCodeBox{C}{D}

\section{Punto 7: considere las valuaciones $\mathbf{v}$ y $\mathbf{w}$ \dots}
\begin{alignat*}{2}
    \mathbf{v} &= \{p \mapsto \mathtt{T},\ q \mapsto \mathtt{F},\ r \mapsto \mathtt{F},\ \dots\}\\
    \mathbf{w} &= \{p \mapsto \mathtt{T},\ q \mapsto \mathtt{F},\ r \mapsto \mathtt{T},\ \dots\}
\end{alignat*}

Demostrar que $\mathbf{v}[(p \equiv (\lnot q))] = \mathbf{w}[(p \equiv (\lnot q))]$

\begin{alignat*}{2}
    \intertext{Tomando $\mathbf{v}$ ...}
    \mathbf{v}[(p \equiv (\lnot q))] &= H_{\equiv}[p, (\lnot q)]\\
    &= H_{\equiv}[\mathbf{v}[p], H_{\lnot}(\mathbf{v}[q])]\\
    &= H_{\equiv}[\mathbf{w}[p], H_{\lnot}(\mathbf{w}[q])]\\
    &= \mathbf{w}[(p \equiv (\lnot q))]
\end{alignat*}

\section{Punto 8: Demuestre que $\mathbf{v}[\phi] \not= \mathbf{v}[(\lnot \phi)]$ para cualquier valuación $\mathbf{v}$}

\begin{alignat*}{2}
    \mathbf{v}[(\phi \equiv (\lnot \phi))] &= \mathtt{F} \tag*{Enunciado/ proposición a probar}\\
    \mathbf{v}[\phi] &\not= \mathbf{v}[(\lnot \phi)] \tag*{negación de Metateorema 2.23}\\[-.4cm]
    \makebox[2.5cm]{\hrulefill}&\makebox[1.9cm]{\hrulefill}\\
    \therefore \mathbf{v}[(\phi \equiv (\lnot \phi))] &= \mathtt{F}
\end{alignat*}

\end{document}
