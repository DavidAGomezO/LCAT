\documentclass[twoside]{article}

% Packages
\usepackage{mathtools}
\usepackage{amssymb}
\usepackage{logicDG}
\usepackage{array}
\usepackage{xcolor}
\usepackage[spanish]{babel}
\usepackage{geometry}
\usepackage{fancyhdr}
\usepackage{graphicx}
\usepackage[hidelinks]{hyperref}
\usepackage{forest}

% Font
\usepackage{lmodern}
\usepackage[T1]{fontenc}

% Page configurations

\geometry{
    a4paper,
    margin = 2.5cm,
    top = 4cm,
    bottom = 2.5cm,
    headheight = 72pt
}
\newcommand{\logo}{C:/Users/usuario/Documents/U/logo-eci.jpg}
\renewcommand{\author}{David Gómez}
\renewcommand{\title}{Tarea 12}

\pagestyle{fancy}
\fancyhf{}
\fancyhead[LO]{\author}
\fancyhead[LE]{\title}
\fancyhead[R]{\includegraphics[width = 4cm]{\logo}}
\fancyfoot[C]{Página \thepage}
\renewcommand{\headrule}{\hbox to \headwidth{\color{rojoEci}\leaders\hrule height \headrulewidth\hfill}}

\hyphenpenalty = 10000

\setlength{\parindent}{0pt}

\newcommand{\sect}[1]{\section*{#1} \addcontentsline{toc}{section}{#1}}
\newcommand{\subsect}[1]{\subsection*{#1} \addcontentsline{toc}{subsection}{#1}}
\newcommand{\subsubsect}[1]{\subsubsection*{#1} \addcontentsline{toc}{subsubsection}{#1}}

\newcommand{\eq}{$=$}
% Colors
\definecolor{rojoEci}{RGB}{225, 70, 49}

% Documento

\begin{document}
\begin{titlepage}
    \begin{center}
        \vspace*{1cm}
 
        \textbf{\fontsize{45}{\baselineskip}\selectfont{\title}}

        \vspace{4cm}

        {\Large Hecho por}

        \vspace{1cm}

        {\textbf{\LARGE\MakeUppercase{\author}}}

        \vspace{2cm}

        \includegraphics[width = .8\textwidth]{\logo}

        \vspace{2cm}

        {\Large Estudiante de Matemáticas\\[5pt]

        Escuela Colombiana de Ingeniería Julio Garavito\\[5pt]

        Colombia\\[5pt]

        \today}
             
    \end{center}
\end{titlepage}

\tableofcontents
\clearpage

\sect{Sección 6.5}
\subsect{Punto 1}
\begin{logicenv}[5]{$F = \{x_0 \mapsto f(x_2), x_1 \mapsto h(x_0, x_3), x_8 \mapsto x_7\}$}
    \begin{align*}
        F(t)    & = F[h(x_0, g(x_1, x_0, x_2))]\\
                & = h(F[x_0], F[g(x_1, x_0, x_2)])\\
                & = h(f(x_2), g(F[x_1], F[x_0], F[x_2]))\\
                & = h(f(x_2, g(h(x_0, x_3), f(x_2), x_2)))
    \end{align*}
\end{logicenv}

\subsect{Punto 2}
\subsubsect{a}
\begin{logicenv}[5]{Árbol de sintaxis de $\phi$}
    \begin{center}
        \begin{forest}
            [
                $\land$
                [
                    $B$
                    [
                        $g$
                        [$x_0$]
                        [$x_1$]
                    ]
                    [
                        $f$
                        [$x_1$]
                    ]
                ]
                [\textit{true}]
            ]
        \end{forest}
    \end{center}
\end{logicenv}

\subsubsect{b}
\begin{logicenv}[5]{Árbol de sintaxis con sustitución asociada de $\phi$}
    \begin{center}
        \begin{forest}
            [
                $\land$
                [
                    $B$
                    [
                        $g$
                        [$x_0$]{
                            \draw[dotted] (-1.8, -2.85) -- (-2.3, -2.85) node[anchor = east] {$f(x_2)$};
                        }
                        [$x_1$]{
                            \draw[dotted] (-0.75, -3) -- (-0.75, -3.5) node[anchor = north] {$h(x_0, x_3)$};
                        }
                    ]
                    [
                        $f$
                        [$x_1$]{
                            \draw[dotted] (0.4, -2.85) -- (0.9, -2.85) node[anchor = west] {$h(x_0, x_3)$};
                        }
                    ]
                ]
                [\textit{true}]
            ]
            % \draw[help lines] (0,0) grid (-3,-5);
        \end{forest}
    \end{center}
\end{logicenv}

\subsubsect{c}
\begin{logicenv}[5]{Árbol de sintaxis de la sustitución de $\phi$}
    \begin{center}
        \begin{forest}
            [
                $\land$
                [
                    $B$
                    [
                        $g$
                        [
                            $f$
                            [$x_2$]
                        ]
                        [
                            $h$
                            [$x_0$]
                            [$x_3$]
                        ]
                    ]
                    [
                        $f$
                        [
                            $h$
                            [$x_0$]
                            [$x_3$]
                        ]
                    ]
                ]
                [\textit{true}]
            ]
        \end{forest}
    \end{center}
\end{logicenv}

\subsect{Punto 3}
\subsubsect{a}
\begin{logicenv}[5]{Árbol de sintaxis}
    \begin{center}
        \begin{forest}
            [
                $\equiv$
                [
                    $\forall x_0$
                    [
                        $P$
                        [$x_0$]
                        [
                            $f$
                            [$x_1$]
                        ]
                    ]
                ]
                [
                    $\land$
                    [
                        $Q$
                        [$x_1$]
                        [
                            $\exists x_1$
                            [
                                $P$
                                [$x_2$]
                                [$x_0$]
                            ]
                        ]
                    ]
                ]
            ]
        \end{forest}
    \end{center}
\end{logicenv}

\subsubsect{b}
\begin{logicenv}[5]{Árbol de sintaxis con la sustitución asociada}
    \begin{center}
        \begin{forest}
            [
                $\equiv$
                [
                    $\forall x_0$
                    [
                        $P$
                        [$x_0$]
                        [
                            $f$
                            [$x_1$]
                        ]
                    ]
                ]
                [
                    $\land$
                    [
                        $Q$
                        [$x_1$]{
                            \draw[dotted] (-0.7, -3.8) -- (-1.2, -3.8) node[anchor = east] {$h(x_0, x_3)$};
                         }
                        [
                            $\exists x_1$
                            [
                                $P$
                                [$x_2$]
                                [$x_0$]{
                                    \draw[dotted] (2, -4.8) -- (2.5, -4.8) node[anchor = west] {$f(x_2)$};
                                }
                            ]
                        ]
                    ]
                ]
            ]
            % \draw[help lines] (3,0) grid (-3,-5);
        \end{forest}
    \end{center}
\end{logicenv}

\subsubsect{c}
\begin{logicenv}[5]{Árbol de sintaxis de la sustitución}
    \begin{center}
        \begin{forest}
            [
                $\equiv$
                [
                    $\forall x_0$
                    [
                        $P$
                        [$x_0$]
                        [
                            $f$
                            [
                                $h$
                                [$x_0$]
                                [$x_3$]
                            ]
                        ]
                    ]
                ]
                [
                    $\land$
                    [
                        $Q$
                        [$x_1$]
                        [
                            $\exists x_1$
                            [
                                $P$
                                [$x_2$]
                                [
                                    $f$
                                    [$x_2$]
                                ]
                            ]
                        ]
                    ]
                ]
            ]
        \end{forest}
    \end{center}
\end{logicenv}

\subsect{Punto 4}
\subsubsect{a}
\begin{logicenv}[5]{$\phi = \forall x_2 (P(x_1, x_2) \to P(x_2, c))$}
    \[\forall x_1 (P(x_1, x_2) \to P(x_2, c))\]
\end{logicenv}

\subsubsect{b}
\begin{logicenv}[5]{$\phi = \forall x_2 P(x_1, x_2) \to P(x_2, c)$}
    \[\forall x_2 P(x_1, x_2) \to P(f(x_1, x_2), c)\]
\end{logicenv}

\subsubsect{c}
\begin{logicenv}[5]{$\phi = Q(x_3) \to \neg \forall x_1 \forall x_2 R(x_1, x_2, c)$}
    \[Q(x_3) \to \neg \forall x_1 \forall x_2 R(x_1, x_2, c)\]
\end{logicenv}

\subsubsect{d}
\begin{logicenv}[5]{$\phi = \forall x_1 Q(x_1) \to \forall x_2 P(x_1, x_2)$}
    \[\forall x_1 Q(x_1) \to \forall x_2 P(x_1, x_2)\]
\end{logicenv}

\subsubsect{e}
\begin{logicenv}[5]{$\phi = \forall x_2 (P(f(x_1, x_2), x_1) \equiv \forall x_1 S(x_3, g(x_1, x_2)))$}
    \[\forall x_2 (P(f(x_1, x_2), x_1) \equiv \forall x_1 S(x_3, g(x_1, x_2)))\]
\end{logicenv}


\subsect{Punto 5}
\subsubsect{a}
$t$ es libre

\subsubsect{b}
$t$ es libre

\subsubsect{c}
$t$ es libre

\subsubsect{d}
$t$ es libre

\subsubsect{e}
$t$ es acotado

\subsect{Punto 6}
\subsubsect{a}
\begin{logicenv}[5]{$\phi = \forall x_2 (P(x_2, f(x_1, x_2)) \lor Q(x_1))$}
    \begin{align*}
        \forall x_2 (P(x_2, f(f(x_1, x_2), x_2)) \lor Q(f(x_1, x_2)))\\
        t \text{ es acotado}
    \end{align*}
\end{logicenv}

\subsubsect{b}
\begin{logicenv}[5]{$\phi = \forall x_2 P(x_2, f(x_1, x_2)) \lor Q(x_1)$}
    \begin{align*}
        \forall x_2 P(f(f(x_1, x_2), x_2), x_2) \lor Q(f(x_1, x_2))\\
        t \text{ es acotada y luego libre}
    \end{align*}
\end{logicenv}

\subsubsect{c}
\begin{logicenv}[5]{$\phi = \forall x_1 \forall x_3 (Q(x_3) \not\equiv Q(x_1))$}
    \begin{align*}
        \forall x_1 \forall x_3 (Q(x_3) \not\equiv Q(x_1))\\
        t \text{ es libre}
    \end{align*}
\end{logicenv}

\subsubsect{d}
\begin{logicenv}[5]{$\phi = \forall x_1 \forall x_3 Q(x_3) \not\equiv Q(x_1)$}
    \begin{align*}
        \forall x_1 \forall x_3 Q(x_3) \not\equiv Q(f(x_1, x_2))\\
        t \text{ es libre}
    \end{align*}    
\end{logicenv}

\subsubsect{e}
\begin{logicenv}[5]{$\phi = \forall x_2 R(x_1, g(x_1), x_2) \to \forall x_3 Q(f(x_1, x_3))$}
    \begin{align*}
        \forall x_2 R(f(x_1, x_2), g(f(x_1, x_2)), x_2) \to \forall x_3 Q(f(f(x_1, x_2), x_3))\\
        t \text{ es acotada y luego libre}
    \end{align*}
\end{logicenv}

\subsect{Punto 10}
\begin{logicenv}[5]{Si $x$ no ocurre libre en $\phi$, entonces $t$ es libre de $x$ en $x$ en $\phi$}
    \begin{logic}
        x \text{ no ocurre libre en } \phi\\%0
        x \text{ es acotada en } \phi & p(0)\\%1
        x \text{ no es sustituida para ningun} F\\%2
        t \text{ es libre de $x$ puesto que no se realiza la sustitución}
    \end{logic}
\end{logicenv}

\subsect{Punto 11}
\begin{logicenv}[5]{$x$ es libre para $x$ en $\phi$}
    \begin{logic}
        \text{Suposición 1}\\
        \text{Suposición 2}\\
        x \text{ es libre de } x \text{ en } \phi & p1, p0
    \end{logic}
\end{logicenv}
\begin{subproof}{Suposición 1}
    \begin{logic}
        x \text{ es libre en } \phi\\
        \phi[x := x] \ x\text{ es libre de } x
    \end{logic}
\end{subproof}
\begin{subproof}{Suposición 2}
    \begin{logic}
        x \text{ es acotada en } \phi\\
        \text{ No se hace la sustitución a } x\\
        x \text{ es libre de} x
    \end{logic}
\end{subproof}

\sect{Sección 6.6}

\subsect{Punto 2}
\subsubsect{a}
No es
\subsubsect{b}
\begin{logicenv}[5]{Árbol de sintaxis}
    \begin{center}
        \begin{forest}
            [
                \texttt{read}
                [$a$]
                [$0$]
            ]
        \end{forest}
    \end{center}
\end{logicenv}

\subsubsect{c}
\begin{logicenv}[5]{Árbol de sintaxis}
    \begin{center}
        \begin{forest}
            [
                \texttt{read}
                [$a$]
                [$0$]
            ]
        \end{forest}
    \end{center}
\end{logicenv}

\subsubsect{d}
\begin{logicenv}[5]{Árbol de sintaxis}
    \begin{center}
        \begin{forest}
            [
                \texttt{read}
                [$a$]
                [$i$]
            ]
        \end{forest}
    \end{center}
\end{logicenv}

\subsubsect{e}
\begin{logicenv}[5]{Árbol de sintaxis}
    \begin{center}
        \begin{forest}
            [
                \texttt{read}
                [$a$]
                [$i$]
            ]
        \end{forest}
    \end{center}
\end{logicenv}

\subsect{Punto 9}
\subsubsect{a}
\begin{logicenv}[5]{$\forall i : I (a[i] = x)$}
    \begin{center}
        \begin{forest}
            [
                $\forall i : I$
                [
                    \eq
                    [
                        \texttt{read}
                        [$a$]
                        [$i$]
                    ]
                    [$x$]
                ]
            ]
        \end{forest}
    \end{center}
\end{logicenv}

\subsubsect{b}
\begin{logicenv}[5]{$\forall i : I \leq i < \texttt{len}(a) \to a[i] = x$}
    \begin{center}
        \begin{forest}
            [
                $\forall i : I$
                [
                    $\to$
                    [
                        $\leq$
                        [$0$]
                        [
                            $<$
                            [$i$]
                            [
                                \texttt{len}
                                [$a$]
                            ]
                        ]
                    ]
                    [
                        \eq
                        [
                            \texttt{read}
                            [$a$]
                            [$i$]
                        ]
                        [$x$]
                    ]
                ]
            ]
        \end{forest}
    \end{center}
\end{logicenv}

\subsubsect{c}
\begin{logicenv}[5]{$\forall b: A (b[i] = x)$}
    \begin{center}
        \begin{forest}
            [
                $\forall b: A$
                [
                    \eq
                    [
                        \texttt{read}
                        [$b$]
                        [$i$]
                    ]
                    [$x$]
                ]
            ]
        \end{forest}
    \end{center}
\end{logicenv}

\subsubsect{d}
\begin{logicenv}[5]{\texttt{len} $(a) > 1 \land \exists j : I \neg(a[0] \cdot a[j] = a[0])$}
    \begin{center}
        \begin{forest}
            [
                $\land$
                [
                    $>$
                    [
                        \texttt{len}
                        [$a$]
                    ]
                    [$1$]
                ]
                [
                    $\exists j : I$
                    [
                        $\neg$
                        [
                            \eq
                            [
                                $\cdot$
                                [
                                    \texttt{read}
                                    [$a$]
                                    [$0$]
                                ]
                                [
                                    \texttt{read}
                                    [$a$]
                                    [$0$]
                                ]
                            ]
                            [
                                \texttt{read}
                                [$a$]
                                [$0$]
                            ]
                        ]
                    ]
                ]
            ]
        \end{forest}
    \end{center}
\end{logicenv}

\subsubsect{e}
\begin{logicenv}[5]{$(\forall i:I \,\vert\, 0 < i < \texttt{len}(a) : a[i - 1] \leq a[i])$}
    \begin{center}
        \begin{forest}
            [
                $\forall i:I$
                [
                    $\to$
                    [
                        $<$
                        [$0$]
                        [
                            $<$
                            [$i$]
                            [
                                \texttt{len}
                                [$a$]
                            ]
                        ]
                    ]
                    [
                        $\leq$
                        [
                            \texttt{read}
                            [$a$]
                            [$i - 1$]
                        ]
                        [
                            \texttt{read}
                            [$a$]
                            [$i$]
                        ]
                    ]
                ]
            ]
        \end{forest}
    \end{center}
\end{logicenv}

\subsect{Punto 11}
\subsubsect{a}
\begin{logicenv}[5]{El arreglo $a$ es decreciente}
    \[a:A \land (\forall i:I \,\vert\, 0 \leq i < \texttt{len}(a) - 1 : a[i] > a[i + 1])\]
\end{logicenv}

\subsubsect{b}
\begin{logicenv}[5]{Los arreglos $a$ y $b$ son distintos}
    \[a, b: A \land (\neg(\texttt{len}(a) = \texttt{len}(b)) \lor (\forall i:I \,\vert\, 0 \leq i < \texttt{len}(a) : \neg(a[i] = b[i])))\]
\end{logicenv}

\subsubsect{c}
\begin{logicenv}[5]{El arreglo $a$ no tiene puntos fijos}
    \[a: A \land (\forall i:I \,\vert\, 0 \leq i < \texttt{len}(a) : \neg(i = a[i]))\]
\end{logicenv}

\subsubsect{d}
\begin{logicenv}[5]{El arreglo $a$ no tiene elementos repetidos}
    \[a:A \land (\forall i, j : I \,\vert\, 0 \leq i < \texttt{len}(a) \land 0 \leq j < \texttt{len}(a) \land \neg(i = j): \neg(a[i] = a[j]))\]
\end{logicenv}

\subsubsect{e}
\begin{logicenv}[5]{El arreglo $a$ es la identidad}
    \begin{align*}
        a:A \land (\forall i:I \,\vert\, 0 \leq i < \texttt{len}(a): a[i] = 0)\\
        a:A \land \texttt{len}(a) = 1 \land a[0] = 1
    \end{align*}
\end{logicenv}

\subsect{Punto 12}
\subsubsect{a}
\begin{logicenv}[5]{$\exists x : V (\texttt{len}(a) = 2 \cdot x + 1)$}
    Existe una variable tal que el doble de su valor más 1 es igual al número de elementos en $a$
\end{logicenv}

\subsubsect{b}
\begin{logicenv}[5]{$\forall i:I \,\vert\, 0 \leq i < \texttt{len}(a) : a[i] = a[0]$}
    Todos los valores de a son iguales
\end{logicenv}

\subsubsect{c}
\begin{logicenv}[5]{$a[1] = 5 \land (\forall i:I \,\vert\, 0 \leq i < \texttt{len}(a) : a[i] = a[0])$}
    El segundo valor de $a$ es $2$ y todos los valores de $a$ son iguales
\end{logicenv}

\subsubsect{d}
\begin{logicenv}[5]{$(\exists i:I \,\vert\, 0 \leq i < \texttt{len}(a) : a[i] = -a[i])$}
    Hay un elemento en $a$ que es igual a $0$
\end{logicenv}

\subsect{15}
\subsubsect{a}
\begin{logicenv}[5]{a}
    El arreglo es creciente no monótono
\end{logicenv}

\subsubsect{b}
\begin{logicenv}[5]{b}
    $a:A$
\end{logicenv}

\subsect{Punto 16}
\subsubsect{a}
Sí
\subsubsect{b}
\begin{verbatim}
    [1, 2, 1] ; [0, 1, 0] ; [0, 0, 0] son
    [1, 2, 3] ; [4, 5, 6] ; [7, 8, 9] no son
\end{verbatim}

\subsubsect{c}
\begin{logicenv}[5]{\texttt{pal}$(a)$}
    \[\texttt{pal}(a) := (\forall i:I \,\vert\, 0 \leq i < \texttt{len}(a) : a[i] = a[\texttt{len}(a) - 1 - i])\]
\end{logicenv}

\subsect{Punto 17}
\subsubsect{a}
Sí
\subsubsect{b}
\begin{verbatim}
    [1, 2, 1, 0, 1, 0] ; [0, 1, 0, 1, 2, 1] ; [0, 0, 0, 2, 3, 2] son
    [1, 2, 3] ; [4, 5, 6] ; [7, 8, 9] no son
\end{verbatim}

\subsubsect{c}
\begin{logicenv}[5]{\texttt{al}$(a)$}
    \begin{align*}
        \texttt{al}(a) :=& \forall i:I \exists j, k : I (0 \leq i \leq j < k < \texttt{len}(a) \land b, c\\
        &:A \land b[i] = a[i] \land c = a[k:\texttt{len}(a) - 1] \to a[i] = a[j - i] \to \texttt{pal}[b] \land \texttt{pal}(c))
    \end{align*}    
\end{logicenv}
\end{document}