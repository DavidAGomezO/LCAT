\documentclass[twoside]{article}

% Packages
\usepackage{mathtools}
\usepackage{amssymb}
\usepackage{logicDG}
\usepackage{array}
\usepackage{xcolor}
\usepackage[spanish]{babel}
\usepackage{geometry}
\usepackage{fancyhdr}
\usepackage{graphicx}
\usepackage[hidelinks]{hyperref}
\usepackage{titlesec}

% Fuente
\usepackage{lmodern}
\usepackage[T1]{fontenc}

% Configuraciones de página

\geometry{
    a4paper,
    margin = 2.5cm,
    top = 4cm,
    bottom = 2.5cm,
    headheight = 72pt
}
\newcommand{\logo}{C:/Users/usuario/Documents/U/logo-eci.jpg}
\renewcommand{\author}{David Gómez}
\renewcommand{\title}{Taller 15}

\pagestyle{fancy}
\fancyhf{}
\fancyhead[LO]{\author}
\fancyhead[LE]{\title}
\fancyhead[R]{\includegraphics[width = 4cm]{\logo}}
\fancyfoot[C]{Página \thepage}
\renewcommand{\headrule}{\hbox to \headwidth{\color{rojoEci}\leaders\hrule height \headrulewidth\hfill}}

\hyphenpenalty = 10000

\setlength{\parindent}{0pt}

\newcommand{\sect}[1]{\section*{#1} \addcontentsline{toc}{section}{#1}}
\newcommand{\subsect}[1]{\subsection*{#1} \addcontentsline{toc}{subsection}{#1}}
\newcommand{\subsubsect}[1]{\subsubsection*{#1} \addcontentsline{toc}{subsubsection}{#1}}


% Colores
\definecolor{rojoEci}{RGB}{225, 70, 49}

% Documento

\begin{document}
\begin{titlepage}
    \begin{center}
        \vspace*{1cm}
 
        \textbf{\fontsize{45}{\baselineskip}\selectfont{\title}}

        \vspace{4cm}

        {\Large Hecho por}

        \vspace{1cm}

        {\textbf{\LARGE\MakeUppercase{\author}}}

        \vspace{2cm}

        \includegraphics[width = .8\textwidth]{\logo}

        \vspace{2cm}

        {\Large Estudiante de Matemáticas\\[5pt]

        Escuela Colombiana de Ingeniería Julio Garavito\\[5pt]

        Colombia\\[5pt]

        \today}
             
    \end{center}
\end{titlepage}

\tableofcontents
\newpage
\sect{Punto 1}
\begin{logicenv}[10]{$limit(f, 0)$}
    Tomando el procedimiento \hyperref[proc741]{7.4.1}
    \[\theo{\left\{\epsilon > 0, n > \frac{1}{\epsilon}\right\}}{\dsl}{\left|f(n) - 0\right| < \epsilon}\]
    \begin{derivation}
            \left|f(n) - 0\right|\\
        Def.($f$)\\
            \left|\frac{1}{n + 1} - 0\right|\\
        Aritmética\\
            \frac{1}{n + 1}\\
        Álgebra\\
            0 < \frac{1}{n + 1} < \frac{1}{n}\\
        $n > m \land m =\frac{1}{\epsilon}$\\
            0 < \frac{1}{n + 1} < \frac{1}{n} < \frac{1}{m} = \epsilon\\
        Transitividad($<$)
            \textit{true}
    \end{derivation}
\end{logicenv}

\sect{Punto 2}
\begin{logicenv}[10]{$limit(f, 1)$}
    Tomando el procedimiento \hyperref[proc741]{7.4.1}
    \[\theo{\left\{\epsilon > 0, n > \frac{1}{\epsilon}\right\}}{\dsl}{\left|f(n) - 1\right| < \epsilon}\]
    \begin{equation*}
        \boxed{f(n) = \frac{n}{n + 1} = 1 - \frac{1}{n + 1}}
    \end{equation*}
    \begin{derivation}
            \left|1 - \frac{1}{n + 1} - 1\right|\\
        Aritmética\\
            \frac{1}{n + 1}\\
        Álgebra\\
            0 < \frac{1}{n + 1} < \frac{1}{n}\\
        $n > m \land m = \frac{1}{\epsilon}$\\
            0 < \frac{1}{n + 1} < \frac{1}{n} < \frac{1}{m} = \epsilon\\
        Transitividad($<$)\\
            \textit{true}
    \end{derivation}
\end{logicenv}

\sect{Punto 3}
\begin{logicenv}[10]{$limit(f, 0)$}
    Tomando el procedimiento \hyperref[proc741]{7.4.1}
    \[\theo{\left\{\epsilon > 0, n > \frac{1}{\epsilon^2}\right\}}{\dsl}{\left|f(n) - 0\right| < \epsilon}\]
    \begin{derivation}
            \left|\frac{1}{\sqrt{n + 1}} - 0 \right|\\
        Aritmética\\
            \frac{1}{\sqrt{n + 1}}\\
        Álgebra\\
            0 < \frac{1}{\sqrt{n + 1}} < \frac{1}{\sqrt{n}}\\
        $n > m \land m = \frac{1}{\epsilon^2}$\\
            0 < \frac{1}{\sqrt{n + 1}} < \frac{1}{\sqrt{n}} < \frac{1}{\sqrt{m}} = \epsilon\\
        Transitividad($<$)\\
            \textit{true}
    \end{derivation}
\end{logicenv}

\sect{Punto 4}
\begin{logicenv}[10]{$limit(f, 1)$}
    Tomando el procedimiento \hyperref[proc741]{7.4.1}
    \[\theo{\left\{\epsilon > 0, n > \frac{1}{\epsilon} + 1\right\}}{\dsl}{\left|f(n) - 1\right| < \epsilon}\]
    \begin{derivation}
            \left|\frac{1}{n - 1}\right|\\
        Aritmética\\
            \frac{1}{n - 1}\\
        Álgebra\\
            0 < \frac{1}{n - 1}\\
        $n > m \land m = \frac{1}{\epsilon} + 1$\\
            0 < \frac{1}{n - 1} < \frac{1}{m - 1} = \epsilon\\
        Transitividad($<$)\\
            \textit{true}
    \end{derivation}
\end{logicenv}

\sect{Punto 5}
\begin{logicenv}[10]{$limit(f, 0)$}
    Tomando el procedimiento \hyperref[proc741]{7.4.1}
    \[\theo{\left\{\epsilon > 0, n > e^{1/\epsilon} + 1\right\}}{\dsl}{\left|f(n) - 0\right| < \epsilon}\]
    \begin{derivation}
            \left|\frac{1}{\ln(n)} - 0\right|\\
        Aritmética\\
            \frac{1}{\ln(n)}\\
        Álgebra\\
            0 < \frac{1}{\ln(n)}\\
        $n > m \land m = e^{1/\epsilon}$\\
            0 < \frac{1}{\ln(n)} < \frac{1}{\ln(m)} = \epsilon
    \end{derivation}
\end{logicenv}

\sect{Punto 6}
\begin{logicenv}[10]{$limit(f, 1)$}
    Tomando el procedimiento \hyperref[proc741]{7.4.1}
    \[\theo{\left\{\epsilon > 0, n > \frac{\epsilon}{3} + 1\right\}}{\dsl}{\left|f(n) - 1\right| < \epsilon}\]
    \begin{equation*}
        \boxed{f(n) = \frac{n^2}{(n + 1)^2} = 1 - \frac{2n + 1}{(n + 1)^2}}
    \end{equation*}
    \begin{derivation}
            \left|1 - \frac{2n + 1}{(n + 1)^2}\right|\\
        Aritmética\\
            \frac{2n + 1}{(n + 1)^2}\\
        Álgebra\\
            0 < \frac{2n + 1}{(n + 1)^2} < \frac{2n + 1}{n + 1} < \frac{2n + 1}{n} < 3n\\
        $n > m \land m = \frac{\epsilon}{3}$\\
            0 < \frac{2n + 1}{(n + 1)^2} < \frac{2n + 1}{n + 1} < \frac{2n + 1}{n} < 3n < 3m = \epsilon\\
        Transitividad($<$)\\
            \textit{true}
    \end{derivation}
\end{logicenv}
\sect{Procedimientos}
\phantomsection
\label{proc741}
\begin{logicenv}{7.4.1}
    \[limit(f, 0) \equiv (\forall \epsilon \in \mathbb{R}\,\vert\, \epsilon > 0: (\exists m \in \mathbb{R} \,\vert\, m \geq 0 : (\forall n \in \mathbb{N} \,\vert\, n > m : |f(n) - 0| < \epsilon)))\]
    Por metateorema 7.22, demostrar
    \[\theo{\{\epsilon > 0\}}{\dsl}{(\exists m \in \mathbb{R} \,\vert\, m \geq 0 : (\forall n \in \mathbb{N} \,\vert\, n > m |f(n) - 0| < \epsilon))}\]

    \begin{align*}
            \res{(\exists m \in \mathbb{R} \,\vert\, m \geq 0 : (\forall n \in \mathbb{N} \,\vert\, n > m : |f(n) - 0| < \epsilon))}\\
        \why{Azúcar sintáctico}\\
            \res{\exists m \in \mathbb{R} (m \geq 0 \land (\forall n \in \mathbb{N} \,\vert\, n > m : |f(n) - 0| < \epsilon))}\\
        \why[\Leftarrow]{instanciación con testigo $a$}\\
            \res{a \geq 0 \land (\forall n \in \mathbb{N} \,\vert\, n > a |f(n) - 0| < \epsilon)}\\
        \why{$a \geq 0 \equiv \textit{true}$, Identidad($\land$)}\\
            \res{(\forall n \in \mathbb{N} \,\vert\, n > a |f(n) - 0| < \epsilon)}
    \end{align*}
    Por metateorema 7.22, demostrar
    \[\theo{\{\epsilon > 0, n > a\}}{\dsl}{|f(n) - 0| < \epsilon}\]
\end{logicenv}
\end{document}