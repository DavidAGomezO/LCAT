\documentclass[twoside]{article}

% Packages
\usepackage{mathtools}
\usepackage{amssymb}
\usepackage{logicDG}
\usepackage{array}
\usepackage{xcolor}
\usepackage[spanish]{babel}
\usepackage{geometry}
\usepackage{fancyhdr}
\usepackage{graphicx}
\usepackage[hidelinks]{hyperref}

% Font
\usepackage{lmodern}
\usepackage[T1]{fontenc}

% Page configurations

\geometry{
    a4paper,
    margin = 2.5cm,
    top = 4cm,
    bottom = 2.5cm,
    headheight = 72pt
}
\newcommand{\logo}{C:/Users/usuario/Documents/U/logo-eci.jpg}
\renewcommand{\author}{David Gómez}
\renewcommand{\title}{Taller 14}

\pagestyle{fancy}
\fancyhf{}
\fancyhead[LO]{\author}
\fancyhead[LE]{\title}
\fancyhead[R]{\includegraphics[width = 4cm]{\logo}}
\fancyfoot[C]{Página \thepage}
\renewcommand{\headrule}{\hbox to \headwidth{\color{rojoEci}\leaders\hrule height \headrulewidth\hfill}}

\hyphenpenalty = 10000

\setlength{\parindent}{0pt}

\newcommand{\sect}[1]{\section*{#1} \addcontentsline{toc}{section}{#1}}
\newcommand{\subsect}[1]{\subsection*{#1} \addcontentsline{toc}{subsection}{#1}}
\newcommand{\subsubsect}[1]{\subsubsection*{#1} \addcontentsline{toc}{subsubsection}{#1}}

% Colors
\definecolor{rojoEci}{RGB}{225, 70, 49}

% Documento

\begin{document}
\begin{titlepage}
    \begin{center}
        \vspace*{1cm}
 
        \textbf{\fontsize{45}{\baselineskip}\selectfont{\title}}

        \vspace{4cm}

        {\Large Hecho por}

        \vspace{1cm}

        {\textbf{\LARGE\MakeUppercase{\author}}}

        \vspace{2cm}

        \includegraphics[width = .8\textwidth]{\logo}

        \vspace{2cm}

        {\Large Estudiante de Matemáticas\\[5pt]

        Escuela Colombiana de Ingeniería Julio Garavito\\[5pt]

        Colombia\\[5pt]

        \today}
             
    \end{center}
\end{titlepage}

\tableofcontents
\clearpage

\sect{Punto 1}
\begin{logicenv}{Si las leyes no existen\dots}
    \begin{gather*}
        L(x) := \text{``$x$ es una ley''}\\
        P(x) := \text{``$x$ está permitido''}\\
        M(x) := \text{``$x$ es una norma moral''}
    \end{gather*}
    \begin{logic}
        \neg (\exists x \,\vert:\, L(x)) \to (\forall y \,\vert:\, P(y)) & Suposición\\
        \neg (\exists x \,\vert:\, L(x)) \to \neg (\exists y \,\vert:\, M(x)) & Suposición\\
        (\exists x \,\vert:\, M(x)) & Suposición\\
        (\exists x\,\vert:\, L(x)) & MTT(p2, p1)
    \end{logic}
\end{logicenv}

\sect{Punto 2}
\begin{logicenv}{Pedro es maestro\dots}
    \begin{gather*}
        p := \text{``Pedro''}\\
        M(x) :=  \text{``$x$ es maestro''}\\
        S(x) := \text{``$x$ es sabio''}\\
        G(x) := \text{``$x$ es generoso''}\\
        P(x) := \text{``$x$ es persona''}\\
        \mathbb{P} = \{x \,\vert\, P(x)\}
    \end{gather*}
    \begin{logic}
        M(p) & Suposición\\ %0
        (\forall x \,\vert\, M(x) : S(x) \land G(x)) & Suposición\\ %1
        (\forall x \,\vert\, M(x) : x \in \mathbb{P}) & Suposición\\ %2
        (\forall x \,\vert\, M(x) : S(x) \land G(x)) \to (M(p) \to S(p) \land G(p)) & Bx4\\ %3
        M(p) \to S(p) \land G(p) & MPP(p3, p1)\\ %4
        S(p) \land G(p) & MPP(p4, p0)\\ %5
        (\forall x \,\vert\, M(x) : x \in \mathbb{P}) \to (M(p) \to p \in \mathbb{P}) & Bx4\\ %6
        (M(p) \to p \in \mathbb{P}) & MPP(p6, p2)\\ %7
        p \in \mathbb{P} & MPP(p7, p0)\\ %8
        (\exists x \,\vert\, x \in \mathbb{P} : G(x)) & (p8, p5)
    \end{logic}
\end{logicenv}

\sect{Punto 3}
\begin{logicenv}[5]{Lancelot ama a la Reina\dots}
    \begin{gather*}
        \text{Lancelot} : l\\
        \text{Reina Ginebra} : g\\
        F(x, y) : \text{$x$ es amigo de $y$}\\
        \text{Rey Arturo} : a\\
        L(x, y): \text{$x$ ama a $y$}
    \end{gather*}
    \begin{logic}
        L(l, g) & Suposición\\ %0
        (\forall x \,\vert\, F(x, l) : \neg L(l, x)) & Suposición\\ %1
        F(a, l) & Suposición\\ %2
        (\forall x, y \,\vert\, L(l, x) \land F(y, l) : \neg L(y, x)) & Suposición\\ %3
        (\forall x, y \,\vert\, L(l, x) \land F(y, l) : \neg L(y, x)) \to (L(l, g) \land F(a, l) \to \neg L(a, g)) & Bx4\\ %4
        L(l, g) \land F(a, l) \to \neg L(a, g) & MPP(p4, p3)\\ %5
        L(l, g) \land F(a, l) & Unión(p2, p0)\\ %6
        \neg L(a, g) & MPP(p6, p5)
    \end{logic}
\end{logicenv}

\sect{Punto 4}
\begin{logicenv}{Hay un hombre\dots}
    \begin{gather*}
        H(x) := \text{``$x$ es un hombre''}\\
        \mathbb{H} = \{x \,\vert\, H(x)\}\\
        D(x, y) = \text{``$x$ desprecia a $y$''}
    \end{gather*}
    \begin{logic}
        (\exists x \,\vert\, H(x) : (\forall y \,\vert\, y \in \mathbb{H} : D(x, y))) & Suposición\\
        (\exists x \,\vert\, H(x) : (\forall y \,\vert\, y \in \mathbb{H} : D(y, x)))
    \end{logic}
    Hace falta que $D$ sea una relación reflexiva, cosa que no se especifica ni es algo deducible de la palabra misma.
\end{logicenv}

\sect{Punto 5}
\begin{logicenv}{Si no es cierto\dots}
    \begin{gather*}
        F(x) := \text{``$x$ es feliz''}\\
    \end{gather*}
    \begin{logic}
        \neg (\forall x \,\vert:\, F(x)) \to (\exists x \,\vert:\, \neg F(x)) & Suposición\\
        \neg (\exists x \,\vert\, \neg F(x)) & Suposición\\
        (\forall x \,\vert:\, F(x)) & MTT(p1, p0)\\
        (\exists x \,\vert:\, F(x)) & Lema 7.13.3
    \end{logic}
\end{logicenv}

\sect{Punto 6}
\begin{logicenv}{Los elefantes son más pesados\dots}
    \begin{gather*}
        E(x) := \text{``$x$ es un elefante''}\\
        R(x) := \text{``$x$ es un ratón''}
        p(x) := \text{peso de $x$}\\
    \end{gather*}
    \begin{logic}
        (\forall x, y \,\vert\, E(x) \land R(y) : p(x) > p(y)) & Suposición\\
        \neg (\forall y, x \,\vert\, R(y) \land E(x) : p(y) > p(x)) & Suposición\\
    \end{logic}
    La conclusión es una proposición sin sentido
\end{logicenv}
\end{document}
