\documentclass{article}

% Librerías:
\usepackage{amsmath, mathtools, amssymb, mathrsfs, amsthm, nicematrix, array, logicDG}
\usepackage{lmodern, graphicx, fancyhdr}
\usepackage[margin = 1cm, top = 2.5cm, bottom = 2.5cm, includefoot]{geometry}
\usepackage{xcolor}
\usepackage[hidelinks]{hyperref}
\usepackage[T1]{fontenc}
\usepackage[spanish]{babel}
\usepackage{listings}
\usepackage{tikz}
\usetikzlibrary{shapes, fit, tikzmark}
\usepackage{color, colortbl}
\usepackage[most]{tcolorbox}

% Configuraciones:
\pagestyle{fancy}
\fancyhf{}
\setlength{\headheight}{72pt}
\rhead{\textit{\author}}
\lhead{\includegraphics[width = 4cm]{\logo}}
\lfoot{Página \thepage}
\rfoot{\titlename}
\renewcommand{\headrule}{\hbox to \headwidth{\color{rojoEci}\leaders\hrule height \headrulewidth\hfill}}
\renewcommand{\footrulewidth}{0.4pt}

\hyphenpenalty=10000

\newcommand{\logo}{C:/Users/usuario/Documents/U/logo-eci.jpg}

\newcommand{\q}[1]{``#1''}
\newcommand{\und}[2]{#1\textbf{\_}#2}
\newcommand{\val}[2]{\mathbf{#1}[#2]}

\setlength{\parindent}{0pt}

%%%%%%%%%%%%%%%%%%%%%%%%%%%%%%%%%%
%%%%%%%%%%%%%%%%%%%%%%%%%%%%%%%%%%
\newcommand{\titlename}{Tarea 09}%
\renewcommand{\author}{David Gomez}%
%%%%%%%%%%%%%%%%%%%%%%%%%%%%%%%%%%
%%%%%%%%%%%%%%%%%%%%%%%%%%%%%%%%%%


% Colores
\definecolor{rojoEci}{RGB}{225, 70, 49}
\definecolor{defini}{HTML}{ede9e6}
\definecolor{deftitlmarg}{HTML}{cfcfcf}

% Documento
\begin{document}

\begin{titlepage}
    \begin{center}
        \vspace*{1cm}

        \textbf{\Huge{\titlename}}

        \vspace{1.5cm}

        \textbf{\large{\author}
}
        \vspace{4cm}

        \includegraphics[width=.8\textwidth]{\logo}

        \vspace{4cm}

        Matemáticas\linebreak
        Escuela Colombiana de Ingeniería Julio Garavito\linebreak
        Colombia\linebreak
        \today

    \end{center}
\end{titlepage}
\clearpage
\tableofcontents
\clearpage

\section{Sección 4.6}
\subsection{Punto 4}
\begin{logicenv}{Teo 4.24.3}
    \begin{derivation}
            (\phi \land \textrm{\textit{true}})\\
        Def.($\land$)\\
            (\phi \equiv (\textrm{\textit{true}} \equiv (\phi \lor \textrm{\textit{true}})))\\
        Teo 4.19.2, Leibniz($\phi = (\phi \equiv (\textrm{\textit{true}} \equiv p))$)\\
            (\phi \equiv (\textrm{\textit{true}} \equiv \textrm{\textit{true}}))\\
        Teo 4.6.2, Leibniz($\phi = (\phi \equiv p)$)\\
            (\phi \equiv \textrm{\textit{true}})\\
        Identidad($\equiv$)\\
            \phi
    \end{derivation}
    Por MT 4.21 se demuestra que\\
    $\vdash_{\text{DS}} ((\phi \land \textrm{\textit{true}}) \equiv \phi)$
\end{logicenv}

\subsection{Punto 5}
\begin{logicenv}{Teo 4.24.4}
    \begin{derivation}
            (\phi \land \textrm{\textit{false}})\\
        Def.($\land$)\\
            (\phi \equiv (\textrm{\textit{false}} \equiv (\phi \lor \textrm{\textit{false}})))\\
        Identidad($\lor$), Conmutativa($\equiv$), Leibniz($\phi = (\phi \equiv (\textrm{\textit{false}} \equiv p)$)\\
            (\phi \equiv (\textrm{\textit{false}} \equiv \phi))\\
        Def.($\neg$), Conmutativa($\equiv$), Leibniz($\phi = (\phi \equiv p)$)\\
            (\phi \equiv (\neg \phi))\\
        Teo 4.15.7\\
            \textrm{\textit{false}}
    \end{derivation}
    Por MT 4.21 se demuestra que\\
    $\vdash_{\text{DS}} ((\phi \land \textrm{\textit{false}}) \equiv \textrm{\textit{false}})$
\end{logicenv}

\subsection{Punto 6}
\begin{logicenv}{Teo 4.24.5}
    \begin{derivation}
            (\phi \land \phi)\\
        Def.($\land$)\\
            (\phi \equiv (\phi \equiv (\phi \lor \phi)))\\
        Asociativa($\equiv$)\\
            ((\phi \equiv \phi) \equiv (\phi \lor \phi))\\
        Idempotencia($\lor$), Leibniz($\phi = ((\phi \equiv \phi) \equiv p)$)\\
            ((\phi \equiv \phi) \equiv \phi)\\\
        Teo 4.6.3, Conmutativa($\equiv$)\\
            (\phi \equiv \textrm{\textit{true}})\\
        Identidad\\
            \phi
    \end{derivation}
    Por MT 4.21 se demuestra que\\
    $\vdash_{\text{DS}} ((\phi \land \phi) \equiv \phi)$
\end{logicenv}

\subsection{Punto 8}
\begin{logicenv}{Teo 4.25.1}
    \begin{derivation}
            (\phi \land (\neg \phi))\\
        Def.($\land$)\\
            (\phi \equiv ((\neg \phi) \equiv (\phi \lor (\neg \phi))))\\
        Asociativa($\equiv$)\\
            ((\phi \equiv (\neg \phi)) \equiv (\phi \lor (\neg \phi)))\\
        Teo 4.19.1, Identidad, Leibniz($\phi = ((\phi \equiv (\neg \phi)) \equiv p)$)\\
            ((\phi \equiv (\neg \phi)) \equiv \textrm{\textit{true}})\\
        Teo 4.15.7, Conmutativa($\equiv$), Leibniz($\phi = (p \equiv \textrm{\textit{true}})$)\\
            (\textrm{\textit{false}} \equiv \textrm{\textit{true}})\\
        Identidad\\
            \textrm{\textit{false}}
    \end{derivation}
    Por MT 4.21 se demuestra que\\
    $\vdash_{\text{DS}} ((\phi \land (\neg \phi)) \equiv \textrm{\textit{false}})$
\end{logicenv}

\subsection{Punto 9}
\begin{logicenv}{Teo 4.25.2}
    \begin{derivation}
        (\neg (\phi \land \psi))\\
    Def.($\land$), Leibniz($\phi = (\neg p)$)\\
        (\neg (\phi \equiv (\psi \equiv (\phi \lor \psi))))\\
    Conmutativa($\lor$), Leibniz($\phi = (\neg (\phi \equiv p))$)\\
        (\neg (\phi \equiv(\psi \equiv (\psi \lor \phi))))\\
    Conmutativa($\equiv$), Leibniz($\phi = (\neg (\phi \equiv p))$)\\
        (\neg (\phi \equiv ((\psi \lor \phi) \equiv \psi)))\\
    Teo 4.15.6, Leibniz($\phi = (\neg (\phi \equiv ((\psi \lor p) \equiv \psi)))$)\\
        (\neg (\phi \equiv ((\psi \lor (\neg (\neg \phi))) \equiv \psi)))\\
    Teo 4.19.4, Leibniz($\phi = (\neg (\phi \equiv p))$)\\
        (\neg (\phi \equiv (\psi \lor (\neg \phi))))\\
    Conmutativa 4.15.4\\
        ((\neg \phi) \equiv (\psi \lor (\neg \phi)))\\
    Conmutativa($\lor$), Leibniz($\phi = ((\neg \phi) \equiv p)$)\\
        ((\neg \phi) \equiv ((\neg \phi) \lor \psi))\\
    Conmutativa($\equiv$)\\
        (((\neg \phi) \lor \psi) \equiv (\neg \phi))\\
    Teo 4.15.6, Leibniz($\phi = (((\neg \phi) \lor p) \equiv (\neg \phi))$)\\
        (((\neg \phi) \lor (\neg(\neg \psi))) \equiv (\neg \phi))\\
    Teo 4.19.4\\
        ((\neg \phi) \lor (\neg \psi))
    \end{derivation}
    Por MT 4.21 se demuestra que\\
    $\vdash_{\text{DS}} ((\neg (\phi \land \psi)) \equiv ((\neg \phi) \lor (\neg \psi)))$
\end{logicenv}

\subsection{Punto 11}
\begin{logicenv}{Teo 4.25.4}
    \begin{derivation}
            (((\phi \land \psi)) \equiv ((\phi \land \tau)) \equiv \phi)\\
        Def.($\land$)\\
            (((\phi \equiv (\psi \equiv (\phi \lor \psi))) \equiv (\phi \equiv (\tau \equiv (\phi \lor \tau)))) \equiv \phi)\\
        Conmutativa($\equiv$)\\
            (\phi \equiv ((\phi \equiv (\psi \equiv (\phi \lor \psi))) \equiv (\phi \equiv (\tau \equiv (\phi \lor \tau)))))\\
        Asociativa($\equiv$), Leibniz($\phi = (\phi \equiv p)$)\\
            (\phi \equiv (\phi \equiv ((\psi \equiv (\phi \lor \psi))) \equiv (\phi \equiv (\tau \equiv (\phi \lor \tau)))))\\
        Asociativa($\equiv$)\\
            ((\phi \equiv \phi) \equiv ((\psi \equiv (\phi \lor \psi)) \equiv (\phi \equiv (\tau \equiv (\phi \lor \tau)))))\\
        Teo 4.6.2, Leibniz($\phi = (p \equiv ((\psi \equiv (\phi \lor \psi)) \equiv (\phi \equiv (\tau \equiv (\phi \lor \tau)))))$)\\
            (\textrm{\textit{true}} \equiv ((\psi \equiv (\phi \lor \psi)) \equiv (\phi \equiv (\tau \equiv (\phi \lor \tau)))))\\
        Conmutativa($\equiv$), Identidad($\equiv$)\\
            ((\psi \equiv (\phi \lor \psi)) \equiv (\phi \equiv (\tau \equiv (\phi \lor \tau))))\\
        Conmutativa($\equiv$), Leibniz($\phi = ((\psi \equiv (\phi \lor \psi)) \equiv p)$)\\
            ((\psi \equiv (\phi \lor \psi)) \equiv ((\tau \equiv (\phi \lor \psi)) \equiv \phi))\\
        Asociativa($\equiv$)\\
            (((\psi \equiv (\phi \lor \psi)) \equiv (\tau \equiv (\phi \lor \psi))) \equiv \phi)\\
        Conmutativa($\equiv$)\\
            (\phi \equiv ((\psi \equiv (\phi \lor \psi)) \equiv (\tau \equiv (\phi \lor \psi))))\\
        Asociativa($\equiv$), Leibniz($\phi = (\phi \equiv (\phi \equiv p))$)\\
            (\phi \equiv (\psi \equiv ((\phi \lor \psi) \equiv (\tau  \equiv (\phi \lor \psi)))))\\
        Conmutativa($\equiv$), Leibniz($\phi = (\phi \equiv (\psi \equiv p))$)\\
            (\phi \equiv (\psi \equiv ((\tau  \equiv (\phi \lor \psi)) \equiv (\phi \lor \psi))))\\
        Asociativa($\equiv$), Leibniz($\phi = (\phi \equiv p)$)\\
            (\phi \equiv ((\phi \equiv \tau) \equiv ((\phi \lor \tau) \equiv (\phi \lor \psi))))\\
        Distribución($\lor, \equiv$), Leibniz($\phi = (\phi \equiv ((\psi \equiv \tau)\equiv p))$)\\
            (\phi \equiv ((\psi \equiv \tau) \equiv (\phi \lor (\tau \equiv \psi))))\\
        Conmutativa($\lor$), Leibniz($\phi = (\phi \equiv ((\psi \equiv \tau) \equiv (\phi \lor p)))$)\\
            (\phi \equiv ((\psi \equiv \tau) \equiv (\phi \lor (\psi \equiv \tau))))\\
        Def.($\land$)\\
            (\phi \land (\psi \equiv \tau))
    \end{derivation}
    Por MT 4.21 y Conmutativa($\equiv$) se demuestra que\\
    $\vdash_{\text{DS}} ((\phi \land (\psi \equiv \tau)) \equiv (((\phi \land \psi) \equiv (\phi \land \psi)) \equiv \phi))$
\end{logicenv}

\subsection{punto 12}
\begin{logicenv}{Teo 4.25.5}
    \begin{derivation}
            ((\phi \land \psi) \not\equiv (\phi \land \tau))\\
        Def.($\land$)\\
            ((\phi \equiv (\psi \equiv (\phi \lor \psi))) \not\equiv (\phi \equiv (\tau \equiv (\phi \lor \tau))))\\
        Def.($\not\equiv$)\\
            ((\neg (\phi \equiv (\psi \equiv (\phi \lor \psi)))) \equiv (\phi \equiv (\tau \equiv (\phi \lor \tau))))\\
        Teo 4.15.4\\
            (\neg ((\phi \equiv (\psi \equiv (\phi \lor \psi))) \equiv (\phi \equiv (\tau \equiv (\phi \lor \tau)))))\\
        Conmutativa($\equiv$)\\
            (\neg (((\psi \equiv (\phi \lor \psi)) \equiv \phi) \equiv (\phi \equiv (\tau \equiv (\phi \lor \tau)))))\\
        Asociativa($\equiv$)\\
            (\neg ((\psi \equiv (\phi \lor \psi)) \equiv (\phi \equiv (\phi \equiv (\tau \equiv (\phi \lor \tau))))))\\
        Asociativa($\equiv$)\\
            (\neg ((\psi \equiv (\phi \lor \psi)) \equiv ((\phi \equiv \phi) \equiv (\tau \equiv (\phi \lor \tau)))))\\
        Conmutativa($\equiv$)\\
            (\neg (((\phi \lor \psi) \equiv \psi) \equiv (\tau \equiv (\phi \lor \tau))))\\
        Asociativa($\equiv$)\\
            (\neg ((\phi \lor \psi) \equiv (\psi \equiv (\tau \equiv (\phi \lor \tau)))))\\
        Asociativa($\equiv$)\\
            (\neg ((\phi \lor \psi) \equiv ((\psi \equiv \tau) \equiv (\phi \lor \tau))))\\
        Conmutativa($\equiv$)\\
            (\neg ((\phi \lor \psi) \equiv ((\phi \lor \tau) \equiv (\psi \equiv \tau))))\\
        Asociativa($\equiv$)\\
            (\neg (((\phi \lor \psi) \equiv (\phi \lor \tau)) \equiv (\psi \equiv \tau)))\\
        Dist.($\lor$, $\equiv$)\\
            (\neg ((\phi \lor (\psi \equiv \tau)) \equiv (\psi \equiv \tau)))\\
        Teo 4.19.4\\
            (\neg (((\neg \phi) \lor (\psi \equiv \tau))))\\
        Dist.($\neg$, $\lor$)\\
            (\phi \land (\neg (\psi \equiv \tau)))\\
        Teo 4.15.4, Def.($\not\equiv$)\\
            (\phi \land (\psi \not\equiv \tau))
    \end{derivation}
    Por MT 4.21 y Conmutativa($\equiv$) se demuestra que\\
    $\vdash_{\text{DS}}((\phi \land (\psi \not\equiv \psi)) \equiv ((\phi \land \psi) \not\equiv (\phi \land \tau)))$
\end{logicenv}

\subsection{Punto 16}
\begin{logicenv}[5]{Debilitamiento}
    \begin{logic}
        (\phi \land \psi) & Hipótesis (Debilitamiento)\\
        (\phi \equiv (\psi \equiv (\phi \lor \psi))) & Def.($\equiv$), Ecuanimidad(p0)\\
        ((\phi \lor \phi) \equiv (\phi \lor (\psi \equiv (\phi \lor \psi)))) & Leibniz($\phi = (p \lor \phi)$)\\
        ((\phi \lor (\psi \equiv (\phi \lor \psi))) \equiv ((\phi \lor \psi) \equiv (\phi \lor (\phi \lor \psi)))) & Dist.($\lor$, $\equiv$)\\
        (((\phi \lor \psi) \equiv (\phi \lor (\phi \lor \psi))) \equiv ((\phi \lor \psi) \equiv ((\phi \lor \phi) \lor \psi))) & Asociativa($\lor$), Leibniz($\phi = ((\phi \lor \psi) \equiv p)$)\\
        (((\phi \lor \psi) \equiv ((\phi \lor \phi) \lor \psi)) \equiv ((\phi \lor \psi) \equiv (\phi \lor \psi))) & Idempotencia($\lor$), Leibniz($\phi = ((\phi \lor \psi) \equiv p)$)\\
        (((\phi \lor \psi) \equiv (\phi \lor \psi)) \equiv \textrm{\textit{true}}) & Teo 4.6.2\\
        ((\phi \lor \phi) \equiv \textrm{\textit{true}}) & Transitividad(p6, p5, p4, p3, p2)\\
        (\phi \lor \phi) & Identidad($\equiv$)(p7)\\
        \phi & Idempotencia($\lor$)(p8)
    \end{logic}
    Debilitamiento permite quitar información de una conjunción. Puesto que esta es verdad únicamente cuando sus dos partes son verdaderas, se puede concluir cualquiera de ellas. 
\end{logicenv}

\subsection{Punto 17}
\begin{logicenv}{Unión}
    \begin{logic}
        \phi & Hipótesis(Unión)\\
        \psi & Hipótesis(Unión)\\
        (\phi \equiv \textrm{\textit{true}}) & Identidad($\equiv$)(p0)\\
        (\phi \lor (\phi \equiv \psi)) & Debilitamiento($\lor$)(p0)\\
        ((\phi \lor (\phi \equiv \psi)) \equiv ((\phi \lor \phi) \equiv (\phi \lor \psi))) & Dist.($\lor$, $\equiv$)\\
        (((\phi \lor \phi) \equiv (\phi \lor \psi)) \equiv (\phi \equiv (\phi \lor \psi))) & Idempotencia($\lor$), Leibniz($\phi = (\phi \lor p)$)\\
        ((\phi \lor (\phi \equiv \psi)) \equiv (\phi \equiv (\phi \lor \psi))) & Transitividad(p5, p4, p3)\\
        (\phi \equiv (\phi \lor \psi)) & Ecuanimidad(p6, p3)\\
        (\psi \equiv \textrm{\textit{true}}) & Identidad($\equiv$)(p1)\\
        ((\phi \equiv (\phi \lor \psi)) \equiv \textrm{\textit{true}}) & Identidad($\equiv$)(p7)\\
        ((\phi \equiv (\phi \lor \psi)) \equiv \psi) & Transitividad(p9, p8)\\
        (\psi \equiv (\phi \equiv (\phi \lor \psi))) & Conmutativa($\equiv$), Ecuanimidad(p10)\\
        (\phi \land \psi) & Def.($\land$), Conmutativa($\lor$), Conmutativa($\land$), Ecuanimidad(p11)
    \end{logic}
    Unión permite juntar varias proposiciones las cuales se tienen como verdaderas. Puesto que la conjunción es verdadera cuando sus dos partes son verdaderas, es posible conectar dos proposiciones verdaderas mediante una conjunción.
\end{logicenv}

\section{Sección 4.7}
\subsection{Punto 3}
\begin{logicenv}{Teo 4.28.2}
    \begin{derivation}
            ((\phi \land \psi) \equiv \phi)\\
        Def.($\land$), Leibniz($\phi = (p \equiv \phi)$)\\
            ((\phi \equiv (\psi \equiv (\phi \lor \psi))) \equiv \phi)\\
        Conmutativa($\equiv$), Asociativa($\equiv$), Identidad($\equiv$)\\
            (\psi \equiv (\phi \lor \psi))\\
        Conmutativa($\equiv$), Def.($\to$)\\
            (\phi \to \psi)
    \end{derivation}
    Por MT 4.21 y Conmutativa($\equiv$) se demuestra que\\
    $\vdash_{\text{DS}} ((\phi \to \psi) \equiv ((\phi \land \psi) \equiv \phi))$
\end{logicenv}

\subsection{punto 7}
\begin{logicenv}{Teo 4.29.4}
    \begin{derivation}
            (\phi \to \textrm{\textit{false}})\\
        Teo 4.28.1\\
            ((\neg \phi) \lor \textrm{\textit{false}})\\
        Identidad($\lor$)\\
            (\neg \phi)
    \end{derivation}
    Por MT 4.21 se demuestra que\\
    $\vdash_{\text{DS}} ((\phi \to \textrm{\textit{false}}) \equiv (\neg \phi))$
\end{logicenv}

\subsection{punto 10}
\begin{logicenv}{Teo 4.30.3}
    \begin{derivation}
            (\phi \to (\psi \land \tau))\\
        Teo 4.28.1\\
            ((\neg \phi) \lor (\psi \land \tau))\\
        Dist.($\lor$, $\land$)\\
            (((\neg \phi) \lor \psi) \land ((\neg \phi) \lor \tau))\\
        Def.($\to$)\\
            ((\phi \to \psi) \land (\phi \to \tau))
    \end{derivation}
    Por MT 4.21 se demuestra que\\
    $\vdash_{\text{DS}} ((\phi \to (\psi \to \tau)) \equiv ((\phi \to \psi) \land (\phi \to \tau)))$
\end{logicenv}

\subsection{Punto 17}
\begin{logicenv}{Teo 4.31.5}
    \begin{derivation}
            (\phi \to (\psi \to \tau))\\
        Teo 4.18.1\\
            ((\neg \phi) \lor (\psi \to \tau))\\
        Teo 4.28.1, Leibniz($\phi = ((\neg \phi) \lor p)$)\\
            ((\neg \phi) \lor ((\neg \psi) \lor \tau))\\
        Asociativa($\lor$)\\
            (((\neg \phi) \lor (\neg \psi)) \lor \tau)\\
        De Morgan, Leibniz($\phi = (p \lor \tau)$)\\
            ((\neg (\phi \land \psi)) \lor \tau)\\
        Teo 4.28.1\\
            ((\phi \land \psi) \to \tau)
    \end{derivation}
    Por MT 4.21 se demuestra que\\
    $\vdash_{\text{DS}} ((\phi \to (\psi \to \tau)) \equiv ((\phi \land \psi) \to \tau))$
\end{logicenv}

\subsection{Punto 18}
\begin{logicenv}{Teo 4.31.6}
    \begin{derivation}
        (\phi \lor (\phi \to \psi))\\
    Teo 4.28.1, Leibniz($\phi = (\phi \lor p)$)\\
        (\phi \lor ((\neg \phi) \lor \psi))\\
    Asociativa($\lor$)\\
        ((\phi \lor (\neg \phi)) \lor \psi)\\
    Teo 4.19.1, Identidad($\equiv$)\\
        (\textrm{\textit{true}} \lor \psi)\\
    Teo 4.19.2\\
        \textrm{\textit{true}}
    \end{derivation}
    Por MT 4.21 e Identidad($\equiv$) se demuestra que\\
    $\vdash_{\text{DS}} (\phi \lor (\phi \to \psi))$
\end{logicenv}

\subsection{Punto 23}
\label{Transitividad}
\begin{logicenv}{Teo 4.33.2}
    \begin{logic}
        ((\phi \to \psi) \land (\psi \to \tau)) & Suposición del antecedente\\
        (\phi \to \psi) & Debilitamiento(p0)\\
        (\psi \to \tau) & Debilitamiento(p0)\\
        ((\neg \phi) \lor \psi) & Teo 4.28.1, Ecuanimidad(p1)\\
        ((\neg \psi) \lor \tau) & Teo 4.28.1, Ecuanimidad(p2)\\
        (\psi \lor (\neg \phi)) & Conmutativa($\lor$)\\
        ((\neg \phi) \lor \tau) & Corte(p5, p4)\\
        (\phi \to \tau) & Teo 4.28.1, Ecuanimidad(p6)
    \end{logic}
    Así, tomando (p7, p0), se demuestra que $\vdash_\text{DS}(((\phi \to \psi) \land (\psi \to \tau)) \to (\phi \to \tau))$
\end{logicenv}

\subsection{Punto 24}
\begin{logicenv}[5]{Teo 4.33.3}
    \begin{logic}
        ((\phi \to \psi) \land (\psi \to \phi)) & Suposición del antecedente\\
        (\phi \to \psi) & Debilitamiento($\land$)(p0)\\
        (\psi \to \phi) & Debilitamiento($\land$)(p0)\\
        ((\phi \lor \psi) \equiv \psi) & Def.($\to$), Ecuanimidad(p1)\\
        ((\psi \land \phi) \equiv \psi) & Teo 4.28.2, Ecuanimidad(p2)\\
        ((\phi \lor \psi) \equiv (\phi \land \psi)) & Transitividad(p4, p3), Conmutativa($\land$)\\
        ((\phi \lor \psi) \equiv (\phi \equiv (\psi \equiv (\phi \lor \psi)))) & Def.($\land$), Leibniz($\phi = ((\phi \lor \psi) \equiv p)$), Ecuanimidad(p5)\\
        ((\phi \lor \psi) \equiv ((\phi \equiv \psi) \equiv (\phi \lor \psi))) & Asociativa($\equiv$), Leibniz($\phi = ((\phi \lor \psi) \equiv p)$), Ecuanimidad(p6)\\
        ((\phi \lor \psi) \equiv ((\phi \lor \psi) \equiv (\phi \equiv \psi))) & Conmutativa($\equiv$), Leibniz($\phi = ((\phi \lor \psi) \equiv p)$), Ecuanimidad(p7)\\
        (((\phi \lor \psi) \equiv (\phi \lor \psi)) \equiv (\phi \equiv \psi)) & Asociativa($\equiv$)\\
        (\textrm{\textit{true}} \equiv (\phi \equiv \psi)) & Teo 4.6.2, Leibniz($\phi = (p \equiv (\phi \equiv \psi))$), Ecuanimidad(p9)\\
        (\phi \equiv \psi) & Identidad($\equiv$), Conmutativa($\equiv$)(p10)
    \end{logic}
    Así, tomando (p0, p11), se demuestra que $\vdash_\text{DS} (((\phi \to \psi) \land (\psi \to \phi)) \to (\phi \equiv \psi))$
\end{logicenv}
\subsection{Punto 35}
\begin{logicenv}{Teo 4.35.5}
    \begin{derivation}
            ((\phi \to \tau) \land (\psi \to \tau))\\
        Teo 4.28.1\\
            (((\neg \phi) \lor \tau) \land ((\neg \psi) \lor \tau))\\
        Conmutativa($\lor$), Distribución($\lor$, $\land$)\\
            (\tau \lor ((\neg \phi) \land (\neg \psi)))\\
        Conmutativa($\lor$)\\
            (((\neg \phi) \land (\neg \psi)) \lor \tau)\\
        De Morgan\\
            ((\neg (\phi \lor \psi)) \lor \tau)\\
        Teo 4.28.1\\
            ((\phi \lor \psi) \to \tau)
    \end{derivation}
    Por MT 4.21 y Conmutativa($\equiv$) se demuestra que\\
    $\vdash_{\text{DS}} (((\phi \lor \psi) \to \tau) \equiv ((\phi \to \tau) \land (\psi \to \tau)))$
\end{logicenv}

\subsection{Punto 38}
    \begin{logicenv}{Teo 4.36.3}
        \begin{logic}
            ((\phi \to \psi) \land (\psi \equiv \tau)) & Suposición del antecedente\\
            (\phi \to \psi) & Debilitamiento(p0)\\
            (\psi \equiv \tau) & Debilitamiento(p0)\\
            ((\phi \to \psi) \equiv (\phi \to \tau)) & Leibniz($\phi = (\phi \to p)$)(p2)\\
            (\phi \to \tau) & Ecuanimidad(p3,p1)
        \end{logic}
        Así, tomando (p4,p0), se demuestra que $(((\phi \to \psi) \land (\psi \equiv \tau)) \to (\phi \to \tau))$
    \end{logicenv}

\subsection{Punto 40}
\subsubsection{a}
%%%%%%%%%%%%%%%%%%%%%
%%%%%%%%%%%%%%%%%%%%%%
\subsubsection{b}
\begin{logicenv}[5]{$\vdash_\text{DS} ((\phi \to \psi) \to ((\phi \lor \tau) \to (\psi \lor \tau)))$}
    \begin{align*}
            \res{(\phi \to \psi)}\\
        \why{Teo 4.28.1}\\
            \res{((\neg \phi) \lor \psi)}\\
        \why[\Rightarrow]{Debilitamiento($\lor$)}\\
            \res{(((\neg \phi) \lor \psi) \lor \tau)}\\
        \why{Asociativa($\lor$)}\\
            \res{((\neg \phi) \lor (\psi \lor \tau))}\\
        \why{Teo 4.19.4}\\
            \res{((\phi \lor (\psi \lor \tau)) \equiv (\psi \lor \tau))}\\
        \why{Asociativa($\lor$), Leibniz($\phi = (p \equiv (\psi \lor \tau))$)}\\
            \res{(((\phi \lor \psi) \lor \tau) \equiv (\phi \lor \tau))}\\
        \why{Idempotencia($\lor$), Leibniz($\phi = (((\phi \lor \psi) \lor p) \equiv (\phi \lor \tau))$)}\\
            \res{(((\phi \lor \psi) \lor (\tau \lor \tau)) \equiv (\phi \lor \tau))}\\
        \why{Asociativa($\lor$), Conmutativa($\lor$), Asociativa($\lor$)}\\
            \res{(((\phi \lor \tau) \lor (\psi \lor \tau)) \equiv (\psi \lor \tau))}\\
        \why{Def.($\to$)}\\
            \res{((\phi \lor \tau) \to (\psi \to \tau))}
    \end{align*}
    Por MT 5.5.1 se demuestra que\\
    $\vdash_\text{DS} ((\phi \to \psi) \to ((\phi \lor \tau) \to (\psi \lor \tau)))$
\end{logicenv}
\subsection{Punto 43}
\begin{logicenv}{Modus Tollens}
    \begin{logic}
        (\phi \to \psi) & Hipótesis MTT\\
        (\neg \psi) & Hipótesis MTT\\
        ((\neg \phi) \lor \psi) & Teo 4.28.1, Ecuanimidad(p0)\\
        (\psi \equiv \textrm{\textit{false}}) & Def.($\neg$), Ecuanimidad(p1)\\
        ((\neg \phi) \lor \textrm{\textit{false}}) & Leibniz($\phi = ((\neg \phi) \lor p)$)(p3)\\
        (\neg \phi) & Identidad($\lor$)
    \end{logic}

    En términos de causas y consecuencias, al tener que $a$ sucede a causa de $b$, y que es cierto que $a$ no sucede, entonces se puede decir que no ha ocurrido $b$. Análogamente, decir que en una fila de 2 fichas de dominó, si se sabe en que sentido se van a tirar, y la segunda no ha caído, se puede concluir que no se ha tirado la primera ficha.
\end{logicenv}

\subsection{Punto 44}
\begin{logicenv}[10]{Transitividad - Silogísmo disyuntivo}
    La regla de transitividad, hablando en términos de causas y consecuencias, dice:\\
    Si $a$ sucede debido a $b$, y $b$ sucede debido a $c$, entonces $a$ sucede debido a $c$\\
    Es decir, conecta el principio de una cadena de causa-consecuencia con su final.
    La demostración y su relación con Corte se puede hallar en el \hyperref[Transitividad]{punto 23}
\end{logicenv}

\section{Sección 5.1}
\subsection{Punto 1}
\subsubsection{a}
\begin{logicenv}[5]{a}
    $\phi \lor \psi \lor \tau \equiv \phi \lor \psi \lor \tau$
\end{logicenv}

\subsubsection{f}
\begin{logicenv}[5]{f}
    $\neg \phi \equiv \phi \equiv \textrm{\textit{false}}$
\end{logicenv}

\subsubsection{m}
\begin{logicenv}[5]{m}
    $\phi \equiv \neg \phi \equiv \textrm{\textit{false}}$
\end{logicenv}

\subsubsection{t}
\begin{logicenv}[5]{t}
    $\phi \to \psi \equiv \phi \lor \psi \equiv \psi$
\end{logicenv}

\subsubsection{w}
\begin{logicenv}[5]{w}
    $\phi \to \psi \lor \tau \equiv (\phi \to \psi) \lor (\phi \to \tau)$
\end{logicenv}

\subsubsection{x}
\begin{logicenv}[5]{x}
    $\phi \to \psi \land \tau \equiv (\phi \to \psi) \land (\phi \to \tau)$
\end{logicenv}

\subsubsection{z}
\begin{logicenv}[5]{z}
    $\phi \lor \psi \to \phi \land \psi \equiv \phi \equiv \psi$
\end{logicenv}

\subsection{Punto 2}
\subsubsection{a}
\begin{logicenv}[5]{Es ambigüa}
    \begin{gather*}
        p \lor (q \land r)\\
        (p \lor q) \land r
    \end{gather*}
\end{logicenv}

\subsubsection{b}
\begin{logicenv}[5]{Es ambigüa}
    \begin{gather*}
        p \land (q \lor r)\\
        (p \land q) \lor r
    \end{gather*}
\end{logicenv}

\subsubsection{c}
\begin{logicenv}[5]{Es ambigüa}
    \begin{gather*}
        p \to (q \to r)\\
        (p \to q) \to r
    \end{gather*}
\end{logicenv}

\subsubsection{d}
\begin{logicenv}[5]{Es ambigüa}
    \begin{gather*}
        p \to (q \gets r)\\
        (p \to q) \gets r
    \end{gather*}
\end{logicenv}

\subsubsection{e}
\begin{logicenv}[5]{Es ambigüa}
    \begin{gather*}
        p \gets (q \to r)\\
        (p \gets q) \to r
    \end{gather*}
\end{logicenv}

\subsection{Punto 5}
\subsubsection{a}
\begin{logicenv}{$\textrm{\textit{true}} \lor p \land q$}
    \[\textrm{\textit{true}} \lor (p \land q)\]
\end{logicenv}

\subsubsection{b}
\begin{logicenv}{$p \equiv p \lor q$}
    \[(p \equiv p) \lor q\]
\end{logicenv}

\subsubsection{c}
\begin{logicenv}{$p \to q \equiv r \equiv p \land q \equiv p \land r$}
    
\end{logicenv}
\end{document}