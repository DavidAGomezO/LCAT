\documentclass{article}

% Librerías:
\usepackage{amsmath, mathtools, amssymb, mathrsfs, amsthm, nicematrix, array}
\usepackage{lmodern, graphicx, fancyhdr, ifthen, etoolbox}
\usepackage[margin = 2cm, top = 2.5cm, includefoot]{geometry}
\usepackage{xcolor}
\usepackage[hidelinks]{hyperref}
\usepackage[T1]{fontenc}
\usepackage[spanish]{babel}
\usepackage{listings}
\usepackage{tikz}
\usetikzlibrary{shapes, fit, tikzmark}
\usepackage{color, colortbl}
\usepackage[most]{tcolorbox}

% Configuraciones:
\pagestyle{fancy}
\fancyhf{}
\setlength{\headheight}{55.34027pt}
\rhead{\textit{David Gómez}}
\lhead{\includegraphics[width = 4cm]{\logo}}
\lfoot{Página \thepage}
\rfoot{Taller 02}
\renewcommand{\headrule}{\hbox to \headwidth{\color{rojoEci}\leaders\hrule height \headrulewidth\hfill}}
\renewcommand{\footrulewidth}{0.4pt}

\hyphenpenalty=10000

\newcommand{\logo}{C:/Users/usuario/Documents/U/logo-eci.jpg}

\newcommand{\q}[1]{``#1''}
\newcommand{\und}[2]{#1\textbf{\_}#2}
\newcommand{\boolun}[2][]{H_{#2}(#1)}
\newcommand{\boolbin}[3]{H_{#1}(#2, #3)}
\newcommand{\val}[2]{\mathbf{#1}[#2]}
\newcommand{\setd}[1]{\{#1\}}

\newlength{\unit}
\setlength{\unit}{1ex}

\newcommand\drawCodeBox[3][1.5]{%
\begin{tikzpicture}[remember picture,overlay]
  \coordinate (start) at ([yshift = 2\unit]pic cs:#2);
  \coordinate (end) at ([yshift = -#1\unit]pic cs:#3);
  \node[inner sep=2pt,draw=black,fit=(start) (end)] {};
\end{tikzpicture}
}

\newcolumntype{L}{>{$}l<{$}} 

\newlength{\logicv}
\setlength{\logicv}{.1cm}
\newenvironment{logicenv}[2][0]{
  %begin
  \begin{tcolorbox}[demo, title = #2]
  \vspace*{#1\logicv}
}{
  %end
  \end{tcolorbox}
  \vspace*{-.5cm}
}

\newenvironment{subproofill}[1][0]{
  %begin
  \begin{tcolorbox}[demo, title = ]
    \vspace*{-#1\logicv}
}{
  %end
  \end{tcolorbox}
  \vspace*{-.5cm}
}


\newenvironment{subproof}[2][0]{
  %begin
  \begin{tcolorbox}[demo, title = #2, colframe = black]
  \vspace*{#1\logicv}
}{
  %end
  \end{tcolorbox}
}
\newenvironment{logic}[1][0]{
    \setlength{\extrarowheight}{3pt}
    \setcounter{row}{-1}
    \begin{center}
    \begin{NiceTabular}{>{\stepcounter{row}\therow.} l *{#1}{r} L r l *{#1}{l} }
}{
    \end{NiceTabular}
    \end{center}
}

\tcbset{demo/.style={
    enhanced, title=title,
    attach boxed title to top
    left = {xshift = .5pt, yshift = -\tcboxedtitleheight -.5pt},
    top = 1pt,
    boxrule = .65pt,
    coltitle = black,
    colback = defini,
    colframe = defini,
    arc = 0mm,
    outer arc = 0mm,
    boxed title style={
        colback = white,
        colframe = deftitlmarg,
        boxrule = .5pt,
        arc = 0mm,
        outer arc = 0mm
    }
}}



% Colores
\definecolor{rojoEci}{RGB}{225, 70, 49}
\definecolor{defini}{HTML}{ede9e6}
\definecolor{deftitlmarg}{HTML}{cfcfcf}

% Documento
\begin{document}



\newcounter{row}

\begin{titlepage}
	\begin{center}
		\vspace*{1cm}

		\textbf{\Huge{Tarea 04}}

		\vspace{1.5cm}

		\textbf{\large{David Gómez}}

		\vspace{4cm}

		\includegraphics[width=\textwidth]{\logo}

		\vspace{5cm}

		Matemáticas\linebreak
		Escuela Colombiana de Ingeniería Julio Garavito\linebreak
		Colombia\linebreak
		\today

	\end{center}
\end{titlepage}
\clearpage
\tableofcontents
\clearpage

\section{Punto 1}
\subsection{a)}
\begin{tcolorbox}[demo, title = a)]
	\vspace*{1cm}
	\begin{logic}[2]
		&&& ((\phi \lor (\psi \equiv \tau)) \equiv ((\phi \lor \psi) \equiv (\phi \lor \psi))) & Enunciado\\
		&&& \val{v}{(\phi \lor (\psi \equiv \tau))} = \val{v}{((\phi \lor \psi) \equiv (\phi \lor \psi))} & MT 2.23 ($\equiv$) (p0)\\[0.4cm]
		&\tikzmark{A}                    && \val{v}{(\phi \lor (\psi \equiv \tau))} = \mathtt{T} & Suposición 1\\
		&&& \val{v}{\phi} = \mathtt{T} \text{ o } \val{v}{(\psi \equiv \tau)} = \mathtt{T}\\[0.4cm]
		&&\tikzmark{A0}       &  \val{v}{\phi} = \mathtt{T} & Suposición 1.1 \\
		&&& \val{v}{((\phi \lor \psi) \equiv (\phi \lor \psi))} = \mathtt{T} & MT 2.23 ($\lor$, $\equiv$) (p2, p4) && \tikzmark{B0}\\[0.4cm]
		&&\tikzmark{A1}       & \val{v}{\phi} = \mathtt{F} \text{ y } \val{v}{(\psi \equiv \tau)} = \mathtt{T} & Suposición 1.2\\
		&&& \tikzmark{A2} \val{v}{\psi} = \mathtt{T} & Suposición 1.2.1\\
		&&& \val{v}{\tau} = \mathtt{T} & MT 2.23 ($\equiv$) (p7)\\
		&&& \val{v}{((\phi \lor \psi) \equiv (\phi \lor \psi))} = \mathtt{T} & MT 2.23 ($\equiv$, $\lor$) (p8, p7) & \tikzmark{B2}\\[0.3cm]
		&&& \tikzmark{A3} \val{v}{\psi} = \mathtt{F} & Suposición 1.2.2\\
		&&& \val{v}{\tau} = \mathtt{F} & MT 2.23 ($\equiv$) (p10)\\
		&&& \val{v}{((\phi \lor \psi) \equiv (\phi \lor \psi))} = \mathtt{F} & MT 2.23 ($\equiv$, $\lor$) (p10, p11, p6) & \tikzmark{B3} & \tikzmark{B1} & \tikzmark{B}\\[0.6cm]
		&\tikzmark{A4}                  && \val{v}{(\phi \lor (\psi \equiv \tau))} = \mathtt{F} & Suposición 2\\
		&&& \val{v}{\phi} = \mathtt{F} \text{ y } \val{v}{(\psi \equiv \tau)} = \mathtt{F} & MT 2.23 ($\lor$) (p13)\\
		&&\tikzmark{A5}     &  \val{v}{\psi} = \mathtt{T} & Suposición 2.1\\
		&&& \val{v}{\tau} = \mathtt{F} & MT 2.23 ($\equiv$) (p15, 14)\\
		&&& \val{v}{((\phi \lor \psi) \equiv (\phi \lor \psi))} = \mathtt{F} & MT 2.23 ($\equiv$, $\lor$) (p16, p15, p14) && \tikzmark{B5}\\[0.6cm]
		&&\tikzmark{A6}     & \val{v}{\psi} = \mathtt{F} & Suposición 2.2\\
		&&& \val{v}{\tau} = \mathtt{T} &  MT 2.23 ($\equiv$) (p18, 14)\\
		&&& \val{v}{((\phi \lor \psi) \equiv (\phi \lor \psi))} = \mathtt{F} & MT 2.23 ($\equiv$, $\lor$) (p19, p18, p14) && \tikzmark{B6} & \tikzmark{B4}\\[0.7cm]
		\hline
		&&& \therefore \qquad ((\phi \lor (\psi \equiv \tau)) \equiv ((\phi \lor \psi) \equiv (\phi \lor \psi)))


	\end{logic}
\end{tcolorbox}
\drawCodeBox[2]{A0}{B0}
\drawCodeBox{A2}{B2}
\drawCodeBox{A3}{B3}
\drawCodeBox[3]{A1}{B1}
\drawCodeBox[3]{A5}{B5}
\drawCodeBox[3]{A6}{B6}
\drawCodeBox[4]{A}{B}
\drawCodeBox[4]{A4}{B4}

\subsection{b)}

\begin{tcolorbox}[demo, title = b)]
	% demostración principal
	\begin{logic}
		&((\phi \to (\psi \to \tau)) \equiv ((\phi \to \psi) \to (\phi \to \tau))) & Enunciado\\
		&\val{v}{(\phi \to (\psi \to \tau))} = \val{v}{((\phi \to \psi) \to (\phi \to \tau))} & MT 2.23 ($\equiv$)(p0)\\
		&\text{Suposición 1}\\
		&\text{Suposición 2}\\
		\hline
		& \therefore \qquad ((\phi \to (\psi \to \tau)) \equiv ((\phi \to \psi) \to (\phi \to \tau)))
	\end{logic}
\end{tcolorbox}
\vspace*{-0.5cm}
% sub-prueba no.1
\begin{tcolorbox}[demo, title = ]
	\begin{tcolorbox}[demo, title = Suposición 1, colframe = black]
		\begin{logic}
			& \val{v}{(\phi  \to (\psi \to \tau))} = \mathtt{F}\\
			& \val{v}{\phi} = \mathtt{T} \text{ y } \val{v}{(\psi \to \tau)} = \mathtt{F} & MT 2.23 ($\to$)(p0)\\
			& \val{v}{\psi} = \mathtt{T} \text{ y } \val{v}{\tau} = \mathtt{F} & MT 2.23 ($\to$)(p1)\\
			& \val{v}{(\phi \to \psi)} = \mathtt{T} & MT 2.23 ($\to$)(p2, p1)\\
			& \val{v}{(\phi \to \tau)} = \mathtt{F} & MT 2.23 ($\to$)(p2, p1)\\
			& \val{v}{((\phi \to \psi) \to (\phi \to \tau))} = \mathtt{F} & MT 2.23 ($\to$)(p3, p4)\\
			& \val{v}{(\phi  \to (\psi \to \tau))} = \val{v}{((\phi \to \psi) \to (\phi \to \tau))} & (p5, p0)
		\end{logic}
	\end{tcolorbox}
\end{tcolorbox}
\vspace*{-.5cm}
% sub-prueba no.2
\begin{tcolorbox}[demo, title = ]
	\begin{tcolorbox}[demo, title = Suposición 2, colframe = black]
		\begin{logic}
			&\val{v}{(\phi \to (\psi \to \tau))} = \mathtt{T}\\
			&\val{v}{\phi} = \mathtt{F} \text{ o } \val{v}{(\psi \to \tau)} = \mathtt{T} & MT 2.23 ($\to$)(p0)\\
			& \text{Suposición 2.1}\\
			& \text{Suposición 2.2}
		\end{logic}
	\end{tcolorbox}
\end{tcolorbox}
\vspace*{-.5cm}

% sub-sub-prueba 2.1
\begin{tcolorbox}[demo, title = ]
	\begin{tcolorbox}[demo, title = ]
		\begin{tcolorbox}[demo, title = suposición 2.1, colframe = black]
			\vspace*{0.3cm}
			\begin{logic}
				&\val{v}{\phi} = \mathtt{F}\\
				& \val{v}{(\phi \to \psi)} = \mathtt{T} & MT 2.23 ($\to$)(p0)\\
				& \val{v}{(\phi \to \tau)} = \mathtt{T} & MT 2.23 ($\to$)(p0)\\
				& \val{v}{((\phi \to \psi) \to (\phi \to \tau))} = \mathtt{T} & MT 2.23 ($\to$)(p2, p1)\\
				& \val{v}{(\phi  \to (\psi \to \tau))} = \val{v}{((\phi \to \psi) \to (\phi \to \tau))} & (p0 Suposición 2, p3)
			\end{logic}
		\end{tcolorbox}
	\end{tcolorbox}
\end{tcolorbox}
\vspace*{-.5cm}


% sub-sub-prueba 2.2
\begin{tcolorbox}[demo, title = ]
	\begin{tcolorbox}[demo, title = ]
		\begin{tcolorbox}[demo, title = Suposición 2.2, colframe = black]
			\vspace*{0.3cm}
			\begin{logic}
				&\val{v}{\phi} = \mathtt{T} \text{ y } \val{v}{(\psi \to \tau)} = \mathtt{T}\\
				&\val{v}{\psi} = \mathtt{F} \text{ o } \val{v}{\tau} = \mathtt{T} & MT 2.23 ($\to$)(p0)\\
				& \text{Suposición 2.2.1}\\
				& \text{Suposición 2.2.2}
			\end{logic}
		\end{tcolorbox}
	\end{tcolorbox}
\end{tcolorbox}
\vspace*{-.5cm}

% sub(3) prueba 2.2.1

\begin{tcolorbox}[demo, title = ]
	\begin{tcolorbox}[demo, title = ]
		\begin{tcolorbox}[demo, title =]
			\begin{tcolorbox}[demo, title = Suposición 2.2.1,  colframe = black]
				\vspace*{0.3cm}
				\begin{logic}
					&\val{v}{\psi} = \mathtt{F}\\
					&\val{v}{(\phi \to \psi)} = \mathtt{F} & MT 2.23 ($\to$)(p0 Suposición 2.2, p0)\\
					& \val{v}{((\phi \to \psi) \to (\phi \to \tau))} = \mathtt{T} & MT 2.23 ($\to$)(p1)\\
					& \val{v}{(\phi  \to (\psi \to \tau))} = \val{v}{((\phi \to \psi) \to (\phi \to \tau))} & (p0 Suposición 2, p2)
				\end{logic}
			\end{tcolorbox}
		\end{tcolorbox}
	\end{tcolorbox}
\end{tcolorbox}
\vspace*{-.5cm}

% sub(3) prueba 2.2.2

\begin{tcolorbox}[demo, title = ]
	\begin{tcolorbox}[demo, title = ]
		\begin{tcolorbox}[demo, title =]
			\begin{tcolorbox}[demo, title = Suposición 2.2.2, colframe = black]
				\vspace*{0.3cm}
				\begin{logic}
					& \val{v}{\psi} = \mathtt{T} \text{ y } \val{v}{\tau} = \mathtt{T}\\
					& \val{v}{(\phi \to \psi)} = \mathtt{T} & MT 2.23 ($\to$)(p0 Suposición 2.2, p0)\\
					& \val{v}{(\phi \to \tau)} = \mathtt{T} & MT 2.23 ($\to$)(p0 Suposición 2.2, p0)\\
					& \val{v}{((\phi \to \psi) \to (\phi \to \tau))} = \mathtt{T} & MT 2.23 ($\to$)(p2, p1)\\
					& \val{v}{(\phi  \to (\psi \to \tau))} = \val{v}{((\phi \to \psi) \to (\phi \to \tau))} & (p0 Suposición 2, p3)
				\end{logic}
			\end{tcolorbox}
		\end{tcolorbox}
	\end{tcolorbox}
\end{tcolorbox}

\section{Punto 2}
\begin{tcolorbox}[demo, title = {$((\neg (p \lor q)) \to p) = \phi$}]
  \begin{logic}
    & \phi \text{ es satisfacible pero no tautología} & Enunciado\\
    & \text{demostración 1}\\
    & \text{demostración 2}\\
    \hline
    & \therefore \qquad \phi \text{ es satisfacible pero no tautología}
  \end{logic}
  % demostración 1
  \begin{tcolorbox}[demo, title = demostración 1, colframe = black]
    \begin{logic}
      & (\exists \mathbf{v}\, \vert\, \val{v}{\phi} = \mathtt{T}) & Enunciado\\
      & \val{v}{((\neg (p \lor q)) \to p)} = \mathtt{T} & Def(p0)\\
      & \val{v}{(\neg (p \lor q))} = \mathtt{F} \text{ o } \val{v}{p} = \mathtt{T} & MT 2.23 ($\to$)(p1)\\
      & \val{v}{(p \lor q)} = \mathtt{T} \text{ o } \val{v}{p} = \mathtt{T} & MT 2.23 ($\neg$)(p2)\\
      & \val{v}{p} = \mathtt{T} \text{ o } \val{v}{q} = \mathtt{T} \text{ o } \val{v}{p} = \mathtt{T} & MT 2.23 ($\lor$)(p3)\\
      & \mathbf{v} = \{p \mapsto \mathtt{T}, q \mapsto \mathtt{T}\} & Suposición (p4)
    \end{logic}
  \end{tcolorbox}
  
  % demostración 2
  \begin{tcolorbox}[demo, title = demostración 2, colframe = black]
    \begin{logic}
      & (\exists \mathbf{v}\, \vert\, \val{v}{\phi} = \mathtt{F}) & Enunciado\\
      & \val{v}{((\neg (p \lor q)) \to p)} = \mathtt{F} & Def. (p0)\\
      & \val{v}{(\neg (p \lor q))} = \mathtt{T} \text{ y } \val{v}{p} = \mathtt{F} & MT 2.23 ($\to$)(p1)\\
      & \val{v}{(p \lor q)} = \mathtt{F} \text{ y } \val{v}{p} = \mathtt{F} & MT 2.23 ($\neg$)(p2)\\
      & \val{v}{p} = \mathtt{F} \text{ y } \val{v}{q} = \mathtt{F} \text{ y } \val{v}{p} = \mathtt{F} & MT 2.23 ($\lor$)(p3)\\
      & \mathbf{v} = \{p \mapsto \mathtt{F}, q \mapsto \mathtt{F}\}
    \end{logic}
  \end{tcolorbox}
\end{tcolorbox}

\section{Punto 3}
\begin{itemize}
  \item $\phi$ tiene a $\lor$ como único conector lógico.
  \begin{tcolorbox}[demo, title = primer inciso]
    \begin{logic}
      & \phi = (\psi \lor \tau) & Enunciado\\
      & \vDash \phi & Enunciado\\
      & (\exists \mathbf{v} \, \vert\, \val{v}{\phi} = \mathtt{F}) & Suposición\\
      & \val{v}{(\psi \lor \tau)} = \mathtt{F} & Def.(p2)\\
      & \val{v}{\psi} = \mathtt{F} \text{ o } \val{v}{\tau} = \mathtt{F} & MT 2.23 ($\lor$)(p3)\\
      & \not\vDash \phi & (p4, p2)
    \end{logic}
  \end{tcolorbox}
  \item $\phi$ tiene a $\land$ como único conector lógico.
  \begin{tcolorbox}[demo, title = segundo inciso]
    \begin{logic}
      & \phi = (\psi \land \tau) & Enunciado\\
      & \vDash \phi & Enunciado\\
      & (\exists \mathbf{v} \, \vert\, \val{v}{\phi} = \mathtt{F}) & Suposición\\
      & \val{v}{(\psi \land \tau)} = \mathtt{F} & Def.(p2)\\
      & \val{v}{\psi} = \mathtt{F} \text{ y } \val{v}{\tau} = \mathtt{F} & MT 2.23 ($\land$)(p3)\\
      & \not\vDash \phi & (p4, p2)
    \end{logic}
  \end{tcolorbox}
\end{itemize}

\section{Punto 4}

\begin{logicenv}[4]{$\vDash (\phi \equiv \psi)$ sii $(\phi \to \psi)$ y $(\phi \gets \psi)$}
  \begin{logic}
    & \text{demostración 1}\\
    & \text{demostración 2}\\
    \hline
    & \therefore \qquad \vDash (\phi \equiv \psi)$ sii $(\phi \to \psi)$ y $(\phi \gets \psi)
  \end{logic}
\end{logicenv}

% demostración 1
\begin{subproofill}
  \begin{subproof}[3]{demostración 1}
    \begin{logic}
      & \text{Si } \vDash (\phi \equiv \psi) \text{ ,entonces } \vDash (\phi \to \psi) \text{ y } \vDash (\phi \gets \psi)\\
      & \vDash (\phi \equiv \psi) & Suposición\\
      & \val{v}{(\phi \equiv \psi)} = \mathtt{T} & Def.(p1)\\
      & \val{v}{\phi} = \val{v}{\psi} & MT 2.23 ($\equiv$)(p2)\\
      & \text{Suposición 1}\\
      & \text{Suposición 2}\\
      \hline
      & \therefore \qquad \text{Si } \vDash (\phi \equiv \psi) \text{ ,entonces } \vDash (\phi \to \psi) \text{ y } \vDash (\phi \gets \psi)
    \end{logic}
  \end{subproof}
\end{subproofill}

%% suposición 1
\begin{subproofill}
  \begin{subproofill}
    \begin{subproof}{Suposición 1}
      \begin{logic}
        & \val{v}{\phi} = \mathtt{T}\\
        & \val{v}{\psi} = \mathtt{T} & (p0, p3 demostración 1)\\
        & \val{v}{(\phi \to \psi)} = \mathtt{T} & MTT 2.23 ($\to$)(p1, p0)\\
        & \val{v}{(\phi \gets \psi)} = \mathtt{T} & MTT 2.23 ($\gets$)(p1, p0)
      \end{logic}
    \end{subproof}
  \end{subproofill}
\end{subproofill}

%% suposición 2
\begin{subproofill}
  \begin{subproofill}
    \begin{subproof}{Suposición 2}
      \begin{logic}
        & \val{v}{\phi} = \mathtt{F}\\
        & \val{v}{\psi} = \mathtt{F} & (p0, p3 demostración 1)\\
        & \val{v}{(\phi \to \psi)} = \mathtt{T} & MTT 2.23 ($\to$)(p1, p0)\\
        & \val{v}{(\phi \gets \psi)} = \mathtt{T} & MTT 2.23 ($\gets$)(p1, p0)
      \end{logic}
    \end{subproof}
  \end{subproofill}
\end{subproofill}

% demostración 2
\begin{subproofill}
  \begin{subproof}[3]{demostración 2}
    \begin{logic}
      & \text{Si } \vDash (\phi \to \psi) \text{ y } \vDash (\phi \gets \psi) \text{ , entonces } \vDash (\phi \equiv \psi)\\
      & \vDash (\phi \to \psi) \text{ y } \vDash (\phi \gets \psi) & Suposición\\
      & \val{v}{(\phi \to \psi)} = \mathtt{T} \text{ y } \val{v}{(\phi \gets \psi)} =\mathtt{T} & Def.(p1)\\
      & (\val{v}{\phi} = \mathtt{F} \text{ o } \val{v}{\psi} = \mathtt{T}) \text{ y } (\val{v}{\phi} = \mathtt{T} \text{ o } \val{v}{\psi} = \mathtt{F}) & MT 2.23 ($\to$, $\gets$)(p2)\\
      & \text{Suposición 1}\\
      & \text{Suposición 2}\\
      \hline
      & \therefore \qquad \text{Si } \vDash (\phi \to \psi) \text{ y } \vDash (\phi \gets \psi) \text{ , entonces } \vDash (\phi \equiv \psi)
    \end{logic}
  \end{subproof}
\end{subproofill}

%% suposición 1
\begin{subproofill}
  \begin{subproofill}
    \begin{subproof}{Suposición 1}
      \begin{logic}
        & \val{v}{\phi} = \mathtt{T}\\
        & \val{v}{\psi} = \mathtt{T} & MT 2.23 ($\gets$)(p0, p3 demostración 2)\\
        & \val{v}{\phi} = \val{v}{\psi} & (p1, p0)\\
        & \val{v}{(\phi \equiv \psi)} = \mathtt{T} & MT 2.23 ($\equiv$)(p2)
      \end{logic}
    \end{subproof}
  \end{subproofill}
\end{subproofill}

%% suposición 2
\begin{subproofill}
  \begin{subproofill}
    \begin{subproof}{Suposición 2}
      \begin{logic}
        & \val{v}{\phi} = \mathtt{F}\\
        & \val{v}{\psi} = \mathtt{F} & MT 2.23 ($\to$)(p0, p3 demostración 2)\\
        & \val{v}{\phi} = \val{v}{\psi} & (p1, p0)\\
        & \val{v}{(\phi \equiv \psi)} = \mathtt{T} & MT 2.23 ($\equiv$)(p2)
      \end{logic}
    \end{subproof}
  \end{subproofill}
\end{subproofill}

\section{Punto 5}

\begin{logicenv}{$\vDash (\phi \land \psi)$ sii $\vDash \phi$ y $\vDash \psi$}
  \begin{logic}
    & \text{demostración 1}\\
    & \text{demostración 2}\\
    \hline
    & \therefore \qquad \vDash (\phi \land \psi) \text{ sii } \vDash \phi \text{ y } \vDash \psi
  \end{logic}
\end{logicenv}

% demostración 1
\begin{subproofill}
  \begin{subproof}{demostración 1}
    \begin{logic}
      & \text{Si } \vDash (\phi \land \psi) \text{ , entonces } \vDash \phi \text{ y } \vDash \psi\\
      & \vDash (\phi \land \psi) & Suposición\\
      & \val{v}{(\phi \land \psi)} & Def.(p1)\\
      & \val{v}{\phi} = \mathtt{T} \text{ y } \val{v}{\psi} = \mathtt{T} & MT 2.23 ($\land$)(p2)\\
      & \vDash \phi \text{ y } \vDash \psi & Def.(p3)\\
      \hline
      & \therefore \qquad \text{Si } \vDash (\phi \land \psi) \text{ , entonces } \vDash \phi \text{ y } \vDash \psi
    \end{logic}
  \end{subproof}
\end{subproofill}

% demostración 2
\begin{subproofill}
  \begin{subproof}{demostración 2}
    \begin{logic}
      & \text{Si } \vDash \phi \text{ y } \vDash \psi \text{ , entonces } \vDash (\phi \land \psi)\\
      & \vDash \phi \text{ y } \vDash \psi & Suposición\\
      & \val{v}{\phi} = \mathtt{T} \text{ y } \val{v}{\psi} = \mathtt{T} & Def.(p1)\\
      & \val{v}{(\phi \land \psi)} = \mathtt{T} & MTT 2.23 ($\land$)(p2)\\
      & \vDash (\phi \land \psi) & Def.(p3)\\
      \hline
      & \text{Si } \vDash \phi \text{ y } \vDash \psi \text{ , entonces } \vDash (\phi \land \psi)
    \end{logic}
  \end{subproof}
\end{subproofill}

\section{Punto 6}
\begin{logicenv}[4]{$\setd{(\phi \lor \psi), ((\neg \phi) \lor \tau)} \vDash (\psi \lor \tau)$}
  \begin{logic}
    & \text{Existe } \mathbf{v} \text{ tal que } \mathbf{v} \text{ satisface }\{(\phi \lor \psi	), ((\neg \phi) \lor \tau)\}\\
    & \val{v}{(\phi \lor \psi)} = \mathtt{T} & Def.(p0)\\
    & \val{v}{((\neg \phi) \lor \tau)} = \mathtt{T} & Def.(p0)\\
    & \val{v}{\phi} =  \mathtt{T} \text{ o } \val{v}{\psi} = \mathtt{T} & MT 2.23 ($\lor$)(p1)\\
    & \val{v}{(\neg \phi)} = \mathtt{T} \text{ o } \val{v}{\tau} = \mathtt{T} & MT 2.23 ($\lor$)(p2)\\
    & \text{Suposición 1}\\
    & \text{Suposición 2}\\
    \hline
    & \therefore \qquad \{(\phi \lor \psi	), ((\neg \phi) \lor \tau)\} \vDash (\psi \lor \tau)
  \end{logic}
\end{logicenv}
  % Suposición 1
  \begin{subproofill}
    \begin{subproof}{Suposición 1}
      \begin{logic}
        & \val{v}{\phi} = \mathtt{T}\\
        & \val{v}{(\neg \phi)} = \mathtt{F} & MT 2.23 ($\neg$)(p0)\\
        & \val{v}{\tau} = \mathtt{T} & MT 2.23 ($\lor$)(p1, p4 Enunciado)\\
        & \val{v}{(\psi \lor \tau)} = \mathtt{T} & MT 2.23 ($\lor$)(p2)
      \end{logic}
    \end{subproof}
  \end{subproofill}

  % Suposición 2
  \begin{subproofill}
    \begin{subproof}{Suposición 2}
      \begin{logic}
        & \val{v}{\phi} = \mathtt{F}\\
        & \val{v}{\psi} = \mathtt{T} & MT 2.23($\lor$)(p0, p3 Enunciado)\\
        & \val{v}{(\psi \lor \tau)} = \mathtt{T} & MT 2.23($\lor$)(p1)
      \end{logic}
    \end{subproof}
  \end{subproofill}

\section{Punto 7}
\begin{logicenv}{$\Gamma \cup \{\phi\} \vDash \psi \text{ sii } \Gamma \vDash (\phi \to \psi)$}
  \begin{logic}
    & \text{demostración 1}\\
    & \text{demostración 2}\\
    \hline
    & \therefore \qquad \Gamma \cup \{\phi\} \vDash \psi \text{ sii } \Gamma \vDash (\phi \to \psi)
  \end{logic}
\end{logicenv}

% demostración 1
\begin{subproofill}
  \begin{subproof}[3]{demostración 1}
    \begin{logic}
      & \text{Si } \Gamma \cup \{\phi\} \vDash \psi \text{ , entonces } \Gamma \vDash (\phi \to \psi)\\
      & \Gamma \cup \{\phi\} \vDash \psi & Suposición\\
      & \text{Existe }\mathbf{v} \text{ tal que} \mathbf{v} { satisface } \Gamma \cup \{\phi\} & Def(p1)\\
      & \mathbf{v} \text{ satisface } \Gamma \text{ y a } \{\phi\} & $\Gamma \cup \{\phi\} := (\exists \xi \,\vert\, \xi \in \Gamma \land \xi \in \{\phi\})$\\
      & \val{v}{\psi} = \mathtt{T} & Def.(p2, p1)\\
      & \val{v}{\phi} = \mathtt{T} & Def.(p3)\\
      & \val{v}{(\phi \to \psi)} = \mathtt{T} & MT 2.23 ($\to$)(p5, p4)\\
      & \mathbf{v} \text{ satisface } \Gamma \text{ y a } (\phi \to \psi) & (p6, p3)\\
      \hline
      & \therefore \text{Si } \Gamma \cup \{\phi\} \vDash \psi \text{ , entonces } \Gamma \vDash (\phi \to \psi)
    \end{logic}
  \end{subproof}
\end{subproofill}

% demostración 2
\begin{subproofill}
  \begin{subproof}[3]{demostración 2}
    \begin{logic}
      & \text{Si } \Gamma \vDash (\phi \to \psi) \text{ , entonces } \Gamma \cup \{\phi\} \vDash \psi\\
      & \Gamma \vDash (\phi \to \psi) & Suposición\\
      & \text{Existe } \mathbf{v} \text{ tal que } \mathbf{v} \text{ satisface } \Gamma\\
      & \val{v}{(\phi \to \psi)} = \mathtt{T} & Def.(p2, p1)\\
      & \val{v}{\phi} = \mathtt{F} \text{ o } \val{v}{\psi} = \mathtt{T} & MT 2.23 ($\to$)(p3)\\
      & \val{v}{\phi} = \mathtt{T} \text{ y } \val{v}{\psi} = \mathtt{T} & Suposición\\
      & \mathbf{v} \text{ satisface } \Gamma \text{ y a} \{\phi\}\\
      \hline
      & \therefore \qquad \text{Si } \Gamma \vDash (\phi \to \psi) \text{ , entonces } \Gamma \cup \{\phi\} \vDash \psi
    \end{logic}
  \end{subproof}
\end{subproofill}


\section{Punto 8}

\begin{logicenv}[4]{$\text{Si } \Delta \not\vDash \phi \text{ y } \Gamma \subset \Delta \text{ , entonces } \Gamma \not\vDash \phi$}
  \begin{logic}
    & \text{Existe } \mathbf{v} \text{ tal que } \mathbf{v} \text{ satisface } \Delta \text{ y } \val{v}{\phi} = \mathtt{F}\\
    & (\forall \xi \, \vert\, \xi \in \Gamma : \xi \in \Delta) & Def. ($\subset$)\\
    & (\forall \xi \, \vert\, \xi \in \Delta : \val{v}{\xi} = \mathtt{T}) & Def. (p0)\\
    & (\forall \xi \, \vert\, \xi \in \Gamma : \val{v}{\xi} = \mathtt{T}) & (p2, p1)\\
    & \mathbf{v} \text{ satisface } \Gamma \text{ y } \val{v}{\phi} = \mathtt{F}\\
    \hline
    & \therefore \qquad \text{Si } \Delta \not\vDash \phi \text{ y } \Gamma \subset \Delta \text{ , entonces } \Gamma \not\vDash \phi
  \end{logic}
\end{logicenv}

\clearpage
\section{Punto 9}

\begin{itemize}
  \item Si Pedro entiende matemáticas, entonces puede entender lógica. Pedro no entiende lógica. Consecuentemente, Pedro no entiende matemáticas.
  \begin{logicenv}{Si Pedro...}
    \begin{center}
      \begin{tabular}{l l}
        $p$ : & Pedro entiende matemáticas\\
        $q$ : & Pedro puede entender lógica\\
        \hline
        & $\Gamma = \{(p \to q), (\neg q)\}, \phi = (\neg p)$
      \end{tabular}
    \end{center}
    \begin{logic}
      & \text{Existe } \mathbf{v} \text{ tal que } \mathbf{v} \text{ satisface } \Gamma & Suposición\\
      & \val{v}{(p \to q)} = \mathtt{T} & Def.(p0)\\
      & \val{v}{(\neg q)} = \mathtt{T} & Def(p0)\\
      & \val{v}{q} = \mathtt{F} & MT 2.23 ($\neg$)(p2)\\
      & \val{v}{p} = \mathtt{F} & MT 2.23 ($\to$)(p3, p1)\\
      & \val{v}{(\neg p)} = \mathtt{T} & MTT 2.23 ($\neg$)(p4)\\
    \end{logic}
  \end{logicenv}
  \vspace*{0.3cm}
  \item Si llueve o cae nieve, entonces no hay electricidad. Llueve. Entonces, no habrá electricidad.
  \begin{logicenv}[3]{Si llueve\dots No habrá electricidad}
    \begin{center}  
      \begin{tabular}{l l}
        $p$ : & Llueve\\
        $q$ : & Nieva\\
        $r$ : & Hay electricidad\\
        \hline
        & $\Gamma = \setd{((p \lor q) \to (\neg r)), p}, \phi = r$
      \end{tabular}
    \end{center}
    \begin{logic}
      & \text{Existe } \mathbf{v} \text{ tal que } \mathbf{v} \text{ satisface } \Gamma & Suposición\\
      & \val{v}{((p \lor q) \to (\neg r))} = \mathtt{T} & Def.(p0)\\
      & \val{v}{p} = \mathtt{T} & Def.(p0)\\
      & \val{v}{(p \lor q)} = \mathtt{T} & MT 2.23 ($\lor$)(p2)\\
      & \val{v}{(\neg r)} = \mathtt{T} & MT 2.23 ($\to$)(p3, p1)\\
      & \val{v}{r} = \mathtt{F} & MT 2.23 ($\neg$)(p4)
    \end{logic}
  \end{logicenv}

  \vspace*{.3cm}
  \item Si llueve o cae nieve, entonces no hay electricidad. Hay electricidad. Entonces no nevó.
  \begin{logicenv}{Si llueve\dots No nevó}
    \begin{center}  
      \begin{tabular}{l l}
        $p$ : & Llueve\\
        $q$ : & Nieva\\
        $r$ : & Hay electricidad\\
        \hline
        & $\Gamma = \setd{((p \lor q) \to (\neg r)), r}, \phi = (\neg q)$
      \end{tabular}
    \end{center}
    \begin{logic}
      & \text{Existe } \mathbf{v} \text{ tal que } \mathbf{v} \text{ satisface } \Gamma & Suposición\\
      & \val{v}{((p \lor q) \to (\neg r))} = \mathtt{T} & Def.(p0)\\
      & \val{v}{(r)} = \mathtt{T} & Def.(p0)\\
      & \val{v}{(p \lor q)} = \mathtt{F} & MT 2.23 ($\to$)(p2, p1)\\
      & \val{v}{p} = \mathtt{F} \text{ y } \val{v}{q} = \mathtt{F} & MT 2.23 ($\lor$)(p3)\\
      & \val{v}{(\neg q)} = \mathtt{T} & MT 2.23 ($\neg$)(p4)
    \end{logic} 
  \end{logicenv}

  \vspace*{.3cm}
  \item Es peligroso conducir cuando está nevando. Esta nevando ahora. Sería peligroso conducir en este momento.
  \begin{logicenv}{Es peligroso conducir\dots}
    \begin{center}  
      \begin{tabular}{l l}
        $p$ : & Es peligroso conducir\\
        $q$ : & Nieva\\
        \hline
        & $\Gamma = \setd{(p \gets q), q}, \phi = p$
      \end{tabular}
    \end{center}
    \begin{logic}
      & \text{Existe } \mathbf{v} \text{ tal que } \mathbf{v} \text{ satisface } \Gamma & Suposición\\
      & \val{v}{(p \gets q)} = \mathtt{T} & Def.(p0)\\
      & \val{v}{q} = \mathtt{T} & Def.(p0)\\
      & \val{v}{p} = \mathtt{T} & MT 2.23($\gets$)(p2)
    \end{logic} 
  \end{logicenv}
  \vspace*{.3cm}

  \item Cuando llueve los árboles se mojan. Los árboles están húmedos esta mañana, así que llovió anoche.
  \begin{logicenv}{Cuando llueve\dots}
    \begin{center}  
      \begin{tabular}{l l}
        $p$ : & Llueve\\
        $q$ : & Los árboles se mojan\\
        \hline
        & $\Gamma = \setd{(p \to q), p}, \phi = q$
      \end{tabular}
    \end{center}
    \begin{logic}
      & \text{Existe } \mathbf{v} \text{ tal que } \mathbf{v} \text{ satisface } \Gamma & Suposición\\
      & \val{v}{(p \to q)} = \mathtt{T} & Def.(p0)\\
      & \val{v}{p} = \mathtt{T} & Def.(p0)\\
      & \val{v}{q} = \mathtt{T} & MT 2.23($\gets$)(p2)
    \end{logic} 
  \end{logicenv}
  \vspace*{.3cm}

  \item Un paraguas evita que se moje bajo la lluvia. Alicia tomó su paraguas y no se mojó. Probablemente estaba lloviendo.
  
  \begin{logicenv}{Un paraguas\dots}
    \begin{center}  
      \begin{tabular}{l l}
        $p$ : & Alicia toma su paraguas\\
        $q$ : & Alicia se moja\\
        $r$ : & Llueve\\
        \hline
        & $\Gamma = \setd{((p \land r) \to (\neg q)), (p \land (\neg q))}, \phi = (r \lor (\neg r))$
      \end{tabular}
    \end{center}
    \begin{logic}
      & \text{Existe } \mathbf{v} \text{ tal que } \mathbf{v} \text{ satisface } \Gamma & Suposición\\
      & \val{v}{((p \land r) \to (\neg q))} = \mathtt{T} & Def.(p0)\\
      & \val{v}{(p \land (\neg q))} = \mathtt{T} & Def.(p0)\\
      & \val{v}{(r \lor (\neg r))} = \mathtt{T} & MT 2.23($\lor$), MT 2.19 N 2.20
    \end{logic} 
  \end{logicenv}
  \vspace*{.3cm}
  \item Las luces rojas previenen accidentes. Miguél no tuvo un accidente, por lo tanto, Miguél se detuvo en una luz roja.
  
  \begin{logicenv}{Las luces rojas\dots}
    \begin{center}  
      \begin{tabular}{l l}
        $p$ : & Miguél tiene un accidente\\
        $q$ : & Miguél se detiene en la luz roja\\
        \hline
        & $\Gamma = \setd{(q \to (\neg p)), q}, \phi = q$
      \end{tabular}
    \end{center}
    \begin{logic}
      & \text{Existe } \mathbf{v} \text{ tal que } \mathbf{v} \text{ satisface } \Gamma & Suposición\\
      & \val{v}{(q \to (\neg p))} = \mathtt{T} & Def.(p0)\\
      & \val{v}{q} = \mathtt{T} & Def.(p0)\\
      & \val{v}{q} = \mathtt{F} & MT 2.23 ($\to$)(p2, p1)
    \end{logic} 
  \end{logicenv}
  \vspace*{.3cm}

  \item Si $sin(x)$ es diferenciable, entonces $sin(x)$ es continua. Si $sin(x)$ es continua, entonces $sin(x)$ es diferenciable. La función $sin(x)$ es diferenciable. Consecuentemente, la función $sin(x)$ es  integrable.
  
  \begin{logicenv}{Si $sin(x)$ \dots}
    \begin{center}  
      \begin{tabular}{l l}
        $p$ : & $sin(x)$ es diferenciable\\
        $q$ : & $sin(x)$ es continua\\
        $r$ : & $sin(x)$ es integrable\\
        \hline
        & $\Gamma = \setd{(p \to q), (q \to p), p}, \phi = r$
      \end{tabular}
    \end{center}
    \begin{logic}
      & \text{Existe } \mathbf{v} \text{ tal que } \mathbf{v} \text{ satisface } \Gamma & Suposición\\
      & \val{v}{(p \to q)} = \mathtt{T} & Def.(p0)\\
      & \val{v}{(q \to p)} = \mathtt{T} & Def.(p0)\\
      & \val{v}{p} = \mathtt{T} & Def.(p0)\\
    \end{logic} 
    Desde el punto de vista de la lógica (hasta el tema que hemos visto), la argumentación no es válida, sin embargo, es válida si es posible añadir el teorema fundamental del cálculo como razón para concluir $r$ .
  \end{logicenv}
  \vspace*{.3cm}
  \item Si Gödel fuera presidente, entonces el Congreso presentaría leyes razonables. Gödel no es presidente. Por lo tanto, el Congreso no presenta leyes razonables.
  
  \begin{logicenv}{Si Gödel fuera presidente\dots}
    \begin{center}  
      \begin{tabular}{l l}
        $p$ : & Gödel es presidente\\
        $q$ : & El Congreso presenta leyes razonables\\
        \hline
        & $\Gamma = \setd{(p \to q), (\neg p)}, \phi = (\neg q)$
      \end{tabular}
    \end{center}
    \begin{logic}
      & \text{Existe } \mathbf{v} \text{ tal que } \mathbf{v} \text{ satisface } \Gamma & Suposición\\
      & \val{v}{(p \to q)} = \mathtt{T} & Def.(p0)\\
      & \val{v}{(\neg p)} = \mathtt{T} & Def.(p0)\\
      & \val{v}{p} = \mathtt{F} & MT 2.23 ($\neg$)\\
      & \val{v}{q} = \mathtt{F} \text{ o } \mathtt{T} & MT 2.23 ($\to$)(p3, p2)
    \end{logic} 
    La argumentación no es válida, y es una falacia.
  \end{logicenv}
  \vspace*{.3cm}
  \item Si llueve, entonces no hay picnic. Si cae nieve, entonces no hay picnic. Llueve o cae nieve. Por lo tanto, no hay picnic.
  
  \begin{logicenv}[4]{Si llueve\dots no hay picnic}
    \begin{center}  
      \begin{tabular}{l l}
        $p$ : & Llueve\\
        $q$ : & Hay picnic\\
        $r$ : & Nieva\\
        \hline
        & $\Gamma = \setd{(p \to (\neg q)), (r \to (\neg q)), (p \lor r)}, \phi = (\neg q)$
      \end{tabular}
    \end{center}
    \begin{logic}
      & \text{Existe } \mathbf{v} \text{ tal que } \mathbf{v} \text{ satisface } \Gamma & Suposición\\
      & \val{v}{(p \to (\neg q))} = \mathtt{T} & Def.(p0)\\
      & \val{v}{(r \to (\neg q))} = \mathtt{T} & Def.(p0)\\
      & \val{v}{(p \lor r)} = \mathtt{T} & Def.(p0)\\
      & \val{v}{p} = \mathtt{T} \text{ o } \val{v}{r} = \mathtt{T} & MT 2.23 ($\lor$)(p3)\\
      & \val{v}{((\neg p) \land (\neg r))} = \mathtt{F} & MT 2.23 ($\lor$)\\
      & (\not\exists \mathbf{v}\, \vert\, \val{v}{p} = \mathtt{F} \text{ y } \val{v}{r} = \mathtt{F})\\
      & \val{v}{(\neg q)} = \mathtt{T} & MT 2.23 ($\to$)(p6, p2, p1)
    \end{logic} 
  \end{logicenv}
  \vspace*{.3cm}
\end{itemize}
\end{document}