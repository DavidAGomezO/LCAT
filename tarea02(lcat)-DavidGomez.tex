\documentclass{article}

% librerías:
\usepackage{amsmath, mathtools, amssymb, mathrsfs, amsthm, multicol, nicematrix}
\usepackage{lmodern, geometry, graphicx, forest}
\usepackage[T1]{fontenc}
\usepackage[spanish]{babel}

%configuraciones:
\setlength{\columnseprule}{1pt}
\geometry{
    left = 20mm,
    right = 20mm,
    top = 20mm
}
\hyphenpenalty=10000
\graphicspath{ {C:/Users/usuario/Documents/U/} }

%texto
\begin{document}
\begin{titlepage}
	\begin{center}
		\vspace*{1cm}

		\textbf{\Large{Tarea No.1}}

		\vspace{1.5cm}

		\textbf{David Gómez}

		\vspace{5cm}

		\includegraphics[width=\textwidth]{logo-eci.jpg}

		\vspace{5cm}

		Matemáticas\\
		Escuela Colombiana de Ingeniería Julio Garavito\\
		Colombia\\
		\today

	\end{center}
\end{titlepage}
\section{Sección 1.1}
\begin{itemize}
	\item \textbf{Punto 1}
	\item[] Proposiciones:
		\begin{itemize}
			\item La lógica estudia las reglas de deducción formales
			\item la lógica proporciona reglas
			\item la lógica matemática se ocupa de la posibilidad de axiomatizar las teorías matemáticas
		\end{itemize}
	\item Frases (no proposiciones):
	      \begin{itemize}
		      \item En un nivel elemental
		      \item En un nivel avanzado
		      \item La teoría de la demostración
	      \end{itemize}
	\item \textbf{Punto 2}
	\item[] Proposiciones:
		\begin{itemize}
			\item Una especificación formal usa notación matemática para describir de manera precisa las propiedades que un sistema de información debe tener
			\item Describe lo que el sistema debe hacer sin decir cómo se va a hacer.
			\item Esta abstracción hace que las especificaciones formales sean útiles en el proceso de desarrollar un sistema
		\end{itemize}
	\item[] Frases (no proposiciones)
		\begin{itemize}
			\item especular sobre el significado de frases en un impreciso Pseudocódigo.
			\item como para aquellos que desarrollan los programas para satisfacer esos requerimientos
			\item promover un entendimiento común entre todos los interesados en el sistema.
		\end{itemize}
	\item \textbf{Punto 3}
	      a) y b) tienen la misma estructura.
\end{itemize}
\section{Sección 1.2}
\begin{itemize}
	\item \textbf{Punto 1}
	      \begin{itemize}
		      \item[] b)
			      \begin{equation*}
				      (\textrm{\textit{true}} \equiv \textrm{\textit{false}}) \tag*{Cumple con la definición de $\phi ::= (\phi \equiv \phi)$}
			      \end{equation*}
		      \item[] d)
			      \begin{equation*}
				      (p \vee (p \equiv (\lnot q))) \tag*{Cumple con la definición de $\phi ::= (\phi \vee \phi)$}
			      \end{equation*}
		      \item[] f)
			      \begin{equation*}
				      ((q \wedge (\lnot q))\leftarrow (\lnot(\lnot(\lnot(p \rightarrow r))))) \tag*{Cumple con la definición de $\phi ::= (\phi \leftarrow \phi)$}
			      \end{equation*}
	      \end{itemize}
	\item \textbf{Punto 2}
	      \begin{itemize}
		      \item[] a)
			      \begin{equation*}
				      (p \vee) \tag*{No se encuentra en la definición de $\phi$ (hace falta algún $\phi$ despues de $\vee$)}
			      \end{equation*}
		      \item[] c)
			      \begin{equation*}
				      \lnot p \tag*{No se encuentra en la definición de $\phi$ (hace falta cerrar todo con "()")}
			      \end{equation*}
		      \item[] e)
			      \begin{equation*}
				      (p \vee q) \vee r \tag*{No se encuentra en la definición de $\phi$ (hace falta cerrar todo con "()")}
			      \end{equation*}
	      \end{itemize}
	\item \textbf{Punto 3}
	      \begin{itemize}
		      \item[] a)
			      \begin{center}
				      \begin{NiceTabular}{l}
					      $p$ : Un número natural es par   \\
					      $q$ : Un número natural es impar \\
					      \makebox[5cm]{\hrulefill}        \\
					      $(p \equiv (\lnot q))$
				      \end{NiceTabular}
			      \end{center}
		      \item[] b)
			      \begin{center}
				      \begin{NiceTabular}{l}
					      $p$ : El sol brilla hoy    \\
					      $q$ : El sol brilla mañana \\
					      \makebox[4cm]{\hrulefill}  \\
					      $(p \rightarrow (\lnot q))$
				      \end{NiceTabular}
			      \end{center}
		      \item[] c)
			      \begin{center}
				      \begin{NiceTabular}{l}
					      $p$ : Juan está celoso       \\
					      $q$ : Juan está de mal genio \\
					      \makebox[4.5cm]{\hrulefill}  \\
					      $p \vee q$
				      \end{NiceTabular}
			      \end{center}
		      \item[] d)
			      \begin{center}
				      \begin{NiceTabular}{l}
					      $p$ : Una petición ocurre               \\
					      $q$ : Una petición será atendida        \\
					      $r$ : El proceso de horarios se bloquea \\
					      \makebox[6cm]{\hrulefill}               \\
					      $((p \Rightarrow q) \vee r)$
				      \end{NiceTabular}
			      \end{center}
		      \item[] e)
			      \begin{center}
				      \begin{NiceTabular}{l}
					      $p$ : Llueve              \\
					      $q$ : Hace sol            \\
					      \makebox[2cm]{\hrulefill} \\
					      $((p \vee q) \wedge (\lnot(p \wedge q)))$
				      \end{NiceTabular}
			      \end{center}
		      \item[] f)
			      \begin{center}
				      \begin{NiceTabular}{l}
					      $p$ : (tener) Zapatos                \\
					      $q$ : (tener) Camisa                 \\
					      $r$ : hay servicio en el restaurante \\
					      \makebox[5cm]{\hrulefill}            \\
					      $((\lnot(p \vee q)) \Rightarrow (\lnot r))$
				      \end{NiceTabular}
			      \end{center}
		      \item[] g)
			      \begin{center}
				      \begin{NiceTabular}{l}
					      $p_0$ : Mi hermana quiere un gato blanco \\
					      $q_0$ : Mi hermana quiere un gato negro  \\
					      $p_1$ : Mi hermana quiere un gato        \\
					      $q_1$ : El gato es blanco                \\
					      $r_1$ : El gato es negro                 \\
					      \makebox[5cm]{\hrulefill}                \\
					      $\Phi \equiv (p_0 \wedge q_0)$           \\
					      $\vee$                                   \\
					      $\Phi \equiv (p_1 \wedge (q_1 \wedge r_1))$
				      \end{NiceTabular}
			      \end{center}
		      \item[] h)
			      \begin{center}
				      \begin{NiceTabular}{l}
					      $p$ : Mi pareja raja            \\
					      $q$ : Mi pareja presta el hacha \\
					      \makebox[5cm]{\hrulefill}       \\
					      $((\lnot p) \wedge (\lnot q))$
				      \end{NiceTabular}
			      \end{center}
	      \end{itemize}
	\item \textbf{Punto 4}
	      $h$ : El cuarteto interpretará a Haydn.
	      $m$ : El cuarteto interpretará a Mozart.
	      \begin{itemize}
		      \item[] a) $(h \vee m)$\\
			      El cuarteto interpretará a Haydn o a Mozart.
		      \item[] b) $(h \equiv (\lnot m))$\\
			      El cuarteto interpreta a Haydn solo si no interpreta a Mozart.
		      \item[] c) $(\lnot (h \equiv m))$\\
			      El cuarteto interpretará a Haydn o a Mozart, pero no ambos.
		      \item[] d) $(\lnot (h \wedge m))$\\
			      El cuarteto no interpreta a Haydn o no interpreta a Mozart.
		      \item[] e) $(((\lnot (h \wedge m)) \wedge (\lnot h)) \rightarrow m)$\\
			      si sucede que, el cuarteto no interpreta a Haydn o no interpreta a Mozart y no interpretan a Haydn, entonces interpretan a Mozart.
	      \end{itemize}
	\item \textbf{Punto 5}
	      \begin{itemize}
		      \item[] a) Si Pedro entiende matemáticas, entonces puede entender lógica. Pedro no entiende lógica.\\Consecuentemente, Pedro no entiende matemáticas.
			      \begin{center}
				      $p$ : Pedro entiende matemáticas.\\
				      $q$ : Pedro puede entender lógica.\\
				      \makebox[5cm]{\hrulefill}\\
				      \begin{NiceTabular}{l l}
					      0. $(p \rightarrow q)$ & axioma/ enunciado \\
					      1. $(\lnot q)$         & axioma/ enunciado \\
					      2. $(\lnot p)$         & MTT, (0,1)
				      \end{NiceTabular}
			      \end{center}
		      \item[] b) Si llueve o cae nieve, entonces no hay electricidad. Llueve. Entonces, no habrá electricidad.
			      \begin{center}
				      $p$ : Llueve\\
				      $q$ : Cae nieve\\
				      $r$ : Hay electricidad\\
				      \makebox[6.5cm]{\hrulefill}\\
				      \begin{NiceTabular}{l l}
					      0. $((p \vee q) \rightarrow (\lnot r))$ & axioma/ enunciado \\
					      1. $(q)$                                & axioma/ enunciado \\
					      2. $(\lnot r)$                          & MPP, (0,1)        \\
				      \end{NiceTabular}
			      \end{center}
		      \item[] c) Si llueve o cae nieve, entonces no hay electricidad. Hay electricidad. Entonces no nevó.
			      \begin{center}
				      $p$ : Llueve\\
				      $q$ : Cae nieve\\
				      $r$ : Hay electricidad\\
				      \makebox[6.5cm]{\hrulefill}\\
				      \begin{NiceTabular}{l l}
					      0. $((p \vee q) \rightarrow (\lnot r))$ & axioma/ enunciado   \\
					      1. $(r)$                                & axioma/ enunciado   \\
					      2. $(\lnot(p \vee q))$                  & MTT, (0, 1)         \\
					      3. $((\lnot p) \wedge (\lnot q))$       & DM, (2)             \\
					      4. $(\lnot q)$                          & Simplificación, (3)
				      \end{NiceTabular}
			      \end{center}
		      \item[] d) Si $\sin x$ es diferenciable, entonces $\sin x$ es continua. Si $\sin x$ es continua, entonces $\sin x$ es diferenciable. La función $\sin x$ es diferenciable. Consecuentemente, la función $\sin x$ es integrable.
			      \begin{center}
				      $p$ : $\sin x$ es diferenciable.\\
				      $q$ : $\sin x$ es continua\\
				      $r$ : $\sin x$ es integrable\\
				      \makebox[7cm]{\hrulefill}\\
				      \begin{NiceTabular}{l l}
					      0. $(p \rightarrow q)$ & axioma/ enunciado               \\
					      1. $(q \rightarrow p)$ & axioma/ enunciado               \\
					      2. $p$                 & axioma/ enunciado               \\
					      3. $(p \equiv q)$      & def. equivalencia, (0, 1)       \\
					      4. $q$                                                   \\
					      5. $r$                 & Teorema fundamental del cálculo
				      \end{NiceTabular}
			      \end{center}
		      \item[] e) Si Gödel fuera presidente, entonces el Congreso presentaría leyes razonables. Gödel no es presidente. Por lo tanto, el Congreso no presenta leyes razonables.
			      \begin{center}
				      $p$ : Gödel es presidente\\
				      $q$ : El congreso presenta leyes razonables\\
				      \makebox[7.5cm]{\hrulefill}\\
				      \begin{NiceTabular}{l l}
					      0. $(p \rightarrow q)$                               \\
					      1. $(\lnot p)$                                       \\
					      2. $(\lnot q)$ & (falacia: negación del antecedente)
				      \end{NiceTabular}
			      \end{center}
		      \item[] f) Si llueve, entonces no hay picnic. Si cae nieve, entonces no hay picnic. Llueve o cae nieve. Por lo tanto, no hay picnic.
			      \begin{center}
				      $p$ : Llueve.\\
				      $q$ : Hay picnic.\\
				      $r$ : Cae nieve.\\
				      \makebox[6.5cm]{\hrulefill}\\
				      \begin{NiceTabular}{l l}
					      0. $(p \rightarrow (\lnot q))$          & axioma/ enunciado    \\
					      1. $(r \rightarrow (\lnot q))$          & axioma/ enunciado    \\
					      2. $(p \vee r)$                         & axioma/ enunciado    \\
					      3. $((p \vee r) \rightarrow (\lnot q))$ & equivalencia, (0, 1) \\
					      4. $(\lnot q)$                          & MPP, (3, 2)
				      \end{NiceTabular}
			      \end{center}
	      \end{itemize}
\end{itemize}
\section{Sección 1.3}
\begin{itemize}
	\item \textbf{Punto 1}
	      \begin{itemize}
		      \item[] b) \textit{true}
			      \begin{center}
				      \textit{true}
			      \end{center}
		      \item[] d) $((p \wedge (\lnot q)) \to r)$
			      \begin{center}
				      \begin{forest}
					      for tree={parent anchor=south,child anchor=north}
					      [$\to$
					      [$\wedge$
							      [$p$]
								      [$\lnot$
									      [$q$]
								      ]
						      ]
						      [$r$]
					      ]
				      \end{forest}
			      \end{center}
		      \item[] f) $(p \to (q \to p))$
			      \begin{center}
				      \begin{forest}
					      for tree={parent anchor=south,child anchor=north}
					      [
					      $\to$
					      [
							      [$p$]
								      [
									      $\to$
									      [
											      [$q$]
												      [$p$]
										      ]
								      ]
						      ]
					      ]
				      \end{forest}
			      \end{center}
	      \end{itemize}
	\item \textbf{Punto 2}
	      \begin{itemize}
		      \item[] a)$p$\\
			      $p$
		      \item[] c) $((p \wedge (\lnot q)) \to r)$\\
			      $p, q, r, (\lnot q), (p \wedge(\lnot q)), ((p \wedge (\lnot q)) \to r) $
		      \item[] e) $(p \to (q \to p))$\\
			      $p, q, (q \to p), (p \to (q \to p))$
	      \end{itemize}
	\item \textbf{Punto 3}
	      \begin{itemize}
		      \item[] a) Una proposición que es una negación de una equivalencia.
			      \begin{center}
				      \begin{forest}
					      for tree={parent anchor=south,child anchor=north}
					      [
					      $\lnot$
					      [
							      $\equiv$
							      [
									      [$\phi$]
										      [$\psi$]
								      ]
						      ]
					      ]
				      \end{forest}
			      \end{center}
		      \item[] b) Una proposición que es una disyunción cuyos disyuntos ambos son conjunciones.
			      \begin{center}
				      \begin{forest}
					      for tree={parent anchor=south,child anchor=north}
					      [
					      $\vee$
					      [
							      $\wedge$
							      [$\phi_0$]
								      [$\phi_1$]
						      ]
						      [
							      $\wedge$
							      [$\phi_2$]
								      [$\phi_3$]
						      ]
					      ]
				      \end{forest}
			      \end{center}
		      \item[] c) Una proposición que es una conjunción de conjunciones.
			      \begin{center}
				      \begin{forest}
					      for tree={parent anchor=south,child anchor=north}
					      [
					      $\wedge$
					      [
							      $\wedge$
							      [$\phi_0$]
								      [$\phi_1$]
						      ]
						      [
							      $\wedge$
							      [$\phi_2$]
								      [$\phi_3$]
						      ]
					      ]
				      \end{forest}
			      \end{center}
		      \item[] d) Una proposición que es una implicación cuyo antecedente es una negación y consecuente es una equivalencia.
			      \begin{center}
				      \begin{forest}
					      for tree={parent anchor=south,child anchor=north}
					      [
					      $\to$
					      [
							      $\lnot$
							      [$\phi_0$]
						      ]
						      [
							      $\equiv$
							      [$\phi_1$]
								      [$\phi_2$]
						      ]
					      ]
				      \end{forest}
			      \end{center}
		      \item[] e)Una proposición que es una consecuencia cuyo antecedente es una disyunción y consecuente una discrepancia.
			      \begin{center}
				      \begin{forest}
					      for tree={parent anchor=south,child anchor=north}
					      [
					      $\gets$
					      [
							      $\not\equiv$
							      [$\phi_2$]
								      [$\phi_3$]
						      ]
						      [
							      $\vee$
							      [$\phi_0$]
								      [$\phi_1$]
						      ]
					      ]
				      \end{forest}
			      \end{center}
	      \end{itemize}
	\item \textbf{Punto 4}
	      $(((p \vee q) \equiv (r \to p)) \gets (q \not\equiv (\lnot q)))$
\end{itemize}
\section{Sección 1.4}
\begin{itemize}
	\item \textbf{Punto 2}
	      \begin{alignat*}{2}
		      \intertext{Con $(\lnot \psi)$}
		      L[(\lnot \psi)]           & = 1 + L[\psi]                                         \\
		                                & = 1 + R[\psi] \tag*{Hipótesis de inducción}           \\
		                                & = R[(\lnot \psi)] \tag*{$\Box$}
		      \intertext{Con $(\psi \not\equiv \tau)$}
		      L[(\psi \not\equiv \tau)] & = 1 + L[\psi] + L[\tau]                               \\
		                                & = 1 + R[\psi] + R[\tau] \tag*{Hipótesis de inducción} \\
		                                & = R[(\psi \not\equiv \tau)] \tag*{$\Box$}
		      \intertext{Con $(\psi \vee \tau)$}
		      L[(\psi \vee \tau)]       & = 1 + L[\psi] + L[\tau]                               \\
		                                & = 1 + R[\psi] + R[\tau] \tag*{Hipótesis de inducción} \\
		                                & = R[(\psi \vee \tau)] \tag*{$\Box$}
		      \intertext{Con $(\psi \wedge \tau)$}
		      L[(\psi \wedge \tau)]     & = 1 + L[\psi] + L[\tau]                               \\
		                                & = 1 + R[\psi] + R[\tau] \tag*{Hipótesis de inducción} \\
		                                & = R[(\psi \wedge \tau)] \tag*{$\Box$}
		      \intertext{Con $(\psi \to \tau)$}
		      L[(\psi \to \tau)]        & = 1 + L[\psi] + L[\tau]                               \\
		                                & = 1 + R[\psi] + R[\tau] \tag*{Hipótesis de inducción} \\
		                                & = R[(\psi \to \tau)] \tag*{$\Box$}
		      \intertext{Con $(\psi \gets \tau)$}
		      L[(\psi \gets \tau)]      & = 1 + L[\psi] + L[\tau]                               \\
		                                & = 1 + R[\psi] + R[\tau] \tag*{Hipótesis de inducción} \\
		                                & = R[(\psi \gets \tau)] \tag*{$\Box$}
	      \end{alignat*}
	\item \textbf{Punto 3}
	      \begin{equation*}
		      L[(x] :=
		      \begin{cases}
			      L[x] + 1 \\
			      L[(] = 1
		      \end{cases}
	      \end{equation*}
	      \begin{equation*}
		      R[x)] :=
		      \begin{cases}
			      R[x] + 1 \\
			      R[)] = 1
		      \end{cases}
	      \end{equation*}
	\item \textbf{Punto 7}\\
	      funciones auxiliares:
	      \begin{equation*}
		      EQ[x \equiv y] :=
		      \begin{cases}
			      EQ[x] + EQ[y] + 1 \\
			      EQ[\equiv] = 1
		      \end{cases}
	      \end{equation*}
	      \begin{equation*}
		      T[x\ \textrm{\textit{true}}\ y] :=
		      \begin{cases}
			      T[x] + T[y] + 1 \\
			      T[\textrm{\textit{true}}] = 1
		      \end{cases}
	      \end{equation*}
	      \begin{equation*}
		      F[x \ \textrm{\textit{false}}\  y] :=
		      \begin{cases}
			      F[x] + F[y] + 1 \\
			      F[\textrm{\textit{false}}] = 1
		      \end{cases}
	      \end{equation*}
	      \begin{equation*}
		      \textrm{Conectores} : \{\lnot, \wedge, \vee, \to, \gets, \not\equiv\}
	      \end{equation*}
	      \begin{alignat*}{2}
		      \textrm{con}          & \in \textrm{Conectores} \\
			  \textrm{otro} & \notin \textrm{Conectores} \\[0.5cm]	
		      S[x\ \textrm{con}\ y] & :=
		      \smash{\left\{\begin{array}{@{}l@{}}
				                    S[x] + S[y] + 1 \\[\jot]
				                    S[\textrm{con}] = 1\\[\jot]
									S[\textrm{otro}] = 0
			                    \end{array}\right.}
	      \end{alignat*}
	      \begin{alignat*}{2}
		      \intertext{Propiedad}
		      M(\phi)&: \smash{\left\{\begin{array}{@{}l@{}}
												(\forall \phi : ((EQ[\phi] \geq 1) \wedge (F[\phi] = 0) \wedge (T[\phi] = 0)))\\[\jot]
			  									M(\phi) \Rightarrow (\exists u\, \vert\, u \in \phi \wedge S[u] = 0)
											\end{array}\right.}
		      \intertext{\textbf{Caso base: $\phi = (p_0 \equiv p_1)$}}
			EQ[\phi] &= 1 \tag*{$\checkmark$}\\
			F[\phi] &= 0\tag*{$\checkmark$}\\
			T[\phi] &= 0\tag*{$\checkmark$}\\
			M(\phi) &\Rightarrow (\exists u\, \vert\, u \in \phi \wedge S[u] = 0) \tag*{\textit{true}}
			\intertext{\textbf{Casos inductivos:} $(\lnot \psi), (\psi \equiv \tau), (\psi \not\equiv \tau), (\psi \vee \tau), (\psi \wedge \tau), (\psi \to \tau), (\psi \gets \tau) $}
			\intertext{Con $\phi = (\lnot \psi)$}
			EQ[(\lnot \psi)] &\geq 1 \tag*{$\checkmark$}\\
			F[(\lnot \psi)] &= 0\tag*{$\checkmark$}\\
			T[(\lnot \psi)] &= 0\tag*{$\checkmark$}\\
			M((\lnot \psi)) &\Rightarrow (\exists u\, \vert\, u \in \phi \wedge S[u] = 0) \tag*{\textit{true}}
			\intertext{Con $\phi = (\psi \equiv \tau)$}
			EQ[(\psi \equiv \tau)] &= 1 + EQ[\psi] + EQ[\tau]\\
			EQ[(\psi \equiv \tau)] &\geq 1 \tag*{$\checkmark$}\\
			F[(\psi \equiv \tau)] &= 0\tag*{$\checkmark$}\\
			T[(\psi \equiv \tau)] &= 0\tag*{$\checkmark$}\\
			M((\psi \equiv \tau)) &\Rightarrow (\exists u\, \vert\, u \in \phi \wedge S[u] = 0) \tag*{\textit{true}}
			\intertext{Con $\phi = (\psi \not\equiv \tau)$}
			EQ[(\psi \not\equiv \tau)] &= EQ[\psi] + EQ[\tau]\\
			EQ[(\psi \not\equiv \tau)] &\geq 1 \tag*{$\checkmark$}\\
			F[(\psi \not\equiv \tau)] &= 0\tag*{$\checkmark$}\\
			T[(\psi \not\equiv \tau)] &= 0\tag*{$\checkmark$}\\
			M((\psi \not\equiv \tau)) &\Rightarrow (\exists u\, \vert\, u \in \phi \wedge S[u] = 0) \tag*{\textit{true}}
			\intertext{Con $\phi = (\psi \not\equiv \tau)$}
			EQ[(\psi \vee \tau)] &= EQ[\psi] + EQ[\tau]\\
			EQ[(\psi \vee \tau)] &\geq 1 \tag*{$\checkmark$}\\
			F[(\psi \vee \tau)] &= 0\tag*{$\checkmark$}\\
			T[(\psi \vee \tau)] &= 0\tag*{$\checkmark$}\\
			M((\psi \vee \tau)) &\Rightarrow (\exists u\, \vert\, u \in \phi \wedge S[u] = 0) \tag*{\textit{true}}
			\intertext{Con $\phi = (\psi \not\equiv \tau)$}
			EQ[(\psi \vee \tau)] &= EQ[\psi] + EQ[\tau]\\
			EQ[(\psi \vee \tau)] &\geq 1 \tag*{$\checkmark$}\\
			F[(\psi \vee \tau)] &= 0\tag*{$\checkmark$}\\
			T[(\psi \vee \tau)] &= 0\tag*{$\checkmark$}\\
			M((\psi \vee \tau)) &\Rightarrow (\exists u\, \vert\, u \in \phi \wedge S[u] = 0) \tag*{\textit{true}}
			\intertext{Con $\phi = (\psi \wedge \tau)$}
			EQ[(\psi \wedge \tau)] &= EQ[\psi] + EQ[\tau]\\
			EQ[(\psi \wedge \tau)] &\geq 1 \tag*{$\checkmark$}\\
			F[(\psi \wedge \tau)] &= 0\tag*{$\checkmark$}\\
			T[(\psi \wedge \tau)] &= 0\tag*{$\checkmark$}\\
			M((\psi \wedge \tau)) &\Rightarrow (\exists u\, \vert\, u \in \phi \wedge S[u] = 0) \tag*{\textit{true}}
			\intertext{Con $\phi = (\psi \to \tau)$}
			EQ[(\psi \to \tau)] &= EQ[\psi] + EQ[\tau]\\
			EQ[(\psi \to \tau)] &\geq 1 \tag*{$\checkmark$}\\
			F[(\psi \to \tau)] &= 0\tag*{$\checkmark$}\\
			T[(\psi \to \tau)] &= 0\tag*{$\checkmark$}\\
			M((\psi \to \tau)) &\Rightarrow (\exists u\, \vert\, u \in \phi \wedge S[u] = 0) \tag*{\textit{true}}
			\intertext{Con $\phi = (\psi \gets \tau)$}
			EQ[(\psi \gets \tau)] &= EQ[\psi] + EQ[\tau]\\
			EQ[(\psi \gets \tau)] &\geq 1 \tag*{$\checkmark$}\\
			F[(\psi \gets \tau)] &= 0\tag*{$\checkmark$}\\
			T[(\psi \gets \tau)] &= 0\tag*{$\checkmark$}\\
			M((\psi \gets \tau)) &\Rightarrow (\exists u\, \vert\, u \in \phi \wedge S[u] = 0) \tag*{\textit{true}}
		  \end{alignat*}
\end{itemize}
\end{document}