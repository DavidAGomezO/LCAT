\documentclass{article}

% Librerías:
\usepackage{amsmath, mathtools, amssymb, mathrsfs, amsthm, nicematrix, array}
\usepackage{lmodern, graphicx, fancyhdr}
\usepackage[margin = 2cm, top = 2.5cm, includefoot]{geometry}
\usepackage{xcolor}
\usepackage[hidelinks]{hyperref}
\usepackage[T1]{fontenc}
\usepackage[spanish]{babel}
\usepackage{listings}
\usepackage{tikz}
\usetikzlibrary{shapes, fit, tikzmark}
\usepackage{color, colortbl}
\usepackage[most]{tcolorbox}

% Configuraciones:
\pagestyle{fancy}
\fancyhf{}
\setlength{\headheight}{55.34027pt}
\rhead{\textit{David Gómez}}
\lhead{\includegraphics[width = 4cm]{\logo}}
\lfoot{Página \thepage}
\rfoot{\titlename}
\renewcommand{\headrule}{\hbox to \headwidth{\color{rojoEci}\leaders\hrule height \headrulewidth\hfill}}
\renewcommand{\footrulewidth}{0.4pt}

\hyphenpenalty=10000

\newcommand{\logo}{C:/Users/usuario/Documents/U/logo-eci.jpg}

\newcommand{\q}[1]{``#1''}
\newcommand{\und}[2]{#1\textbf{\_}#2}
\newcommand{\boolun}[2][]{H_{#2}(#1)}
\newcommand{\boolbin}[3]{H_{#1}(#2, #3)}
\newcommand{\val}[2]{\mathbf{#1}[#2]}

\setlength{\parindent}{0pt}

%%%%%%%%%%%%%%%%%%%%%%%%%%%%%%%%%%
%%%%%%%%%%%%%%%%%%%%%%%%%%%%%%%%%%
\newcommand{\titlename}{Tarea 04}%
%%%%%%%%%%%%%%%%%%%%%%%%%%%%%%%%%%
%%%%%%%%%%%%%%%%%%%%%%%%%%%%%%%%%%


\newlength{\unit}
\setlength{\unit}{1ex}

\newcommand\drawCodeBox[3][1.5]{%
\begin{tikzpicture}[remember picture,overlay]
  \coordinate (start) at ([yshift = 2\unit]pic cs:#2);
  \coordinate (end) at ([yshift = -#1\unit]pic cs:#3);
  \node[inner sep=2pt,draw=black,fit=(start) (end)] {};
\end{tikzpicture}
}

\newcolumntype{L}{>{$}l<{$}} 


\newlength{\logicv}
\setlength{\logicv}{.1cm}
\newenvironment{logicenv}[2][0]{
  %begin
  \begin{tcolorbox}[demo, title = #2]
  \vspace*{#1\logicv}
}{
  %end
  \end{tcolorbox}
  \vspace*{-.5cm}
}

\newenvironment{subproofill}[1][0]{
  %begin
  \begin{tcolorbox}[demo, title = ]
    \vspace*{-#1\logicv}
}{
  %end
  \end{tcolorbox}
  \vspace*{-.5cm}
}


\newenvironment{subproof}[2][0]{
  %begin
  \begin{tcolorbox}[demo, title = #2, colframe = black]
  \vspace*{#1\logicv}
  \begin{logic}
}{
  %end
  \end{logic}
  \end{tcolorbox}
}
\newenvironment{logic}{
    \setlength{\extrarowheight}{3pt}
    \setcounter{row}{-1}
    \begin{center}
    \begin{NiceTabular}{>{\stepcounter{row}\therow.\hspace*{5pt}} L r }
}{
    \end{NiceTabular}
    \end{center}
}

\tcbset{demo/.style={
    enhanced, title=title,
    attach boxed title to top
    left = {xshift = .5pt, yshift = -\tcboxedtitleheight -.5pt},
    top = 1pt,
    boxrule = .65pt,
    coltitle = black,
    colback = defini,
    colframe = defini,
    arc = 0mm,
    outer arc = 0mm,
    boxed title style={
        colback = white,
        colframe = deftitlmarg,
        boxrule = .5pt,
        arc = 0mm,
        outer arc = 0mm
    }
}}

% Colores
\definecolor{rojoEci}{RGB}{225, 70, 49}
\definecolor{defini}{HTML}{ede9e6}
\definecolor{deftitlmarg}{HTML}{cfcfcf}

% Documento
\begin{document}



\newcounter{row}

\begin{titlepage}
    \begin{center}
        \vspace*{1cm}

        \textbf{\Huge{\titlename}}

        \vspace{1.5cm}

        \textbf{\large{David Gómez}}

        \vspace{4cm}

        \includegraphics[width=\textwidth]{\logo}

        \vspace{5cm}

        Matemáticas\linebreak
        Escuela Colombiana de Ingeniería Julio Garavito\linebreak
        Colombia\linebreak
        \today

    \end{center}
\end{titlepage}
\clearpage
\begin{center}
    \textbf{Sección 2.6}
\end{center}
\tableofcontents
\clearpage

\section{Punto 2}

\begin{logicenv}[5]{Punto 2}
    $A$ dice: \q{Soy escudero o $B$ es un caballero}\\
    $\Gamma_0 = \{(a \equiv ((\neg a) \lor b)), a\}$\\
    $\Gamma_1 = \{(a \equiv ((\neg a) \lor b)), (\neg a)\}$\\
    \makebox[7cm]{\hrulefill}\\
    $\therefore \qquad A$ es caballero y $B$ es escudero
\end{logicenv}
\begin{subproofill}
    \begin{subproof}{Con $\Gamma_0$}
        (\exists \mathbf{v} \, \vert\, \mathbf{v} \text{ satisface } \Gamma_0)\\
        \val{v}{(a \equiv ((\neg a) \lor b))} = \mathtt{T} & Def.(p0)\\
        \val{v}{a} = \mathtt{T} & Def.(p0)\\
        \val{v}{((\neg a) \lor b)} = \mathtt{T} & MT 2.23($\equiv$)(p2, p1)\\
        \val{v}{(\neg a)} = \mathtt{T} \text{ o } \val{v}{b} = \mathtt{T} & MT 2.23 ($\lor$)(p3)\\
        \val{v}{b} = \mathtt{T} & MT 2.23($\lor$)(p4, p2)
    \end{subproof}
\end{subproofill}
\begin{subproofill}
    \begin{subproof}{Con $\Gamma_1$}
        (\exists \mathbf{v} \, \vert\, \mathbf{v} \text{ satisface } \Gamma_0)\\
        \val{v}{(a \equiv ((\neg a) \lor b))} = \mathtt{T} & Def.(p0)\\
        \val{v}{(\neg a)} = \mathtt{T} & Def.(p0)\\
        \val{v}{((\neg a) \lor b)} = \mathtt{F} & MT 2.23($\equiv$)(p2, p1)\\
        \val{v}{(\neg a)} = \mathtt{F} \text{ y } \val{v}{b} = \mathtt{F} & MT 2.23($\lor$)(p3)\\
        \val{v}{(\neg a)} = \mathtt{F} \text{ y } \val{v}{(\neg a)} = \mathtt{T} & Contradicción (p4, p2)
    \end{subproof}
\end{subproofill}

\section{Punto 11}
\begin{logicenv}[5]{Punto 11}
  $B$ dijo: \q{$A$ dijo que es escudero}\\
  $C$ dijo: \q{No le crea a $B$ porque está mintiendo}\\
  $\Gamma_0 = \{(b \equiv (a \equiv (\neg a))), (c \equiv (\neg b)), b\}$\\
  $\Gamma_1 = \{(b \equiv (a \equiv (\neg a))), (c \equiv (\neg b)), (\neg b)\}$\\
  \makebox[7cm]{\hrulefill}\\
  $ \therefore \qquad B$ es escudero y $C$ es caballero
\end{logicenv}
\begin{subproofill}
  \begin{subproof}{Con $\Gamma_0$}
    (\exists \mathbf{v} \, \vert\, \mathbf{v} \text{ satisface } \Gamma_0)\\
    \val{v}{(b \equiv( a \equiv (\neg a)))} = \mathtt{T} & Def.(p0)\\
    \val{v}{b} = \mathtt{T} & Def.(p0)\\
    \val{v}{(a \equiv (\neg a))} = \mathtt{F} & MT 2.23 ($\equiv$)\\
    \val{v}{b} =\mathtt{F} & MT 2.23($\equiv$)(p1)\\
    \val{v}{b} = \mathtt{T} \text{ y } \val{v}{b} = \mathtt{F} & Contradicción (p4, p2)
  \end{subproof}
\end{subproofill}
\begin{subproofill}
  \begin{subproof}{Con $\Gamma_1$}
    (\exists \mathbf{v} \, \vert\, \mathbf{v} \text{ satisface } \Gamma_0)\\
    \val{v}{(b \equiv (a \equiv (\neg a)))} = \mathtt{T} & Def.(p0)\\
    \val{v}{b} = \mathtt{T} & Def.(p0)\\
    \val{v}{c} = \mathtt{T}
  \end{subproof}
\end{subproofill}

\section{Punto 12}
\begin{logicenv}[5]{Punto 12}
  $B$ dijo: \q{$A$ dijo que hay al menos un caballero entre nosotros}\\
  $C$ dijo: \q{$B$ miente}\\
  $\Gamma_0 = \{(b \equiv (a \equiv (a \lor b \lor c))), (c \equiv (\neg b)), b\}$\\
  $\Gamma_1 = \{(b \equiv (a \equiv (a \lor b \lor c))), (c \equiv (\neg b)), (\neg b)\}$\\
  \makebox[8cm]{\hrulefill}\\
  $\therefore \qquad$No es posible determinar su naturaleza
\end{logicenv}
\begin{subproofill}
  \begin{subproof}{Con $\Gamma_0$}
    (\exists \mathbf{v} \, \vert \, \mathbf{v} \text{ satisface } \Gamma_0)\\
    \val{v}{(b \equiv (a \equiv (a \lor b \lor c)))} = \mathtt{T} & Def.(p0)\\
    \val{v}{(c \equiv (\neg b))} = \mathtt{T} & Def.(p0)\\
    \val{v}{b} = \mathtt{T} & Def.(p0)\\
    \val{v}{(\neg b)} = \mathtt{F} & MT 2.23($\neg$)(p3)\\
    \val{v}{c} = \mathtt{F} & MT 2.23($\equiv$)(p4,p2)
  \end{subproof}
\end{subproofill}
\begin{subproofill}
  \begin{subproof}{Con $\Gamma_1$}
    (\exists \mathbf{v} \, \vert \, \mathbf{v} \text{ satisface } \Gamma_0)\\
    \val{v}{(b \equiv (a \equiv (a \lor b \lor c)))} = \mathtt{T} & Def.(p0)\\
    \val{v}{(c \equiv (\neg b))} = \mathtt{T} & Def.(p0)\\
    \val{v}{(\neg b)} = \mathtt{T} & Def.(p0)\\
    \val{v}{c} = \mathtt{T} & MT 2.23($\equiv$)(p3, p2)
  \end{subproof}
\end{subproofill}

\section{Punto 13}
\begin{logicenv}[5]{Punto 13}
  $A$ dice: \q{Todos nosotros somos escuderos}\\
  $B$ dice: \q{Exactamente uno de nosotros es caballero}
  \begin{alignat*}{2}
    \Gamma_0 &= \{(a \equiv ((\neg a) \land (\neg b) \land (\neg c))),\\
    & \quad (b \equiv (a \land (\neg b) \land (\neg c)) \lor ((\neg a) \land b \land (\neg c)) \lor ((\neg a) \land (\neg b) \land c)), ((\neg a) \land (\neg b))\}\\
    \Gamma_1 &= \{(a \equiv ((\neg a) \land (\neg b) \land (\neg c))),\\
    & \quad (b \equiv (a \land (\neg b) \land (\neg c)) \lor ((\neg a) \land b \land (\neg c)) \lor ((\neg a) \land (\neg b) \land c)), b\}\\
    \Gamma_2 &= \{(a \equiv ((\neg a) \land (\neg b) \land (\neg c))),\\
    & \quad (b \equiv (a \land (\neg b) \land (\neg c)) \lor ((\neg a) \land b \land (\neg c)) \lor ((\neg a) \land (\neg b) \land c)), (\neg b)\}
  \end{alignat*}
  \makebox[11.5cm]{\hrulefill}\\
  $\therefore \qquad A$ es escudero, no se puede determinar la naturaleza de los demás
\end{logicenv}
\begin{subproofill}
  \begin{subproof}[5]{Con $\Gamma_0$}
    (\exists \mathbf{v}\, \vert\, \mathbf{v} \text{ satisface } \Gamma_0)\\
    \val{v}{(a \equiv ((\neg a) \land (\neg b) \land (\neg c)))} = \mathtt{T} & Def.(p0)\\
    \val{v}{(b \equiv (a \land (\neg b) \land (\neg c)) \lor ((\neg a) \land b \land (\neg c)) \lor ((\neg a) \land (\neg b) \land c))} & Def.(p0)\\
    \val{v}{((\neg a) \land (\neg b))}\\
    \val{v}{(\neg c)} = \mathtt{F} & MT 2.23($\equiv$, $\land$)(p3, p1)\\
  \end{subproof}
\end{subproofill}
\begin{subproofill}
  \begin{subproof}[5]{Con $\Gamma_1$}
    (\exists \mathbf{v}\, \vert\, \mathbf{v} \text{ satisface } \Gamma_1)\\
    \val{v}{(a \equiv ((\neg a) \land (\neg b) \land (\neg c)))} = \mathtt{T} & Def.(p0)\\
    \val{v}{(b \equiv (a \land (\neg b) \land (\neg c)) \lor ((\neg a) \land b \land (\neg c)) \lor ((\neg a) \land (\neg b) \land c))} & Def.(p0)\\
    \val{v}{b} = \mathtt{T}\\
    \val{v}{((\neg a) \land b \land (\neg c))} = \mathtt{F} & MT 2.23($\equiv$, $\land$) (p2)\\
    \val{v}{a} = \mathtt{F} & MT 2.23($\land$, $\neg$)\\
    \val{v}{c} = \mathtt{F} & MT 2.23($\land$, $\neg$)
  \end{subproof}
\end{subproofill}
\begin{subproofill}
  \begin{subproof}[5]{Con $\Gamma_2$}
    (\exists \mathbf{v}\, \vert\, \mathbf{v} \text{ satisface } \Gamma_2)\\
    \val{v}{(a \equiv ((\neg a) \land (\neg b) \land (\neg c)))} = \mathtt{T} & Def.(p0)\\
    \val{v}{(b \equiv (a \land (\neg b) \land (\neg c)) \lor ((\neg a) \land b \land (\neg c)) \lor ((\neg a) \land (\neg b) \land c))} & Def.(p0)\\
    \val{v}{(\neg b)} = \mathtt{T} & Def.(p0)\\
    \val{v}{(a \land (\neg b) \land (\neg c))} = \mathtt{F} & MT 2.23($\neg$, $\lor$)(p3, p2)\\
    \val{v}{(\neg b)} = \mathtt{F} & MT 2.23($\land$)(4)\\
    \val{v}{(\neg b)} = \mathtt{F} \text{ y } \val{v}{(\neg b)} = \mathtt{T} & Contradicción (p5, p3)
  \end{subproof}
\end{subproofill}

\section{Punto 14}
\begin{logicenv}[5]{Punto 14}
  $A$ dice: \q{Todos somos escuderos}\\
  $B$ dice: \q{Exactamente uno de nosotros es escudero}\\
  $\Gamma_0 = \{(a \equiv ((\neg a) \land (\neg b) \land (\neg c))),$\\
  \hspace*{1cm}$(b \equiv (((\neg a) \land b \land c) \lor (a \land (\neg b) \land c) \lor (a \land b \land (\neg c)))), (\neg a)\}$\\
  $\Gamma_1 = \{(a \equiv ((\neg a) \land (\neg b) \land (\neg c))),$\\
  \hspace*{1cm}$(b \equiv (((\neg a) \land b \land c) \lor (a \land (\neg b) \land c) \lor (a \land b \land (\neg c)))), a\}$\\
  \makebox[10cm]{\hrulefill}\\
  El ejercicio está mal planteado
\end{logicenv}
\begin{subproofill}
  \begin{subproof}[5]{Con $\Gamma_0$}
    (\exists \mathbf{v}\, \vert\, \mathbf{v} \text{ satisface } \Gamma_0)\\
    \val{v}{(a \equiv ((\neg a) \land (\neg b) \land (\neg c)))} = \mathtt{T} & Def.(p0)\\
    \val{v}{(b \equiv (((\neg a) \land b \land c) \lor (a \land (\neg b) \land c) \lor (a \land b \land (\neg c))))} = \mathtt{T} & Def.(p0)\\
    \val{v}{(\neg a)} = \mathtt{T} & Def.(p0)\\
    \val{v}{a} = \mathtt{F} & MTT 2.23($\neg$)(p3)\\
    \val{v}{(\neg a)} = \mathtt{F} \text{ y } \val{v}{(\neg b)}  = \mathtt{F} \text{ y } \val{v}{(\neg c)} = \mathtt{F} & MT 2.23 ($\equiv$, $\land$)(p4, p1)\\
    \val{v}{(\neg a)} = \mathtt{F} \text{ y } \val{v}{(\neg a)} = \mathtt{T} & Contradicción (p5, p3)
  \end{subproof}
\end{subproofill}
\begin{subproofill}
  \begin{subproof}{Con $\Gamma_1$}
    (\exists \mathbf{v}\, \vert\, \mathbf{v} \text{ satisface } \Gamma_0)\\
    \val{v}{(a \equiv ((\neg a) \land (\neg b) \land (\neg c)))} = \mathtt{T} & Def.(p0)\\
    \val{v}{(b \equiv (((\neg a) \land b \land c) \lor (a \land (\neg b) \land c) \lor (a \land b \land (\neg c))))} = \mathtt{T} & Def.(p0)\\
    \val{v}{a} = \mathtt{T} & Def.(p0)\\
    \val{v}{(\neg a)} = \mathtt{T} \text{ y } \val{v}{(\neg b)} = \mathtt{T} \text{ y } \val{v}{(\neg c)} = \mathtt{T} & MT 2.23($\equiv$, $\land$)\\
    \val{v}{a} = \mathtt{F} & MT 2.23($\neg$)(p4)\\
    \val{v}{a} = \mathtt{T} \text{ y } \val{v}{a} = \mathtt{F} & (p5, p3)
  \end{subproof}
\end{subproofill}

\section{Punto 18}

El habitante $A$ dice \q{Yo dije que si soy caballero entonces soy escudero, y si soy escudero entonces soy caballero}

\section{Punto 19}

El habitante $A$ dice \q{Si $B$ es caballero, entonces soy caballero}\\
El habitante $B$ dice \q{Si soy caballero, entonces $A$ es caballero}

\section{Punto 23}

\begin{logicenv}[5]{Punto 23}
  $C$ dijo: \q{A lo sumo uno de nosotros es caballero}\\
  $\Gamma_0 = \{(c \equiv (a \land (\neg b) \land (\neg c)) \lor ((\neg a) \land b \land (\neg c)) \lor ((\neg a) \land (\neg b) \land c) \lor ((\neg a) \land (\neg b) \land (\neg c))), c\}$\\
  $\Gamma_1 = \{(c \equiv (a \land (\neg b) \land (\neg c)) \lor ((\neg a) \land b \land (\neg c)) \lor ((\neg a) \land (\neg b) \land c) \lor ((\neg a) \land (\neg b) \land (\neg c))), (\neg c)\}$
  \makebox[16.5cm]{\hrulefill}\\
  $C$ es el único caballero, y por ende el único hombre lobo (tomando la suposición dada por el enunciado).
\end{logicenv}
\begin{subproofill}
  \dots Tomando lo que dijo $C$ como $\phi$\dots
  \begin{subproof}[5]{Con $\Gamma_0$}
    (\exists \mathbf{v} \, \vert\, \mathbf{v} \text{ satisface } \Gamma_0)\\
    \val{v}{(c \equiv \phi} = \mathtt{T} & Def.(p0)\\
    \val{v}{c} = \mathtt{T} & Def(p0)\\
    \val{v}{((\neg a) \land (\neg b) \land c)} = \mathtt{T} & MT 2.23($\equiv$, $\land$)(p2, p1)\\
  \end{subproof}
\end{subproofill}
\begin{subproofill}
  \begin{subproof}[5]{Con $\Gamma_1$}
    (\exists \mathbf{v} \, \vert\, \mathbf{v} \text{ satisface } \Gamma_0)\\
    \val{v}{(c \equiv \phi)} = \mathtt{T} & Def.(p0)\\
    \val{v}{(\neg c)} = \mathtt{T} & Def(p0)\\
    \val{v}{(a \land (\neg b) \land (\neg c))} = \mathtt{F} & MT 2.23($\equiv$, $\land$) (p2, p1)\\
    \val{v}{(\neg c)} = \mathtt{F} & MTT 2.23 ($\land$)(p3)\\ 
    \val{v}{(\neg c)} = \mathtt{F} \text{ y } \val{v}{(\neg c)} = \mathtt{T} & Contradicción (p4, p2)
  \end{subproof} 
\end{subproofill}
\end{document}