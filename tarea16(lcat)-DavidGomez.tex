\documentclass[twoside]{article}

% Packages
\usepackage{mathtools}
\usepackage{amssymb}
\usepackage{logicDG}
\usepackage{array}
\usepackage{xcolor}
\usepackage[spanish]{babel}
\usepackage{geometry}
\usepackage{fancyhdr}
\usepackage{graphicx}
\usepackage[hidelinks]{hyperref}

% Fuente
\usepackage{lmodern}
\usepackage[T1]{fontenc}

% Configuraciones

\geometry{
    a4paper,
    margin = 2.5cm,
    top = 4cm,
    bottom = 2.5cm,
    headheight = 72pt
}
\newcommand{\logo}{C:/Users/usuario/Documents/U/logo-eci.jpg}
\renewcommand{\author}{David Gómez}
\renewcommand{\title}{Tarea 15}

\pagestyle{fancy}
\fancyhf{}
\fancyhead[LO]{\author}
\fancyhead[LE]{\title}
\fancyhead[R]{\includegraphics[width = 4cm]{\logo}}
\fancyfoot[C]{Página \thepage}
\renewcommand{\headrule}{\hbox to \headwidth{\color{rojoEci}\leaders\hrule height \headrulewidth\hfill}}

\hyphenpenalty = 10000

\setlength{\parindent}{0pt}

\newcommand{\sect}[1]{\section*{#1} \addcontentsline{toc}{section}{#1}}
\newcommand{\subsect}[1]{\subsection*{#1} \addcontentsline{toc}{subsection}{#1}}
\newcommand{\subsubsect}[1]{\subsubsection*{#1} \addcontentsline{toc}{subsubsection}{#1}}

\newcommand{\divs}[2]{#1 \cdot\hspace*{-.1cm}\mid #2}

% Colores
\definecolor{rojoEci}{RGB}{225, 70, 49}

% Documento

\begin{document}
\begin{titlepage}
    \begin{center}
        \vspace*{1cm}
 
        \textbf{\fontsize{45}{\baselineskip}\selectfont{\title}}

        \vspace{4cm}

        {\Large Hecho por}

        \vspace{1cm}

        {\textbf{\LARGE\MakeUppercase{\author}}}

        \vspace{2cm}

        \includegraphics[width = .8\textwidth]{\logo}

        \vspace{2cm}

        {\Large Estudiante de Matemáticas\\[5pt]

        Escuela Colombiana de Ingeniería Julio Garavito\\[5pt]

        Colombia\\[5pt]

        \today}
             
    \end{center}
\end{titlepage}

{\centering Sección 7.5}
\tableofcontents
\clearpage

% Sección 7.5: 1,2,3,4,5,7,8,9,10
\sect{Punto 1}
\begin{logicenv}{Reflexividad de ``$=$''}
    \[(\forall x \in \mathbb{R} \,\vert:\, x = x)\]
    Basta demostrar $x = x$ en $\mathbb{R}$ (MT 7.20)
    \begin{logic}
        x = x & Bx6
    \end{logic}
\end{logicenv}

\sect{Punto 2}
\begin{logicenv}{Simetría de ``$=$''}
    \[(\forall x, y \in \mathbb{R} \,\vert\, x = y : y = x)\]
    Basta demostrar $x = y \to y = x$ en $\mathbb{R}$
    Suponga $\Gamma = \{x = y, \neg y = x\}$
    \begin{logic}
        x = y & $\Gamma$\\
        \neg y = x & $\Gamma$\\
        \neg y = x \to (\neg y = x)[x := y] & Bx4(p0)\\
        (\neg y = x)[x := y] & MPP(p2, p1)\\
        \neg y = y & (p3)\\
        \textit{false} & $y = y \equiv \textit{true}$
    \end{logic}
\end{logicenv}

\sect{Punto 3}
\begin{logicenv}{Transitividad de ``$=$''}
    \[(\forall x, y, z \in \mathbb{R} \,\vert\, x = y \land y = z : )\]
    Basta con demostrar $x = y \land y = x \to x = z$ en $\mathbb{R}$
    
\end{logicenv}

\sect{Punto 4}
\begin{logicenv}{``Leibniz'' con igualdad}
    \begin{logic}
        t = u & Suposición\\
        u = t & Reflexividad($=$)(p0)\\
        t = u \to (\phi \equiv \phi[t := u]) & Bx7(p0)\\
        \phi \equiv \phi[t := u] & MPP(p2, p0)\\
        u = t \to (\phi \equiv \phi[u := t]) & Bx7(p1)\\
        \phi \equiv \phi[u := t] & MPP(p4, p1)\\
        \phi[t := u] \equiv \phi[x := t][t := u] \equiv \phi[x := u] & $t$, $u$ libres para $x$ en $\phi$\\
        \phi[u := t]  \equiv \phi[x := u][u := t] \equiv \phi[x := t] & $t$, $u$ libres para $x$ en $\phi$\\
        \phi[u := t] \equiv \phi[t := u] & Transitividad(p3, p5)\\
        \phi[x := t] \equiv \phi[t := u] & (p8, p6, p5)
    \end{logic}
\end{logicenv}

\sect{Punto 5}
\begin{logicenv}{Teo 7.28.2}
    \begin{derivation}
            (\exists x \,\vert\, x = t : \phi)\\
        Azúcar sintáctico\\
            \exists x (x = t \land \phi)\\
        Def.($\forall x$), Dist.($\neg$, $\land$)\\
            \neg \forall x (\neg x = t \lor \neg \phi)\\
        (alt.)Def.($\to$)\\
            \neg \forall x (x = t \to \neg \phi)\\
        Regla de un punto ($\forall x$)\\
            \neg \neg \phi[x := t]\\
        Doble negación\\
            \phi[x := t]
    \end{derivation}
\end{logicenv}

\sect{Punto 7}
\begin{logicenv}{$a$ divide a sus múltiplos}
    \[(\forall a \in \mathbb{N} \,\vert:\, a \cdot\mid ab)\]
    Basta demostrar $a \cdot\mid ab$ en $\mathbb{N}$
    \begin{align*}
            \res{a \cdot\mid ab}\\
        \why{Def.($\cdot\mid$)}\\
            \res{\exists x (ax = ab)}\\
        \why[\Leftarrow]{Instanciación con testigo $x = b$}\\
            \res{ab  = ab}\\
        \why{Reflexividad($=$)}\\
            \res{\textit{true}}
    \end{align*}
\end{logicenv}

\sect{Punto 8}
\begin{logicenv}{si $a$ y $b$ se dividen entre si, $a = b$ o $a = -b$}
    \[(\forall a, b \in \mathbb{N} \,\vert\, \divs{a}{b} \land \divs{b}{a} : a = b \lor a = -b)\]
    Basta demostrar $\divs{a}{b} \land \divs{b}{a} \to a = b \lor a = -b$ en $\mathbb{N}$

    \begin{derivation}
            \divs{a}{b} \land \divs{b}{a} \to a = b \lor a = -b\\
        Def.($\divs{}{}$)\\
            \exists x (ax = b) \land \exists y (by = a) \to a = b \lor a = -b\\
        Dist.($\exists x$, $\land$)\\
            \exists x (ax = b \land \exists y (by = a)) \to a = b \lor a = -b\\
        Teo 7.20.2\\
            \forall x (ax = b \land \exists y (by = a) \to a = b \lor a = -b)
    \end{derivation}
    \[\theo{\{ax = b \land \exists y (by = a)\}}{\dsl}{a = b \lor a = -b}\]
    \begin{derivation}
            ax = b \land \exists y (by = a) \to a = b \lor a = -b\\
        Dist.($\exists x$, $\land$)\\
            \exists y(ax = b \land by = a) \to a = b \lor a = -b\\
        Teo 7.20.2\\
            \forall y (ax = b \land by = a \to a = b \lor a = -b)\\
    \end{derivation}
    \[\theo{\{ax = b \land by = a\}}{\dsl}{a = b \lor a = -b}\]
    \begin{logic}
        ax = b & Suposición\\
        by = a & Suposición\\
        a = \frac{b}{x} & Álgebra(p0)\\
        by = \frac{b}{x} & Transitividad($=$)(p2, p1)\\
        y = \frac{1}{x} & Álgebra(p3)\\
        x = y = -1 \lor x = y = 1 & Álgebra(p4), ($x, y \in \mathbb{N}$)\\
        -a = b \lor a = b & Transitividad($=$)(p5, p0)
    \end{logic}
    
\end{logicenv}

\sect{Punto 9}
\sect{Punto 10}
\end{document}