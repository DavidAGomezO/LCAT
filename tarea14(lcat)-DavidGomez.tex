\documentclass[twoside]{article}

% Packages
\usepackage{mathtools}
\usepackage{amssymb}
\usepackage{logicDG}
\usepackage{array}
\usepackage{xcolor}
\usepackage[spanish]{babel}
\usepackage{geometry}
\usepackage{fancyhdr}
\usepackage{graphicx}
\usepackage[hidelinks]{hyperref}

% Font
\usepackage{lmodern}
\usepackage[T1]{fontenc}

% Page configurations

\geometry{
    a4paper,
    margin = 2.5cm,
    top = 4cm,
    bottom = 2.5cm,
    headheight = 72pt
}
\newcommand{\logo}{C:/Users/usuario/Documents/U/logo-eci.jpg}
\renewcommand{\author}{David Gómez}
\renewcommand{\title}{Tarea 14}

\pagestyle{fancy}
\fancyhf{}
\fancyhead[LO]{\author}
\fancyhead[LE]{\title}
\fancyhead[R]{\includegraphics[width = 4cm]{\logo}}
\fancyfoot[C]{Página \thepage}
\renewcommand{\headrule}{\hbox to \headwidth{\color{rojoEci}\leaders\hrule height \headrulewidth\hfill}}

\hyphenpenalty = 10000

\setlength{\parindent}{0pt}

\newcommand{\sect}[1]{\section*{#1} \addcontentsline{toc}{section}{#1}}
\newcommand{\subsect}[1]{\subsection*{#1} \addcontentsline{toc}{subsection}{#1}}
\newcommand{\subsubsect}[1]{\subsubsection*{#1} \addcontentsline{toc}{subsubsection}{#1}}

% Colors
\definecolor{rojoEci}{RGB}{225, 70, 49}

% Documento

\begin{document}
\begin{titlepage}
    \begin{center}
        \vspace*{1cm}
 
        \textbf{\fontsize{45}{\baselineskip}\selectfont{\title}}

        \vspace{4cm}

        {\Large Hecho por}

        \vspace{1cm}

        {\textbf{\LARGE\MakeUppercase{\author}}}

        \vspace{2cm}

        \includegraphics[width = .8\textwidth]{\logo}

        \vspace{2cm}

        {\Large Estudiante de Matemáticas\\[5pt]

        Escuela Colombiana de Ingeniería Julio Garavito\\[5pt]

        Colombia\\[5pt]

        \today}
             
    \end{center}
\end{titlepage}

\begin{center}
    \textbf{Sección 7.3}
    (Lemas al final)
\end{center}
\tableofcontents
\clearpage

\sect{Punto 1}
\subsect{a}
\begin{logicenv}{Si hay alguien que pague impuestos\dots}
    \begin{gather*}
        I(x) := \text{``$x$ paga impuestos''}\\
        P(x) := \text{``$x$ es político''}\\
        F(x) := \text{``$x$ es filántropo''}
    \end{gather*}
    \begin{logic}
        (\exists x \,\vert :\, I(x) \to (\forall y \,\vert\, P(y) : I(y))) & Suposición\\
        (\exists x \,\vert :\, F(x) \to (\forall y \,\vert\, I(y) : F(y))) & Suposición\\
        \exists x I(x) \to \forall y (P(y) \to I(y)) & Azúcar sintáctico(p0)\\
        \exists x F(x) \to \forall y (I(y) \to F(y)) & Azúcar sintáctico(p1)\\
        \exists x I(x) \land \exists x F(x) \to \forall y (P(y) \to I(y)) & Lema 7.3.1(p3, p2)\\
        \exists x (\exists x I(x) \land F(x)) \to \forall y (P(y) \to I(y)) & Dist.($\exists x$, $\land$)
    \end{logic}
\end{logicenv}

\subsect{b}
\begin{logicenv}{Si hay alguien que pague impuestos\dots}
    \begin{logic}
        (\exists x \,\vert:\, I(x) \to (\forall y \,\vert\, P(y) : I(y))) & Suposición\\
        (\exists x \,\vert :\, F(x) \to (\forall y \,\vert\, I(y) : F(y))) & Suposición\\
        \exists x I(x) \to \forall y (P(y) \to I(y)) & Azúcar sintáctico(p0)\\
        \exists x F(x) \to \forall y (I(y) \to F(y)) & Azúcar sintáctico(p1)\\
        \exists x I(x) \land \exists x F(x) \to \forall y (P(y) \to I(y)) \land \forall y (I(y) \to F(y)) & Lema 7.3.2(p3, p2)\\
        \exists x (\exists x I(x) \land F(x)) \to \forall x ((P(y) \to I(y)) \land(I(y) \to F(y))) & Dist($\exists x$, $\land$), Dist($\forall x$, $\land$) (p4)\\
        \exists x (\exists x I(x) \land F(x)) \to \forall x (P(y) \to F(y)) & Transitividad($\to$)
    \end{logic}
\end{logicenv}

\sect{Punto 2}
\begin{logicenv}{$\theo{}{\dsl}{\exists x \textit{true} \equiv \textit{true}}$}
    \begin{derivation}
            \exists x \textit{true}\\
        Def.($\exists x \phi$)\\
            \neg \forall x \neg \textit{true}\\
        $\neg \textit{true} \equiv \textit{false}$\\
            \neg \forall x \textit{false}\\
        $\forall x \textit{false} \equiv \textit{false}$\\
            \neg \textit{false}\\
        $\neg \textit{false} \equiv \textit{true}$\\
            \textit{true}
    \end{derivation}
    Por metateorema de derivación se demuestra que $\theo{}{\dsl}{\exists x \textit{true} \equiv \textit{true}}$
\end{logicenv}

\sect{Punto 3}
\begin{logicenv}{$\theo{}{\dsl}{\exists x \textit{false} \equiv \textit{false}}$}
    \begin{derivation}
        \exists x \textit{false}\\
    Def.($\exists x \phi$)\\
        \neg \forall x \neg \textit{false}\\
    $\neg \textit{false} \equiv \textit{true}$\\
        \neg \forall x \textit{true}\\
    $\forall x \textit{true} \equiv \textit{true}$\\
        \neg \textit{true}\\
    $\neg \textit{true} \equiv \textit{false}$\\
        \textit{false}
\end{derivation}
Por metateorema de derivación se demuestra que $\theo{}{\dsl}{\exists x \textit{false} \equiv \textit{false}}$
\end{logicenv}

\sect{Punto 4}
\begin{logicenv}{$\exists x \exists x \phi \equiv \exists x \phi$}
    \begin{derivation}
            \exists x \exists x \phi\\
        Def.($\exists x \phi$)\\
            \neg \forall x \exists x \phi\\
        Generalización, $x$ no libre en $\exists x \phi$\\
            \neg \neg \exists x \phi\\
        Doble negación\\
            \exists x \phi
    \end{derivation}
    Por metateorema de derivación se demuestra que $\exists x \exists x \phi \equiv \exists x \phi$
\end{logicenv}

\sect{Punto 5}
\begin{logicenv}{$\theo{}{\dsl}{\phi[x:= t] \equiv \exists x \phi}$, $t$ libre para $x$ en $\phi$}
    \begin{align*}
            \res{\phi[x := t]}\\
        \why{Generalización, x no aparece libre en $\phi[x := t]$}\\
            \res{\forall x \phi[x := t]}\\
        \why[\Rightarrow]{Lema 7.3.3}\\
            \res{\exists x \phi[x := t]}
    \end{align*}
\end{logicenv}

\sect{Punto 6}
\begin{logicenv}{$\theo{}{\dsl}{\exists x \phi \lor \exists x \psi \equiv \exists x (\phi \lor \psi)}$}
    \begin{derivation}
            \exists x \phi \lor \exists x \psi\\
        Def.($\exists x \phi$)\\
            \neg \forall x \neg \phi \lor \neg \forall x \neg \psi\\
        Dist.($\neg$, $\land$)\\
            \neg (\forall x \neg \phi \land \forall x \neg \psi)\\
        Dist.($\forall x$, $\land$)\\
            \neg \forall x (\neg \phi \land \neg \psi)\\
        Def.($\exists x \phi$)\\
            \exists x \neg (\neg \phi \land \neg \psi)\\
        Dist.($\neg$, $\land$)\\
            \exists x (\phi \lor \psi)
    \end{derivation}
\end{logicenv}

\sect{Punto 7}
\begin{logicenv}{$\theo{}{\dsl}{\psi \lor \exists x \phi \equiv \exists x (\psi \lor \phi)}$, $x$ no libre en $\psi$}
    \begin{derivation}
            \exists x (\psi \lor \phi)\\
        Def.($\exists x \phi$)\\
            \neg \forall x \neg(\psi \lor \phi)\\
        Dist.($\neg$, $\lor$)\\
            \neg \forall x (\neg \psi \land \neg \phi)\\
        Dist.($\forall x$, $\land$)\\
            \neg (\forall x \neg \psi \land \forall x \neg \phi)\\
        Generalización, $x$ no libre en $\psi$\\
            \neg (\neg \psi \land \forall x \neg \phi)\\
        Dist.($\neg$, $\land$)\\
            \psi \lor \neg \forall x \neg \phi\\
        Def.($\exists x \phi$)\\
            \psi \lor \exists x \phi
    \end{derivation}
    Por Conmutativa($\equiv$) y metateorema de derivación se demuestra que $\theo{}{\dsl}{\psi \lor \exists x \phi \equiv \exists x (\psi \lor \phi)}$, $x$ no libre en $\psi$
\end{logicenv}

\sect{Punto 8}
\begin{logicenv}{$\theo{}{\dsl}{\psi \land \exists x \phi \equiv \exists x (\psi \land \phi)}$, $x$ no libre en $\psi$}
    \begin{derivation}
            \exists x (\psi \land \phi)\\
        Def.($\exists x \phi$)\\
            \neg \forall x \neg (\psi \land \phi)\\
        Dist.($\neg$, $\land$)\\
            \neg \forall x (\neg \psi \lor \neg \phi)\\
        Dist.($\forall x$, $\lor$), $x$ no libre en $\psi$\\
            \neg (\neg \psi \lor \forall x \neg \phi)\\
        Dist.($\neg$, $\lor$)\\
            \psi \land \neg \forall x \neg \phi\\
        Def.($\exists x \phi$)\\
            \psi \land \exists x \phi
    \end{derivation}
    Por Conmutativa($\equiv$) y metateorema de derivación se demuestra que $\theo{}{\dsl}{\psi \land \exists x \phi \equiv \exists x (\psi \land \phi)}$, $x$ no libre en $\psi$
\end{logicenv}

\sect{Punto 9}
\begin{logicenv}{$\theo{}{\dsl}{\psi \to \exists x \phi \equiv \exists x (\psi \to \phi)}$, $x$ no libre en $\psi$}
    \begin{derivation}
            \exists x (\psi \to \phi)\\
        (alt.)Def.($\to$)\\
            \exists x (\neg \psi \lor \phi)\\
        Dist.($\lor$, $\exists x$), $x$ no libre en $\psi$\\
            \neg \psi \lor \exists x \phi\\
        (alt.)Def.($\to$)\\
            \psi \to \exists x \phi
    \end{derivation}
    Por Conmutativa($\equiv$) y metateorema de derivación se demuestra que
    $\theo{}{\dsl}{\psi \to \exists x \phi \equiv \exists x (\psi \to \phi)}$, $x$ no libre en $\psi$
\end{logicenv}

\sect{Punto 10}
\begin{logicenv}{$\theo{}{\dsl}{(\exists x \,\vert\, \textit{false} : \phi) \equiv \textit{false}}$}
    \begin{derivation}
            (\exists x \,\vert\, \textit{false} : \phi)\\
        Azúcar sintáctico\\
            \exists x (\textit{false} \land \textit{false})\\
        $\textit{false} \land \phi \equiv \textit{false}$\\
            \exists x \textit{false}\\
        $\exists x \textit{false} \equiv \textit{false}$\\
            \textit{false}
    \end{derivation}
    Por metateorema de derivación se demuestra que $\theo{}{\dsl}{(\exists x \,\vert\, \textit{false} : \phi) \equiv \textit{false}}$
\end{logicenv}

\sect{Punto 12}
\begin{logicenv}{$\theo{}{\dsl}{(\exists x \,\vert\, \psi \lor \tau : \phi) \equiv (\exists x \,\vert\, \psi : \phi) \lor (\exists x \,\vert\, \tau : \phi)}$}
    \begin{derivation}
            (\exists x \,\vert\, \psi \lor \tau : \phi)\\
        Azúcar sintáctico\\
            \exists x ((\psi \lor \tau) \land \phi)\\
        Dist.($\land$, $\lor$)\\
            \exists x ((\psi \land \phi) \lor (\tau \land \phi))\\
        Dist.($\exists x$, $\lor$)\\
            \exists x (\psi \land \phi) \lor \exists x (\tau \land \phi)\\
        Azúcar sintáctico\\
            (\exists x \,\vert\, \psi : \phi) \lor (\exists x \,\vert\, \tau : \phi)
    \end{derivation}
    Por metateorema de derivación se demuestra que $\theo{}{\dsl}{(\exists x \,\vert\, \psi \lor \tau : \phi) \equiv (\exists x \,\vert\, \psi : \phi) \lor (\exists x \,\vert\, \tau : \phi)}$
\end{logicenv}


\sect{Punto 13}
\begin{logicenv}{$\theo{}{\dsl}{\exists x \phi \equiv \exists y (\phi[x := y])}$, $y$ no libre en $\phi$}
    \begin{derivation}
            \exists x \phi\\
        Def.($\exists x \phi$)\\
            \neg \forall x \neg \phi\\
        $\forall x \phi \equiv \forall y (\phi[x := y])$, $y$ no libre en $\phi$\\
            \neg \forall y \neg (\phi[x := y])\\
        Def.($\exists x \phi$)\\
            \exists y (\phi[x := y])
    \end{derivation}
    Por metateorema de derivación se demuestra que $\theo{}{\dsl}{\exists x \phi \equiv \exists y (\phi[x := y])}$, $y$ no libre en $\phi$
\end{logicenv}

\sect{Punto 16}
\begin{logicenv}{$\theo{}{\dsl}{\exists x \psi \to \phi \equiv \forall x (\psi \to \phi)}$, $x$ no libre en $\phi$}
    \begin{derivation}
            \exists x \psi \to \phi\\
        (alt.)Def.($\to$)\\
            \neg \exists x \psi \lor \phi\\
        $\neg \exists x \phi \equiv \forall x \neg \phi$\\
            \forall x \neg \psi \lor \phi\\
        Dist.($\forall x$, $\lor$), $x$ no libre en $\phi$\\
            \forall x (\neg \psi \lor \phi)\\
        (alt.)Def.($\to$)\\
            \forall x (\psi \to \phi)
    \end{derivation}
    Por metateorema de derivación se demuestra que $\theo{}{\dsl}{\exists x \psi \to \phi \equiv \forall x (\psi \to \phi)}$, $x$ no libre en $\phi$
\end{logicenv}


\sect{Punto 17}
\begin{logicenv}{$\theo{}{\dsl}{\exists x (\phi \to \psi) \equiv \forall x \phi \to \exists x \psi}$, $x$ no libre en $\phi$}
    \begin{derivation}
            \exists x (\phi \to \psi)\\
        Def.($\exists x \phi$)\\
            \neg \forall x \neg (\phi \to \psi)\\
        Dist.($\neg$, $\to$)\\
            \neg \forall x (\phi \land \neg \psi)\\
        Dist.($\forall$, $\land$)\\
            \neg (\forall x \phi \land \forall x \neg \psi)\\
        Dist.($\neg$, $\land$)\\
            \neg \forall x \phi \lor \neg \forall x \neg \psi\\
        Def.($\exists x \phi$)\\
            \neg \forall x \phi \lor \exists x \psi\\
        (alt.)Def.($\to$)\\
            \forall x \phi \to \exists x \psi            
    \end{derivation}
    Por metateorema de derivación se demuestra que $\theo{}{\dsl}{\exists x (\phi \to \psi) \equiv \forall x \phi \to \exists x \psi}$, $x$ no libre en $\phi$
\end{logicenv}

\sect{Lemas}
\subsect{Lema 7.3.1}
\begin{logicenv}{$(\phi \to \psi) \land (\tau \to \xi) \to (\phi \land \tau \to \psi)$}
    \[\Gamma = \{\phi \to \psi, \tau \to \xi\}\]
    \begin{logic}
        \phi \to \psi & $\Gamma$\\ %0
        \tau \to \xi & $\Gamma$\\ %1
        \neg (\phi \land \tau \to \psi) & Intento por reducción al absurdo\\ %2
        \phi \land \tau \land \psi & Dist.($\neg$, $\to$)(p2)\\ %3
        \phi & Debilitamiento(p3)\\ %4
        \neg \psi & Debilitamiento(p3)\\ %5
        \psi & MPP(p4, p0)\\ %6
        \neg \psi \land \psi & Unión(p6, p5)\\ %7
        \textit{false} & (p7)
    \end{logic}
    Dado que $\Gamma = \{\phi \to \psi, \tau \to \xi\}\cup \{\neg (\phi \land \tau \to \psi)\} \vDash \textit{false}$ se demuestra que 
    \[(\phi \to \psi) \land (\tau \to \xi) \to (\phi \land \tau \to \psi)\]
\end{logicenv}

\subsect{Lema 7.3.2}
\begin{logicenv}{$(\phi \to \psi) \land (\tau \to \xi) \to (\phi \land \tau \to \psi \land \xi)$}
    \[\Gamma = \{\phi \to \psi, \tau \to \xi\}\]
    \begin{logic}
        \phi \to \psi & $\Gamma$\\ %0
        \tau \to \xi & $\Gamma$\\ %1
        \neg (\phi \land \tau \to \psi \land \xi) & Intento por reducción al absurdo\\ %2
        \phi \land \tau \land \neg (\psi \land \xi) & Dist.($\neg$, $\to$)\\ %3
        \phi & Debilitamiento(p3)\\ %4
        \tau & Debilitamiento(p3)\\ %5
        \psi & MPP(p4, p0)\\ %6
        \xi & MPP(p5, p1)\\ %7
        \psi \land \xi & Unión(p7, p6) \\ %8
        \psi \land \xi \land \neg (\psi \land \xi) & Unión(p8, Debilitamiento(p3))\\ %9
        \textit{false} & (p9)
    \end{logic}
    Dado que $\Gamma = \{\phi \to \psi, \tau \to \xi\}\cup \{\neg (\phi \land \tau \to \psi \land \xi)\} \vDash \textit{false}$ se demuestra que 
    \[(\phi \to \psi) \land (\tau \to \xi) \to (\phi \land \tau \to \psi \land \xi)\]
\end{logicenv}

\subsect{Lema 7.3.3}
\begin{logicenv}{$\forall x \phi \to \exists x \phi$}
    \begin{derivation}
            \forall x \phi \to \exists x \phi\\
        (alt.)Def.($\to$)\\
            \neg \forall x \phi \lor \exists x \phi\\
        $\neg \forall x \phi \equiv \exists x \neg \phi$\\
            \exists x \neg \phi \lor \exists x \phi\\
        Dist.($\exists x$, $\lor$)\\
            \exists x (\neg \phi \lor \phi)\\
        $\neg \phi \lor \phi \equiv \textit{true}$\\
            \exists x \textit{true}\\
        $\exists x \textit{true} \equiv \textit{true}$\\
            \textit{true}
    \end{derivation}
    Por Identidad($\equiv$) y metateorema de derivación se demuestra que $\forall x \phi \to \exists x \phi$
\end{logicenv}
\end{document}