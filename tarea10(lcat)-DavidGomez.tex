\documentclass[twoside]{article}

% Packages
\usepackage{mathtools}
\usepackage{amssymb}
\usepackage{logicDG}
\usepackage{array}
\usepackage{xcolor}
\usepackage[spanish]{babel}
\usepackage{geometry}
\usepackage{fancyhdr}
\usepackage{graphicx}
\usepackage[hidelinks]{hyperref}

% Font
\usepackage{lmodern}
\usepackage[T1]{fontenc}

% Page configurations

\geometry{
    a4paper,
    margin = 1cm,
    top = 3cm,
    bottom = 2.5cm,
    headheight = 72pt
}
\newcommand{\logo}{C:/Users/usuario/Documents/U/logo-eci.jpg}
\newcommand{\titlename}{Tarea 10}
\renewcommand{\author}{David Gómez}

\pagestyle{fancy}
\fancyhf{}
\fancyhead[LO]{\author}
\fancyhead[LE]{\nouppercase{\leftmark\hfill\rightmark}}
\fancyhead[R]{\includegraphics[width = 4cm]{\logo}}
\fancyfoot[C]{Página \thepage}
\renewcommand{\headrule}{\hbox to \headwidth{\color{rojoEci}\leaders\hrule height \headrulewidth\hfill}}

\hyphenpenalty = 10000

\setlength{\parindent}{0pt}

% Colors
\definecolor{rojoEci}{RGB}{225, 70, 49}

% Documento

\begin{document}
\begin{titlepage}
    \begin{center}
        \vspace*{1cm}
 
        \textbf{\fontsize{45}{\baselineskip}\selectfont{\titlename}}

        \vspace{4cm}

        {\Large Hecho por}

        \vspace{1cm}

        {\textbf{\LARGE\MakeUppercase{\author}}}

        \vspace{2cm}

        \includegraphics[width = .8\textwidth]{\logo}

        \vspace{2cm}

        {\Large Estudiante de Matemáticas\\[5pt]

        Escuela Colombiana de Ingeniería Julio Garavito\\[5pt]

        Colombia\\[5pt]

        \today}
             
    \end{center}
\end{titlepage}

\tableofcontents
\clearpage

\section{Sección 5.3}

\subsection{Punto 5}
\subsubsection{a}
\begin{logicenv}{$\theo{}{DS}{\phi \to (\psi \to \phi)}$}
    \begin{align*}
            \res{\phi}\\
        \why[\Rightarrow]{Debilitamiento($\lor$)}\\
            \res{\phi \lor \neg \psi}\\
        \why{Conmutativa($\lor$)}\\
            \res{\neg \psi \lor \phi}\\
        \why{Teo 4.28.1}\\
            \res{\psi \to \phi}
    \end{align*}
    Por MT 5.5.1 se demuestra que\\
    $\theo{}{DS}{\phi \to (\psi \to \phi)}$
\end{logicenv}

\subsubsection{b}
\begin{logicenv}{$\theo{}{DS}{((\phi \to \psi) \to \phi) \to \phi}$}
    \begin{align*}
            \res{(\phi \to \psi) \to \phi}\\
        \why{Teo 4.28.1}\\
            \res{\neg(\phi \to \psi) \lor \phi}\\
        \why{Dist($\neg$, $\to$), Leibniz($\phi = p \lor \phi)$}\\
            \res{(\phi \land \neg \psi) \lor \phi}\\
        \why{Dist($\lor$, $\land$)}\\
            \res{(\phi \lor \phi) \land (\phi \lor \neg \psi)}\\
        \why{Idempotencia($\lor$)}\\
            \res{\phi \land (\phi \lor \neg \psi)}\\
        \why[\Rightarrow]{Debilitamiento($\land$)}\\
            \res{\phi}
    \end{align*}
    Por MT 5.5.1 se demuestra que\\
    $\theo{}{DS}{((\phi \to \psi) \to \phi) \to \phi}$
\end{logicenv}

\subsubsection{c}
\begin{logicenv}{$\theo{}{DS}{(\phi \to \psi) \to ((\psi \to \tau) \to (\phi \to \tau))}$}
    \begin{align*}
            \res{(\psi \to \tau) \to (\phi \to \tau)}\\
        \why{Teo 4.28.1}\\
            \res{\neg(\psi \to \tau) \lor (\phi \to \tau)}\\
        \why{Teo 4.28.1 Leibniz($\phi = \neg(\psi \to \tau) \lor p$)}\\
            \res{\neg(\psi \to \tau) \lor (\neg\phi \lor \tau)}\\
        \why{Dist($\neg$, $\to$), Leibniz($\phi = p \lor (\neg\phi \lor \tau)$)}\\
            \res{(\psi \land \neg\tau) \lor (\neg\phi \lor \tau)}\\
        \why{Dist($\lor$, $\land$)}\\
            \res{(\neg\phi \lor \tau \lor \psi) \land (\neg\phi \lor \tau \lor \neg\tau)}\\
        \why{Teo 4.19.1, Identidad($\equiv$), Leibniz($\phi = (\neg\phi \lor \tau \lor \psi) \land (\neg\phi \lor p)$)}\\
            \res{(\neg\phi \lor \tau \lor \psi) \land (\neg\phi \lor \text{\textit{true}})}\\
        \why{Teo 4.19.1, Identidad($\equiv$), Leibniz($\phi = (\neg\phi \lor \tau \lor \psi) \land p$)}\\
            \res{(\neg\phi \lor \tau \lor \psi) \land \text{\textit{true}}}\\
        \why{Teo 4.24.3}\\
            \res{\neg\phi \lor \tau \lor \psi}\\
        \why[\Leftarrow]{Debilitamiento($\lor$)}\\
            \res{\neg\phi \lor \psi}\\
        \why{Teo 4.28.1}\\
            \res{\phi \to \psi}
    \end{align*}
    Por MT 5.5.2, y Def($\gets$) se demuestra que\\
    $\theo{}{DS}{(\phi \to \psi) \to ((\psi \to \tau) \to (\phi \to \tau))}$
\end{logicenv}


\subsection{Punto 6}

\subsubsection{a}
\begin{logicenv}{$\theo{}{DS}{(\phi \to \psi) \to (\phi \lor \tau \to \psi \lor \tau)}$}
    \begin{align*}
            \res{\phi \to \psi}\\
        \why{Teo 4.28.1}\\
            \res{\neg\phi \lor \psi}\\
        \why[\Rightarrow]{Debilitamiento($\lor$)}\\
            \res{\neg\phi \lor \psi \lor \tau}\\
        \why{Teo 4.19.4}\\
            \res{\phi \lor \psi  \lor \tau \equiv \psi \lor \tau}\\
        \why{Idempotencia($\lor$), Leibniz($\phi = \phi \lor \psi \lor p \equiv \psi \lor \tau)$}\\
            \res{\phi \lor \psi \lor \tau \lor \tau \equiv \psi \lor \tau}\\
        \why{Conmutativa($\lor$)}\\
            \res{\phi \lor \tau \lor \psi \lor \tau \equiv \psi \lor \tau}\\
        \why{Def($\to$)}\\
            \res{\phi \lor \tau \to \psi \lor \tau}
    \end{align*}
    Por MT 5.5.1 se demuestra que\\
    $\theo{}{DS}{(\phi \to \psi) \to (\phi \lor \tau \to \psi \lor \tau)}$
\end{logicenv}

\subsubsection{b}
\begin{logicenv}[2]{$\theo{}{DS}{(\phi \to \psi) \to (\phi \land \tau \to \psi \land \tau)}$}
    \begin{align*}
            \res{(\phi \land \tau) \to (\psi \land \tau)}\\
        \why{Teo 4.28.1}\\
            \res{\neg(\phi \land \tau) \lor (\psi \land \tau)}\\
        \why{Dist($\neg$, $\land$), Leibniz($\phi = p \lor (\psi \land \tau)$)}\\
            \res{\neg\phi \lor \neg\tau \lor (\psi \land \tau)}\\
        \why{Dist($\lor$, $\land$), Leibniz($\phi = \neg\phi \lor p$)}\\
            \res{\neg\phi \lor ((\neg\tau \lor \psi) \land (\neg\tau \lor \tau))}\\
        \why{Teo 4.19.1, Identidad($\equiv$), Leibniz($\phi = \neg\phi \lor ((\neg\tau \lor \psi) \land p)$)}\\
            \res{\neg\phi \lor ((\neg\tau \lor \psi) \land \text{\textit{true}})}\\
        \why{Teo 4.24.3, Leibniz($\phi = \neg\phi \lor p$)}\\
            \res{\neg\phi \lor \neg\tau \lor \psi}\\
        \why[\Leftarrow]{Debilitamiento($\lor$)}\\
            \res{\neg\phi \lor \psi}\\
        \why{Teo 4.28.1}\\
            \res{\phi \to \psi}
    \end{align*}
    Por MT 5.5.2, Def($\gets$) se demuestra que\\
    $\theo{}{DS}{(\phi \to \psi) \to (\phi \land \tau \to \psi \land \tau)}$
\end{logicenv}


\section{Sección 5.4}
\subsection{Punto 8}
\subsubsection{a}
\begin{logicenv}{Si {$\theo{\Gamma}{DS}{\phi}$ , entonces $\theo{\Gamma}{DS}{\phi \lor \psi}$}}
    \begin{logic}
        \theo{\Gamma}{DS}{\phi} & Suposición\\
        \phi \to (\phi \lor \psi) & Def(Debilitamiento($\lor$))\\
        \phi \lor \psi & Modus Ponens(p1, p0)
    \end{logic}
    El uso de Modus Ponens requiere que $\phi$ se tenga, pero se sabe que $\theo{\Gamma}{DS}{\phi}$, Por lo que, usando Transitividad($\to$)
    $\theo{\Gamma}{DS}{\phi \lor \psi}$
\end{logicenv}

\subsubsection{b}
\begin{logicenv}{Si $\theo{\Gamma}{DS}{\phi \land \psi}$ , entonces $\theo{\Gamma}{DS}{\phi}$}
    \begin{logic}
        \theo{\Gamma}{DS}{\phi \land \psi} & Suposición\\
        (\phi \land \psi) \to \phi & Def(Debilitamiento($\land$))\\
        \phi & Modus Ponens(p1, p0)
    \end{logic}
    El uso de Modus Ponens requiere que $\phi \land \psi$ se tenga, pero se sabe que $\theo{\Gamma}{DS}{\phi \land \psi}$, Por lo que, usando Transitividad($\to$) $\theo{\Gamma}{DS}{\phi}$
\end{logicenv}

\subsubsection{c}
\begin{logicenv}[5]{Si $\theo{\Gamma}{DS}{\phi}$ y $\theo{\Gamma}{DS}{\psi}$ , entonces $\theo{\Gamma}{DS}{\phi \land \psi}$}
    \begin{logic}
        \theo{\Gamma}{DS}{\phi} \text{ y } \theo{\Gamma}{DS}{\psi} & Suposición\\
        \Gamma \vDash \phi \text{ y } \Gamma \vDash \psi & Coherencia(p0)\\
        (\exists \mathbf{v} \,\vert\, \forall x \in \Gamma : \val{v}{x} = \mathtt{T}) & Suposición($\Gamma$ es satisfacible)\\
        \val{v}{\phi} = \mathtt{T} \text{ y } \val{v}{\psi} = \mathtt{T} & Def(p2)(p1)\\
        \val{v}{\phi \land \psi} = \mathtt{T} & MT 2.23($\land$)(p3)\\
        \Gamma \vDash \phi \land \psi & (p4, p3)\\
        \theo{\Gamma}{DS}{\phi \land \psi} & Completitud(p5)
    \end{logic}
\end{logicenv}

\subsection{Punto 10}
\begin{logicenv}[5]{$\theo{\Gamma}{DS}{\phi}$ sii $\Gamma \cup \{\neg\phi\}$ es insatisfacible}
    \begin{logic}
        \Gamma \text{ es satisfacible} & Suposición\\
        \text{Demostración 1}\\
        \text{Demostración 2}\\
        \theo{\Gamma}{DS}{\phi} \text{ sii } \Gamma \cup \{\neg\phi\} \text{ es insatisfacible }
    \end{logic}
\end{logicenv}
\begin{subproof}{Demostración 1}
    Si $\theo{\Gamma}{DS}{\phi}$ , entonces $\Gamma \cup \{\neg\phi\}$ es insatisfacible
    \begin{logic}
        \theo{\Gamma}{DS}{\phi} & Suposición\\
        \Gamma\vDash\phi & Coherencia(p0)\\
        (\exists \mathbf{v} \,\vert\, \forall x \in \Gamma : \val{v}{x} = \mathtt{T}) & Def(p1)\\
        \val{v}{\phi} = \mathtt{T} & (p2, p1)\\
        \val{v}{\neg \phi} = \mathtt{F} & MT 2.23($\neg$)\\
        \text{Demostración 1.1}\\
        \Gamma \cup \{\neg\phi\} \text{ es insatisfacible} & (p5)
    \end{logic}
\end{subproof}
\begin{subproof}[2]{Demostración 1.1}
    \begin{logic}
        \Gamma \cup \{\neg\phi\} \text{ es satisfacible} & Suposición\\
        \Gamma \text{ es satisfacible y } \{\neg\phi\} \text{ es satisfacible} & (p0)\\
        (\exists \mathbf{w} \,\vert\, \forall x \in \Gamma : \val{w}{x} = \mathtt{T}) & Def(p1)\\
        (\exists \mathbf{w} \,\vert\, \forall x \in \{\neg\phi\} : \val{w}{x} = \mathtt{T}) & Def(p1)\\
        \val{w}{\phi} = \mathtt{T} & (p1 Demostración 1)\\
        \val{w}{\neg\phi} = \mathtt{T} & (p3)\\
        \val{w}{\phi} = \mathtt{F} & MT 2.23($\neg$)(p5)\\
        \val{w}{\phi} = \mathtt{T} \text{ y } \val{w}{\phi} = \mathtt{F} & Absurdo(p6, p4)
    \end{logic}
\end{subproof}

\begin{subproof}{Demostración 2}
    Si $\Gamma \cup \{\neg\phi\}$ es insatisfacible entonces $\theo{\Gamma}{DS}{\phi}$
    \begin{logic}
        \Gamma \cup \{\neg\phi\} \text{ es insatisfacible} & Suposición\\
        (\forall \mathbf{v} : (\exists x \in \Gamma \,\vert\, \val{v}{x} = \mathtt{F})) & Def(p0)\\
        \Gamma \text{ es satisfacible} & (p0 punto10)\\
        \{\neg\phi\} \text{es insatisfacible} & (p2 ,p0)\\
        \val{v}{\neg\phi} = \mathtt{F} & (p3)\\
        \val{v}{\phi} = \mathtt{T} & MT 2.23($\neg$)(p4)\\
        \Gamma\vDash\phi & (p5, p2)\\
        \theo{\Gamma}{DS}{\phi} & Completitud(p6)
    \end{logic}
\end{subproof}

\subsection{Punto 11}
\subsubsection{a}
\begin{logicenv}[5]{Si $\theo{\Gamma}{DS}{\phi \lor \psi}$ , entonces $\Gamma \cup \{\neg \phi\}$ o $\Gamma \cup \{\neg \psi\}$ es satisfacible}
    \begin{logic}
        \Gamma \text{ es satisfacible} & Suposición\\
        \theo{\Gamma}{DS}{\phi \lor \psi} & Suposición\\
        (\exists \mathbf{v} \,\vert\, \forall x \in \Gamma : \val{v}{x} = \mathtt{T}) & Def.(p0)\\
        \Gamma\vDash \phi \lor \psi & Coherencia(p1)\\
        \val{v}{\phi \lor \psi} = \mathtt{T} & (p3, p2)\\
        \val{v}{\phi} = \mathtt{T} \text{ o } \val{v}{\psi} = \mathtt{T}\\
        \text{Suposición 1}\\
        \text{Suposición 2}\\
        \text{Suposición 3}\\
        \Gamma \cup \{\neg\phi\} \text{ o } \Gamma \cup \{\neg\psi\} \text{ es insatisfacible}
    \end{logic}
\end{logicenv}

\begin{subproof}{Suposición 1}
    \begin{logic}
        \val{v}{\phi} = \mathtt{T} \text{ y } \val{v}{\psi} = \mathtt{T}\\
        \Gamma \cup \{\neg\phi\} \text{ es insatisfacible} & (p0, p3 Punto 11)\\
        \Gamma \cup \{\neg\psi\} \text{ es insatisfacible} & (p0, p3 Punto 11)
    \end{logic}
\end{subproof}
\begin{subproof}{Suposición 2}
    \begin{logic}
        \val{v}{\phi} = \mathtt{T} \text{ y } \val{v}{\psi} = \mathtt{F}\\
        \Gamma \cup \{\neg\phi\} \text{ es insatisfacible} & (p0, p3 Punto 11)\\
        \Gamma \cup \{\neg\psi\} \text{ es satisfacible} & (p0, p3 Punto 11)
    \end{logic}
\end{subproof}
\begin{subproof}{Suposición 3}
    \begin{logic}
        \val{v}{\phi} = \mathtt{F} \text{ y } \val{v}{\psi} = \mathtt{T}\\
        \Gamma \cup \{\neg\phi\} \text{ es satisfacible} & (p0, p3 Punto 11)\\
        \Gamma \cup \{\neg\psi\} \text{ es insatisfacible} & (p0, p3 Punto 11)
    \end{logic}
\end{subproof}


\subsubsection{b}
\begin{logicenv}[5]{Si $\theo{\Gamma}{DS}{\neg(\phi \lor \psi)}$ , entonces $\Gamma \cup \{\neg\phi, \neg\psi\}$ es satisfacible}
    \begin{logic}
        \Gamma \text{ es satisfacible} & Suposición\\
        \theo{\Gamma}{DS}{\neg(\phi \lor \psi)} & Suposición\\
        \Gamma\vDash \neg(\phi \lor \psi) & Coherencia(p1)\\
        (\exists \mathbf{v} \,\vert\, \forall x \in \Gamma : \val{v}{x} = \mathtt{T}) & Def(p0)\\
        \val{v}{\neg(\phi \lor \psi)} = \mathtt{T} & (p3, p2)\\
        \val{v}{\neg\phi \land \neg\psi} = \mathtt{T} & Dist($\neg$, $\lor$)(p4)\\
        \val{v}{\neg\phi} = \mathtt{T} \text{ y } \val{v}{\neg\psi} = \mathtt{T} & MT 2.23($\land$)(p5)\\
        \Gamma \cup \{\neg\phi, \neg\psi\} \text{ es satisfacible} & (p4, p0)
    \end{logic}
\end{logicenv}


\section{Sección 5.5}
\subsection{Punto 19}
\begin{logicenv}[5]{$\theo{\Gamma}{DS}{\psi \to \phi}$ sii $\theo{\Gamma}{DS}{\psi \land \neg\phi \to \neg\psi}$}
    \begin{logic}
        \Gamma \text{ es satisfacible} & Suposición\\
        \text{Demostración 1}\\
        \text{Demostración 2}\\
        \theo{\Gamma}{DS}{\psi \to \phi} \text{ sii } \theo{\Gamma}{DS}{\psi \land \neg\phi \to \neg\psi}
    \end{logic}
\end{logicenv}
\begin{subproof}{Si $\theo{\Gamma}{DS}{\psi \to \phi}$ , entonces $\theo{\Gamma}{DS}{\psi \land \neg\phi \to \neg\psi}$}
    \begin{logic}
        \theo{\Gamma}{DS}{\psi \to \phi} & Suposición\\ %0
        \Gamma\vDash \psi \to \phi & Coherencia(p0)\\ %1
        (\exists \mathbf{v} \,\vert\, \forall x \in \Gamma : \val{v}{x} = \mathtt{T}) & Def(p0 Punto 19)\\ %2
        \val{v}{\psi \to \phi} = \mathtt{T} & (p2, p1)\\ %3
        \val{v}{\psi \land \neg\phi \to \neg\psi} = \mathtt{T} & Lema 19.1.2(p3)\\ %4
        \Gamma\vDash \psi \land \neg\phi \to \neg\psi (p4, p0)\\%5
        \theo{\Gamma}{DS}{\psi \land \neg\phi \to \neg\psi} & Completitud(p5)
    \end{logic}
\end{subproof}
\begin{subproof}{Si $\theo{\Gamma}{DS}{\psi \land \neg\phi \to \neg\psi}$ , entonces $\theo{\Gamma}{DS}{\psi \to \phi}$}
    \begin{logic}
        \theo{\Gamma}{DS}{\psi \land \neg\phi \to \neg\psi} & Suposición\\ %0
        \Gamma\vDash \psi \land \neg\phi \to \neg\psi & Coherencia(p0)\\%1
        (\exists \mathbf{v} \,\vert\, \forall x \in \Gamma : \val{v}{x} = \mathtt{T}) & Def(p0 Punto 19)\\%2
        \val{v}{\psi \land \neg\phi \to \neg\psi} = \mathtt{T} & (p2, p1)\\%3
        \val{v}{\psi \to \phi} = \mathtt{T} & Lema 19.1.1 (p3)\\%4
        \Gamma\vDash \psi \to \phi & (p4, p0)\\%5
        \theo{\Gamma}{DS}{\psi \to \phi} & Completitud(p5)
    \end{logic}
\end{subproof}
\begin{subproof}{Lema 19.1}
    \begin{derivation}
            \psi \land \neg\phi \to \neg\psi\\
        Dist($\neg$, $\to$)\\
            \neg(\psi \to \phi) \to \neg\psi\\
        Contrapositiva\\
            \psi \to (\psi \to \phi)\\
        Teo 4.31.5\\
            (\psi \land \psi) \to \phi\\
        Teo 4.24.5, Leibniz($\phi = p \to \phi$)\\
            \psi \to \phi
    \end{derivation}
    \begin{itemize}
        \item[.1] Por MT 4.21 se demuestra que $(\psi \land \neg\phi \to \neg\psi) \equiv (\psi \to \phi)$
        \item[.2] Por Conmutativa($\equiv$) se demuestra que $(\psi \to \phi) \equiv (\psi \land \neg\phi \to \neg\psi)$
    \end{itemize}
\end{subproof}


\subsection{Punto 26}
\begin{logicenv}[5]{26}
    ``Si $a, b \in \mathbb{Z}$ son tales que $par(ab)$, entonces al menos uno de $a$ y $b$ es par''

    Sea $S_n$ el termino general de la sucesión que describe a los pares, se define el conjunto de los pares como:
    \[\{S_n\} = \{2n, n \in \mathbb{Z}\}\]

    Sea $U_n$ el termino general de la sucesión que describe a los impares, se define el conjunto de los impares como:
    \[\{U_n\} = \{2n + 1, n \in \mathbb{Z}\}\]

    \begin{logic}
        a, b \in \mathbb{Z} \land ab \in \{S_n\} & Suposición\\%0
        a, b \in \{U_n\} & Suposición\\%1
        ab \in \{S_n\} & Debilitamiento($\land$)(p0)\\%2
        a, b \in \mathbb{Z} & Debilitamiento($\land$)(p0)\\%3
        ab = 2n \text(para algún $n$ entero) & (p2, p1)\\%4
        a = 2c_0 + 1 & (p1)\\%5
        b = 2c_1 + 1 & (p1)\\%6
        ab = (2c_0 + 1)(2c_1 + 1) = 2(2c_0c_1 + c_0 + c_1) + 1 = 2c_2 + 1 &(p6, p5)\\%7
        ab \in \{U_n\} & (p7)\\%8
        ab \in \{U_n\} \land ab \in \{S_n\} & Contradicción (p8, p2)
    \end{logic}
    Por lo que se concluye $\neg(a, b \in \{U_n\})$, es decir\\
    $a \in \{S_n\} \lor b \in \{S_n\}$
\end{logicenv}

\subsection{Punto 27}
\begin{logicenv}[5]{27}
    ``Si $a, b \in \mathbb{Z}$ son tales que $par(ab)$, entonces $a$ y $b$ son impares''

    Sea $U_n$ el termino general de la sucesión que describe a los impares, se define el conjunto de los impares como:
    \[\{U_n\} = \{2n + 1, n \in \mathbb{Z}\}\]

    \begin{logic}
        a, b \in \mathbb{Z} \land ab \in \{U_n\} & Suposición\\%0
        a, b \in \{U_n\} & Debilitamiento($\land$)\\%1
        a = 2k_0 + 1 & (p1)\\%2
        b = 2k_1 + 1 & (p1)\\%3
        ab = (2k_0 + 1)(2k_1 + 1) = 2(2k_0k_1 + k_0 + k_1) + 1 =2k_2 + 1 & (p3, p1)\\%4
        ab \in \{U_n\} 
    \end{logic}
\end{logicenv}
\end{document}
