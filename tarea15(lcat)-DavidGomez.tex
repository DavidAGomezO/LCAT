\documentclass[twoside]{article}

% Packages
\usepackage{mathtools}
\usepackage{amssymb}
\usepackage{logicDG}
\usepackage{array}
\usepackage{xcolor}
\usepackage[spanish]{babel}
\usepackage{geometry}
\usepackage{fancyhdr}
\usepackage{graphicx}
\usepackage[hidelinks]{hyperref}

% Font
\usepackage{lmodern}
\usepackage[T1]{fontenc}

% Page configurations

\geometry{
    a4paper,
    margin = 2.5cm,
    top = 4cm,
    bottom = 2.5cm,
    headheight = 72pt
}
\newcommand{\logo}{C:/Users/usuario/Documents/U/logo-eci.jpg}
\renewcommand{\author}{David Gómez}
\renewcommand{\title}{Titulo}

\pagestyle{fancy}
\fancyhf{}
\fancyhead[LO]{\author}
\fancyhead[LE]{\title}
\fancyhead[R]{\includegraphics[width = 4cm]{\logo}}
\fancyfoot[C]{Página \thepage}
\renewcommand{\headrule}{\hbox to \headwidth{\color{rojoEci}\leaders\hrule height \headrulewidth\hfill}}

\hyphenpenalty = 10000

\setlength{\parindent}{0pt}

\newcommand{\sect}[1]{\section*{#1} \addcontentsline{toc}{section}{#1}}
\newcommand{\subsect}[1]{\subsection*{#1} \addcontentsline{toc}{subsection}{#1}}
\newcommand{\subsubsect}[1]{\subsubsection*{#1} \addcontentsline{toc}{subsubsection}{#1}}

% Colors
\definecolor{rojoEci}{RGB}{225, 70, 49}

% Documento

\begin{document}
\begin{titlepage}
    \begin{center}
        \vspace*{1cm}
 
        \textbf{\fontsize{45}{\baselineskip}\selectfont{\title}}

        \vspace{4cm}

        {\Large Hecho por}

        \vspace{1cm}

        {\textbf{\LARGE\MakeUppercase{\author}}}

        \vspace{2cm}

        \includegraphics[width = .8\textwidth]{\logo}

        \vspace{2cm}

        {\Large Estudiante de Matemáticas\\[5pt]

        Escuela Colombiana de Ingeniería Julio Garavito\\[5pt]

        Colombia\\[5pt]

        \today}
             
    \end{center}
\end{titlepage}

{\centering Sección 7.4}
\tableofcontents
\clearpage

\sect{Punto 1}
\subsect{a}
\begin{logicenv}{Demostración con suposición en $\dsl$}
    \begin{itemize}
        \item[(i)] Si no hay cuantificadores que afecten globalmente a la conclusión entonces se puede definir de la misma forma que en DS
        \item[(ii)] En caso de un cuantificador afectando la conclusión (o que sea referente a un objeto específico) hace falta ver las suposiciones y lograr llevarlas a un nivel del subconjunto al que se hace referencia (ej: tomar suposiciones de los pares para llegar a una conclusión sobre estos, sería lo mismo que reducir el conjunto a los pares y aplicar la definición para DS)  
    \end{itemize}
\end{logicenv}

\subsect{b}
\begin{logicenv}{Derivaciones son suposiciones}
    \begin{itemize}
        \item[(i)] Si no hay cuantificadores que afecten la conclusión globalmente entonces funciona tal como en DS
        \item[(ii)] Se toma todo $\psi \in \Gamma$ como verdadero y aplicando lo que aporta $\dsl$ a lo que ya se tenía en DS más las suposiciones 
    \end{itemize}
\end{logicenv}

\subsect{c}
igual a (b) pero con pasos en los que se justifique con $\Rightarrow$
\subsect{d}
igual a (b) pero con pasos en los que se justifique con $\Leftarrow$

\sect{Punto 2}
\begin{logicenv}{Refutar $\forall x \phi  \equiv \phi$}
    Al añadir un cuantificador, ya sea implícita o explícitamente, se está trabajando sobre los elementos de un conjunto, digamos $A$

    Si $x$ es libre en $\phi$, $\forall x \phi$, hace referencia a una propiedad que cumplen todos los elementos $A$

    En caso de que $\phi$ no tenga cuantificador ni mención de la variable $x$ entonces decir $\forall x \phi$ no es algo que realmente tenga mucho significado, pues nos dice que una proposición la cual en sí misma es verdad, se cumple para todos los elementos.
    
    En el caso de que $x$ sea libre en $\phi$, decir $\forall x \phi \equiv \phi$ nos dice que la propiedad $\phi$ que se cumple para todos los elementos de $A$ se cumple también para todos los elementos de todos los conjuntos, cosa que no es cierta.

    ej:
    \begin{logic}
        \forall x (\sqrt{x^2} = x) & Aquí el conjunto es $\mathbb{R}^+$\\
        \forall x (\sqrt{x^2} = x) \equiv \sqrt{x^2} = x & Proposición a refutar(p0)\\
        \sqrt{x^2} = x & Ecuanimidad(p1, p0)\\
        \forall y (\sqrt{x^2} = x) & Aquí el conjunto es $\mathbb{R}$, Generalización(p2)\\
        \forall y (\sqrt{x^2} = x) \to \sqrt{(-1)^2} = (-1) & $x$ no libre en $(-1)$\\
        \sqrt{(-1)^2} = (-1) & MPP(p4, p3)\\
        \sqrt{(-1)^2} = 1 & Álgebra\\
        -1 = 1 & \textit{false}
    \end{logic}
\end{logicenv}

\sect{Punto 6}
Tomando procedimiento 7.4.1
\begin{logicenv}{$f(n) = \frac{1}{n}, n > 0$}
    \begin{logic}
        f(n) > f(n + 1) & Demostración 1\\
        \theo{\{\epsilon > 0, n > 100\}}{\dsl}{|f(n) - 0| < \epsilon} & Demostración 2\\
        limit(f, 0) = \textit{true}
    \end{logic}
\end{logicenv}
\begin{subproof}{Demostración 1}
    \begin{derivation}
            \textit{true}\\
        Aritmética\\
            1 > 0\\
        Aritmética\\
            n + 1 > n\\
        Aritmética\\
            \frac{1}{n} >  \frac{1}{n + 1}\\
        Def.($f$)\\
            f(n) > f(n + 1)
    \end{derivation}
\end{subproof}
\begin{subproof}{Demostración 2}
    \begin{derivation}
            |f(n) - 0| < \epsilon\\
        Def.($f$)\\
            \frac{1}{n} < \epsilon\\
        $n > 100$, transitividad($<$), Aritmética\\
            0 < \frac{1}{100} < \frac{1}{n} < \epsilon\\
        transitividad($>$), $\epsilon > 0$\\
            \textit{true}
    \end{derivation}
\end{subproof}

\sect{Punto 7}
Tomando procedimiento 7.4.1
\begin{logicenv}{$f(n) = \frac{1}{n + 1}, n \in \mathbb{N}$}
    \begin{logic}
        f(n) > f(n + 1) & Demostración 1\\
        \theo{\{\epsilon > 0, n > 100\}}{\dsl}{|f(n) - 0| < \epsilon} & Demostración 2\\
        limit(f, 0) = \textit{true}
    \end{logic}
\end{logicenv}
\begin{subproof}{Demostración 1}
        \begin{derivation}
                \textit{true}\\
            Aritmética\\
                1 > 0\\
            Aritmética\\
                n + 2 > n + 1\\
            Aritmética\\
                \frac{1}{n + 1} > \frac{1}{n + 2}\\
            Def($f$)\\
                f(n) > f(n + 1)
        \end{derivation}
\end{subproof}
\begin{subproof}{Demostración 2}
    \begin{derivation}
            |f(n) - 0| < \epsilon\\
        Def.($f$)\\
            \frac{1}{n + 1} < \epsilon\\
        $n > 100$, transitividad($<$), Aritmética\\
            0 < \frac{1}{101} < \frac{1}{n + 1} < \epsilon\\
        transitividad($<$), $\epsilon > 0$\\
            \textit{true}
    \end{derivation}
\end{subproof}

\sect{Punto 8}
\begin{logicenv}{$f(n) = \frac{1}{n^2}, n > 0$}
    \begin{logic}
        f(n) > f(n + 1) & Demostración 1\\
        \theo{\{\epsilon > 0, n > 100\}}{\dsl}{|f(n) - 0| < \epsilon} & Demostración 2\\
        limit(f, 0) = \textit{true}
    \end{logic}
\end{logicenv}
\begin{subproof}{Demostración 1}
    \begin{derivation}
            \textit{true}\\
        n > 0\\
            n > -\frac{1}{2}\\
        Álgebra\\
            2n + 1 > 0\\
        Álgebra\\
            n^2 + 2n + 1 > n^2\\
        Álgebra\\
            (n + 1)^2 > n^2\\
        Álgebra\\
            \frac{1}{n^2} > \frac{1}{(n + 1)^2}
    \end{derivation}
\end{subproof}
\begin{subproof}{Demostración 2}
    \begin{derivation}
            |f(n) - 0| < \epsilon\\
        Def.($f$)\\
            \frac{1}{n^2} < \epsilon\\
        n > 100, transitividad($<$), Aritmética\\
            0 < \frac{1}{100^2} < \frac{1}{n^2} < \epsilon\\
        transitividad($<$), $\epsilon >  0$\\
            \textit{true}
    \end{derivation}
\end{subproof}

\sect{Punto 9}
\begin{logicenv}{$f(n) = n, n \in \mathbb{N}$}
    \begin{logic}
        f(n + 1) > f(n) & Demostración 1\\
        \theo{\{\epsilon > 0, n > 10\}}{\dsl}{|f(n) - 0| < \epsilon} & Demostración 2\\
        limit(f, 0) = \textit{false}
    \end{logic}
\end{logicenv}
\begin{subproof}{Demostración 1}
    \begin{derivation}
            \textit{true}\\
        Aritmética\\
            1 > 0\\
        Álgebra\\
            n + 1 > n\\
        Def.($f$)\\
            f(n + 1) > f(n)
    \end{derivation}
\end{subproof}
\begin{subproof}{Demostración 2}
    Suponga que existe el límite y dicho límite es $\mathcal{L}$
    \begin{derivation}
            |f(n) - \mathcal{L}| < \epsilon\\
        
    \end{derivation}
\end{subproof}


\sect{Procedimientos}
\begin{logicenv}{7.4.1}
    \[limit(f, 0) \equiv (\forall \epsilon \in \mathbb{R}\,\vert\, \epsilon > 0: (\exists m \in \mathbb{R} \,\vert\, m \geq 0 : (\forall n \in \mathbb{N} \,\vert\, n > m : |f(n) - 0| < \epsilon)))\]
    Por metateorema 7.22, demostrar
    \[\theo{\{\epsilon > 0\}}{\dsl}{(\exists m \in \mathbb{R} \,\vert\, m \geq 0 : (\forall n \in \mathbb{N} \,\vert\, n > m |f(n) - 0| < \epsilon))}\]

    \begin{align*}
            \res{(\exists m \in \mathbb{R} \,\vert\, m \geq 0 : (\forall n \in \mathbb{N} \,\vert\, n > m : |f(n) - 0| < \epsilon))}\\
        \why{Azúcar sintáctico}\\
            \res{\exists m \in \mathbb{R} (m \geq 0 \land (\forall n \in \mathbb{N} \,\vert\, n > m : |f(n) - 0| < \epsilon))}\\
        \why[\Leftarrow]{instanciación con testigo $a$}\\
            \res{a \geq 0 \land (\forall n \in \mathbb{N} \,\vert\, n > a |f(n) - 0| < \epsilon)}\\
        \why{$a \geq 0 \equiv \textit{true}$, Identidad($\land$)}\\
            \res{(\forall n \in \mathbb{N} \,\vert\, n > a |f(n) - 0| < \epsilon)}
    \end{align*}
    Por metateorema 7.22, demostrar
    \[\theo{\{\epsilon > 0, n > a\}}{\dsl}{|f(n) - 0| < \epsilon}\]
\end{logicenv}
\end{document}