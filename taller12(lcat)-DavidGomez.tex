\documentclass[twoside]{article}

% Packages
\usepackage{mathtools}
\usepackage{amssymb}
\usepackage{logicDG}
\usepackage{array}
\usepackage{xcolor}
\usepackage[spanish]{babel}
\usepackage{geometry}
\usepackage{fancyhdr}
\usepackage{graphicx}
\usepackage[hidelinks]{hyperref}

% Font
\usepackage{lmodern}
\usepackage[T1]{fontenc}

% Page configurations

\geometry{
    a4paper,
    margin = 2.5cm,
    top = 4cm,
    bottom = 2.5cm,
    headheight = 72pt
}
\newcommand{\logo}{C:/Users/usuario/Documents/U/logo-eci.jpg}
\renewcommand{\author}{David Gómez}
\renewcommand{\title}{Taller 12}

\pagestyle{fancy}
\fancyhf{}
\fancyhead[LO]{\author}
\fancyhead[LE]{\nouppercase{\leftmark\hfill\rightmark}}
\fancyhead[R]{\includegraphics[width = 4cm]{\logo}}
\fancyfoot[C]{Página \thepage}
\renewcommand{\headrule}{\hbox to \headwidth{\color{rojoEci}\leaders\hrule height \headrulewidth\hfill}}

\hyphenpenalty = 10000

\setlength{\parindent}{0pt}

\newcommand{\sect}[1]{\section*{#1} \addcontentsline{toc}{section}{#1}}
\newcommand{\subsect}[1]{\subsection*{#1} \addcontentsline{toc}{subsection}{#1}}
\newcommand{\subsubsect}[1]{\subsubsection*{#1} \addcontentsline{toc}{subsubsection}{#1}}

% Colors
\definecolor{rojoEci}{RGB}{225, 70, 49}

% Documento

\begin{document}
\begin{titlepage}
    \begin{center}
        \vspace*{1cm}
 
        \textbf{\fontsize{45}{\baselineskip}\selectfont{\title}}

        \vspace{4cm}

        {\Large Hecho por}

        \vspace{1cm}

        {\textbf{\LARGE\MakeUppercase{\author}}}

        \vspace{2cm}

        \includegraphics[width = .8\textwidth]{\logo}

        \vspace{2cm}

        {\Large Estudiante de Matemáticas\\[5pt]

        Escuela Colombiana de Ingeniería Julio Garavito\\[5pt]

        Colombia\\[5pt]

        \today}
             
    \end{center}
\end{titlepage}

\tableofcontents
\clearpage

\sect{Punto 1}

\begin{logicenv}[5]{Los arreglos a y b contienen los mismos valores}
    \begin{equation*}
        a,b:A \land \texttt{len}(a) = \texttt{len}(b) \land (\forall i:I \,\vert\, 0 \leq i < \texttt{len}(a) : a[i] = b[i])
    \end{equation*}
\end{logicenv}

\sect{Punto 2}
\begin{logicenv}[5]{Si algun valor del arreglo a es igual a 0, algun otro valor del arreglo es 1}
    \begin{equation*}
        a:A \land (\forall i:I \,\vert\, 0 \leq i < \texttt{len}(a) \land a[i] = 0 : (\exists j:I \,\vert\, 0 \leq j < \texttt{len}(a)) : a[j] = 1)
    \end{equation*}
\end{logicenv}

\sect{Punto 3}
\begin{logicenv}[5]{El mínimo valor del arreglo a es un índice del arreglo b}
    \begin{align*}
        a,b:A \land (&\exists i:I \,\vert\, 0 \leq i < \texttt{len}(a) \land (\forall j:I \,\vert\, 0 \leq j < \texttt{len}(a) : a[i] \leq a[j])\\
        &\quad : (\exists k:I \,\vert\, 0 \leq k < \texttt{len}(b)) : k = a[i])
    \end{align*}
\end{logicenv}

\sect{Punto 4}
\begin{logicenv}[5]{Todos los índices del arreglo a son valores del arreglo b}
    \begin{equation*}
        a, b : A \land (\forall i:I (0 \leq i < \texttt{len}(a)) : (\exists j:I \,\vert\, 0 \leq j < \texttt{len}(b)) : i = b[j])
    \end{equation*}
\end{logicenv}

\sect{Punto 5}
\begin{logicenv}[5]{El mínimo valor del arreglo a es el máximo valor del arreglo b}
    \begin{align*}
        a, b : A \land (&\exists i:I \,\vert\, 0 \leq i < \texttt{len}(a) \land (\forall j:I \,\vert\, 0 \leq j < \texttt{len}(a) : a[i] \leq a[j])\\
        &\quad \land (\exists k:I \,\vert\, 0 \leq k < \texttt{len}(b) \land (\forall l:I \,\vert\, 0 \leq l < \texttt{len}(a) b[i] \geq b[j]))\\
        &\quad : a[i] = b[k])
    \end{align*}
\end{logicenv}
\end{document}