\documentclass[twoside]{article}

% Packages
\usepackage{mathtools}
\usepackage{amssymb}
\usepackage{logicDG}
\usepackage{array}
\usepackage{xcolor}
\usepackage[spanish]{babel}
\usepackage{geometry}
\usepackage{fancyhdr}
\usepackage{graphicx}
\usepackage{tikz}
\usetikzlibrary{trees}
\usepackage{forest}
\usepackage[hidelinks]{hyperref}

% Font
\usepackage{lmodern}
\usepackage[T1]{fontenc}

% Page configurations

\geometry{
    a4paper,
    margin = 2.5cm,
    top = 4cm,
    bottom = 2.5cm,
    headheight = 72pt
}
\newcommand{\logo}{C:/Users/usuario/Documents/U/logo-eci.jpg}
\renewcommand{\author}{David Gómez}
\renewcommand{\title}{Tarea 11}

\pagestyle{fancy}
\fancyhf{}
\fancyhead[LO]{\author}
\fancyhead[LE]{\title}
\fancyhead[R]{\includegraphics[width = 4cm]{\logo}}
\fancyfoot[C]{Página \thepage}
\renewcommand{\headrule}{\hbox to \headwidth{\color{rojoEci}\leaders\hrule height \headrulewidth\hfill}}

\hyphenpenalty = 10000

\setlength{\parindent}{0pt}

\newcommand{\sect}[1]{\section*{#1} \addcontentsline{toc}{section}{#1}}
\newcommand{\subsect}[1]{\subsubsection*{#1} \addcontentsline{toc}{subsubsection}{#1}}
\newcommand{\subsubsect}[1]{\subsubsection*{#1} \addcontentsline{toc}{subsubsection}{#1}}

% Colors
\definecolor{rojoEci}{RGB}{225, 70, 49}

% Documento

\begin{document}
\begin{titlepage}
    \begin{center}
        \vspace*{1cm}
 
        \textbf{\fontsize{45}{\baselineskip}\selectfont{\title}}

        \vspace{4cm}

        {\Large Hecho por}

        \vspace{1cm}

        {\textbf{\LARGE\MakeUppercase{\author}}}

        \vspace{2cm}

        \includegraphics[width = .8\textwidth]{\logo}

        \vspace{2cm}

        {\Large Estudiante de Matemáticas\\[5pt]

        Escuela Colombiana de Ingeniería Julio Garavito\\[5pt]

        Colombia\\[5pt]

        \today}
             
    \end{center}
\end{titlepage}

\tableofcontents
\clearpage

\sect{Sección 6.1}
\subsect{Punto 4}
\begin{logicenv}[5]{Lógica aristotélica}
    La lógica aristotélica tiene únicamente 4 posibles fórmulas, y en ellas se usa únicamente una variable junto a una propiedad que cumple o no.
    \begin{gather*}
        \mathcal{L} = (\mathcal{F}, \mathcal{P}, \mathcal{X})\\
        \mathcal{F} = \varnothing\\
        \mathcal{P} = P\\
        \mathcal{X} = S
    \end{gather*}
    Ejemplo:\\
    Universal afirmativo(A) = Todo $S$ es $P$\\
    que escrito como fórmula queda: $\forall S (P(S))$
\end{logicenv}

\sect{Sección 6.2}
% Sección 6.2: 2, 3
\subsect{Punto 2}
\begin{logicenv}[5]{Arbol de sintaxis de $n$ \hyperref[ej63]{Ejemplo 6.3}}
    \begin{center}
        \begin{tikzpicture}
            \node{$n$};
        \end{tikzpicture}
    \end{center}
\end{logicenv}
\begin{logicenv}[5]{Árbol de sintaxis de $g(f(n), n)$ \hyperref[ej63]{Ejemplo 6.3}}
    \begin{center}
        \begin{tikzpicture}
            \node {$g$}
                child {node {$f$}
                    child {node {$n$} }}
                child {node {$n$}};
        \end{tikzpicture}
    \end{center}
\end{logicenv}

\begin{logicenv}[5]{Árbol de sintaxis de $f(g(f(n), n))$, \hyperref[ej63]{Ejemplo 6.3}}
    \begin{center}
        \begin{tikzpicture}
            \node {$f$}
                child {node {$g$}
                    child {node {$f$}
                        child {node{$n$}}}
                    child {node {$n$}}};
        \end{tikzpicture}
    \end{center}
\end{logicenv}

\begin{logicenv}[10]{Ejemplo 6.3}
    \label{ej63}
    Sea $\mathcal{F} \ =\  \{n, f, g\}$ con \ $ar(n) \ =\  0$, \ $ar(f) \ =\  1$ y \ $ar(g) \ =\  2$. Entonces $n$, $g(f(n), n)$ y $f(g(f(n), n))$ son términos. Sin embargo las expresiones $n(f)$ , $g(f(n))$ y $g(f(n), n, n)$ no lo son ¿Por qué?
\end{logicenv}

\subsect{Punto 3}
\begin{logicenv}[5]{3}
    \begin{gather*}
        a\\
        b\\
        c
    \end{gather*}
\end{logicenv}

% Sección 6.3: 8, 10, 11, 12, 14, 16, 19
\sect{Sección 6.3}
\subsect{8}
\subsubsect{a}
\begin{logicenv}[10]{8.a}
    María admira a todos los profesores.
    \[(\forall x \,\vert\, P(x) : A(m, x))\]
\end{logicenv}

\subsubsect{b}
\begin{logicenv}[10]{8.b}
    Algún profesor admira a María.
    \[(\exists x \,\vert\, P(x) : A(x, m))\]
\end{logicenv}

\subsubsect{c}
\begin{logicenv}[10]{8.c}
    \[A(m, m)\]
\end{logicenv}

\subsubsect{d}
\begin{logicenv}[10]{8.d}
    No todos los estudiantes asisten a todas las clases.
    \[(\forall x \,\vert\, C(x) : (\exists y \,\vert\, E(x) : \neg B(y, x)))\]
\end{logicenv}

\subsubsect{e}
\begin{logicenv}[10]{8.e}
    Ninguna clase tuvo como asistentes a todos los estudiantes
    \[\forall x \exists y (C(x) \land E(y) \land \neg B(y, x))\]
\end{logicenv}

\subsubsect{f}
\begin{logicenv}[10]{8.f}
    Ninguna clase tuvo como asistentes a estudiante alguno
    \[(\forall x \forall y\,\vert\, C(x) \land E(y): \neg B(y, x))\]
\end{logicenv}
\subsect{10}
\subsubsect{a}
\begin{logicenv}{Todos tienen una madre}
    \[\forall x \exists y (M(y, x))\]
\end{logicenv}
\subsubsect{b}
\begin{logicenv}[5]{Todos tienen una madre y un padre}
    \[\forall x \exists y \exists z (M(y, x) \land P(z, x))\]
\end{logicenv}
\subsubsect{c}
\begin{logicenv}[5]{Quien sea que tiene una madre tiene un padre}
    \[(\forall x\,\vert\, M(y, x) : P(z, x))\]
\end{logicenv}
\subsubsect{d}
\begin{logicenv}{Juan es abuelo}
    \begin{gather*}
        \text{Juan} : j\\
        P(x, y) \land P(j, x)
    \end{gather*}
\end{logicenv}

\subsubsect{e}
\begin{logicenv}{Ana y Jaime son primos}
    \begin{gather*}
        \text{Ana} : a\\
        \text{Jaime} : j\\
        (A(x, y) \lor H(x, y)) \land (M(x, a) \lor P(x, a)) \land (M(y, j) \lor P(y, j))
    \end{gather*}
\end{logicenv}

\subsubsect{f}
\begin{logicenv}[5]{Algunas madres son tias}
    \[(\exists x \,\vert\, (M(y, z) \lor P(y, z)) : A(x, y) \land M(x, w))\]
\end{logicenv}
\subsubsect{g}
\begin{logicenv}[5]{Ningún tío es padre}
    \[(\forall x \,\vert\, ((P(y, z) \lor M(y, z)) \land H(x, y)) : \neg P(x, w))\]
\end{logicenv}
\subsubsect{h}
\begin{logicenv}[5]{La abuela de nadie es padre de alguien}
    \[\forall x \exists y (M(x, z) \land \neg M(z, w) \land P(x, y))\]
\end{logicenv}
\subsubsect{i}
\begin{logicenv}{Juan y Juana son marido y mujer}
    \begin{gather*}
        \text{Juan} : j\\
        \text{Juana} : ja\\
        E(j, ja)
    \end{gather*}
\end{logicenv}
\subsubsect{j}
\begin{logicenv}{Carlos es el cuñado de Mónica}
    \begin{gather*}
        \text{Carlos} : c\\
        \text{Mónica} : m\\
        (H(x, m) \lor A(x, m)) \land E(c, m)
    \end{gather*}
\end{logicenv}

\subsect{11}
\subsubsect{a}
\begin{logicenv}{Hay al menos dos elementos}
    \[\exists x \exists y\]
\end{logicenv}
\subsubsect{b}
\begin{logicenv}{Hay a lo sumo dos elementos}
    \[\forall x (\exists y \land \neg \exists w)\]
\end{logicenv}
\subsubsect{c}
\begin{logicenv}{Hay exactamente tres elementos}
    \[\forall x \forall y \forall z (\neg \exists w)\]
\end{logicenv}
\subsubsect{d}
\begin{logicenv}[5]{Para cualquier par de elementos, hay otro elemento distinto a ellos}
    \[\forall x, \forall y (\exists w \land \neg P(x, w) \land \neg P(y, w))\]
\end{logicenv}
\subsect{12}
\subsubsect{a}
\begin{logicenv}[5]{Exactamente un elemento tiene la propiedad R}
    \[(\exists x \,\vert\, P(x) : \neg \exists y \,\vert\,: P(x))\]
\end{logicenv}
\subsubsect{b}
\begin{logicenv}[5]{Todos, excepto dos elementos tienen la propiedad R}
    \[\exists y, z (\neg P(y) \land \neg P(z))\]
\end{logicenv}
\subsect{14}
\begin{logicenv}{$\mathcal{L}$}
    \begin{gather*}
        \mathcal{F} = \emptyset\\
        \mathcal{P} = \{E\}\\
    \end{gather*}
    Donde $ar(E) = 1$ y $E(x)$ simboliza ``la persona $x$ es egoísta''
\end{logicenv}
\subsubsect{a}
\begin{logicenv}[5]{Todos los humanos son egoístas}
    \[(\forall x\,\vert\,: E(x))\]
\end{logicenv}
\subsubsect{b}
\begin{logicenv}[5]{Ningún humano es egoísta}
    \[(\forall x \,\vert\,: \neg E(x))\]
\end{logicenv}
\subsubsect{c}
\begin{logicenv}[5]{Algunos humanos son egoístas}
    \[(\exists x\,\vert\, : E(x))\]
\end{logicenv}
\subsubsect{d}
\begin{logicenv}[5]{Algunos humanos no son egoístas}
    \[(\exists x \,\vert\,: \neg E(x))\]
\end{logicenv}
\subsect{16}
\begin{logicenv}{$\mathcal{L}$}
    \begin{gather*}
        \mathcal{F} = \emptyset\\
        \mathcal{P} = \{E, O\}
    \end{gather*}
    Donde $ar(E) = 1$, $ar(O) = 1$, $E(y)$ simboliza ``Usted engaña a $y$'' y $O(x)$ simboliza ``$x$ es una ocasión''.
\end{logicenv}
\subsubsect{a}
\begin{logicenv}[5]{Usted puede engañar a algunos algunas veces}
    \[\exists x \,\vert\, O(x) : E(y)\]
\end{logicenv}
\subsubsect{b}
\begin{logicenv}[5]{Usted puede engañar a todos algunas veces}
    \[\forall x \exists y (O(y) \land E(x))\]
\end{logicenv}
\subsubsect{c}
\begin{logicenv}[5]{Usted no puede engañarlos a todos algunas veces}
    \[\forall x \exists y (\neg E(x) \land O(y))\]
\end{logicenv}
\subsubsect{d}
\begin{logicenv}[5]{Usted no puede engañar a alguien todas las veces}
    \[\forall x \exists y (\neg E(y) \land O(x))\]
\end{logicenv}
\subsect{19}
\begin{logicenv}[5]{19}
    Tomando $\mathcal{L}_{19}$

    \[(\forall t_0 \,\vert\, T(t_0) \land s_0 = \text{tarea}(t_0): s_1 = \text{inicio}(t_0) \land (\forall t_1 \,\vert\, t_1 \in \text{prer}(t_0) : C(t_1)))\]
\end{logicenv}
\vspace*{-.01cm}
\begin{subproof}{$\mathcal{L}_{19}$}
    \begin{gather*}
        \mathcal{F} = \{\text{tarea}, \text{inicio}, \text{prer}\}\\
    \mathcal{P} = \{T, C\}
    \end{gather*}
\end{subproof}

% Sección 6.4: 3,4,5,6
\sect{Sección 6.4}
\subsect{Punto 3}
\subsubsect{a}

\subsubsect{b}
\begin{logicenv}[5]{$S(m, x)$}
    $x$ es libre
\end{logicenv}
\subsubsect{c}
\begin{logicenv}[5]{$B(m, f(m))$}
    No hay variables
\end{logicenv}
\subsubsect{d}
\begin{logicenv}[5]{$B(x, y) \to \exists z S(z, y)$}
    $x$ , $y$ son libres\\
    $z$ es acotada
\end{logicenv}
\subsubsect{e}
\begin{logicenv}[5]{$S(x, y) \to S(y, f(f(x)))$}
    Todas las variables son libres
\end{logicenv}
\subsect{Punto 4}
\subsubsect{a}
\begin{logicenv}[5]{$P(c, c, d) \lor \forall x P(f(d), h(h(c, x), d), y)$}
    todas las apariciones de $x$ son acotadas y todas las de $y$ son libres\\
    el alcance de $\forall$ es a $x$
\end{logicenv}
\subsubsect{b}
\begin{logicenv}[5]{$\exists y (P(x, y, x) \to \exists z Q(z, y, f(z)))$}
    todas las apariciones de $y$ son acotadas por $\exists y$\\
    todas las apariciones de $z$ son acotadas por $\exists z$\\
    todas las apariciones de $x$ son libres
\end{logicenv}
\subsubsect{c}
\begin{logicenv}[5]{$\exists y P(x, y, Z) \not\equiv \forall y Q(z, y, f(z))$}
    $y$ es acotada en un momento por $\exists y$ y luego por $\forall y$\\
    Todas las apariciones de $x$, $z$ son libres
\end{logicenv}
\subsubsect{d}
\begin{logicenv}[5]{$\exists y (P(x, y, x) \not\equiv \forall y Q(z, y, f(z)))$}
    $y$ es acotada en un momento por $\exists y$ y luego por $\forall y$\\
    Todas las apariciones de $x$, $z$ son libres
\end{logicenv}
\subsubsect{e}
\begin{logicenv}[5]{$\forall x \exists y P(x, y, x) \to \exists z Q(z, y, f(x))$}
    $x$ es acotada por $\forall x$ y luego es libre\\
    $z$ es acotada por $\exists z$\\
    $y$ es libre
\end{logicenv}
\subsubsect{f}
\begin{logicenv}[5]{$\forall z \exists y P(x, y, x) \to \exists z Q(z, y, f(x))$}
    $y$ es acotada por $\exists y$  y luego es libre\\
    $z$ es acotada por $\exists z$\\
    $x$ es libre
\end{logicenv}
\subsubsect{g}
\begin{logicenv}[5]{$\forall x (\exists y P(x, y, x) \land \exists z Q(z,  y, f(x)))$}
    $x$ es acotada por $\forall x$\\
    $y$ es acotada por $\exists y$ y luego es libre\\
    $z$ es acotada por $\exists z$
\end{logicenv}
\subsubsect{h}
\begin{logicenv}[5]{$\forall z (\exists y P(x, y, x) \land \exists z Q(z, y, f(x)))$}
    $x$ es libre\\
    $y$ es acotada por $\exists y$ y luego es libre\\
    $z$ es acotada por $\exists z$
\end{logicenv}
\subsect{Punto 5}
\subsubsect{a}
\begin{logicenv}[5]{Arbol de sintaxis}
    \begin{center}
        \begin{forest}
            for tree={parent anchor=south,child anchor=north}
            [
                $\exists x$
                [
                    $\land$
                    [
                        $P$
                        [$y$]
                        [$z$]
                    ]
                    [
                        $\forall y$
                        [
                            $\lor$
                            [
                                $\neg$
                                [
                                    $Q$
                                    [$y$]
                                    [$x$]
                                ]
                            ]
                            [
                                $P$
                                [$y$]
                                [$z$]
                            ]
                        ]
                    ]
                ]
            ]
        \end{forest}
    \end{center}
\end{logicenv}
\subsubsect{b}
\begin{logicenv}[5]{apariciones libres y acotadas}
    $x$ es acotada por $\exists x$\\
    $y$ es libre y luego acotada por $\forall y$\\
    $z$ es libre
\end{logicenv}
\subsubsect{c}
\begin{logicenv}[5]{c}
    Sí, $y$ pues primero no se ve afectada en $P(y, z)$ por ningún cuantificador, sin embargo, luego se ve afectada en $\neg Q(y, x) \lor P(y, z)$ por $\forall y$
\end{logicenv}
\subsubsect{d}
\begin{logicenv}[5]{d}
    El alcance de $\exists x$ es $P(y, z) \land \forall y (\neg Q(y, x) \lor P(y, z))$
\end{logicenv}
\subsubsect{e}
\begin{logicenv}[5]{e}
    El alcance de $\forall y$ es $\neg Q(y, x) \lor P(y, z)$
\end{logicenv}
\subsubsect{f}
\begin{logicenv}[5]{f}
    $\exists x P(y, z) \land \forall y (\neg Q(y, x) \lor P(y, z))$
    \begin{center}
        \begin{forest}
            for tree={parent anchor=south,child anchor=north}
            [
                $\land$
                [
                    $\exists x$
                    [
                        $P$
                        [$y$]
                        [$z$]
                    ]
                ]
                [
                    $\forall y$
                    [
                        $\lor$
                        [
                            $\neg$
                            [
                                $Q$
                                [$y$]
                                [$x$]
                            ]
                        ]
                        [
                            $P$
                            [$y$]
                            [$z$]
                        ]
                    ]
                ]
            ]
        \end{forest}
    \end{center}
\end{logicenv}
\subsect{Punto 6}
\begin{logicenv}[10]{Definición de $quant(x, \phi)$}
    Sea $\mathbb{C} = \{\equiv, \not\equiv, \land, \lor, \to, \gets\}$\\
    Sea $fun(x)$ una función que encuentre el cuantificador más cercano a la izquierda de $x$ que tenga a $x$\\
    Sea $op()$ una función que encuentre cualquier $c \in \mathbb{C}$\\
    Sea $par(x)$ una función que encuentra la parentización de la sub-fórmula en la que se encuentra $x$
    Si $op()$ retorna verdadero en el recorrido que hace $fun(x)$ y la posición retornada por $fun(x)$ no se encuentra inmediatamente a la izquierda de el primer paréntesis encontrado por $par(x)$, entonces $x$ es libre. 
\end{logicenv}
\end{document}