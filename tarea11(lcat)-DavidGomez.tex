\documentclass[twoside]{article}

% Packages
\usepackage{mathtools}
\usepackage{amssymb}
\usepackage{logicDG}
\usepackage{array}
\usepackage{xcolor}
\usepackage[spanish]{babel}
\usepackage{geometry}
\usepackage{fancyhdr}
\usepackage{graphicx}
\usepackage{tikz}
\usetikzlibrary{trees}
\usepackage[hidelinks]{hyperref}

% Font
\usepackage{lmodern}
\usepackage[T1]{fontenc}

% Page configurations

\geometry{
    a4paper,
    margin = 2.5cm,
    top = 4cm,
    bottom = 2.5cm,
    headheight = 72pt
}
\newcommand{\logo}{C:/Users/usuario/Documents/U/logo-eci.jpg}
\renewcommand{\author}{David Gómez}
\renewcommand{\title}{Tarea 11}

\pagestyle{fancy}
\fancyhf{}
\fancyhead[LO]{\author}
\fancyhead[LE]{\title}
\fancyhead[R]{\includegraphics[width = 4cm]{\logo}}
\fancyfoot[C]{Página \thepage}
\renewcommand{\headrule}{\hbox to \headwidth{\color{rojoEci}\leaders\hrule height \headrulewidth\hfill}}

\hyphenpenalty = 10000

\setlength{\parindent}{0pt}

\newcommand{\sect}[1]{\section*{#1} \addcontentsline{toc}{section}{#1}}
\newcommand{\subsect}[1]{\subsubsection*{#1} \addcontentsline{toc}{subsubsection}{#1}}
\newcommand{\subsubsect}[1]{\subsubsection*{#1} \addcontentsline{toc}{subsubsection}{#1}}

% Colors
\definecolor{rojoEci}{RGB}{225, 70, 49}

% Documento

\begin{document}
\begin{titlepage}
    \begin{center}
        \vspace*{1cm}
 
        \textbf{\fontsize{45}{\baselineskip}\selectfont{\title}}

        \vspace{4cm}

        {\Large Hecho por}

        \vspace{1cm}

        {\textbf{\LARGE\MakeUppercase{\author}}}

        \vspace{2cm}

        \includegraphics[width = .8\textwidth]{\logo}

        \vspace{2cm}

        {\Large Estudiante de Matemáticas\\[5pt]

        Escuela Colombiana de Ingeniería Julio Garavito\\[5pt]

        Colombia\\[5pt]

        \today}
             
    \end{center}
\end{titlepage}

\tableofcontents
\clearpage

\sect{Sección 6.1}
\subsect{Punto 4}
\begin{logicenv}[5]{Lógica aristotélica}
    La lógica aristotélica tiene únicamente 4 posibles fórmulas, y en ellas se usa únicamente una variable junto a una propiedad que cumple o no.
    \begin{gather*}
        \mathcal{L} = (\mathcal{F}, \mathcal{P}, \mathcal{X})\\
        \mathcal{F} = \varnothing\\
        \mathcal{P} = P\\
        \mathcal{X} = S
    \end{gather*}
    Ejemplo:\\
    Universal afirmativo(A) = Todo $S$ es $P$\\
    que escrito como fórmula queda: $\forall S (P(S))$
\end{logicenv}

\sect{Sección 6.2}
% Sección 6.2: 2, 3
\subsect{Punto 2}
\begin{logicenv}[5]{Arbol de sintaxis de $n$ \hyperref[ej63]{Ejemplo 6.3}}
    \begin{center}
        \begin{tikzpicture}
            \node{$n$};
        \end{tikzpicture}
    \end{center}
\end{logicenv}
\begin{logicenv}[5]{Árbol de sintaxis de $g(f(n), n)$ \hyperref[ej63]{Ejemplo 6.3}}
    \begin{center}
        \begin{tikzpicture}
            \node {$g$}
                child {node {$f$}
                    child {node {$n$} }}
                child {node {$n$}};
        \end{tikzpicture}
    \end{center}
\end{logicenv}

\begin{logicenv}[5]{Árbol de sintaxis de $f(g(f(n), n))$, \hyperref[ej63]{Ejemplo 6.3}}
    \begin{center}
        \begin{tikzpicture}
            \node {$f$}
                child {node {$g$}
                    child {node {$f$}
                        child {node{$n$}}}
                    child {node {$n$}}};
        \end{tikzpicture}
    \end{center}
\end{logicenv}

\begin{logicenv}[10]{Ejemplo 6.3}
    \label{ej63}
    Sea $\mathcal{F} \ =\  \{n, f, g\}$ con \ $ar(n) \ =\  0$, \ $ar(f) \ =\  1$ y \ $ar(g) \ =\  2$. Entonces $n$, $g(f(n), n)$ y $f(g(f(n), n))$ son términos. Sin embargo las expresiones $n(f)$ , $g(f(n))$ y $g(f(n), n, n)$ no lo son ¿Por qué?
\end{logicenv}

\subsect{Punto 3}
\begin{logicenv}[5]{3}
    \begin{gather*}
        a\\
        b\\
        c
    \end{gather*}
\end{logicenv}

% Sección 6.3: 8, 10, 11, 12, 14, 16, 19
\sect{Sección 6.3}
\subsect{8}
\subsubsect{a}
\begin{logicenv}[10]{8.a}
    María admira a todos los profesores.
    \[(\forall x \,\vert\, P(x) : A(m, x))\]
\end{logicenv}

\subsubsect{b}
\begin{logicenv}[10]{8.b}
    Algún profesor admira a María.
    \[(\exists x \,\vert\, P(x) : A(x, m))\]
\end{logicenv}

\subsubsect{c}
\begin{logicenv}[10]{8.c}
    \[A(m, m)\]
\end{logicenv}

\subsubsect{d}
\begin{logicenv}[10]{8.d}
    No todos los estudiantes asisten a todas las clases.
    \[(\forall x \,\vert\, C(x) : (\exists y \,\vert\, E(x) : \neg B(y, x)))\]
\end{logicenv}

\subsubsect{e}
\begin{logicenv}[10]{8.e}
    Ninguna clase tuvo como asistentes a todos los estudiantes
    \[\forall x \exists y (C(x) \land E(y) \land \neg B(y, x))\]
\end{logicenv}

\subsubsect{f}
\begin{logicenv}[10]{8.f}
    Ninguna clase tuvo como asistentes a estudiante alguno
    \[(\forall x \forall y\,\vert\, C(x) \land E(y): \neg B(y, x))\]
\end{logicenv}
\subsect{10}
\subsubsect{a}
\begin{logicenv}{Todos tienen una madre}
    \[\forall x \exists y (M(y, x))\]
\end{logicenv}
\subsubsect{b}
\begin{logicenv}[5]{Todos tienen una madre y un padre}
    \[\forall x \exists y \exists z (M(y, x) \land P(z, x))\]
\end{logicenv}
\subsubsect{c}
\begin{logicenv}[5]{Quien sea que tiene una madre tiene un padre}
    \[(\forall x\,\vert\, M(y, x) : P(z, x))\]
\end{logicenv}
\subsubsect{d}
\begin{logicenv}{Juan es abuelo}
    \begin{gather*}
        \text{Juan} : j\\
        P(x, y) \land P(j, x)
    \end{gather*}
\end{logicenv}

\subsubsect{e}
\begin{logicenv}{Ana y Jaime son primos}
    \begin{gather*}
        \text{Ana} : a\\
        \text{Jaime} : j\\
        (A(x, y) \lor H(x, y)) \land (M(x, a) \lor P(x, a)) \land (M(y, j) \lor P(y, j))
    \end{gather*}
\end{logicenv}

\subsubsect{f}
\begin{logicenv}[5]{Algunas madres son tias}
    \[(\exists x \,\vert\, (M(y, z) \lor P(y, z)) : A(x, y) \land M(x, w))\]
\end{logicenv}
\subsubsect{g}
\begin{logicenv}[5]{Ningún tío es padre}
    \[(\forall x \,\vert\, ((P(y, z) \lor M(y, z)) \land H(x, y)) : \neg P(x, w))\]
\end{logicenv}
\subsubsect{h}
\begin{logicenv}[5]{La abuela de nadie es padre de alguien}
    \[\forall x \exists y (M(x, z) \land \neg M(z, w) \land P(x, y))\]
\end{logicenv}
\subsubsect{i}
\begin{logicenv}{Juan y Juana son marido y mujer}
    \begin{gather*}
        \text{Juan} : j\\
        \text{Juana} : ja\\
        E(j, ja)
    \end{gather*}
\end{logicenv}
\subsubsect{j}
\begin{logicenv}{Carlos es el cuñado de Mónica}
    \begin{gather*}
        \text{Carlos} : c\\
        \text{Mónica} : m\\
        (H(x, m) \lor A(x, m)) \land E(c, m)
    \end{gather*}
\end{logicenv}

\subsect{11}
\subsubsect{a}
\begin{logicenv}{Hay al menos dos elementos}
    \[\exists x \exists y\]
\end{logicenv}
\subsubsect{b}
\begin{logicenv}{Hay a lo sumo dos elementos}
    \[\forall x (\exists y \land \neg \exists w)\]
\end{logicenv}
\subsubsect{c}
\begin{logicenv}{Hay exactamente tres elementos}
    \[\forall x \forall y \forall z (\neg \exists w)\]
\end{logicenv}
\subsubsect{d}
\begin{logicenv}[5]{Para cualquier par de elementos, hay otro elemento distinto a ellos}
    \[\forall x, \forall y (\exists w \land \neg P(x, w) \land \neg P(y, w))\]
\end{logicenv}
\subsect{12}
\subsubsect{a}
\begin{logicenv}[5]{Exactamente un elemento tiene la propiedad R}
    \[(\exists x \,\vert\, P(x) : \neg \exists y \,\vert\,: P(x))\]
\end{logicenv}
\subsubsect{b}
\begin{logicenv}[5]{Todos, excepto dos elementos tienen la propiedad R}
    \[\exists y, z (\neg P(y) \land \neg P(z))\]
\end{logicenv}
\subsect{14}
\begin{logicenv}{$\mathcal{L}$}
    \begin{gather*}
        \mathcal{F} = \emptyset\\
        \mathcal{P} = \{E\}\\
    \end{gather*}
    Donde $ar(E) = 1$ y $E(x)$ simboliza ``la persona $x$ es egoísta''
\end{logicenv}
\subsubsect{a}
\begin{logicenv}[5]{Todos los humanos son egoístas}
    \[(\forall x\,\vert\,: E(x))\]
\end{logicenv}
\subsubsect{b}
\begin{logicenv}[5]{Ningún humano es egoísta}
    \[(\forall x \,\vert\,: \neg E(x))\]
\end{logicenv}
\subsubsect{c}
\begin{logicenv}[5]{Algunos humanos son egoístas}
    \[(\exists x\,\vert\, : E(x))\]
\end{logicenv}
\subsubsect{d}
\begin{logicenv}[5]{Algunos humanos no son egoístas}
    \[(\exists x \,\vert\,: \neg E(x))\]
\end{logicenv}
\subsect{16}

\subsect{19}
\end{document}