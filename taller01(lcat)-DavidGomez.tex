\documentclass{article}

% librerías:
\usepackage{amsmath, mathtools, amssymb, mathrsfs, amsthm, multicol, nicematrix}
\usepackage{lmodern, geometry, graphicx}
\usepackage[T1]{fontenc}
\usepackage[spanish]{babel}

%configuraciones:
\setlength{\columnseprule}{1pt}
\geometry{
    left = 20mm,
    right = 20mm,
    top = 20mm
}
\hyphenpenalty=10000
\graphicspath{ {C:/Users/usuario/Documents/U/} }

%texto
\begin{document}
\begin{titlepage}
	\begin{center}
		\vspace*{1cm}

		\textbf{\Large{Taller No.1}}

		\vspace{1.5cm}

		\textbf{David Gómez}

		\vspace{5cm}

		\includegraphics[width=\textwidth]{logo-eci.jpg}

		\vspace{5cm}

		Matemáticas\\
		Escuela Colombiana de Ingeniería Julio Garavito\\
		Colombia\\
		\today

	\end{center}
\end{titlepage}

\section{Punto 1}
\begin{center}
    \includegraphics[width=10cm]{diagonal.png}
\end{center}
La idea de este proceso es ver cómo se generan las diagonales de un polígono. Por lo que observé, en los primeros dos pasos de cualquier polígono de lados mayores que o iguales a 4, la cantidad de diagonales es $(l-3)$, de ahí en adelante, el número de diagonales va disminuyendo hasta llegar a $1$. La cantidad de pasos es $(l-2)$.

Si se realiza el paso a paso con una figura de más lados (en mi caso usé un dodecaedro), y se escribe el número de diagonales obtenidas en cada paso se puede ver que la operación a realizar es la siguiente:
\begin{alignat*}{2}
    (l-3) + \sum_{i = 1}^{l-3} i &= (l-3) + \frac{(l-3)(l-2)}{2}\\
    &= \frac{l^2 -2l -3l +6 + 2l -6}{2}\\
    &= \frac{l(l-3)}{2}
    \intertext{Por otro lado, si se intuye una función recursiva a partir de procesos como el ilustrado en la imágen, se llega a que la función $D$, la cual tiene de entrada el número de lados, y salida el número de diagonales, se puede definir como:}
\end{alignat*}
\vspace*{-1cm}
\begin{equation*}
    D(l) :=
    \begin{cases}
        D(l-1) + l-2\\
        D(4) = 2
    \end{cases}
\end{equation*}
En base a esto, procedo a afirmar que:
\begin{alignat*}{2}
    D(l) &= (l-3) + \sum_{i = 1}^{l-3} i\ ,\ \forall l\, \vert\, l\in \mathbb{N} \wedge l \geq 5\\
    \intertext{Caso base, $n=5$}
    D(5) &= (5-3) + \sum_{i = 1}^{5-3} i\\
    D(4) + 3 &= (2) + \sum_{i = 1}^{2} i\\
    2 + 3 &= 2 + 1 + 2\\
    5 &= 5 \tag*{\textit{true}}
    \intertext{Caso inductivo, $n = k+1$}
    D(k+1) &= D(k) + (k-1)\\
    &= \frac{k(k-3)}{2} + (k-1)\\
    &= \frac{k^2 -3k +2k -2}{2}\\
    &= \frac{k^2 - k - 2}{2}\\
    &= \frac{k^2 + k - 2k - 2}{2}\\
    &= \frac{k(k+1) - 2(k+1)}{2}\\
    &= \frac{(k+1)(k-2)}{2} \tag*{$\Box$}
\end{alignat*}
\section{Punto 2}
\begin{equation*}
    v(n) :=
    \begin{cases}
        v(n-1)\,\left[1-\frac{K}{100}\right]\\
        v(0) = v_0
    \end{cases}
\end{equation*}
\begin{alignat*}{2}
    v(0) &= v_0\\
    v(1) &= v_0\,\left(1-\frac{K}{100}\right)\\
    v(2) &= v_0\,\left(1-\frac{K}{100}\right)\,\left(1-\frac{K}{100}\right)\\
    &\vdots\\
    v(n) &= v_0\,\left(1-\frac{K}{100}\right)^n \ ,\ \forall n\,\vert\, n \in \mathbb{N} \wedge n \geq 1\\
    \intertext{Caso base, $n=1$}
    v(1) &= v_0 \,\left(1-\frac{K}{100}\right)\\
    v_0\,\left(1-\frac{K}{100}\right) &= v_0 \,\left(1-\frac{K}{100}\right)\\
    \intertext{Caso inductivo, $n = c+1$}
    v(c+1) &= v(c)\,\left[1 - \frac{K}{100}\right]\\
    &= v_0\,\left(1-\frac{K}{100}\right)^c\,\left(1-\frac{K}{100}\right)\\
    &= v_0\,\left(1-\frac{K}{100}\right)^{c+1} \tag*{$\Box$}\\
\end{alignat*}
\section{Punto 3}
\begin{equation*}
    v(n) :=
    \begin{cases}
        v(n-1)\,\left[1-\frac{K}{100}\right] - C\\
        v(0) = v_0
    \end{cases}
\end{equation*}
\begin{alignat*}{2}
    v(0) &= v_0\\
    v(1) &= v_0\,\left(1-\frac{K}{100}\right) - C\\
    v(2) &= v_0\left(1 - \frac{k}{100}\right)^2 - C\left(1 - \frac{k}{100}\right) - C\\
    &\vdots\\
    v(n) &= v_0\,\left(1-\frac{K}{100}\right)^n - C \sum_{i=0}^{n-1}\left(1 - \frac{k}{100}\right)^i \ ,\ \forall n\,\vert\, n \in \mathbb{N} \wedge n \geq 1\\
    \intertext{Caso base, $n=1$}
    v(1) &= v_0 \,\left(1-\frac{K}{100}\right) - C\\
    v_0\,\left(1-\frac{K}{100}\right) -C &= v_0 \,\left(1-\frac{K}{100}\right) - C\\
    \intertext{Caso inductivo, $n = u+1$}
    v(u+1) &= v(u)\,\left[1 - \frac{K}{100}\right]-C\\
    &=\left(v_0\left(1-\frac{K}{100}\right)^u - C \sum_{i=0}^{u-1}\left(1 - \frac{k}{100}\right)^i\right)\left(1-\frac{K}{100}\right) - C\\
    &= v_0\left(1-\frac{K}{100}\right)^{u+1} - C \sum_{i=0}^{u-1}\left(1 - \frac{k}{100}\right)^{i+1} - C\\
    &= v_0\left(1-\frac{K}{100}\right)^{u+1} - C \sum_{i=0}^{u}\left(1 - \frac{k}{100}\right)^{i} \tag*{$\Box$}
\end{alignat*}
\section{Punto 4}
\begin{equation*}
    C(\psi I) :=
    \begin{cases}
        C(\psi) + 1\\
        C(I) = 1
    \end{cases}
\end{equation*}
\begin{alignat*}{2}
    P(x\phi y)&: C(y) = 2(C(x)) - 1\\
    \intertext{En axiomas}
    IDI \tag*{Axioma}\\
    C(I) &= 2C(I) - 1\\
    1 &= 2 - 1\\
    1 &= 1 \tag*{\textit{true}}\\
    \intertext{En R1}
    xIDyII \tag*{Definición de R1}
    C(yII) &= 2C(xI) - 1\\
    C(y) + 2 &= 2C(x) + 2 - 1\\
    C(y) &= 2C(x) - 1 \tag*{\textit{true} (por hipótesis de inducción)}
\end{alignat*}
\end{document}