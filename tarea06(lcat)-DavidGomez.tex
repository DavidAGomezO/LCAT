\documentclass{article}

% Librerías:
\usepackage{amsmath, mathtools, amssymb, mathrsfs, amsthm, nicematrix, array}
\usepackage{lmodern, graphicx, fancyhdr}
\usepackage[margin = 2cm, top = 2.5cm, includefoot]{geometry}
\usepackage{xcolor}
\usepackage[hidelinks]{hyperref}
\usepackage[T1]{fontenc}
\usepackage[spanish]{babel}
\usepackage{listings}
\usepackage{tikz}
\usetikzlibrary{shapes, fit, tikzmark}
\usepackage{color, colortbl}
\usepackage[most]{tcolorbox}

% Configuraciones:
\pagestyle{fancy}
\fancyhf{}
\setlength{\headheight}{55.34027pt}
\rhead{\textit{David Gómez}}
\lhead{\includegraphics[width = 4cm]{\logo}}
\lfoot{Página \thepage}
\rfoot{\titlename}
\renewcommand{\headrule}{\hbox to \headwidth{\color{rojoEci}\leaders\hrule height \headrulewidth\hfill}}
\renewcommand{\footrulewidth}{0.4pt}

\hyphenpenalty=10000

\newcommand{\logo}{C:/Users/usuario/Documents/U/logo-eci.jpg}

\newcommand{\q}[1]{``#1''}
\newcommand{\und}[2]{#1\textbf{\_}#2}
\newcommand{\boolun}[2][]{H_{#2}(#1)}
\newcommand{\boolbin}[3]{H_{#1}(#2, #3)}
\newcommand{\val}[2]{\mathbf{#1}[#2]}

\setlength{\parindent}{0pt}

%%%%%%%%%%%%%%%%%%%%%%%%%%%%%%%%%%
%%%%%%%%%%%%%%%%%%%%%%%%%%%%%%%%%%
\newcommand{\titlename}{Tarea 06}%
%%%%%%%%%%%%%%%%%%%%%%%%%%%%%%%%%%
%%%%%%%%%%%%%%%%%%%%%%%%%%%%%%%%%%


\newlength{\unit}
\setlength{\unit}{1ex}

\newcommand\drawCodeBox[3][1.5]{%
\begin{tikzpicture}[remember picture,overlay]
  \coordinate (start) at ([yshift = 2\unit]pic cs:#2);
  \coordinate (end) at ([yshift = -#1\unit]pic cs:#3);
  \node[inner sep=2pt,draw=black,fit=(start) (end)] {};
\end{tikzpicture}
}

\newcolumntype{L}{>{$}l<{$}} 


\newlength{\logicv}
\setlength{\logicv}{.1cm}
\newenvironment{logicenv}[2][0]{
  %begin
  \begin{tcolorbox}[demo, title = #2]
  \vspace*{#1\logicv}
}{
  %end
  \end{tcolorbox}
  \vspace*{-.5cm}
}

\newenvironment{subproofill}[1][0]{
  %begin
  \begin{tcolorbox}[demo, title = ]
    \vspace*{-#1\logicv}
}{
  %end
  \end{tcolorbox}
  \vspace*{-.5cm}
}


\newenvironment{subproof}[2][0]{
  %begin
  \begin{tcolorbox}[demo, title = #2, colframe = black]
  \vspace*{#1\logicv}
  \begin{logic}
}{
  %end
  \end{logic}
  \end{tcolorbox}
}
\newenvironment{logic}{
    \setlength{\extrarowheight}{3pt}
    \setcounter{row}{-1}
    \begin{center}
    \begin{NiceTabular}{>{\stepcounter{row}\therow.\hspace*{5pt}} L r }
}{
    \end{NiceTabular}
    \end{center}
}

\tcbset{demo/.style={
    enhanced, title=title,
    attach boxed title to top
    left = {xshift = .5pt, yshift = -\tcboxedtitleheight -.5pt},
    top = 1pt,
    boxrule = .65pt,
    coltitle = black,
    colback = defini,
    colframe = defini,
    arc = 0mm,
    outer arc = 0mm,
    boxed title style={
        colback = white,
        colframe = deftitlmarg,
        boxrule = .5pt,
        arc = 0mm,
        outer arc = 0mm
    }
}}

% Colores
\definecolor{rojoEci}{RGB}{225, 70, 49}
\definecolor{defini}{HTML}{ede9e6}
\definecolor{deftitlmarg}{HTML}{cfcfcf}

% Documento
\begin{document}



\newcounter{row}

\begin{titlepage}
    \begin{center}
        \vspace*{1cm}

        \textbf{\Huge{\titlename}}

        \vspace{1.5cm}

        \textbf{\large{David Gómez}}

        \vspace{4cm}

        \includegraphics[width=\textwidth]{\logo}

        \vspace{5cm}

        Matemáticas\linebreak
        Escuela Colombiana de Ingeniería Julio Garavito\linebreak
        Colombia\linebreak
        \today

    \end{center}
\end{titlepage}
\clearpage
\tableofcontents
\clearpage

\section{Sección 3.1}
\subsection{Punto 2}
$\mathbf{F} = \{p \mapsto (p \equiv q), q \mapsto (r \to s), r \mapsto \textrm{\textit{false}}\}$
\subsubsection{b) $(p \equiv q)$}

\begin{logicenv}{punto b)}
    $$((p \equiv q) \equiv (r \to s))$$
\end{logicenv}

\subsubsection{punto d)}

\begin{logicenv}{punto d)}
    $$(((p \equiv q) \land (r \to s)) \lor ((\neg(p \equiv q)) \land (\neg(r \to s))))$$
\end{logicenv}

\subsubsection{punto f)}

\begin{logicenv}{punto f)}
    $$(((p \equiv q) \lor \textrm{\textit{false}}) \gets ((p \equiv q) \land (r \to s)))$$
\end{logicenv}

\subsubsection{punto g)}

\begin{logicenv}{punto g)}
    $$(\neg((\textrm{\textit{false}} \land (\textrm{\textit{false}} \gets ((p \equiv q) \lor s))) \equiv (\neg(((p \equiv q) \to (r \to s)) \lor (\textrm{\textit{false}} \land (\neg \textrm{\textit{false}}))))))$$
\end{logicenv}

\subsection{Punto 3}

\subsubsection{a)}

\begin{logicenv}{punto a)}
    $$\mathbf{F} = \{p \mapsto \textrm{\textit{true}}\}$$
    $$\vDash \textrm{\textit{true}}$$
\end{logicenv}

\subsubsection{punto c)}

\begin{logicenv}{punto c)}
    \begin{alignat*}{1}
        \mathbf{F} = \{r \mapsto p\}\\
        ((p \land (\neg q)) \to p)\\
        (((\neg p) \lor q) \lor p) \tag*{demostrado en clase}\\
        \vDash (p \lor (\neg p)) \tag*{demostrado en clase}\\
        \vDash (\textrm{\textit{true}} \lor q) \tag*{MT 2.23 ($\lor$)}\\
        \vDash ((p \land (\neg q)) \to p)
    \end{alignat*}
\end{logicenv}

\subsubsection{punto e)}

\begin{logicenv}{punto e)}
    $$\mathbf{F} = \{q \mapsto p\}$$
    $$ \vDash ((p \to (p \to p)))$$
\end{logicenv}

\subsubsection{punto g)}

\begin{logicenv}{punto g)}
    $$\mathbf{F} = \{p \mapsto \textrm{\textit{true}}, r \mapsto \textrm{\textit{true}}, q \mapsto \textrm{\textit{true}}\}$$
    $$\vDash (\neg((\textrm{\textit{true}} \land (\textrm{\textit{true}} \gets (\textrm{\textit{true}} \lor s))) \equiv (\neg ((\textrm{\textit{true}} \to \textrm{\textit{true}}) \lor (\textrm{\textit{true}} \land (\neg \textrm{\textit{true}}))))))$$
\end{logicenv}

\subsection{punto 5}

\subsubsection{punto a)}

\begin{logicenv}{punto a)}
    \begin{alignat*}{2}
        \phi &= p\\
        \psi &= q\\
        \tau &= r\\
        \phi[q := \tau][p := \psi] &= q\\
        \phi[p := \psi][q := \tau] &= r
    \end{alignat*}
\end{logicenv}

\subsubsection{punto b)}

\begin{logicenv}{punto b)}
    \begin{alignat*}{2}
        \phi &= (p \land q)\\
        \psi &= q\\
        \tau &= s\\
        \phi[p, q := \psi, \tau] &= (q \land s)\\
        \phi[p := \psi][q := \tau] &= (s \land s)
    \end{alignat*}
\end{logicenv}

\section{Sección 3.2}

\subsection{Punto 1}

\begin{logicenv}{punto 1}
    \begin{alignat*}{1}
        \phi &= \textrm{\textit{true}}\\
        \psi &= (((p \equiv q) \land (r \to s)) \lor ((\neg(p \equiv q)) \land (\neg(r \to s))))\\
        \phi[p := \psi] &= \phi
    \end{alignat*}
\end{logicenv}
\end{document}