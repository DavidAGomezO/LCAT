\documentclass{article}

% Librerías:
\usepackage{amsmath, mathtools, amssymb, mathrsfs, amsthm, nicematrix, array}
\usepackage{lmodern, graphicx, fancyhdr}
\usepackage[margin = 2cm, top = 2.5cm, includefoot]{geometry}
\usepackage{xcolor}
\usepackage[hidelinks]{hyperref}
\usepackage[T1]{fontenc}
\usepackage[spanish]{babel}
\usepackage{listings}
\usepackage{tikz}
\usetikzlibrary{shapes, fit, tikzmark}
\usepackage{color, colortbl}
\usepackage[most]{tcolorbox}

% Configuraciones:
\pagestyle{fancy}
\fancyhf{}
\setlength{\headheight}{55.34027pt}
\rhead{\textit{David Gómez}}
\lhead{\includegraphics[width = 4cm]{\logo}}
\lfoot{Página \thepage}
\rfoot{\titlename}
\renewcommand{\headrule}{\hbox to \headwidth{\color{rojoEci}\leaders\hrule height \headrulewidth\hfill}}
\renewcommand{\footrulewidth}{0.4pt}

\hyphenpenalty=10000

\newcommand{\logo}{C:/Users/usuario/Documents/U/logo-eci.jpg}

\newcommand{\q}[1]{``#1''}
\newcommand{\und}[2]{#1\textbf{\_}#2}
\newcommand{\boolun}[2][]{H_{#2}(#1)}
\newcommand{\boolbin}[3]{H_{#1}(#2, #3)}
\newcommand{\val}[2]{\mathbf{#1}[#2]}

\setlength{\parindent}{0pt}

%%%%%%%%%%%%%%%%%%%%%%%%%%%%%%%%%%
%%%%%%%%%%%%%%%%%%%%%%%%%%%%%%%%%%
\newcommand{\titlename}{Taller 05}%
%%%%%%%%%%%%%%%%%%%%%%%%%%%%%%%%%%
%%%%%%%%%%%%%%%%%%%%%%%%%%%%%%%%%%


\newlength{\unit}
\setlength{\unit}{1ex}

\newcommand\drawCodeBox[3][1.5]{%
\begin{tikzpicture}[remember picture,overlay]
  \coordinate (start) at ([yshift = 2\unit]pic cs:#2);
  \coordinate (end) at ([yshift = -#1\unit]pic cs:#3);
  \node[inner sep=2pt,draw=black,fit=(start) (end)] {};
\end{tikzpicture}
}

\newcolumntype{L}{>{$}l<{$}} 


\newlength{\logicv}
\setlength{\logicv}{.1cm}
\newenvironment{logicenv}[2][0]{
  %begin
  \begin{tcolorbox}[demo, title = #2]
  \vspace*{#1\logicv}
}{
  %end
  \end{tcolorbox}
  \vspace*{-.5cm}
}

\newenvironment{subproofill}[1][0]{
  %begin
  \begin{tcolorbox}[demo, title = ]
    \vspace*{-#1\logicv}
}{
  %end
  \end{tcolorbox}
  \vspace*{-.5cm}
}


\newenvironment{subproof}[2][0]{
  %begin
  \begin{tcolorbox}[demo, title = #2, colframe = black]
  \vspace*{#1\logicv}
  \begin{logic}
}{
  %end
  \end{logic}
  \end{tcolorbox}
}
\newenvironment{logic}{
    \setlength{\extrarowheight}{3pt}
    \setcounter{row}{-1}
    \begin{center}
    \begin{NiceTabular}{>{\stepcounter{row}\therow.\hspace*{5pt}} L r }
}{
    \end{NiceTabular}
    \end{center}
}

\tcbset{demo/.style={
    enhanced, title=title,
    attach boxed title to top
    left = {xshift = .5pt, yshift = -\tcboxedtitleheight -.5pt},
    top = 1pt,
    boxrule = .65pt,
    coltitle = black,
    colback = defini,
    colframe = defini,
    arc = 0mm,
    outer arc = 0mm,
    boxed title style={
        colback = white,
        colframe = deftitlmarg,
        boxrule = .5pt,
        arc = 0mm,
        outer arc = 0mm
    }
}}

% Colores
\definecolor{rojoEci}{RGB}{225, 70, 49}
\definecolor{defini}{HTML}{ede9e6}
\definecolor{deftitlmarg}{HTML}{cfcfcf}

% Documento
\begin{document}



\newcounter{row}

\begin{titlepage}
    \begin{center}
        \vspace*{1cm}

        \textbf{\Huge{\titlename}}

        \vspace{1.5cm}

        \textbf{\large{David Gómez}}

        \vspace{4cm}

        \includegraphics[width=\textwidth]{\logo}

        \vspace{5cm}

        Matemáticas\linebreak
        Escuela Colombiana de Ingeniería Julio Garavito\linebreak
        Colombia\linebreak
        \today

    \end{center}
\end{titlepage}
\clearpage
\tableofcontents
\clearpage

\section{Punto 1}

\begin{logicenv}[5]{Punto 1{ , ($A$ , $B$)}}
    $A$ dice: \q{nosotros tenemos la misma naturaleza}\\
    $\Gamma_0 = \{(a \equiv (a \equiv b)), a\}$\\
    $\Gamma_1 = \{(a \equiv (a \equiv b)), (\neg a)\}$\\
    \makebox[8cm]{\hrulefill}\\
    No es posible determinar la naturaleza de $A$ y $B$
\end{logicenv}
% Sub(1) prueba 1
\begin{subproofill}
  \begin{subproof}{Con $\Gamma_0$}
    (\exists \mathbf{v}\, \vert\, \mathbf{v} \text{ satisface } \Gamma_0)\\
    \val{v}{(a \equiv (a \equiv b))} = \mathtt{T} & Def.(p0)\\
    \val{v}{a} = \mathtt{T} & Def.(p0)\\
    \val{v}{(a \equiv b)} = \mathtt{T} & MT 2.23($\equiv$)(p1, p2)\\
    \val{v}{b} = \mathtt{T} & MTT 2.23 ($\equiv$)(p3, p2)
  \end{subproof}
\end{subproofill}

% Sub(1) prueba 2
\begin{subproofill}
  \begin{subproof}{Con $\Gamma_1$}
    (\exists \mathbf{v}\, \vert\, \mathbf{v} \text{ satisface } \Gamma_1)\\
    \val{v}{(a \equiv (a \equiv b))} = \mathtt{T} & Def.(p0)\\
    \val{v}{(\neg a)} = \mathtt{T} & Def.(p0)\\
    \val{v}{(a \equiv b)} = \mathtt{F} & MT 2.23($\equiv$)(p2, p1)\\
    \val{v}{a} \not= \val{v}{b} & MT 2.23($\equiv$)(p3)\\
    \val{v}{b} = \mathtt{T} & (p4, p2)
  \end{subproof}
\end{subproofill}

\section{Punto 2}

\begin{logicenv}[5]{Punto 2 { , ($A$ , $B$)}}
  $A$ dice: \q{al menos uno de nosotros es caballero}\\
  $\Gamma_0 = \{(a \equiv (a \lor b)), a\}$\\
  $\Gamma_1 = \{(a \equiv (a \lor b)), (\neg a)\}$\\
  \makebox[7.5cm]{\hrulefill}\\
  No se puede determinar la naturaleza de $A$ y $B$
\end{logicenv}
\begin{subproofill}
  \begin{subproof}{Con $\Gamma_0$}
    (\exists \mathbf{v}\, \vert\, \mathbf{v} \text{ satisface } \Gamma_0)\\
    \val{v}{(a \equiv (a \lor b))} = \mathtt{T} & Def.(p0)\\
    \val{v}{a} = \mathtt{T} & Def.(p0)\\
  \end{subproof}
\end{subproofill}
\begin{subproofill}
  \begin{subproof}{Con $\Gamma_1$}
    (\exists \mathbf{v}\, \vert\, \mathbf{v} \text{ satisface } \Gamma_0)\\
    \val{v}{(a \equiv (a \lor b))} = \mathtt{T} & Def.(p0)\\
    \val{v}{(\neg a)} = \mathtt{T} & Def.(p0)\\
    \val{v}{a}  = \mathtt{F} \text{ y } \val{v}{b} = \mathtt{F}
  \end{subproof}
\end{subproofill}

\section{Punto 3}

\begin{logicenv}[5]{punto 3{, ($A$ , $B$ , $C$)}}
  $A$ dice: \q{$B$ es escudero}\\
  $B$ dice: \q{$A$ y $C$ son del mismo tipo}\\
  $\Gamma_0 = \{(a \equiv (\neg b)), (b \equiv (a \equiv c)), a\}$\\
  $\Gamma_1 = \{(a \equiv (\neg b)), (b \equiv (a \equiv c)), (\neg a)\}$\\
  \makebox[7cm]{\hrulefill}\\
  $C$ es escudero
\end{logicenv}
\begin{subproofill}
  \begin{subproof}{Con $\Gamma_0$}
    (\exists \mathbf{v}, \vert\, \mathbf{v} \text{ satisface } \Gamma_0)\\
    \val{v}{(a \equiv (\neg b))} = \mathtt{T} & Def.(p0)\\
    \val{v}{(b \equiv (a \equiv c))} = \mathtt{T} & Def.(p0)\\
    \val{v}{a} = \mathtt{T} & Def.(p0)\\
    \val{v}{a} = \val{v}{(\neg b)} & MT 2.23($\equiv$)(p2)\\
    \val{v}{b} = \val{v}{(a \equiv c)} & MT 2.23($\equiv$)(p3)\\
    \val{v}{b} = \mathtt{F} & MT 2.23($\equiv$, $\neg$)(p4, p3)\\
    \val{v}{a} \not= \val{v}{c} & MT 2.23($\equiv$)(p6, p5)\\
    \val{v}{c} = \mathtt{F} & (p7, p3)
  \end{subproof}
\end{subproofill}
\begin{subproofill}
  \begin{subproof}{Con $\Gamma_1$}
    (\exists \mathbf{v}, \vert\, \mathbf{v} \text{ satisface } \Gamma_1)\\
    \val{v}{(a \equiv (\neg b))} = \mathtt{T} & Def.(p0)\\
    \val{v}{(b \equiv (a \equiv c))} = \mathtt{T} & Def.(p0)\\
    \val{v}{(\neg a)} = \mathtt{T} & Def.(p0)\\
    \val{v}{b} = \mathtt{T} & MT 2.23($\equiv$, $\neg$)(p3, p1)\\
    \val{v}{a} = \val{v}{c} & MT 2.23($\equiv$)(p4, p2)\\
    \val{v}{c} = \mathtt{F} & MT 2.23($\neg$)(p5, p3)
  \end{subproof}
\end{subproofill}

\section{Punto 4}

\begin{logicenv}[5]{Punto 4{, ($A$, $B$, $C$)}}
  $A$ dice: \q{$B$ y $C$ son de la misma naturaleza}\\
  $\Gamma_0 = \{(a \equiv (b \equiv c)), a, b\}$\\
  $\Gamma_1 = \{(a \equiv (b \equiv c)), a, (\neg b)\}$\\
  $\Gamma_2 = \{(a \equiv (b \equiv c)), (\neg a), b\}$\\
  $\Gamma_0 = \{(a \equiv (b \equiv c)), (\neg a), (\neg b)\}$\\
  \makebox[7cm]{\hrulefill}\\
  $C$ responderá \q{sí}
\end{logicenv}
\begin{subproofill}
  \begin{subproof}{Con $\Gamma_0$}
    (\exists \mathbf{v} \, \vert\, \mathbf{v} \text{ satisface } \Gamma_0)\\
    \val{v}{(a \equiv (b \equiv c))} = \mathtt{T} & Def.(p0)\\
    \val{v}{a} = \mathtt{T} & Def.(p0)\\
    \val{v}{b} = \mathtt{T} & Def.(p0)\\
    \val{v}{c} = \mathtt{T} & MT 2.23($\equiv$)(p3, p2, p1)\\
    \text{$C$ respondería sí}
  \end{subproof}
\end{subproofill}

\begin{subproofill}
  \begin{subproof}{Con $\Gamma_1$}
    (\exists \mathbf{v} \, \vert\, \mathbf{v} \text{ satisface } \Gamma_1)\\
    \val{v}{(a \equiv (b \equiv c))} = \mathtt{T} & Def.(p0)\\
    \val{v}{a} = \mathtt{T} & Def.(p0)\\
    \val{v}{(\neg b)} = \mathtt{T} & Def.(p0)\\
    \val{v}{c} = \mathtt{F} & MT 2.23($\equiv$, $\neg$)(p3, p2, p1)\\
    \text{$C$ respondería sí}
  \end{subproof}
\end{subproofill}

\begin{subproofill}
  \begin{subproof}{Con $\Gamma_2$}
    (\exists \mathbf{v} \, \vert\, \mathbf{v} \text{ satisface } \Gamma_0)\\
    \val{v}{(a \equiv (b \equiv c))} = \mathtt{T} & Def.(p0)\\
    \val{v}{(\neg a)} = \mathtt{T} & Def.(p0)\\
    \val{v}{b} = \mathtt{T} & Def.(p0)\\
    \val{v}{c} = \mathtt{F} & MT 2.23($\equiv$, $\neg$)(p3, p2, p1)\\
    \text{$C$ respondería sí}
  \end{subproof}
\end{subproofill}

\begin{subproofill}
  \begin{subproof}{Con $\Gamma_3$}
    (\exists \mathbf{v} \, \vert\, \mathbf{v} \text{ satisface } \Gamma_0)\\
    \val{v}{(a \equiv (b \equiv c))} = \mathtt{T} & Def.(p0)\\
    \val{v}{(\neg a)} = \mathtt{T} & Def.(p0)\\
    \val{v}{(\neg b)} = \mathtt{T} & Def.(p0)\\
    \val{v}{c} = \mathtt{T} & MT 2.23($\equiv$, $\neg$)(p3, p2, p1)\\
    \text{$C$ respondería sí}
  \end{subproof}
\end{subproofill}

\section{Punto 5}

El habitante $A$ dice \q{Yo dije que si no soy caballero entonces soy escudero, y si soy caballero entonces no soy escudero} 
\end{document}