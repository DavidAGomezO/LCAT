\documentclass{article}

% Librerías:
\usepackage{amsmath, mathtools, amssymb, mathrsfs, amsthm, nicematrix, array, logicDG}
\usepackage{lmodern, graphicx, fancyhdr}
\usepackage[margin = 1cm, top = 2.5cm, bottom = 2.5cm, includefoot]{geometry}
\usepackage{xcolor}
\usepackage[hidelinks]{hyperref}
\usepackage[T1]{fontenc}
\usepackage[spanish]{babel}
\usepackage{listings}
\usepackage{tikz}
\usetikzlibrary{shapes, fit, tikzmark}

% Configuraciones:
\pagestyle{fancy}
\fancyhf{}
\setlength{\headheight}{72pt}
\rhead{\textit{\author}}
\lhead{\includegraphics[width = 4cm]{\logo}}
\lfoot{Página \thepage}
\rfoot{\titlename}
\renewcommand{\headrule}{\hbox to \headwidth{\color{rojoEci}\leaders\hrule height \headrulewidth\hfill}}
\renewcommand{\footrulewidth}{0.4pt}

\hyphenpenalty=10000

\newcommand{\logo}{C:/Users/usuario/Documents/U/logo-eci.jpg}

\newcommand{\q}[1]{``#1''}
\newcommand{\und}[2]{#1\textbf{\_}#2}
\newcommand{\boolun}[2][]{H_{#2}(#1)}
\newcommand{\boolbin}[3]{H_{#1}(#2, #3)}
\newcommand{\val}[2]{\mathbf{#1}[#2]}

\setlength{\parindent}{0pt}

%%%%%%%%%%%%%%%%%%%%%%%%%%%%%%%%%%
%%%%%%%%%%%%%%%%%%%%%%%%%%%%%%%%%%
\newcommand{\titlename}{Taller 09}%
\renewcommand{\author}{David Gómez}%
%%%%%%%%%%%%%%%%%%%%%%%%%%%%%%%%%%
%%%%%%%%%%%%%%%%%%%%%%%%%%%%%%%%%%


% Colores
\definecolor{rojoEci}{RGB}{225, 70, 49}
\definecolor{defini}{HTML}{ede9e6}
\definecolor{deftitlmarg}{HTML}{cfcfcf}

% Documento
\begin{document}

\begin{titlepage}
    \begin{center}
        \vspace*{1cm}

        \textbf{\Huge{\titlename}}

        \vspace{1.5cm}

        \textbf{\large{\author}
}
        \vspace{4cm}

        \includegraphics[width=.8\textwidth]{\logo}

        \vspace{4cm}

        Matemáticas\linebreak
        Escuela Colombiana de Ingeniería Julio Garavito\linebreak
        Colombia\linebreak
        \today

    \end{center}
\end{titlepage}
\clearpage
\tableofcontents
\clearpage

\section{Punto 1}

\subsection{Teo 4.29.3}

\begin{logicenv}{Teo 4.29.3}
    \begin{derivation}
            (\textrm{\textit{true}} \to \phi)\\
        Teo 4.28.1\\
            ((\neg \textrm{\textit{true}}) \lor \phi)\\
        Teo 4.15.2\\
            (\textit{false} \lor \phi)\\
        Identidad($\lor$)\\
            \phi
    \end{derivation}
    Por MT 4.21 se demuestra que\\
    $\vdash_{\text{DS}} ((\textrm{\textit{true}} \to \phi) \equiv \phi)$
\end{logicenv}

\subsection{Teo 4.30.2}
\begin{logicenv}{Teo 4.30.2}
    \begin{derivation}
            (\phi \to (\psi \lor \tau))\\
        Teo 4.28.1\\
            ((\neg \phi) \lor (\psi \lor \tau))\\
        Idempotencia($\lor$), Leibniz($\phi = (p lor (\psi \lor \tau))$)\\
            (((\neg \phi) \lor (\neg \phi)) \lor (\psi \lor \tau))\\
        Asociativa($\lor$)\\
            ((\neg \phi) \lor ((\neg \phi) \lor (\psi \lor \tau)))\\
        Asociativa($\lor$), Conmutativa($\lor$), Leibniz($\phi = ((\neg \phi) \lor p)$)\\
            ((\neg \phi) \lor (\tau \lor ((\neg \phi) \lor \psi)))\\
        Asociativa($\lor$), Conmutativa($\lor$)\\
            (((\neg \phi) \lor \psi) \lor ((\neg \phi) \lor \tau))\\
        Teo 4.28.1\\
            ((\phi \to \psi) \lor (\phi \lor \tau))
    \end{derivation}
    Por MT 4.21 se demuestra que\\
    $\vdash_{\text{DS}} ((\phi \to (\psi \lor \tau)) \equiv ((\phi \to \psi) \lor (\phi \lor \tau)))$
\end{logicenv}

\subsection{Teo 4.31.2}
\begin{logicenv}{Teo 4.31.2}
    \begin{derivation}
            ((\neg (\phi \to \psi)))\\
        Teo 4.28.1, Leibniz($\phi = (\neg p)$)\\
            ((\neg ((\neg \phi) \lor \psi)))\\
        Dist.($\neg$, $\lor$)\\
            ((\neg (\neg \phi)) \land (\neg \psi))\\
        Teo 4.15.6, Leibniz($\phi = (p \land (\neg \psi))$)\\
            (\phi \land (\neg \psi))
    \end{derivation}
    Por MT 4.21 se demuestra que\\
    $\vdash_{\text{DS}} (((\neg (\phi \to \psi))) \equiv (\phi \land (\neg \psi)))$
\end{logicenv}

\subsection{Teo 4.33.3}
\begin{logicenv}[5]{Teo 4.33.3}
    \begin{align*}
            \res{((\phi \to \psi) \land (\psi \to \phi))}\\
        \why{{Def.($\to$), Teo 4.28.2}}\\
            \res{(((\phi \lor \psi) \equiv \psi) \land ((\psi \land \phi) \equiv \psi))}\\
        \why[\Rightarrow]{Transitividad($\equiv$)}\\
            \res{((\phi \lor \psi) \equiv (\psi \land \phi))}\\
        \why{Def($\land$)}\\
            \res{((\phi \lor \psi) \equiv (\psi \equiv (\phi \equiv (\psi \lor \phi))))}\\
        \why{Asociativa($\equiv$)}\\
            \res{((\phi \lor \psi) \equiv ((\psi \equiv \phi) \equiv (\psi \lor \phi)))}\\
        \why{Conmutativa($\equiv$, Asociativa($\equiv$), Conmutativa($\equiv$))}\\
            \res{(((\phi \lor \psi) \equiv (\phi \lor \psi)) \equiv (\phi \equiv \psi))}\\
        \why{Teo 4.6.2, Conmutativa($\equiv$), Identidad($\equiv$)}\\
            \res{(\phi \equiv \psi)}
    \end{align*}
    Por MT 5.5.1 se demuestra que\\
    $\vDash_\text{DS} (((\phi \to \psi) \land (\psi \to \phi)) \to (\phi \equiv \psi))$
\end{logicenv}

\section{Punto 2}

\subsection{Teo 4.15.5}
\begin{logicenv}{Teo 4.15.5}
    \begin{derivation}
            (((\neg \phi) \equiv \psi) \equiv (\phi \equiv (\neg \psi)))\\
        Teo 4.14.4, Leibniz($\phi = (p \equiv (\phi \equiv (\neg \psi)))$)\\
            ((\neg (\phi \equiv \psi)) \equiv (\phi \equiv (\neg \psi)))\\
        Conmutativa($\equiv$), Leibniz($\phi = ((\neg p) \equiv (\phi \equiv (\neg \psi)))$)\\
            ((\neg (\psi \equiv \phi)) \equiv (\phi \equiv (\neg \psi)))\\
        Teo 4.14.4, Leibniz($\phi = (p \equiv (\phi \equiv (\neg \psi)))$)\\
            (((\neg \psi) \equiv \phi) \equiv (\phi \equiv (\neg \psi)))\\
        Conmutativa($\equiv$), Leibniz($\phi = (p \equiv (\phi \equiv (\neg \psi)))$)\\
            ((\phi \equiv (\neg \psi)) \equiv (\phi \equiv (\neg \psi)))\\
        Teo 4.6.2\\
            \textrm{\textit{true}}
    \end{derivation}
    Por MT 4.21, y Identidad($\equiv$) se demuestra que\\
    $\vdash_{\text{DS}} (((\neg \phi) \equiv \psi) \equiv (\phi \equiv (\neg \psi)))$
\end{logicenv}

\subsection{Teo 4.16.1}
\begin{logicenv}{Teo 4.16.1}
    \begin{derivation}
            ((\phi \not\equiv (\psi \not\equiv \tau)) \equiv ((\phi \not\equiv \psi) \not\equiv \tau))\\
        Def($\not\equiv$)\\
            (((\neg \phi) \equiv ((\neg \psi) \equiv \tau)) \equiv ((\neg ((\neg \phi) \equiv \psi)) \equiv \tau))\\
        Teo 4.15.4\\
            ((\neg (\phi \equiv (\neg (\psi \equiv \tau)))) \equiv ((\neg (\neg (\phi \equiv \psi))) \equiv \tau))\\
        Teo 4.15.4\\
            ((\neg (\phi \equiv (\neg (\psi \equiv \tau)))) \equiv ((\neg (\neg ((\phi \equiv \psi))) \equiv \tau)))\\
        Teo 4.15.6\\
            ((\neg (\phi \equiv (\neg (\psi \equiv \tau)))) \equiv ((\phi \equiv \psi) \equiv \tau))\\
        Teo 4.15.5, Teo 4.15.4\\
            ((\neg (\neg (\phi \equiv (\psi \equiv \tau)))) \equiv ((\phi \equiv \psi) \equiv \tau))\\
        Teo 4.14.6\\
            ((\phi \equiv (\psi \equiv \tau)) \equiv ((\phi \equiv \psi) \equiv \tau))\\
        Asociativa($\equiv$)\\
            (((\phi \equiv \psi) \equiv \tau) \equiv ((\phi \equiv \psi) \equiv \tau))\\
        Teo 4.6.2\\
        \textrm{\textit{true}}
    \end{derivation}
    Por MT 4.21 y Identidad($\equiv$) se demuestra que\\
    $\vdash_{\text{DS}} ((\phi \not\equiv (\psi \not\equiv \tau)) \equiv ((\phi \not\equiv \psi) \not\equiv \tau))$
\end{logicenv}

\subsection{Teo 4.24.1}
\begin{logicenv}{Teo 4.24.1}
    \begin{derivation}
            ((\phi \land (\psi \land \tau)) \equiv ((\phi \land \psi) \land \tau))\\
        Def.($\land$)\\
            ((\phi \equiv ()))
    \end{derivation}
\end{logicenv}
\end{document}