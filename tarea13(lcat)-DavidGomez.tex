\documentclass[twoside]{article}

% Packages
\usepackage{mathtools}
\usepackage{amssymb}
\usepackage{logicDG}
\usepackage{array}
\usepackage{xcolor}
\usepackage[spanish]{babel}
\usepackage{geometry}
\usepackage{fancyhdr}
\usepackage{graphicx}
\usepackage[hidelinks]{hyperref}

% Font
\usepackage{lmodern}
\usepackage[T1]{fontenc}

% Page configurations

\geometry{
    a4paper,
    margin = 2.5cm,
    top = 4cm,
    bottom = 2.5cm,
    headheight = 72pt
}
\newcommand{\logo}{C:/Users/usuario/Documents/U/logo-eci.jpg}
\renewcommand{\author}{David Gómez}
\renewcommand{\title}{Tarea 13}

\pagestyle{fancy}
\fancyhf{}
\fancyhead[LO]{\author}
\fancyhead[LE]{\title}
\fancyhead[R]{\includegraphics[width = 4cm]{\logo}}
\fancyfoot[C]{Página \thepage}
\renewcommand{\headrule}{\hbox to \headwidth{\color{rojoEci}\leaders\hrule height \headrulewidth\hfill}}

\hyphenpenalty = 10000

\setlength{\parindent}{0pt}

\newcommand{\sect}[1]{\section*{#1} \addcontentsline{toc}{section}{#1}}
\newcommand{\subsect}[1]{\subsection*{#1} \addcontentsline{toc}{subsection}{#1}}
\newcommand{\subsubsect}[1]{\subsubsection*{#1} \addcontentsline{toc}{subsubsection}{#1}}

% Colors
\definecolor{rojoEci}{RGB}{225, 70, 49}

% Documento

\begin{document}
\begin{titlepage}
    \begin{center}
        \vspace*{1cm}
 
        \textbf{\fontsize{45}{\baselineskip}\selectfont{\title}}

        \vspace{4cm}

        {\Large Hecho por}

        \vspace{1cm}

        {\textbf{\LARGE\MakeUppercase{\author}}}

        \vspace{2cm}

        \includegraphics[width = .8\textwidth]{\logo}

        \vspace{2cm}

        {\Large Estudiante de Matemáticas\\[5pt]

        Escuela Colombiana de Ingeniería Julio Garavito\\[5pt]

        Colombia\\[5pt]

        \today}
             
    \end{center}
\end{titlepage}

\tableofcontents
\clearpage

\sect{Sección 7.1}
% Sección 7.1: 1,2,4,5,6,7

\subsect{Punto 1}
\[\mathbf{F} = \{p_0 \mapsto p_1, p_2 \mapsto \text{\textit{true}}, p_3 \mapsto H(x), p_4 \mapsto p_4\}\]

\subsubsect{a}
\begin{logicenv}{$\phi = p_0$}
    \[\mathbf{F}[p_0] = p_1\]
\end{logicenv}

\subsubsect{b}
\begin{logicenv}{$\phi = H(y) = \forall x H(x) \land \textit{false}$}
    \[\mathbf{F}[H(y) = \forall x H(x) \land \textit{false}] = H(y) = \forall x H(x) \land \textit{false}\]
\end{logicenv}

\subsubsect{c}
\begin{logicenv}{$\phi = \forall x \forall y (H(f(x, y)) \lor p_3)$}
    \[\mathbf{F}[\forall x \forall y (H(f(x, y)) \lor p_3)] = \forall x \forall y (H(f(x, y)) \lor H(x))\]
\end{logicenv}


\subsect{Punto 2}
\subsubsect{a}
Es libre
\subsubsect{b}
Es libre
\subsubsect{c}
Es libre

\subsect{Punto 3}
\subsubsect{a}
\begin{logicenv}{Demostración de $\phi$}
    \begin{itemize}
        \item[(i)] Si no hay cuantificadores que afecten globalmente en $\phi$, la definición de demostración es exactamente la misma que en DS.
        \item[(ii)] Si $\phi$ es de la forma $\forall x \psi$ , entonces una demostración es una secuencia no vacía de proposiciones tales que el último elemento de la secuencia es $\psi$ y los anteriores son axiomas o deducciones de pasos anteriores mediante reglas de inferencia.
        \item[(iii)] Si $\phi$ es de la forma $\exists x \psi$, entonces una demostración es una secuencia no vacía de proposiciones tales que el último elemento de la secuencia es $\psi$ y los anteriores son axiomas, deducciones de pasos anteriores mediante reglas de inferencia o suposiciones. 
    \end{itemize}
\end{logicenv}

\subsubsect{b}
\begin{logicenv}{Derivación}
    \begin{itemize}
        \item[(i)] Si no hay cuantificadores que afecten globalmente en $\phi$, la definición de derivación es exactamente la misma que en DS.
        \item[(ii)] Si $\phi$ es de la forma $\forall x (\psi \equiv \tau)$ , entonces una derivación es una secuencia finita de proposiciones, en la cual cada elemento es obtenido mediante transitividades de $\equiv$ del elemento anterior.
        \item[(iii)] Si $\phi$ es de la forma $\exists x (\psi \equiv \tau)$ , entonces una derivación es una secuencia finita de proposiciones en la cual cada elemento anterior es obtenido mediante transitividades de $\equiv$ del elemento anterior y además se tienen en cuenta suposiciones dadas antes de iniciar la derivación como verdades ``locales''.  
    \end{itemize}
\end{logicenv}

\subsubsect{c}
\begin{logicenv}{Derivación de debilitamiento}
    \begin{itemize}
        \item[(i)] Si no hay cuantificadores que afecten globalmente en $\phi$, la definición de derivación de debilitamiento es exactamente la misma que en DS.
        \item[(ii)] Si $\phi$ es de la forma $\forall x (\psi \to \tau)$ ,  entonces una derivación es una secuencia finita de proposiciones, en la cual cada elemento es obtenido mediante transitividades de $\equiv$ y al menos una de $\to$ del elemento anterior.
        \item[(iii)] Si $\phi$ es de la forma $\exists x (\psi \to \tau)$ , entonces una derivación es una secuencia finita de proposiciones en la cual cada elemento anterior es obtenido mediante transitividades de $\equiv$ y al menos una de $\to$ del elemento anterior y además se tienen en cuenta suposiciones dadas antes de iniciar la derivación como verdades ``locales''.
    \end{itemize}
\end{logicenv}


\subsubsect{d}
\begin{logicenv}{Derivación de fortalecimiento}
    \begin{itemize}
        \item[(i)] Si no hay cuantificadores que afecten globalmente en $\phi$, la definición de derivación de fortalecimiento es exactamente la misma que en DS.
        \item[(ii)] Si $\phi$ es de la forma $\forall x (\psi \gets \tau)$ ,  entonces una derivación es una secuencia finita de proposiciones, en la cual cada elemento es obtenido mediante transitividades de $\gets$ del elemento anterior.
        \item[(iii)] Si $\phi$ es de la forma $\exists x (\psi \gets \tau)$ , entonces una derivación es una secuencia finita de proposiciones en la cual cada elemento anterior es obtenido mediante transitividades de $\equiv$ y al menos una de $\gets$ del elemento anterior y además se tienen en cuenta suposiciones dadas antes de iniciar la derivación como verdades ``locales''.
    \end{itemize}
\end{logicenv}

\subsect{Punto 4}
\begin{logicenv}[10]{Si $\theo{}{DS}{\phi}$ entonces $\theo{}{DS($\mathcal{L}$)}{\phi}$}
    Ya que todos los axiomas de DS están contenidos en DS($\mathcal{L}$), usando la definición de demostración, y puesto que no hay cuantificadores en ningún teorema de DS. Se puede demostrar cualquier teorema de DS usando los axiomas de DS($\mathcal{L}$).
\end{logicenv}
\sect{Sección 7.2}
% Sección 7.2: 2,3,4,5,7,9,10,12,14,16

\end{document}