\documentclass{article}

% Librerías:
\usepackage{amsmath, mathtools, amssymb, mathrsfs, amsthm, nicematrix, array, logicDG}
\usepackage{lmodern, graphicx, fancyhdr}
\usepackage[margin = 1cm, top = 2.5cm, bottom = 2.5cm, includefoot]{geometry}
\usepackage{xcolor}
\usepackage[hidelinks]{hyperref}
\usepackage[T1]{fontenc}
\usepackage[spanish]{babel}
\usepackage{listings}
\usepackage{tikz}
\usetikzlibrary{shapes, fit, tikzmark}
\usepackage{color, colortbl}
\usepackage[most]{tcolorbox}

% Configuraciones:
\pagestyle{fancy}
\fancyhf{}
\setlength{\headheight}{72pt}
\rhead{\textit{\author}}
\lhead{\includegraphics[width = 4cm]{\logo}}
\lfoot{Página \thepage}
\rfoot{\titlename}
\renewcommand{\headrule}{\hbox to \headwidth{\color{rojoEci}\leaders\hrule height \headrulewidth\hfill}}
\renewcommand{\footrulewidth}{0.4pt}

\hyphenpenalty=10000

\newcommand{\logo}{C:/Users/usuario/Documents/U/logo-eci.jpg}

\newcommand{\q}[1]{``#1''}
\newcommand{\und}[2]{#1\textbf{\_}#2}
\newcommand{\boolun}[2][]{H_{#2}(#1)}
\newcommand{\boolbin}[3]{H_{#1}(#2, #3)}
\newcommand{\val}[2]{\mathbf{#1}[#2]}

\setlength{\parindent}{0pt}

%%%%%%%%%%%%%%%%%%%%%%%%%%%%%%%%%%
%%%%%%%%%%%%%%%%%%%%%%%%%%%%%%%%%%
\newcommand{\titlename}{Tarea 08}%
\renewcommand{\author}{David Gómez}%
%%%%%%%%%%%%%%%%%%%%%%%%%%%%%%%%%%
%%%%%%%%%%%%%%%%%%%%%%%%%%%%%%%%%%


% Colores
\definecolor{rojoEci}{RGB}{225, 70, 49}
\definecolor{defini}{HTML}{ede9e6}
\definecolor{deftitlmarg}{HTML}{cfcfcf}

% Documento
\begin{document}

\begin{titlepage}
    \begin{center}
        \vspace*{1cm}

        \textbf{\Huge{\titlename}}

        \vspace{1.5cm}

        \textbf{\large{\author}
}
        \vspace{4cm}

        \includegraphics[width=.8\textwidth]{\logo}

        \vspace{4cm}

        Matemáticas\linebreak
        Escuela Colombiana de Ingeniería Julio Garavito\linebreak
        Colombia\linebreak
        \today

    \end{center}
\end{titlepage}
\clearpage
\tableofcontents
\clearpage

\section{Sección 4.3}
\subsection{Punto 5}
\begin{logicenv}[5]{$\vdash_{\text{DS}}(((\neg \phi) \equiv \psi) \equiv (\phi \equiv (\neg \psi)))$}
    \begin{logic}
        (((\neg \phi) \equiv \psi) \equiv ((\neg \phi) \equiv \psi)) & Teo 4.6.3$[\phi := ((\neg \phi) \equiv \psi)]$\\
        (((\neg \phi) \equiv \psi) \equiv ((\phi \equiv \textrm{\textit{false}}) \equiv \psi)) & Ax9, Leibinz ($\phi = (((\neg \phi) \equiv \psi) \equiv (p \equiv \psi))$), Ecuanimidad (p0)\\
        (((\neg \phi) \equiv \psi) \equiv ((\textrm{\textit{false}} \equiv \phi) \equiv \psi)) & Conmutativa($\equiv$), Leibinz ($\phi = (((\neg \phi) \equiv \psi) \equiv (p \equiv \psi))$), Ecuanimidad (p1)\\
        (((\neg \phi) \equiv \psi) \equiv (\textrm{\textit{false}} \equiv (\phi \equiv \psi))) & Asociativa($\equiv$), Leibinz ($\phi = (((\neg \phi) \equiv \psi) \equiv p)$), Ecuanimidad (p2)\\
        (((\neg \phi) \equiv \psi) \equiv ((\phi \equiv \psi) \equiv \textrm{\textit{false}})) & Conmutativa($\equiv$), Leibinz ($\phi = (((\neg \phi)\equiv \psi) \equiv p)$), Ecuanimidad (p3)\\
        (((\neg \phi) \equiv \psi) \equiv (\phi \equiv (\psi \equiv \textrm{\textit{false}}))) & Asociativa($\equiv$), Leibinz ($\phi = (((\neg \phi) \equiv \psi) \equiv p)$), Ecuanimidad (p4)\\
        (((\neg \phi) \equiv \psi) \equiv (\phi \equiv (\neg \psi))) & Def($\neg$), Leibinz ($\phi = (((\neg \phi) \equiv \psi) \equiv (\phi \equiv p))$), Ecuanimidad (p5)
    \end{logic}
\end{logicenv}

\subsection{Punto 6}
\begin{logicenv}[5]{$\vdash_{\text{DS}} ((\phi \equiv (\neg \phi)) \equiv \textrm{\textit{false}})$}
    \begin{logic}
        ((\textrm{\textit{false}} \equiv \textrm{\textit{false}}) \equiv \textrm{\textit{true}}) & Teo 4.6.2$[\phi := \textrm{\textit{false}}]$\\
        ((\phi \equiv \phi) \equiv \textrm{\textit{true}}) & Teo 4.6.2\\
        ((\phi \equiv \phi) \equiv (\textrm{\textit{false}} \equiv \textrm{\textit{false}})) & Transitividad (p1, p0)\\
        (((\phi \equiv \phi) \equiv \textrm{\textit{false}}) \equiv \textrm{\textit{false}}) & Asociativa($\equiv$), Ecuanimidad (p2)\\
        ((\phi \equiv (\phi \equiv \textrm{\textit{false}})) \equiv \textrm{\textit{false}}) & Asociativa($\equiv$), Leibinz ($\phi = (p \equiv \textrm{\textit{false}})$), Ecuanimidad (p3)\\
        ((\phi \equiv (\neg \phi)) \equiv \textrm{\textit{false}}) & Ax9, Leibinz($\phi = ((\phi \equiv p) \equiv \textrm{\textit{false}})$), Ecuanimidad (p4)
    \end{logic}
\end{logicenv}

\subsection{Punto 8}
\begin{logicenv}[5]{$\vdash_{\text{DS}} ((\phi \not\equiv \psi) \equiv (\psi \not\equiv \phi))$}
    \begin{logic}
        ((\phi \not\equiv \psi) \equiv ((\neg \phi) \equiv \psi)) & Ax10\\
        ((\phi \not\equiv \psi) \equiv ((\phi \equiv \textrm{\textit{false}}) \equiv \psi)) & Ax9, Leibinz ($\phi = ((\phi \not\equiv \psi) \equiv (p \not\equiv \psi))$), Ecuanimidad (p0)\\
        ((\phi \not\equiv \psi) \equiv (\phi \equiv (\textrm{\textit{false}} \equiv \psi))) & Asociativa($\equiv$), Leibinz ($\phi = ((\phi \not\equiv \psi) \equiv p)$), Ecuanimidad (p1)\\
        ((\phi \not\equiv \psi) \equiv (\phi \equiv (\psi \equiv \textrm{\textit{false}}))) & Conmutativa($\equiv$), Leibinz  ($\phi = ((\phi \not\equiv \psi) \equiv (\phi \equiv p))$), Ecuanimidad (p2)\\
        ((\phi \not\equiv \psi) \equiv (\phi \equiv (\neg \psi))) & Def($\neg$), Leibinz ($\phi = ((\phi \not\equiv \psi) \equiv (\phi \equiv p))$), Ecuanimidad (p3)\\
        ((\phi \not\equiv \psi) \equiv ((\neg \psi) \equiv \phi)) & Conmutativa($\equiv$), Leibinz ($\phi = ((\phi \not\equiv \psi) \equiv p)$), Ecuanimidad (p4)\\
        ((\phi \not\equiv \psi) \equiv (\psi \not\equiv \phi)) & Def($\not\equiv$), Leibinz ($\phi = ((\phi \not\equiv \psi) \equiv p)$), Ecuanimidad (p5)
    \end{logic}
\end{logicenv}

\subsection{Punto 9}
\begin{logicenv}[5]{$\vdash_{\text{DS}} ((\phi \neg\equiv \textrm{\textit{false}}) \equiv \phi)$}
    \begin{logic}
        ((\phi \not\equiv \textrm{\textit{false}}) \equiv ((\neg \phi) \equiv \textrm{\textit{false}})) & Def($\not\equiv$)\\
        ((\phi \not\equiv \textrm{\textit{false}}) \equiv (\neg(\neg \phi))) & Def($\neg$), Leibinz ($\phi = ((\phi \not\equiv \textrm{\textit{false}}) \equiv p)$), Ecuanimidad (p0)\\
        ((\phi \not\equiv \textrm{\textit{false}}) \equiv \phi) & Teo 4.15.6, Leibinz ($\phi = ((\phi \not\equiv \textrm{\textit{false}}) \equiv p)$), Ecuanimidad (p1)
    \end{logic}
\end{logicenv}

\subsection{Punto 10}
\begin{logicenv}[5]{$\vdash_{\text{DS}} (((\phi \not\equiv \psi) \not\equiv \psi) \equiv \phi)$}
    \begin{logic}
        ((\phi \not\equiv \textrm{\textit{false}}) \equiv \phi) & Teo 4.16.5\\
        ((\phi \not\equiv (\psi \not\equiv \psi)) \equiv \phi) & Teo 4.16.4$[\phi := \psi]$, Leibinz ($\phi = ((\phi \not\equiv \textrm{\textit{false}}) \equiv \phi)$), Ecuanimidad (p0)\\
        (((\phi \not\equiv \psi) \not\equiv \psi) \equiv \phi) & Asociativa($\not\equiv$), Leibinz ($\phi = (p \equiv \phi)$), Ecuanimidad(p1)
    \end{logic}
\end{logicenv}

\section{Sección 4.4}
\subsection{Punto 1}

\begin{logicenv}[5]{$\vdash_{\text{DS}} ((\phi \not\equiv \psi) \equiv (\psi \not\equiv \phi))$}
    \begin{logic}
        ((\phi \lor \textrm{\textit{false}}) \equiv \phi) & Ax6\\
        ((\textrm{\textit{false}} \equiv (\neg \textrm{\textit{false}})) \equiv \textrm{\textit{false}}) & Teo 4.16.4$[\phi := \textrm{\textit{false}}]$\\
        ((\phi \lor (\textrm{\textit{false}} \equiv (\neg \textrm{\textit{false}}))) \equiv \phi) & Leibinz ($\phi = ((\phi \lor p) \equiv \phi)$)(p1), Ecuanimidad (p0)\\
        (((\phi \lor \textrm{\textit{false}}) \equiv (\phi \lor (\neg \textrm{\textit{false}}))) \equiv \phi) & Distributiva($\lor, \equiv$), Leibinz ($\phi = (p \equiv \phi)$), Ecuanimidad (p2)\\
        ((\phi \equiv (\phi \lor (\neg \textrm{\textit{false}}))) \equiv \phi) & Ax6, Leibinz ($\phi = ((p \equiv (\phi \lor (\neg \textrm{\textit{false}}))) \equiv \phi)$), Ecuanimidad (p3)\\
        ((\phi \equiv (\phi \lor \textrm{\textit{true}})) \equiv \phi) & Teo 4.15.2, Leibinz ($\phi = ((\phi \equiv (\phi \lor p)) \equiv \phi)$), Ecuanimidad (p4)\\
        (((\phi \lor \textrm{\textit{true}}) \equiv \phi) \equiv \phi) & Conmutativa($\equiv$), Leibinz($\phi = (p \equiv \phi)$), Ecuanimidad (p5)\\
        ((\phi \equiv \textrm{\textit{true}}) \equiv (\phi \equiv \phi)) & Asociativa($\equiv$), Ecuanimidad (p6)\\
        ((\phi \lor \textrm{\textit{true}}) \equiv \textrm{\textit{true}}) & Teo 4.6.2, Leibinz ($\phi = ((\phi \lor \textrm{\textit{true}}) \equiv p)$), Ecuanimidad (p7)
    \end{logic}
\end{logicenv}

\subsection{Punto 2}
\begin{logicenv}{$\vdash_{\text{DS}} (\phi \lor \textrm{\textit{true}})$}
    \begin{logic}
        ((\phi \lor \textrm{\textit{true}}) \equiv \textrm{\textit{true}}) & Teo 4.19.2\\
        (\phi \lor \textrm{\textit{true}}) & Identidad (p0)
    \end{logic}
\end{logicenv}

\subsection{Punto 3}
\begin{logicenv}{$\vdash_{\text{DS}} ((\phi \lor \psi) \equiv ((\phi \lor (\neg \psi)) \equiv \phi))$}
    \begin{logic}
        ((\phi \lor \psi) \equiv (\phi \lor \psi))\\
        (((\phi \lor \psi) \equiv (\phi \lor \psi)) \equiv \textrm{\textit{true}})\\
        ((\phi \lor \psi) \equiv ((\phi \equiv \psi) \equiv \textrm{\textit{true}}))\\
        ((\phi \lor \psi) \equiv ((\phi \lor \psi) \equiv (\phi \equiv \phi)))\\
        ((\phi \lor \psi) \equiv (((\phi \lor \psi) \equiv \phi) \equiv \phi))\\
        ((\phi \lor \psi) \equiv (((\phi \lor \psi) \equiv (\phi \lor \textrm{\textit{false}})) \equiv \phi))\\
        ((\phi \lor \psi) \equiv ((\phi \lor (\psi \equiv \psi)) \equiv \phi))\\
        ((\phi \lor \psi) \equiv ((\phi \lor (\neg \psi)) \equiv \phi))
    \end{logic}
\end{logicenv}
\pagebreak
\subsection{Punto 7}

\begin{alignat*}{2}
    &((a \in \mathbb{N} \land a^2 = 2k) \Rightarrow a = 2c)
    \intertext{$\equiv$}
    &((a \in \mathbb{N} \land a^2 = 2k \land a = 2b + a) \Rightarrow \textrm{\textit{false}})
    \intertext{$\equiv$}
    &((a \in \mathbb{N} \land a^2 = 2k \land a^2 = 2(b^2 + b) + 1) \Rightarrow \textrm{\textit{false}})
    \intertext{$\equiv$}
    &((a \in \mathbb{N} \land a^2 = 2k \land a^2 = 2d + a) \Rightarrow \textrm{\textit{false}})
    \intertext{$\equiv$}
    &((a \in \mathbb{N} \land \textrm{\textit{false}}) \Rightarrow \textrm{\textit{false}})\\
    \intertext{$\equiv$}
    &(\textrm{\textit{false}} \Rightarrow \textrm{\textit{false}})
    \intertext{$\equiv$}
    &\textrm{\textit{true}}
\end{alignat*}

\subsection{Punto 10}

\begin{logicenv}[5]{Silogismo disyuntivo}
    Al tener una disyunción, y tener que una de sus partes es falsa, se puede concluir que la otra parte de la disyunción debe ser verdadera.

    Demostración:
    \begin{logic}
        (\phi \lor \psi)\\
        (\neg \phi)\\
        (\phi \equiv \textrm{\textit{false}})\\
        (\textrm{\textit{false}} \lor \psi)\\
        \psi
    \end{logic}
\end{logicenv}

\subsection{Punto 11}

\begin{logicenv}{Corte}
    Al tener dos disyunciones las cuales comparten una única variable/ proposición, con signo opuesto, se puede concluir que alguna de las otras partes de las disyunciones debe ser cierta

    Demostración:
    \begin{logic}
        (\phi \lor \psi)
        ((\neg \phi) \lor \tau)\\
        \text{Suposición 1}\\
        \text{Suposición 2}\\
        (\psi \lor \tau)
    \end{logic}
\end{logicenv}
\begin{subproof}{Suposición 2}
    \begin{logic}
        \phi\\
        \tau
    \end{logic}
\end{subproof}
\begin{subproof}{Suposición 2}
    \begin{logic}
        (\neg \phi)\\
        (\phi \equiv \textrm{\textit{false}})\\
        (\textrm{\textit{false}} \lor \psi)\\
        \psi 
    \end{logic}
\end{subproof}

\subsection{Punto 12}

\begin{logicenv}{Debilitamiento}
    Al tener una proposición como verdadera, cualquier disyunción que la tenga será verdadera
    \begin{logic}
        \phi\\
        (\phi \equiv \textrm{\textit{true}})\\
        (\psi \lor \textrm{\textit{true}})
    \end{logic}
\end{logicenv}

\subsection{Punto 13}

\begin{logicenv}[5]{Debilitamiento?}
    Es incorrecta pues, de una disyunción no se puede saber cual de sus partes la hace verdadera, o incluso si son ambas.
\end{logicenv}

\section{Sección 4.5}

\end{document}