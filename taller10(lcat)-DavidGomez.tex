\documentclass[twoside]{article}

% Packages
\usepackage{mathtools}
\usepackage{amssymb}
\usepackage{logicDG}
\usepackage{array}
\usepackage{xcolor}
\usepackage[spanish]{babel}
\usepackage{geometry}
\usepackage{fancyhdr}
\usepackage{graphicx}
\usepackage[hidelinks]{hyperref}

% Font
\usepackage{lmodern}
\usepackage[T1]{fontenc}

% Page configurations

\geometry{
    a4paper,
    margin = 1cm,
    top = 3cm,
    bottom = 2.5cm,
    headheight = 72pt
}
\newcommand{\logo}{C:/Users/usuario/Documents/U/logo-eci.jpg}
\title{Titulo}
\author{Autor}

\pagestyle{fancy}
\fancyhf{}
\fancyhead[LO]{\author}
\fancyhead[LE]{\nouppercase{\leftmark\hfill\rightmark}}
\fancyhead[R]{\includegraphics[width = 4cm]{\logo}}
\fancyfoot[C]{Página \thepage}
\renewcommand{\headrule}{\hbox to \headwidth{\color{rojoEci}\leaders\hrule height \headrulewidth\hfill}}

\hyphenpenalty = 10000

\setlength{\parindent}{0pt}

\newcommand{\sect}[1]{\section*{#1} \addcontentsline{toc}{section}{#1}}
\newcommand{\subsect}[1]{\subsubsection*{#1} \addcontentsline{toc}{subsubsection}{#1}}
\newcommand{\subsubsect}[1]{\subsubsection*{#1} \addcontentsline{toc}{subsubsection}{#1}}

% Colors
\definecolor{rojoEci}{RGB}{225, 70, 49}

% Documento

\begin{document}
\begin{titlepage}
    \begin{center}
        \vspace*{1cm}
 
        \textbf{\fontsize{45}{\baselineskip}\selectfont{\@title}}

        \vspace{4cm}

        {\Large Hecho por}

        \vspace{1cm}

        {\textbf{\LARGE\MakeUppercase{\@author}}}

        \vspace{2cm}

        \includegraphics[width = .8\textwidth]{\logo}

        \vspace{2cm}

        {\Large Estudiante de Matemáticas\\[5pt]

        Escuela Colombiana de Ingeniería Julio Garavito\\[5pt]

        Colombia\\[5pt]

        \today}
             
    \end{center}
\end{titlepage}

\tableofcontents
\clearpage

\sect{Punto 1}
\subsect{Teorema 4.33.3}
\begin{logicenv}{Teo 4.33.3}
    \begin{align*}
            \res{(\phi \to \psi) \land (\psi \to \tau)}\\
        \why{Teo 4.28.1}\\
            \res{((\neg\phi \to \psi) \land (\neg\psi \lor \tau))}\\
        \why[\Rightarrow]{Corte}\\
            \res{\neg\phi \lor \tau}\\
        \why{Teo 4.28.1}\\
            \res{\phi \to \tau}
    \end{align*}
    Por MT 5.5.1 se demuestra que\\
    $\theo{}{DS}{((\phi \to \psi) \land (\psi \to \tau)) \to (\phi \to \tau)}$
\end{logicenv}

\subsect{Teorema 4.36.2}
\begin{logicenv}{Teo 4.36.2}
    \begin{align*}
            \res{(\phi \equiv \psi) \land (\psi \to \tau)}\\
        \why{Def($\equiv$)}\\
            \res{(\phi \to \psi) \land (\psi \to \phi) \land (\psi \to \tau)}\\
        \why{Def.Transitividad($\to$)}\\
            \res{(\psi \to \phi) \land (\phi \to \tau)}\\
        \why{Debilitamiento($\land$)}\\
            \res{\phi \to \tau}
    \end{align*}
    Por Mt 5.5.1 se demuestra que\\
    $\theo{}{DS}{(\phi \equiv \psi) \land (\psi \to \tau) \to (\phi \to \tau)}$
\end{logicenv}

\sect{Punto 2}
\subsect{Teorema 4.31.4}
\begin{logicenv}{Teo 4.31.4}
    \begin{logic}
        \phi \equiv \psi & Suposición del antecedente\\%0
        (\phi \to \psi) \land (\psi \to \phi) & Def($\equiv$), Ecuanimidad(p0)\\%1
        \phi \to \psi & Debilitamiento($\land$)
    \end{logic}
    Por suposición del antecedente se demuestra que\\
    $\theo{}{DS}{(\phi \equiv \psi) \to (\phi \to \psi)}$
\end{logicenv}

\subsect{Teorema 4.33.1}
\begin{logicenv}{Teo 4.33.1}
    \begin{logic}
        \phi & Suposición del antecedente
    \end{logic}
    Por suposición del antecedente se demuestra que\\
    $\theo{}{DS}{\phi \to \phi}$    
\end{logicenv}
\end{document}