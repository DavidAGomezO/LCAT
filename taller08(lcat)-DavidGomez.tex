\documentclass{article}

% Librerías:
\usepackage{amsmath, mathtools, amssymb, mathrsfs, amsthm, nicematrix, array, logicDG}
\usepackage{lmodern, graphicx, fancyhdr}
\usepackage[margin = 1cm, top = 2.5cm, bottom = 2.5cm, includefoot]{geometry}
\usepackage{xcolor}
\usepackage[hidelinks]{hyperref}
\usepackage[T1]{fontenc}
\usepackage[spanish]{babel}
\usepackage{listings}
\usepackage{tikz}
\usetikzlibrary{shapes, fit, tikzmark}
\usepackage{color, colortbl}
\usepackage[most]{tcolorbox}

% Configuraciones:
\pagestyle{fancy}
\fancyhf{}
\setlength{\headheight}{72pt}
\rhead{\textit{\author}}
\lhead{\includegraphics[width = 4cm]{\logo}}
\lfoot{Página \thepage}
\rfoot{\titlename}
\renewcommand{\headrule}{\hbox to \headwidth{\color{rojoEci}\leaders\hrule height \headrulewidth\hfill}}
\renewcommand{\footrulewidth}{0.4pt}

\hyphenpenalty=10000

\newcommand{\logo}{C:/Users/usuario/Documents/U/logo-eci.jpg}

\newcommand{\q}[1]{``#1''}
\newcommand{\und}[2]{#1\textbf{\_}#2}
\newcommand{\boolun}[2][]{H_{#2}(#1)}
\newcommand{\boolbin}[3]{H_{#1}(#2, #3)}
\newcommand{\val}[2]{\mathbf{#1}[#2]}

\setlength{\parindent}{0pt}

%%%%%%%%%%%%%%%%%%%%%%%%%%%%%%%%%%
%%%%%%%%%%%%%%%%%%%%%%%%%%%%%%%%%%
\newcommand{\titlename}{Taller 08}%
\renewcommand{\author}{David Gómez}%
%%%%%%%%%%%%%%%%%%%%%%%%%%%%%%%%%%
%%%%%%%%%%%%%%%%%%%%%%%%%%%%%%%%%%


% Colores
\definecolor{rojoEci}{RGB}{225, 70, 49}
\definecolor{defini}{HTML}{ede9e6}
\definecolor{deftitlmarg}{HTML}{cfcfcf}

% Documento
\begin{document}

\begin{titlepage}
    \begin{center}
        \vspace*{1cm}

        \textbf{\Huge{\titlename}}

        \vspace{1.5cm}

        \textbf{\large{\author}
}
        \vspace{4cm}

        \includegraphics[width=.8\textwidth]{\logo}

        \vspace{4cm}

        Matemáticas\linebreak
        Escuela Colombiana de Ingeniería Julio Garavito\linebreak
        Colombia\linebreak
        \today

    \end{center}
\end{titlepage}
\clearpage
\tableofcontents
\clearpage

\section{Punto 1}
\begin{logicenv}{$\vdash_{\text{DS}} (\neg \textrm{\textit{false}})$}
    \begin{logic}
        ((\neg \textrm{\textit{false}}) \equiv \textrm{\textit{true}}) & Teo 4.15.2\\
        (\neg \textrm{\textit{false}}) & Identidad (p0)
    \end{logic}
\end{logicenv}

\section{Punto 2}
\begin{logicenv}[5]{$\vdash_{\text{DS}} ((\phi \not\equiv (\psi \not\equiv \tau)) \equiv ((\phi \not\equiv \psi) \not\equiv \tau))$}
    \begin{logic}
        ((\phi \not\equiv (\psi \not\equiv \tau)) \equiv (\phi \not\equiv (\psi \not\equiv \tau)))  & Teo 4.6.3\\
        ((\phi \not\equiv (\psi \not\equiv \tau)) \equiv ((\neg \phi) \equiv (\psi \not\equiv \tau))) & Def($\not\equiv$)\\
        (((\neg \phi) \equiv (\psi \not\equiv \tau)) \equiv ((\neg \phi) \equiv ((\neg \psi) \equiv \tau))) & Def($\not\equiv$), Leibniz ($\phi = ((\neg \phi) \equiv p)$)\\
        (((\neg \phi) \equiv ((\neg \psi) \equiv \tau)) \equiv ((\neg \phi) \equiv (\psi \equiv (\neg \tau)))) & Teo 4.15.5, Leibniz ($\phi = ((\neg \phi) \equiv p)$)\\
        (((\neg \phi) \equiv (\psi \equiv (\neg \tau))) \equiv (((\neg \phi) \equiv \psi) \equiv (\neg \tau))) & Asociativa($\equiv$)\\
        ((((\neg \phi) \equiv \psi) \equiv (\neg \tau)) \equiv ((\neg ((\neg \phi) \equiv \psi)) \equiv \tau)) & Teo 4.15.5\\
        (((\neg ((\neg \phi) \equiv \psi)) \equiv \tau) \equiv ((\neg (\phi \not\equiv \psi)) \equiv \tau)) & Def($\not\equiv$)\\
        (((\neg (\phi \not\equiv \psi)) \equiv \tau) \equiv ((\phi \not\equiv \psi) \equiv \tau)) & Def($\equiv$)\\
        ((\phi \not\equiv (\psi \not\equiv \tau)) \equiv ((\phi \not\equiv \psi) \equiv \tau)) & Transitividad(p7,p6,p5,p4,p3,p2,p1,p0)
    \end{logic}
\end{logicenv}

\section{Punto 3}
\begin{logicenv}{$\vdash_{\text{DS}} ((\phi \not\equiv (\psi \not\equiv \tau)) \equiv ((\phi \not\equiv \psi) \not\equiv \tau))$}
    \begin{derivation}
            (\phi \not\equiv (\psi \not\equiv \tau))\\
            Def.($\neg)$\\
            ((\neg \phi) \equiv (\psi \not\equiv \tau))\\
            Def.($\not\equiv$), Leibniz ($\phi = ((\neg \phi) \equiv p)$)\\
            ((\neg \phi) \equiv ((\neg \psi) \equiv \tau))\\
            Teo 4.15.5, Leibniz ($\phi = ((\neg \phi) \equiv p)$)\\
            ((\neg \phi) \equiv (\psi \equiv (\neg \tau)))\\
            Asociativa($\equiv$)\\
            (((\neg \phi) \equiv \psi) \equiv (\neg \tau))\\
            Teo 4.15.5\\
            ((\neg ((\neg \phi) \equiv \psi)) \equiv \tau)\\
            Def.($\not\equiv$)\\
            ((\neg (\phi \not\equiv \psi)) \equiv \tau)\\
            Def.($\not\equiv$)\\
            ((\phi \not\equiv \psi) \not\equiv \tau)
    \end{derivation}
    Por MT 4.21 se demuestra que\\
    $\vdash_{\text{DS}} ((\phi \not\equiv (\psi \not\equiv \tau)) \equiv ((\phi \not\equiv \psi) \not\equiv \tau))$
\end{logicenv}

\section{Punto 4}
\begin{logicenv}{$\vdash_{\text{DS}} ((\phi \lor \textrm{\textit{true}}) \equiv \textrm{\textit{true}})$}
    \begin{derivation}
        (\phi \lor \textrm{\textit{true}})\\
        Teo 4.6.2\\
        (\phi \lor (\textrm{\textit{true}} \equiv \textrm{\textit{true}}))\\
        Distribución ($\lor$, $\equiv$)\\
        ((\phi \lor \textrm{\textit{true}}) \equiv (\phi \lor \textrm{\textit{true}}))\\
        Teo 4.6.2\\
        \textrm{\textit{true}}
    \end{derivation}
    Por MT 4.21 se demuestra que\\
    $\vdash_{\text{DS}} ((\phi \lor \textrm{\textit{true}}) \equiv \textrm{\textit{true}})$
\end{logicenv}

\section{Punto 5}
\begin{logicenv}{$\vdash_{\text{DS}} ((\phi \lor \psi) \equiv ((\phi \lor (\neg \psi)) \equiv \phi))$}
    \begin{derivation}
        (\phi \lor \psi)\\
        Identidad\\
        ((\phi \lor \psi) \equiv \textrm{\textit{true}})\\
        Teo 4.6.2\\
        ((\phi \lor \psi) \equiv (\phi \equiv \phi))\\
        Asociativa($\equiv$)\\
        (((\phi \lor \psi) \equiv \phi) \equiv \phi)\\
        Identidad($\lor$)\\
        (((\phi \lor \psi) \equiv (\phi \lor \textrm{\textit{false}})) \equiv \phi)\\
        Distribución($\lor$, $\equiv$)\\
        ((\phi \lor (\psi \equiv \textrm{\textit{false}})) \equiv \phi)\\
        Def.($\neg$)\\
        ((\phi \lor (\neg \psi)) \equiv \phi)
    \end{derivation}
    Por MT 4.21 se demuestra que\\
    $\vdash_{\text{DS}} ((\phi \lor \psi) \equiv ((\phi \lor (\neg \psi)) \equiv \phi))$
\end{logicenv}

\section{Punto 6}
\begin{logicenv}[2]{$\vdash_{\text{DS}} ((\neg  (\phi \lor \psi)) \equiv ((\neg \phi) \land (\neg \psi)))$}
    \begin{derivation}
        (\neg(\phi \lor \psi))\\
        Teo 4.19.4\\
        (\neg ((\phi \lor (\neg \psi)) \equiv \phi))\\
        Teo 4.15.4\\
        ((\neg(\phi \lor (\neg \psi))) \equiv \phi)\\
        Teo 4.15.5\\
        ((\phi \lor (\neg \psi)) \equiv (\neg \phi))\\
        Conmutativa($\equiv$)\\
        ((\neg \phi) \equiv (\phi \lor (\neg \psi)))\\
        Conmutativa($\equiv$), Leibniz ($\phi = ((\neg \phi) \equiv \phi)$)\\
        ((\neg \phi) \equiv ((\neg \psi) \lor \phi))\\
        Teo 4.19.4, Leibniz ($\phi = ((\neg \phi) \equiv p)$)\\
        ((\neg \phi) \equiv (((\neg \psi) \lor (\neg \phi)) \equiv (\neg \psi)))\\
        Conmutativa($\equiv$), Leibniz ($\phi = ((\neg \phi) \equiv p)$)\\
        ((\neg \phi) \equiv ((\neg \psi) \equiv ((\neg \psi) \lor (\neg \phi))))\\
        Conmutativa($\equiv$), Leibniz ($\phi = (\phi = ((\neg \phi) \equiv ((\neg \psi) \equiv p)))$)\\
        ((\neg \phi) \equiv ((\neg \psi) \equiv ((\neg \phi) \lor (\neg \psi))))\\
        Def.($\land$)\\
        ((\neg \phi) \land (\neg \psi))
    \end{derivation}
    Por MT 4.21 se demuestra que\\
    $\vdash_{\text{DS}} ((\neg  (\phi \lor \psi)) \equiv ((\neg \phi) \land (\neg \psi)))$
\end{logicenv}

\section{Punto 7}
\begin{logicenv}[2]{$\vdash_{\text{DS}} ((\phi \land (\psi \not\equiv \tau)) \equiv ((\phi \land \psi) \not\equiv (\phi \land \tau)))$}
    \begin{derivation}
            ((\phi \land \psi) \not\equiv (\phi \land \tau))\\
        Def.($\not\equiv$)\\
            ((\neg (\phi \land \psi)) \equiv (\phi \land \tau))\\
        Teo 4.15.4\\
            (\neg ((\phi \land \psi) \equiv (\phi \land \psi)))\\
        Def.($\land$), Lbz($\phi = (\neg (p \equiv (\phi \land \tau)))$), Lbz($\phi = (\neg ((\phi \equiv (\psi \equiv (\phi \lor \psi))) \equiv p))$)\\
            (\neg ((\phi \equiv (\psi \equiv (\phi \lor \psi))) \equiv (\phi \equiv (\tau \equiv (\phi \lor \tau)))))\\
        Asociativa($\equiv$), Leibniz($\phi = (\neg p)$)\\
            (\neg (\phi \equiv ((\psi \equiv (\phi \lor \psi)) \equiv (\phi \equiv (\tau \equiv (\phi \lor \tau))))))\\
        Conmutativa($\equiv$), Leibniz($\phi = (\neg (\phi \equiv p))$)\\
            (\neg (\phi \equiv ((\phi \equiv (\tau \equiv (\phi \lor \tau))) \equiv (\psi \equiv (\phi \lor \psi)))))\\
        Asociativa($\equiv$), Leibniz($\phi = (\neg (\phi \equiv p))$)\\
            (\neg (\phi \equiv (\phi \equiv ((\tau \equiv (\phi \lor \tau)) \equiv (\psi \equiv (\phi \lor \psi))))))\\
        Asociativa($\equiv$), Leibniz($\phi = (\neg p)$)\\
            (\neg ((\phi \equiv \phi) \equiv ((\tau \equiv (\phi \lor \tau)) \equiv (\psi \equiv (\phi \lor \psi)))))\\
        Teo 4.15.6, Teo 4.6.4\\
            (\neg (\textrm{\textit{true}} \equiv ((\tau \equiv ((\neg (\neg \phi)) \lor \tau)) \equiv (\psi \equiv (\neg (\neg \phi)) \lor \psi))))\\
        Distribución($\lor$, $\equiv$), Leibniz($\phi = (\neg (\textrm{\textit{true}} \equiv p))$)\\
            (\neg (\textrm{\textit{true}} \equiv ((\tau \lor (\neg \phi)) \equiv (\psi \lor (\neg \phi)))))\\
        Teo 4.15.2, Teo 4.15.4\\
            (\textrm{\textit{false}} \equiv ((\tau \lor (\neg \phi)) \equiv (\psi \lor (\neg \phi))))\\
        Def($\neg$), Conmutativa($\equiv$)\\
            (\neg ((\tau \lor (\neg \phi)) \equiv (\psi \lor (\neg \phi))))\\
        Distribución($\lor$, $\equiv$)\\
            (\neg ((\neg \phi) \lor (\tau \equiv \psi)))\\
        Teo 4.25.4, Teo 4.16.6\\
            (\phi \land (\neg (\tau \equiv \psi)))\\
        Conmutativa($\equiv$), Leibniz($\phi = (\phi \land (\neg p))$)\\
            (\phi \land (\neg (\psi \equiv \tau)))\\
        Def.($\equiv$), Teo 4.15.4\\
            (\phi \land (\psi \not\equiv \tau))
    \end{derivation}
    Por MT 4.21 y por conmutatividad de ($\equiv$) se demuestra que\\
    $\vdash_{\text{DS}} ((\phi \land (\psi \not\equiv \tau)) \equiv ((\phi \land \psi) \not\equiv (\phi \land \tau)))$
\end{logicenv}

\section{Punto 8}
\begin{logicenv}[2]{$\vdash_{\text{DS}} ((\phi \lor (\psi \land \tau)) \equiv ((\phi \lor \psi) \land (\phi \lor \tau)))$}
    \begin{derivation}
            (\phi \lor (\psi \land \tau))\\
        Def.($\land$), Leibniz($\phi = (\phi \lor p)$)\\
            (\phi \lor (\psi \equiv (\tau \equiv (\psi \lor \tau))))\\
        Distribución($\lor$, $\equiv$)\\
            ((\phi \lor \psi) \equiv (\phi \lor (\tau \equiv (\psi \lor \tau))))\\
        Idempotencia($\lor$), Leibniz($\phi = ((\phi \lor \psi) \equiv ((\phi \lor \psi) \equiv (p \lor (\psi \lor \tau))))$)\\
            ((\phi \lor \psi) \equiv ((\phi \lor \tau) \equiv ((\phi \lor \phi) \lor (\psi \lor \tau))))\\
        Asociativa($\lor$), Leibniz($\phi = ((\phi \lor \psi) \equiv ((\phi \lor \psi) \equiv p))$)\\
            ((\phi \lor \psi) \equiv ((\phi \lor \psi) \equiv (\phi \lor (\phi \lor (\psi \lor \tau)))))\\
        Asociativa($\lor$), Leibniz($\phi = ((\phi \lor \psi) \equiv ((\phi \lor \psi) \equiv (\phi \lor p)))$)\\
            ((\phi \lor \psi) \equiv ((\phi \lor \psi) \equiv (\phi \lor ((\phi \lor \psi) \lor \tau))))\\
        Conmutativa($\lor$), Leibniz($\phi = ((\phi \lor \psi) \equiv ((\phi \lor \psi) \equiv (\phi \lor p)))$)\\
            ((\phi \lor \psi) \equiv ((\phi \lor \psi) \equiv (\phi \lor (\tau \lor (\phi \lor \psi)))))\\
        Asociativa($\lor$), Leibniz($\phi = ((\phi \lor \psi) \equiv ((\phi \lor \psi) \equiv p))$)\\
            ((\phi \lor \psi) \equiv ((\phi \lor \psi) \equiv ((\phi \lor \tau) \lor (\phi \lor \psi))))\\
        Def.($\land$)\\
            ((\phi \lor \psi) \land (\phi \lor \psi))
    \end{derivation}
    Por MT 4.21 se demuestra que\\
    $\vdash_{\text{DS}} ((\phi \lor (\psi \land \tau)) \equiv ((\phi \lor \psi) \land (\phi \lor \tau)))$
\end{logicenv}
\end{document}