\documentclass{article}

% Librerías:
\usepackage{amsmath, mathtools, amssymb, mathrsfs, amsthm, nicematrix}
\usepackage{lmodern, graphicx, fancyhdr}
\usepackage[margin = 2cm, top = 2.5cm, includefoot]{geometry}
\usepackage{xcolor}
\usepackage[hidelinks]{hyperref}
\usepackage[T1]{fontenc}
\usepackage[spanish]{babel}
\usepackage{listings}
\usepackage{tikz}
\usetikzlibrary{shapes, snakes, fit, tikzmark}
\usepackage{color, colortbl}

% Configuraciones:
\pagestyle{fancy}
\fancyhf{}
\setlength{\headheight}{55.34027pt}
\rhead{\textit{David Gómez}}
\lhead{\includegraphics[width = 4cm]{\logo}}
\lfoot{Página \thepage}
\rfoot{Taller 02}
\renewcommand{\headrule}{\hbox to \headwidth{\color{rojoEci}\leaders\hrule height \headrulewidth\hfill}}
\renewcommand{\footrulewidth}{0.4pt}

\hyphenpenalty=10000

\newcommand{\logo}{C:/Users/usuario/Documents/U/logo-eci.jpg}

\newcommand{\q}[1]{``#1''}
\newcommand{\und}[2]{#1\textbf{\_}#2}

\newcommand\drawCodeBox[2]{%
\begin{tikzpicture}[remember picture,overlay]
  \coordinate (start) at ([yshift = 2ex]pic cs:#1);
  \coordinate (end) at ([yshift = -1.5ex]pic cs:#2);
  \node[inner sep=2pt,draw=black,fit=(start) (end)] {};
\end{tikzpicture}
}

% Colores
\definecolor{rojoEci}{RGB}{225, 70, 49}


% Documento
\begin{document}
\begin{titlepage}
	\begin{center}
		\vspace*{1cm}

		\textbf{\Huge{Tarea 03}}

		\vspace{1.5cm}

		\textbf{\large{David Gómez}}

		\vspace{4cm}

		\includegraphics[width=\textwidth]{\logo}

		\vspace{5cm}

		Matemáticas\\
		Escuela Colombiana de Ingeniería Julio Garavito\\
		Colombia\\
		\today

	\end{center}
\end{titlepage}
\clearpage
\tableofcontents
\clearpage
\section{Sección 2.1}
\subsection{Punto 1: Nombre todas las funciones booleanas unarias}
\begin{itemize}
	\item $H_{\lnot}$
	\item $H_{\textrm{\textit{false}}}$
	\item $H_{\textrm{\textit{true}}}$
\end{itemize}
\subsection{Punto 3: Demuestre que $H_{\not\equiv} = (H_{\lnot} \circ H_{\equiv})$}
\begin{NiceTabular}{| c | c | c |}
	\hline
	$\phi$      & $\psi$      & $(\phi \not\equiv \psi)$ \\
	\hline
	\ttfamily T & \ttfamily T & \ttfamily F              \\
	\ttfamily T & \ttfamily F & \ttfamily T              \\
	\ttfamily F & \ttfamily T & \ttfamily T              \\
	\ttfamily F & \ttfamily F & \ttfamily F              \\
	\hline
\end{NiceTabular}
\hspace*{1cm}
\begin{NiceTabular}{| c | c | c |}
	\hline
	$\phi$      & $\psi$      & $(\phi \equiv \psi)$ \\
	\hline
	\ttfamily T & \ttfamily T & \ttfamily T          \\
	\ttfamily T & \ttfamily F & \ttfamily F          \\
	\ttfamily F & \ttfamily T & \ttfamily F          \\
	\ttfamily F & \ttfamily F & \ttfamily T          \\
	\hline
\end{NiceTabular}
\hspace*{1cm}
\begin{NiceTabular}{| c | c | c | c |}
	\hline
	$\phi$      & $\psi$      & $(\phi \equiv \psi)$ & $(\lnot(\phi \equiv \psi))$ \\
	\hline
	\ttfamily T & \ttfamily T & \ttfamily T          & \ttfamily F                 \\
	\ttfamily T & \ttfamily F & \ttfamily F          & \ttfamily T                 \\
	\ttfamily F & \ttfamily T & \ttfamily F          & \ttfamily F                 \\
	\ttfamily F & \ttfamily F & \ttfamily T          & \ttfamily T                 \\
	\hline
\end{NiceTabular}
\\[-0.4cm]
\begin{figure}[h]
	\centering
	\begin{tikzpicture}[node distance = 2cm, every node/.style={transform shape}]
		\node[draw] (a) at (-6.5,0) {$H_{\not\equiv}$};
		\node[draw] (b) at (3.7,0) {$H_{\lnot} \circ H_{\equiv}$};
		\node[draw = none] (c) at (7,0){};
		%
		\draw (a) -- (b);
		\draw (a) -- (-6.5, 1);
		\draw (b) -- (3.7, 1);
	\end{tikzpicture}
\end{figure}
\\
\subsection{Punto 4: Tabla de verdad}
\subsubsection{a) $(\textrm{\textit{true}} \not\equiv \textrm{\textit{false}})$}
\begin{center}
    \begin{NiceTabular}{|c|c|c|}
        \hline
        \textit{true} & \textit{false} & $(\textrm{\textit{true}} \not\equiv \textrm{\textit{false}})$ \\
        \hline
        \ttfamily T   & \ttfamily F    & \ttfamily T                                                   \\
        \hline
    \end{NiceTabular}
\end{center}
\subsubsection{d) $(p \wedge (\lnot q))$}
\begin{center}
    \begin{NiceTabular}{|c|c|c|c|}
        \hline
        $p$         & $q$         & $(\lnot q)$ & $(p \wedge (\lnot q))$ \\
        \hline
        \ttfamily T & \ttfamily F & \ttfamily T & \ttfamily T            \\
        \ttfamily T & \ttfamily T & \ttfamily F & \ttfamily F            \\
        \ttfamily F & \ttfamily F & \ttfamily T & \ttfamily F            \\
        \ttfamily F & \ttfamily T & \ttfamily F & \ttfamily F            \\
        \hline
    \end{NiceTabular}
\end{center}
\subsubsection{j) $((p \wedge q) \to p)$}
\begin{center}
    \begin{NiceTabular}{|c|c|c|c|c|}
        \hline
        $p$         & $q$         & $(p \wedge q)$ & $((p \wedge q) \to p)$ \\
        \hline
        \ttfamily T & \ttfamily T & \ttfamily T    & \ttfamily T            \\
        \ttfamily T & \ttfamily F & \ttfamily F    & \ttfamily T            \\
        \ttfamily F & \ttfamily T & \ttfamily F    & \ttfamily F            \\
        \ttfamily F & \ttfamily F & \ttfamily F    & \ttfamily T            \\
        \hline
    \end{NiceTabular}
\end{center}
\subsubsection{k) $(p \to (p \wedge q))$}
\begin{center}
    \begin{NiceTabular}{|c|c|c|c|}
        \hline
        $p$         & $q$         & $(p \wedge q)$ & $(p \to (p \wedge q))$ \\
        \hline
        \ttfamily T & \ttfamily T & \ttfamily T    & \ttfamily T            \\
        \ttfamily T & \ttfamily F & \ttfamily F    & \ttfamily F            \\
        \ttfamily F & \ttfamily T & \ttfamily F    & \ttfamily T            \\
        \ttfamily F & \ttfamily F & \ttfamily F    & \ttfamily T            \\
        \hline
    \end{NiceTabular}
\end{center}
\subsubsection{m) $((p \equiv (q \equiv r)) \equiv ((p \equiv q) \equiv r))$}
\begin{center}
    \begin{NiceTabular}{|c|c|c|c|c|c|c|c|}
        \hline
        $p$         & $q$         & $r$         & $(q \equiv r)$ & $(p \equiv q)$ & $(p \equiv (q \equiv r))$ & $((p \equiv q) \equiv r)$ & $((p \equiv (q \equiv r)) \equiv ((p \equiv q) \equiv r))$ \\
        \hline
        \ttfamily T & \ttfamily T & \ttfamily T & \ttfamily T    & \ttfamily T    & \ttfamily T               &  \ttfamily T              & \ttfamily T\\
        \ttfamily T & \ttfamily T & \ttfamily F & \ttfamily F    & \ttfamily T    & \ttfamily F               &  \ttfamily F              & \ttfamily T\\
        \ttfamily T & \ttfamily F & \ttfamily T & \ttfamily F    & \ttfamily F    & \ttfamily F               &  \ttfamily F              & \ttfamily T\\
        \ttfamily T & \ttfamily F & \ttfamily F & \ttfamily T    & \ttfamily F    & \ttfamily T               &  \ttfamily T              & \ttfamily T\\
        \ttfamily F & \ttfamily T & \ttfamily T & \ttfamily T    & \ttfamily F    & \ttfamily F               &  \ttfamily F              & \ttfamily T\\
        \ttfamily F & \ttfamily T & \ttfamily F & \ttfamily F    & \ttfamily F    & \ttfamily T               &  \ttfamily T              & \ttfamily T\\
        \ttfamily F & \ttfamily F & \ttfamily T & \ttfamily F    & \ttfamily T    & \ttfamily T               &  \ttfamily T              & \ttfamily T\\
        \ttfamily F & \ttfamily F & \ttfamily F & \ttfamily T    & \ttfamily T    & \ttfamily F               &  \ttfamily F              & \ttfamily T\\
        \hline
    \end{NiceTabular}
\end{center}
\subsection{Punto 5: Comparar $(p \wedge q)$ y $((p \vee q) \equiv (p \equiv q))$}
\begin{center}
    \begin{NiceTabular}{|c|c|c|c|c|c|c|}
        \hline
        $p$ & $q$ & $(p\wedge q)$ & $(p\vee q)$ & $(p\equiv q)$ & $((p \vee q) \equiv (p \equiv q))$ & $((p\wedge q) \equiv ((p \vee q) \equiv (p \equiv q)))$\\
        \hline
        \ttfamily T & \ttfamily T & \ttfamily T & \ttfamily T & \ttfamily T & \ttfamily T & \ttfamily T\\
        \ttfamily T & \ttfamily F & \ttfamily F & \ttfamily T & \ttfamily F & \ttfamily F & \ttfamily T\\
        \ttfamily F & \ttfamily T & \ttfamily F & \ttfamily T & \ttfamily F & \ttfamily F & \ttfamily T\\
        \ttfamily F & \ttfamily F & \ttfamily F & \ttfamily F & \ttfamily T & \ttfamily F & \ttfamily T\\
        \hline
    \end{NiceTabular}
\end{center}
\subsection{Punto 8: Una no tautología}
\hspace*{1cm}
$(((p \vee q) \to r) \equiv ((p \to q) \wedge (q \to r)))$
\begin{center}
    \begin{NiceTabular}{|c|c|c|c|c|c|c|c|c|}
        \hline
        $p$ & $q$ & $r$ & $(p \vee q)$ & $((p \vee q) \to r)$  & $(p \to q)$ & $(q \to r)$ & $((p \to q) \wedge (q \to r))$ & $(((p \vee q) \to r) \equiv ((p \to q) \wedge (q \to r)))$\\
        \hline
        \ttfamily T &\ttfamily T &\ttfamily T &        \ttfamily T &                \ttfamily T &    \ttfamily T &       \ttfamily T &             \ttfamily T & \ttfamily T \\
        \ttfamily T &\ttfamily T &\ttfamily F &        \ttfamily T &                \ttfamily F &    \ttfamily T &       \ttfamily F &             \ttfamily F & \ttfamily T \\
        \ttfamily T &\ttfamily F &\ttfamily T &        \ttfamily T &                \ttfamily T &    \ttfamily F &       \ttfamily T &             \ttfamily F & \ttfamily F \\
        \ttfamily T &\ttfamily F &\ttfamily F &        \ttfamily T &                \ttfamily F &    \ttfamily F &       \ttfamily T &             \ttfamily F & \ttfamily T \\
        \ttfamily F &\ttfamily T &\ttfamily T &        \ttfamily T &                \ttfamily T &    \ttfamily T &       \ttfamily T &             \ttfamily T & \ttfamily T \\
        \ttfamily F &\ttfamily T &\ttfamily F &        \ttfamily T &                \ttfamily F &    \ttfamily T &       \ttfamily F &             \ttfamily F & \ttfamily T \\
        \ttfamily F &\ttfamily F &\ttfamily T &        \ttfamily F &                \ttfamily T &    \ttfamily T &       \ttfamily T &             \ttfamily T & \ttfamily T \\
        \ttfamily F &\ttfamily F &\ttfamily F &        \ttfamily F &                \ttfamily T &    \ttfamily T &       \ttfamily T &             \ttfamily T & \ttfamily T \\
        \hline
    \end{NiceTabular}
\end{center}
\subsection{Punto 9}
\subsubsection{a) Definir las relaciones}
\begin{itemize}
    \item $R_{\equiv}$
    \begin{equation*}
        R_{\equiv} := \{(x, y) \in \mathbb{B}^2\, \vert\, H_{\equiv}(x, y) = \texttt{T}\}
    \end{equation*}
    \item $R_{\not\equiv}$
    \begin{equation*}
        R_{\not\equiv} := \{(x, y) \in \mathbb{B}^2\, \vert\, H_{\not\equiv}(x, y) = \texttt{T}\}
    \end{equation*}
    \item $R_{\vee}$
    \begin{equation*}
        R_{\vee} := \{(x, y) \in \mathbb{B}^2\, \vert\, H_{\vee}(x, y) = \texttt{T}\}
    \end{equation*}
    \item $R_{\wedge}$
    \begin{equation*}
        R_{\wedge} := \{(x, y) \in \mathbb{B}^2\, \vert\, H_{\wedge}(x, y) = \texttt{T}\}
    \end{equation*}
    \item $R_{\to}$
    \begin{equation*}
        R_{\to} := \{(x, y) \in \mathbb{B}^2\, \vert\, H_{\to}(x, y) = \texttt{T}\}
    \end{equation*}
    \item $R_{\gets}$
    \begin{equation*}
        R_{\gets} := \{(x, y) \in \mathbb{B}^2\, \vert\, H_{\gets}(x, y) = \texttt{T}\}
    \end{equation*}
\end{itemize}
\subsubsection{b) Investigar cuando una relación binaria es...}

Supóngase $\mathbb{A}$ un conjunto no vacío y se define la relación $R_{\sim}(x, y) := \{(x, y) \in \mathbb{A}^2 \, \vert\, x \sim y\}$
\begin{itemize}
    \item Asociativa:
    $(\forall x, y, z \in \mathbb{A} : (x \sim y) \sim z = x \sim (y \sim z))$
    \item Conmutativa:
    $(\forall x, y \in \mathbb{A} : x \sim y = y \sim x)$
    \item Reflexiva:
    $(\forall x \in \mathbb{A} : x \sim x)$
    \item Irreflexiva:
    $(\forall x \in \mathbb{A} : (x, x) \notin R_{\sim})$
    \item Asimétrica:
    $(\forall x, y \in \mathbb{A} : ((x, y) \in R_{\sim}) \Rightarrow ((y, x) \notin R_{\sim}))$
    \item Antisimétrica:
    $(\forall x, y \in \mathbb{A} : (x \sim y) \Rightarrow x = y)$
    \item Idempotente:
    $(\forall x \in \mathbb{A} : x \sim x = x)$
    \item Transitiva:
    $(\forall x, y, z \in \mathbb{A} : ((x \sim y) \wedge (y \sim z)) \Rightarrow (x \sim z))$
\end{itemize}
\subsubsection{c) Clasificar las relaciones lógicas}
\begin{itemize}
    \item $R_{\equiv}$
    \begin{itemize}
        \item Asociativa
        \item Conmutativa
        \item Reflexiva
        \item Transitiva
    \end{itemize}
    \item $R_{\not\equiv}$
    \begin{itemize}
        \item Conmutativa
        \item Irreflexiva
    \end{itemize}
    \item $R_{\vee}$
    \begin{itemize}
        \item Asociativa
        \item Conmutativa
        \item Reflexiva
        \item Idempotente
        \item Transitiva
    \end{itemize}
    \item $R_{\wedge}$
    \begin{itemize}
        \item Asociativa
        \item Conmutativa
        \item Reflexiva
        \item Idempotente
        \item Transitiva
    \end{itemize}
    \item $R_{\to}$
    \begin{itemize}
        \item Reflexiva
        \item Transitiva
    \end{itemize}
    \item $R_{\gets}$
    \begin{itemize}
        \item Reflexiva
        \item Transitiva
    \end{itemize}
\end{itemize}
\subsection{Punto 13: Considere 4 cartas...}
Si suponemos el conjunto letras en las cartas $\mathbb{L} = \{\textrm{A}, \textrm{B}, \textrm{C}, \dots , \textrm{Z}\}$, el conjunto de las vocales $\mathbb{V} = \{\textrm{A}, \textrm{E}, \textrm{I}, \textrm{O}, \textrm{U}\}$ , el conjunto de números en las cartas (naturales) $\mathbb{N}$ , la sucesión de los pares $\{S_n\}\ ,\ S_n = 2n\ ,\ n\in \mathbb{N}$ y la sucesión de los impares $\{U_n\}\ ,\ U_n = 2n - 1\ ,\ n \in \mathbb{N}$ entonces el enunciado dice que:
$$(\forall x, y \in \mathbb{L} \times \mathbb{N} : (x \in \mathbb{V}) \to (y \in \{S_n\}))$$
Por contradicción se tendría entonces
\begin{alignat*}{2}
    (\lnot (\forall x, y \in \mathbb{L} \times \mathbb{N} &: (x \in \mathbb{V}) \to (y \in \{S_n\})))\\
    (\exists x, y \in \mathbb{L} \times \mathbb{N} &: (\lnot ((x \in \mathbb{V}) \to (y \in \{S_n\}))))\\
    (\exists x, y \in \mathbb{L} \times \mathbb{N} &: (x \in \mathbb{V}) \wedge (y \notin \{S_n\}))\\
    (\exists x, y \in \mathbb{L} \times \mathbb{N} &: (x \in \mathbb{V}) \wedge (y \in \{U_n\}))
\end{alignat*}
Ya que también se tiene que $((p \to q) \wedge p) \therefore q$ , Entonces hay que voltear la carta de la cual se puede ver la \q{A} y la carta de la cual se puede ver el 3.
La primera debido a que se debe comprobar que el antecedente con valor verdadero equivalga al consecuente con valor verdadero.
La segunda debido a que se debe comprobar que la negación a la proposición tenga un valor falso, y a su vez cumpla con $((p \to q) \wedge (\lnot q)) \therefore (\lnot p)$

\subsection{Punto 14: Juana quiere ir de compras...}
\begin{center}
    \begin{NiceTabular}{l l}
        $p$ &: podar el césped\\
        $l$ &: lavar y secar los platos\\
        $t$ &: doblar las toallas de la cocina\\
        $d$ &: limpiar el polvo\\
        $f$ &: fregar los pisos\\
        $h$ &: hacer mercado\\
        $r$ &: recoger la ropa de la lavandería\\
        $q$ &: ir de compras\\
    \end{NiceTabular}
\end{center}
\subsubsection{a) Especificar las opciones}
$((p \vee (l \wedge t) \vee d \vee f \vee (h \wedge r)) \equiv q)$
\subsubsection{b) Suponiendo que...}
Se tiene entonces:
\begin{center}
    \begin{NiceTabular}{l l}
        $((p \vee (l \wedge t) \vee d \vee f \vee (h \wedge r)) \equiv q)$ & Hipótesis\\
        $(\lnot d)$ & Hipótesis\\
        $(\lnot f)$ & Hipótesis\\
        $(\lnot p)$ & Hipótesis\\
        $(\lnot l)$ & Hipótesis\\
        $t$ & Hipótesis\\
        $(h)$ & Hipótesis\\
        $(\lnot r)$ & Hipótesis\\
        \hline
        $(((\textrm{\textit{false}} \vee (\textrm{\textit{false}} \wedge \textrm{\textit{true}}) \vee \textrm{\textit{false}} \vee \textrm{\textit{false}} \vee (\textrm{\textit{true}} \wedge \textrm{\textit{false}}) )) \equiv q)$ & Aplicación del valor de verdad de las hipótesis\\
        $((\textrm{\textit{false}} \vee \textrm{\textit{false}} \vee \textrm{\textit{false}} \vee \textrm{\textit{false}} \vee \textrm{\textit{false}}) \equiv q)$ & Aplicación de tabla de verdad\\
        $(\textrm{\textit{false}} \equiv q)$ & Aplicación de tabla de verdad\\
        $(\lnot q)$ & definición de equivalencia\\
        \hline
        \hline
        $\therefore$ Juana no puede ir de compras
    \end{NiceTabular}
\end{center}
\section{Sección 2.2}
\subsection{Punto 1: Considere la proposición $\phi$ ...}
$$\phi = ((((\lnot p)\vee q)\equiv(r \to p))\gets(q \vee (\lnot (q \wedge q))))$$
\subsubsection{a) proponga una valuación $\mathbf{v}$ tal que $\mathbf{v}(\phi) = \mathtt{T}$ }

\begin{alignat*}{2}
    (\mathbf{v}[(((\lnot p)\vee q)\equiv(r \to p))] = \mathtt{T}) &\vee (\mathbf{v}[(q \vee (\lnot (q \wedge q)))] = \mathtt{F}) \tag*{negación de Metateorema 2.23}\\
    (\mathbf{v}[((\lnot p) \vee q)] = \mathbf{v}[(r \to p)]) &\vee (\mathbf{v}[q] = \mathbf{v}[(\lnot (q \wedge q))] = \mathtt{F})\\
    \intertext{El antecedente siempre será verdad, por lo que hay que llegar a que la consecuencia también lo sea...}
    (\mathbf{v}[(\lnot p)] = \mathtt{T} \vee \mathbf{v}(q) = \mathtt{T})  &\wedge (\mathbf{v}[r] = \mathtt{F} \vee \mathbf{v}[p] = \mathtt{T})\\
    \intertext{Tomando $p \mapsto \mathtt{T}$ una posible valuación sería:}
    \bar{\mathbf{v}}[\phi] &= \{p \mapsto \mathtt{T}, q \mapsto \mathtt{T}, r \mapsto \mathtt{T}\}
\end{alignat*}
\subsubsection{b)  proponga una valuación $\mathbf{w}$ tal que $\mathbf{w}(\phi) = \mathtt{F}$ }
Tomando lo obtenido en el punto anterior, entonces:
$$\bar{\mathbf{w}}[\phi] = \{p \mapsto \mathtt{T}, q \mapsto \mathtt{F}, r \mapsto \mathtt{T} \}$$
\subsection{Punto 6: Demostrar}
\begin{alignat*}{2}
    \mathbf{v}[(\phi \equiv \phi)] &= \mathtt{T} \tag*{Enunciado/ proposición a probar}\\
    \mathbf{v}[\phi] &= \mathbf{v}[\phi] \tag*{Metateorema 2.23}\\[-0.4cm]
    \makebox[1.5cm]{\hrulefill}&\makebox[1.3cm]{\hrulefill}\\
    \therefore \mathbf{v}[(\phi \equiv \phi)] &= \mathtt{T}
\end{alignat*}
\subsection{Punto 7: Demostrar}
\begin{alignat*}{2}
    \mathbf{v}[(\phi \equiv (\lnot \phi))] &= \mathtt{F} \tag*{Enunciado/ proposición a probar}\\
    \mathbf{v}[\phi] &\not= \mathbf{v}[(\lnot \phi)] \tag*{negación de Metateorema 2.23}\\[-.4cm]
    \makebox[2cm]{\hrulefill}&\makebox[1.9cm]{\hrulefill}\\
    \therefore \mathbf{v}[(\phi \equiv (\lnot \phi))] &= \mathtt{F}
\end{alignat*}
\clearpage
\subsection{Punto 8: Demostrar}
\begin{alignat*}{2}
    \mathbf{v}[(\phi \vee (\lnot \phi))] &= \mathtt{T} \tag*{Enunciado/ proposición a probar}\\
    (\mathbf{v}[\phi] = \mathtt{T}) &\vee (\mathbf{v}[\lnot \phi] = \mathtt{T})\tag*{negación de Metateorema 2.23}
    \intertext{Tomando $\phi \mapsto \mathtt{T}$}
    \mathbf{v}[\lnot \phi] &= \mathtt{F}\\
    H_{\vee}(\mathtt{T}, \mathtt{F}) &= \mathtt{T}
    \intertext{Tomando $\phi \mapsto \mathtt{F}$}
    \mathbf{v}[\lnot \phi] &= \mathtt{T}\\
    H_{\vee}(\mathtt{F}, \mathtt{T}) &= \mathtt{T}\\[-.4cm]
    \makebox[2cm]{\hrulefill}&\makebox[1cm]{\hrulefill}\\
    \therefore \mathbf{v}[(\phi \vee (\lnot \phi))] &= \mathtt{T}
\end{alignat*}
\subsection{Punto 9: Demostrar}
\begin{alignat*}{2}
    \mathbf{v}[(\phi \wedge (\lnot \phi))] &= \mathtt{T} \tag*{Enunciado/ proposición a probar}\\
    (\mathbf{v}[\phi] = \mathtt{F}) &\vee (\mathbf{v}[\lnot \phi] = \mathtt{F})\tag*{negación de Metateorema 2.23}
    \intertext{Tomando $\phi \mapsto \mathtt{T}$}
    \mathbf{v}[\lnot \phi] &= \mathtt{F}\\
    H_{\wedge}(\mathtt{T}, \mathtt{F}) &= \mathtt{F}
    \intertext{Tomando $\phi \mapsto \mathtt{F}$}
    \mathbf{v}[\lnot \phi] &= \mathtt{T}\\
    H_{\wedge}(\mathtt{F}, \mathtt{T}) &= \mathtt{F}\\[-.4cm]
    \makebox[2cm]{\hrulefill}&\makebox[1cm]{\hrulefill}\\
    \therefore \mathbf{v}[(\phi \vee (\lnot \phi))] &= \mathtt{F}
\end{alignat*}
\end{document} 